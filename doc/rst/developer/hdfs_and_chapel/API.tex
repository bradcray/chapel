
\section{API}\label{s:API}

There are four APIs exposed. 
\begin{itemize}
\item A Systems Level API that is the glue code between the Chapel runtime
and the filesystem. ({\tt qio\_plugin\_hdfs.c/h})
\item A user-level IO API that interfaces with the system API, as well as
provides the interface for the MapReduce API.
({\tt HDFS.chpl})
\item A way to parse from the IO interface in Chapel to records in Chapel.
({\tt RecordParser.chpl})
\item A MapReduce API that interfaces with the IO system and
RecordParser in Chapel. ({\tt HDFSiterator.chpl})
\end{itemize}
%
\subsection{Runtime API}
Throughout this section, the names that we use in the description of the function pointer are the names of those
function pointers in the {\tt qio\_file\_functions\_t} struct.
The Runtime API currently consists of Several functions:
\begin{itemize}
\item {\tt readv} Implements a function that has the same semantics as {\tt readv} in
\href{http://pubs.opengroup.org/onlinepubs/9699919799/}{POSIX.1-2008}. It takes
in a user defined filesystem struct which contains all the information that the file system needs
in order to work. The {\tt iovec} argument is already allocated.
\begin{lstlisting}
typedef qioerr (*qio_readv_fptr)
                (void*, // plugin file pointer
                const struct iovec*,   
                // Data to write into
                int,                    
                // number of elements in iovec
                ssize_t*);              
                // Amount that was written into iovec
\end{lstlisting}
\item {\tt writev} Implements the same semantics as {\tt writev} in
\href{http://pubs.opengroup.org/onlinepubs/9699919799/}{POSIX.1-2008}. 
\begin{lstlisting}
typedef qioerr (*qio_writev_fptr) 
                (void*, // plugin fp
                const struct iovec*,      
                // data to write from
                int,                      
                // Number of elements in iovec
                ssize_t*);                
                // Amount written on return
\end{lstlisting}
\item {\tt preadv} Implements a function which does a positional read of the file 
and puts the results in the {\tt iovec} argument which is already allocated. 
\begin{lstlisting}
typedef qioerr (*qio_preadv_fptr) 
                (void*, // plugin fp
                const struct iovec*,      
                // Data to write into
                int,                      
                // number of elements in iovec
                off_t,                    
                // Offset to read from
                ssize_t*);                
                // Amount that was written into iovec
\end{lstlisting}
\item {\tt pwritev} does a positional write of the file that is referenced by the plugin
filepointer from the contents stored in the argument to {\tt iovec}. It returns the
number of bytes written on return.
\begin{lstlisting}
typedef qioerr (*qio_pwritev_fptr) 
                (void*,//plugin fp
                const struct iovec*,      
                // data to write from
                int,                      
                // Number of elements in iovec 
                off_t,                    
                // offset to write 
                ssize_t*);                
                // Amount written on return
\end{lstlisting}
\item {\tt seek} Implements the semantics of {\tt lseek} in 
\href{http://pubs.opengroup.org/onlinepubs/9699919799/}{POSIX.1-2008}.
\begin{lstlisting}
typedef qioerr (*qio_seek_fptr)
                (void*,  // plugin fp
                 off_t,  // offset to seek from
                 int,    // Amount to seek
                 off_t*);// Offset on return from seek
\end{lstlisting}
\item {\tt filelength} Returns the length of the file in bytes that is referenced by the
plugin filepointer.
\begin{lstlisting}
typedef qioerr (*qio_filelength_fptr)
                (void*,     // file information 
                 int64_t*); // length on return
\end{lstlisting}
\item {\tt getpath} Returns the path to the file that is referenced by the plugin
filepointer.
\begin{lstlisting}
typedef qioerr (*qio_getpath_fptr)
                (void*, // file information
                 const char**);// string/path on return
\end{lstlisting}
\item {\tt open} Opens the file on the configured filesystem pointed to by the path passed in as
the second argument. The plugin filepointer (which is user defined) is finished being
populated here. The user also at this point needs to also set the flags for the file
(for more information see ~\ref{s:flags} and ~\ref{s:hints}). {\tt open} expects a configured filesystem
passed in as the last argument to the function.
\begin{lstlisting}
typedef qioerr (*qio_open_fptr)
                (void**,  // the plugin fp on return
                 const char*, // pathname to file
                 int*, // flags out
                 mode_t,  // mode
                 qio_hint_t, // Hints for opening the file
                 void*);  // The configured filesystem
\end{lstlisting}
\item {\tt close} Closes the file pointed to by the plugin filepointer. Has the same
semantics as {\tt close} in
\href{http://pubs.opengroup.org/onlinepubs/9699919799/}{POSIX.1-2008}.
\begin{lstlisting}
typedef qioerr (*qio_close_fptr)
                (void*); // file fp
\end{lstlisting}
\item {\tt fsync} Provides the same functionality as {\tt fsync} in 
\href{http://pubs.opengroup.org/onlinepubs/9699919799/}{POSIX.1-2008}.
\begin{lstlisting}
typedef qioerr (*qio_fsync_fptr)
                (void*); // file information
\end{lstlisting}
\item {\tt getcwd} replicates the semantics of {\tt getcwd} in 
\href{http://pubs.opengroup.org/onlinepubs/9699919799/}{POSIX.1-2008}.
\begin{lstlisting}
typedef qioerr (*qio_getcwd_fptr)
                (void*, // file information
                 const char**); // path on return
\end{lstlisting}
\item The struct {\tt qio\_file\_functions\_t} represents all the functions needed within
QIO in order to implement the functionality needed for file IO in Chapel. This is
loaded into the QIO representation of the file at initialization and the user is
responsible for populating this struct before passing it into the QIO runtime code.
\begin{lstlisting}
typedef struct qio_file_functions_s {
  qio_writev_fptr  writev; 
  qio_readv_fptr   readv;

  qio_pwritev_fptr pwritev;
  qio_preadv_fptr  preadv;

  qio_close_fptr   close;
  qio_open_fptr    open;

  qio_seek_fptr   seek;

  qio_filelength_fptr filelength;
  qio_getpath_fptr getpath;

  qio_fsync_fptr fsync;
  qio_getcwd_fptr getcwd;

  void* fs; // Holds the configured filesystem

} qio_file_functions_t;
\end{lstlisting}
\end{itemize}
Where the {\tt void* fs} in {\tt qio\_file\_functions\_t} holds the configured file system. This way we can support
calling functions that are not dependent upon opening the file on a file system (\eg
calling {\tt getpath} or {\tt getcwd}). 

The various types of information needed by the file system in order to read and write
files is passed around as a user defined struct in a {\tt void*} (the plugin fp). This way we can support
any filesystem since we no longer have to worry about the number of
arguments to these functions. Therefore the library writer is responsible for packing
the arguments into - and extracting the arguments from - the plugin fp. 

The library writer is also responsible for writing the wrapper functions
around the calls to the filesystem so that it conforms with this API.
This way we can report appropriate errors as well as supporting as many
filesystems as possible.

The only other function needed in order to create an interface with QIO is to
implement a function that populates the {\tt qio\_file\_functions\_t} struct (the number of
arguments to such a function can be arbitrary and user defined). Hereafter we will
call this function {\tt create\_qio\_functions}. This will then be an
interface to the module level code. The way in which we pass this information through to the QIO runtime is via the function
in runtime called {\tt qio\_file\_open\_access\_usr} which is called by {\tt open} in the Chapel
module level code and is the last argument to the function
(the rest of the arguments are the same as {\tt qio\_file\_open\_access}). 

This now brings us to a discussion of the module level API to the runtime.

The module-level API for the runtime is almost trivial and depends only on
{\tt create\_qio\_functions}. The way the library writer would interface with the runtime 
at the module level would be along the lines of 
\begin{lstlisting}
extern proc create_qio_functions(...):
                 qio_file_functions_t;

proc myOpen(...): file {
  var err: syserr = ENOERR;
  ...
  var fsfns = create_qio_functions(...);
  ...
  err = qio_file_open_access(..., fsfns);
  ...
  }
\end{lstlisting}

\subsection{User API for HDFS}
\subsubsection{Types}
The types defined by the HDFS module are as follows:

\begin{lstlisting}
record hdfsChapelFile {
  var files: [rcDomain] file;
}
\end{lstlisting}
Which is a wrapper around a replicated array of files; one per locale. (\ie a
``Global file'' in a sense)

\begin{lstlisting}
record hdfsChapelFileSystem {
  var home: locale;
  var _internal_file: [rcDomain] c_ptr; 
  // contains hdfsFS
}
\end{lstlisting}
This is almost the same as {\tt hdfsChapelFile}, except this time instead of replicating
a file across each locale, it replicates the configured file system across each locale.

\begin{lstlisting}
record hdfsChapelFile_local {
  var home: locale = here;
  var _internal_:qio_locale_map_ptr_t 
      = QIO_LOCALE_MAP_PTR_T_NULL;
}
\end{lstlisting}
Represents a mapping of a localeId to a specific byte range in the file.

\begin{lstlisting}
record hdfsChapelFileSystem_local {
  var home: locale;
  var _internal_: c_ptr;
}
\end{lstlisting}
Represents a configured file system pointer. 

\subsubsection{Functions}
\begin{lstlisting}
hdfsChapelConnect(name: string, port: int): fs;
\end{lstlisting}
Connects to HDFS with name {\tt name} and port {\tt port} and replicates across all locales
on the machine. This way there is a valid way to reference this other then from the
locale it was called on.

\begin{lstlisting}
fs.hdfsChapelDisconnect();
\end{lstlisting}
Disconnects (on each locale) from the file system {\tt fs} connected to
by {\tt hdfsChapelConnect}

\begin{lstlisting}
fs.hdfsOpen(filename: string, 
                flags: iomode): hdfsChapelFile;
\end{lstlisting}
Opens a file with path {\tt pathname} and in mode {\tt flags} on each locale from the file system
that was connected to via {\tt hdfsConnect}. The only possible iomodes are
{\tt iomode.r} and {\tt iomode.cw} (due to HDFS constraints).

\begin{lstlisting}
hdfsChapelFile.hdfsClose();
\end{lstlisting}
Closes the files created by {\tt fs.hdfsOpen}.

\begin{lstlisting}
hdfsChapelFile.getLocal: file;
\end{lstlisting}
Returns the file for the current locale that you are on when this function is called.

\begin{lstlisting}
hdfs_chapel_connect(path:string, port: int): 
                  hdfsChapelFileSystem_local;
\end{lstlisting}
Same as {\tt hdfsChapelConnect} except that this only creates a valid file system on the
locale it was on when it was called.

\begin{lstlisting}
hdfsChapelFileSystem_local.hdfs_chapel_disconnect();
\end{lstlisting}
Disconnects from HDFS.

\begin{lstlisting}
getHosts(f: file);
\end{lstlisting}
Returns a C array of structs of the form 
\begin{lstlisting}
{(locale_id, start_byte, length), ...}
\end{lstlisting}
which can be accessed via 
\begin{lstlisting}
getLocaleBytes(g: hdfsChapelFile_locale, i: int);
\end{lstlisting}
The only other things added to the current IO functionality is a convenience function
\begin{lstlisting}
hdfsChapelFile.hdfsReader(...): channel;
\end{lstlisting}
Which takes in the same arguments as the standard {\tt file.reader} function.

After this, all other functionality is supported. An example of this is:
\begin{lstlisting}{chapel}
use HDFS;

var gfl: hdfsChapelFile;
var hdfs: fs;

hdfs = hdfsChapelConnect("default", 0);
gfl  = hdfs.hdfsOpen("/tmp/advo.txt", iomode.r);

for loc in Locales {
  on loc {
    var r = gfl.hdfsReader(start=50);
    // same as:
    // var r = gfl.getLocal.reader(start=50);
    var str: string;
    r.readline(str);
    writeln("on locale ", here.id, " string: " + str);
    r.close();
  }
}
on Locales[2] {
  gfl.hdfsClose();
}

on Locales[1] {
  hdfs.hdfsChapelDisconnect();
}

/* outputs:

   on locale 0 string: 0325

   on locale 1 string: 0325

   on locale 2 string: 0325

   on locale 3 string: 0325

 */
\end{lstlisting}

\subsection{Record parser API}
The API for the record parser works with the IO interface in Chapel and
\emph{is not} dependent upon using HDFS or anything else except for the
functions provided in {\tt IO.chpl} and regexp support. The API is as follows:
\begin{lstlisting}
new recordReader(type recordType, reader: channel, 
                 regex: string): recordReader;
\end{lstlisting}
Creates a record reader that parses into the record of type {\tt recordType}
from the channel {\tt reader} using the regexp {\tt regex}
\begin{lstlisting}
new recordReader(type recordType, 
                 reader: channel): recordReader;
\end{lstlisting}
The same as the first one, however this time the regex is inferred from the
field names in the record {\tt recordType}. The regex created this way is very
lax in terms of how much whitespace there is between records. This could lead to naming problems as
well as there might be problems parsing into the record. 

\begin{lstlisting}
recordReader.get(): recordType
\end{lstlisting}
Returns one record and advances the position in the file to the end of
where it read. Will return error if it cannot return one.

\begin{lstlisting}
recordReader.stream(): iter(recordType)
\end{lstlisting}
Returns a stream of records of type {\tt recordType} until the regex no longer
matches. At end, leaves the channel position at the place where it read to.  

An example of how to use this is as follows:
\begin{lstlisting}
use RecordParser;

var fl = open("test.txt", iomode.r);
var ch = fl.reader();

record Test {
  var name: string;
  var id:   int;
}

var regex = "Name: (.*)\\s*Id: (.*)\\n\\n";

var M = new RecordReader(Test, ch, regex);

writeln("get() = ", M.get());

writeln("Now testing stream()");

for m in M.stream {
  writeln(m);
}

ch.close();
fl.close();

/*  Outputs: 
    get() = (name = one, id = 1)
    Now testing stream
    (name = two, id = 2)
    (name = three, id = 3)

    For test.txt = 
    Name: one
    Id: 1

    Name: two
    Id: 2

    Name: three
    Id: 3
*/
\end{lstlisting}

\subsection{MapReduce API}
The API here is the simplest of them all and consists of only one function
\begin{lstlisting}
HDFSiter(path: string, type recordType,
         regex: string): iter(RecordType)
\end{lstlisting}
This is a leader-follower iterator that is locale aware in terms of data
locality (\eg  if blocks 0 and 1 reside on locales 0 and 1 respectively,
it will read on and work on locales 0 and 1 while using those blocks).

Many times what things might look like is:
\begin{lstlisting}
use HDFSiterator;

record someRecord {
  var id1: real;
  var id2: real;
}

var regex = "ID1(.*)\\s*ID2(.*)\\n";

forall r in HDFSiter("/tmp/test.txt", someRecord, regex) {
  <do something with r in here>
}
\end{lstlisting}

\subsection{Flags in QIO}\label{s:flags}
The flags for QIO are fairly straightforward and consist of:
\begin{itemize}
\item {\tt QIO\_FDFLAG\_READABLE} Specifies that this file has been opened in a mode that
supports reading. 
\item {\tt QIO\_FDFLAG\_WRITABLE} Specifies that this file has been opened in a mode that
supports writing
\item {\tt QIO\_FDFLAG\_SEEKABLE} Specifies that this file is seekable.
\end{itemize}

\subsection{Hints in QIO}\label{s:hints}
There are 5 types of hints:
\begin{itemize}
\item {\tt IOHINT\_NONE} Normal operation. We expect to use this most of the time.
\item {\tt QIO\_HINT\_RANDOM} We expect random access to this file.
\item {\tt QIO\_HINT\_SEQUENTIAL} We expect sequential access to this file.
\item {\tt QIO\_HINT\_CACHED} We expect the entire file to be cached and/or pulled in all
at once.
\item {\tt QIO\_HINT\_PARALLEL} We expect many channels to work on this file concurrently.
\end{itemize}
