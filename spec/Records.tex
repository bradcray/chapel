\sekshun{Records}
\label{Records}
\index{records}

A record is a data structure that is similar to a class but instead has value
semantics, similar to primitive types.  Value semantics mean that assignment, argument passing and function
return values are by default all done by copying.  Value semantics also imply that a
variable of record type is associated with only one piece of storage and has
only one type throughout its lifetime.  Storage is allocated for a variable of
record type when the variable declaration is executed, and the record variable
is also initialized at that time.

A record declaration creates a record type~\rsec{Record_Declarations}.  A
variable of record type contains all and only the fields defined by that type
(\rsec{Record_Types}).  Value semantics imply that the type of a record variable
is known at compile time (i.e. it is statically typed).  

Records can be created through an explicit call to a record
constructor, which allocates storage, initializes
it and returns it.  
A record is also created upon a variable declaration of a record type.

A record type is generic if it contains generic fields.  Generic record types
are discussed in detail in~\rsec{Generic_Types}.

\section{Record Declarations}
\label{Record_Declarations}
\index{records!declarations}
\index{declarations!records}
\index{record@\chpl{record}}

A record is defined with the following syntax:
\begin{syntax}
record-declaration-statement:
  simple-record-declaration-statement
  external-record-declaration-statement

simple-record-declaration-statement:
  `record' identifier record-inherit-list[OPT] { record-statement-list }

record-inherit-list:
  : record-type-list

record-type-list:
  record-type
  record-type , record-type-list

record-statement-list:
  record-statement
  record-statement record-statement-list

record-statement:
  variable-declaration-statement
  method-declaration-statement
  type-declaration-statement
  empty-statement
\end{syntax}

A \sntx{record-declaration-statement} defines a new type symbol specified by the
identifier.  A record inherits data and methods from other records
if the \sntx{record-inherit-list} is specified.

\begin{future}
Allowing a record to inherit from more than one record is future work.
\end{future}

\begin{rationale}
We do not allow records to inherit from classes because of the following.

Inheritance implies that the derived type can be cast to one of its base types.
If the base type is a record type, casting to the base type has the effect of
removing all of the data fields and all of functions that are not associated
with the base type.  Thereafter, the record variable has the base record type,
in both compile-time and run-time interpretations.

If the base type were a class type, the result of the cast would have the static
type of the base class while its run-time type was a record type.  Since a
record's type is supposed to be determined at compile time, this is a bit
incongruous with the definition of a record.  Moreover, space would have to be
allocated in this special case, to store the record's run-time type.
\end{rationale}

As in a class declarations, the body of a record declaration can contain
variable, iterator and method declarations as well as nested type declarations.

If a record declaration contains a type alias or parameter field, or it contains
a variable or constant without a specified type and without an initialization
expression, then it declares a generic record type.  Generic record types are
described in~\rsec{Generic_Types}.

If the \chpl{extern} keyword appears before the \chpl{record} keyword, then an
external record type is declared.  An external record type declaration must not
contain a \sntx{record-inherit-list}.  An external record is used within Chapel
for type and field resolution, but no corresponding backend definition is
generated.  It is presumed that the definition of an external record is supplied
by a library or the execution environment.  See the chapter on interoperability
(\rsec{Interoperability}) for more information on external records.
% External record inheritance is not currently supported.

\begin{future}
Privacy controls for classes and records are currently not specified,
as discussion is needed regarding its impact on inheritance, for
instance.
\end{future}

\subsection{Record Types}
\label{Record_Types}
\index{records!record types}
\index{records!types}
\index{types!records}

A record type specifier simply names a record type, using
the following syntax:
\begin{syntax}
record-type:
  identifier
  identifier ( named-expression-list )
\end{syntax}
A record type specifier may appear anywhere a type specifier is permitted.

For non-generic records, the record name by itself is sufficient to specify the
type.  Generic records must be instantiated to serve as a fully-specified
type, for example to declare a variable.  This is done with
type constructors, which are defined in Section~\ref{Type_Constructors}.

\subsection{Record Fields}
\label{Record_Fields}
\index{records!fields}
\index{fields!records}

Variable declarations within a record type declaration define fields within that
record type.  The presence of at least one parameter field causes the record
type to become generic.  Variable fields define the storage associated with a
record.

\begin{chapelexample}{defineActorRecord.chpl}
The code
\begin{chapel}
record ActorRecord {
  var name: string;
  var age: uint;
}
\end{chapel}
\begin{chapeloutput}
\end{chapeloutput}
defines a new record type called \chpl{ActorRecord} that has two fields: the
string field \chpl{name} and the unsigned integer field \chpl{age}.  The data
contained by a record of this type is exactly the same as that contained by
an instance of the \chpl{Actor} class defined in the preceding chapter~\rsec{Class_Fields}.
\end{chapelexample}

\subsection{Record Methods}
\label{Record_Methods}
\index{records!methods}
\index{methods!records}

A record method is a function or iterator that is bound to a record.  Unlike
functions that take a record as an argument, record methods access the record by
reference, so that persistent field updates are possible.

The syntax for record method declarations is identical to that for class method
declarations (\rsec{Class_Methods}).

\subsection{Nested Record Types}
\label{Nested_Record_Types}
\index{nested records}
\index{records!nested}

Record type declarations may be nested within other class, record and union
declarations.  Methods defined in a nested record type may access fields
declared in the containing aggregate type either implicitly, or explicitly by
means of an \chpl{outer} reference.

\section{Record Inheritance}
\label{Record_Inheritance}
\index{records!inheritance}
\index{inheritance!records}

A \emph{derived} record type is a type that inherits from other record types.  For each named
base record type, inheritance effectively inserts all of its fields and methods
into the new record type.

Since record types are resolved statically, there is no type hierarchy implied
by record inheritance.  It is merely a shorthand for including a list of fields
in the record (or class) type being defined.  Record inheritance can be useful
for grouping data common to several or class or record types.

\begin{future}
From the definition of record inheritance, it is apparent that a record of a
derived type can be cast legally to any of its base record types.  But given
their semantics, records can also be legally cast to types with which they have
no inheritance relationship.  Thus, records do not induce a well-defined type
hierarchy.
\end{future}

\begin{chapelexample}{recordInheritance.chpl}
\begin{chapel}
record Center { var x, y: real; }
record Circle : Center {
  var radius: real;
}
record Ellipse : Center {
  var major, minor: real;
}
\end{chapel}
\begin{chapeloutput}
\end{chapeloutput}
The record \chpl{Center} is defined and used as a shorthand in defining
the \chpl{Circle} and \chpl{Ellipse} records.  The \chpl{Circle} record contains
three \chpl{real} fields named \chpl{x}, \chpl{y} and \chpl{radius}.  The
\chpl{Ellipse} record contains four \chpl{real} fields named \chpl{x}, \chpl{y},
\chpl{major} and \chpl{minor}.
\end{chapelexample}

The syntax and semantics for accessing methods (including getter methods and
hence fields) in a base
record type is the same as for accessing fields in a base class (\rsec{Accessing_Base_Class_Fields}).

\subsection{Shadowing Base Record Fields}
\label{Shadowing_Base_Record_Fields}
\index{records!fields!shadowing}

A field in the derived record can be declared with the same name as a
field in a base record.  Such a field shadows the field in the base
record, meaning that the field by the same name in the base record is not
directly accessible.

\begin{openissue}
A syntax for accessing shadowed fields has not yet been specified.
\end{openissue}

\subsection{Overriding Base Record Methods}
\label{Overriding_Base_Record_Methods}
\index{records!base method!overriding}

\begin{future}
If a method in a derived record is declared with a signature identical to that
of a method in a base record, then it is said to override the
base record's method.  Since records do not support dynamic dispatch, method
overriding is the same as method shadowing: When referenced via the derived
record type, the derived type's version of the method is called; when referenced
via the base record type, the base record type's version of the method is called.

The identical signature requires that the names, types, and order of
the formal arguments be identical. The return type of the overriding
method must be the same as the return type of the base record's method,
or must be a subrecord of the base record method's return type.
\end{future}

\section{Record Variable Declarations}
\label{Record_Variable_Declarations}
\index{records!variable declarations}
\index{variables!records}

A record variable declaration is a variable declaration using a record type.
When a variable of record type is declared, storage is allocated sufficient to
store all of the fields defined in that record type.  

In the context of a class or record or union declaration, the fields are
allocated within the object as if they had been declared individually.  In this
sense, records provide a way to group related fields within a containing class
or record type. 

In the context of a function body, a record variable declaration
causes storage to be allocated sufficient to store all of the fields in that
record type.  The record variable is initialized through a call to its
default initializer.  The default initializer for a record is defined in the
same way as the default initializer for a class (\rsec{Default_Initialization}).

\subsection{Storage Allocation}
\label{Record_Storage}
\index{records!allocation}

Storage for a record variable directly contains the data associated
with the fields in the record, in the same manner as variables
of primitive types directly contain the primitive values.
Record storage is reclaimed when the record variable goes out of scope.
No additional storage for a record is allocated or reclaimed.
Field data of one variable's record is not shared with data
of another variable's record.

\subsection{Record Initialization}
\label{Record_Initialization}
\index{records!initialization}
\index{initialization!record}

A variable of a record type declared without an initialization expression
is initialized through a call to the record's default initializer, passing no arguments.
The default initializer for a record is defined in the same way as the default
initializer for a class (\rsec{Default_Initialization}).

If the new record type is derived from other record types, the
default initializer for each base record will be called in lexical order before default
initializer for the record itself.

To construct a record as an expression,
i.e. without binding it to a variable, the \chpl{new} operator is
required.  In this case, storage is allocated and reclaimed as for a record
variable declaration (\rsec{Record_Storage}), except that the temporary record
goes out of scope at the end of the enclosing expression.

To initialize a record variable with a non-default value, it can be assigned
the value of a constructor call expression.  The constructors for a record are
defined in the same way as those for a class (\rsec{Class_Constructors}).

\begin{rationale}
The \chpl{new} keyword disambiguates types from values. This is needed because of the close
relationship between constructors and type specifiers for classes and
records.
\end{rationale}

\begin{chapelexample}{recordCreation.chpl}
The program
\begin{chapel}
record TimeStamp {
  var time: string = "1/1/1011";
}

var timestampDefault: TimeStamp;                  // use the default for 'time'
var timestampCustom = new TimeStamp("2/2/2022");  // ... or a different one
writeln(timestampDefault);
writeln(timestampCustom);

var idCounter = 0;
record UniqueID {
  var id: int;
  proc UniqueID() { idCounter += 1; id = idCounter; }
}

writeln(new UniqueID());  // create and use a record value without a variable
writeln(new UniqueID());
\end{chapel}
produces the output
\begin{chapelprintoutput}{}
(time = 1/1/1011)
(time = 2/2/2022)
(id = 1)
(id = 2)
\end{chapelprintoutput}
The variable \chpl{timestampDefault} is initialized with \chpl{TimeStamp}'s
default initializer. The expression \chpl{new TimeStamp} creates a record that
is assigned to \chpl{timestampCustom}.  It effectively
initializes \chpl{timestampCustom} via a call to the constructor with desired
arguments. The records created with \chpl{new UniqueID()} are discarded after
they are used.
\end{chapelexample}

As with classes, the user can provide his own constructors
(\rsec{User_Defined_Constructors}).  If any user-defined constructors are
supplied, the default initializer cannot be called directly.  

\subsection{Record Destructor}
\label{Record_Destructor}
\index{records!destructor}
\index{destructor!records}

A record author may specify addditional actions to be performed before record storage is
reclaimed by defining a record destructor.  A record destructor is a method that has the
same name as the record prefixed by a tilde.  A record destructor takes no arguments
(aside from the implicit \chpl{this} argument).  If defined, the destructor is called
on a record object after it goes out of scope and before its memory is reclaimed.

% TODO: The above ambiguous language is intended to allow optimizations involving extending
% the lifetime of an object.  However, we leave unspecified the means by which a user may
% demand avid running of the destructor and reclamation of memory (as in C\#).  We need to
% specify this so the above language can be tightened up for that case.

% For now, the actual lifetime of a record object is under the control of the compiler.  For
% example, as an optimization, ownership of an object may be transferred between variables with
% non-overlapping lifetimes.  When this happens, there will be no observable destruction of
% one of those variables.  The compiler may also choose to insert temporary copies e.g. of
% record formals or of a record return value.

% The compiler guarantees that every record constructor call will have exactly one
% corresponding record destructor call.  However, the exact number of constructor-destructor
% pairs is determined by the compiler, and may also be influenced by various compiler
% options.

\begin{chapelexample}{recordDestructor.chpl}
\begin{chapel}
class C { var x: int; } // A class with nonzero size.
// If the class were empty, whether or not its memory was reclaimed 
// would not be observable.

// Defines a record implementing simple memory management.
record R {
  var c: C;
  proc R() { c = new C(0); }
  proc ~R() { delete c; c = nil; }
}

proc foo()
{
  var r: R; // Initialized using default constructor.
  writeln(r);
  // r will go out of scope here.
  // Its destructor will be called to free the C object it contains.
}

foo();
\end{chapel}
\begin{chapeloutput}
(c = {x = 0})

====================
Leaked Memory Report
==============================================================
Number of leaked allocations
           Total leaked memory (bytes)
                      Description of allocation
==============================================================
==============================================================
\end{chapeloutput}
\begin{chapelexecopts}
--memLeaksByType
\end{chapelexecopts}
\end{chapelexample}


\section{Record Arguments}
\label{Record_Arguments}
\index{records!arguments}
\index{arguments!records}

When records are copied into or out of a function's formal argument,
the copy is performed consistently with the semantics described for
record assignment (\rsec{Record_Assignment}).

\begin{chapelexample}{paramPassing.chpl}
The program
\begin{chapel}
record MyColor {
  var color: int;
}
proc printMyColor(in mc: MyColor) {
  writeln("my color is ", mc.color);
  mc.color = 6;   // does not affect the caller's record
}
var mc1: MyColor;        // 'color' defaults to 0
var mc2: MyColor = mc1;  // mc1's value is copied into mc2
mc1.color = 3;           // mc1's value is modified
printMyColor(mc2);       // mc2 is not affected by assignment to mc1
printMyColor(mc2);       // ... or by assignment in printMyColor()

proc modifyMyColor(inout mc: MyColor, newcolor: int) {
  mc.color = newcolor;
}
modifyMyColor(mc2, 7);   // mc2 is affected because of the 'inout' intent
printMyColor(mc2);
\end{chapel}
produces
\begin{chapelprintoutput}{}
my color is 0
my color is 0
my color is 7
\end{chapelprintoutput}
The assignment to \chpl{mc1.color} affects only the record stored
in \chpl{mc1}. The record in \chpl{mc2} is not affected by
the assignment to \chpl{mc1} or by the assignment in \chpl{printMyColor}.
\chpl{mc2} is affected by the assignment in \chpl{modifyMyColor}
because the intent \chpl{inout} is used.
\end{chapelexample}

\section{Record Field Access}
\label{Record_Field_Access}
\index{records!field access}
\index{field access}

A record field is accessed the same way as a class field
(\rsec{Class_Field_Accesses}).  When a field access is used as an
rvalue, the value of that field is returned.  When it is used as
an lvalue, the value of the record field is updated.

Member access expressions that access parameter fields
produce a parameter.

\subsection{Field Getter Methods}
\label{Field_Getter_Methods}
\index{records!getters}

As in classes, field accesses are performed via getter methods
(\rsec{Getter_Methods}).  By default, these methods simply return a reference to
the specified field (so they can be written as well as read).  The user may
redefine these as needed.

\section{Record Method Calls}
\label{Record_Method_Access}
\index{records!method calls}
\index{method calls}

A record method may be invoked the same way as a class method
(\rsec{Class_Method_Calls}).  Unlike class methods, record methods are
resolved at compile time.  

\subsection{The Method Receiver and the {\em this} Argument}
\label{The_this_Reference}
\index{this@\chpl{this}}
\index{records!receiver}
\index{receiver}

The \emph{receiver} of a record method is similar to and is determined in the
same way as the receiver of a class method (\rsec{The_em_this_Reference}).
The type of the receiver is the record in which the method is defined.
The receiver formal argument can be referred to within the method
using the identifier \chpl{this}.

The difference from a class method is that the receiver actual argument,
which must be a record value, is passed to the record method by reference,
rather than by copying. Therefore updates to the receiver made in the
method, if any, are visible outside the method.

\section{The {\em this} Method}
\index{records!indexing}

As with classes, records can be supplied with a \chpl{this} method.  This method
defines the behavior of the indexing operator \chpl{[]}.

\section{The {\em these} Method}
\index{records!iterating}

A \chpl{these} method can be defined for records as well as classes (\rsec{The_these_Method}).  It
provides an iterator which iterates over the contents of the record in a
user-defined manner.

\section{Common Operations}

\subsection{Record Assignment}
\label{Record_Assignment}
\index{records!assignment}

A variable of record type may be updated by assignment.  The compiler provides
a default assignment operator for each record type \chpl{R} having the signature
\begin{example}
\begin{chapel}
proc =(ref lhs:R, rhs:R) : void ;
\end{chapel}
\end{example}
\noindent
In it, the value of each field of the record on the right-hand side is assigned
to the corresponding field of the record on the left-hand side.

The compiler-provided assignment operator may be overridden.

The following example demonstrates record assignment.
\begin{chapelexample}{assignment.chpl}
\begin{chapel}
record R {
  var i: int;
  var x: real;
  proc print() { writeln("i = ", this.i, ", x = ", this.x); }
}
var A: R;
A.i = 3;
A.print();	// "i = 3, x = 0.0"

var C: R;
A = C;
A.print();	// "i = 0, x = 0.0"

C.x = 3.14;
A.print();	// "i = 0, x = 0.0"
\end{chapel}
\begin{chapeloutput}
i = 3, x = 0.0
i = 0, x = 0.0
i = 0, x = 0.0
\end{chapeloutput}
Prior to the first call to \chpl{R.print}, the record \chpl{A} is created and
initialized to all zeroes.  Then, its \chpl{i} field is set to \chpl{3}.
For the second call to \chpl{R.print}, the record \chpl{C} is created assigned
to \chpl{A}.  Since \chpl{C} is default-initialized to all zeroes, those zero
values overwrite both values in \chpl{A}.

The next clause demonstrates that \chpl{A} and \chpl{C} are distinct entities,
rather than two references to the same object.  Assigning \chpl{3.14}
to \chpl{C.x} does not affect the \chpl{x} field in \chpl{A}.
\end{chapelexample}

%REVIEW: vass: need to define "reference assignment"
% and ideally remove the reference to C++.
% Also, this seems not specific to records and so should be moved
% to the "assignment statement" section.
\begin{openissue}
Whether reference assignment is to be supported is an open question.
If so, it would work like reference assignment in C++ -- basically creating an
alias for the RHS.
References can be used to reduce the length of dereference expression, and also
improve performance -- especially if that expression is used repeatedly.
\end{openissue}

\subsection{Default Comparison Operators}
\label{Record_Comparison_Operators}
\index{records!equality}
\index{records!inequality}
\index{records!==@\chpl{==}}
\index{records!"!=@\chpl{"\"!=}}
\index{== (record)@\chpl{==} (record)}
\index{"!= (record)@\chpl{"\"!=} (record)}
Default functions to overload \chpl{==} and \chpl{\!=} are defined for
records if none are explicitly defined.
The default implementation of \chpl{==} applies \chpl{==} to each
field of the two argument records and reduces the result with
the \chpl{&&} operator.  The default implementation of \chpl{\!=}
applies \chpl{\!=} to each field of the two argument records and
reduces the result with the \chpl{||} operator.

\subsection{Implicit Record Conversions}
\label{Implicit_Record_Conversions}
\index{conversions!records}
\index{conversions!implicit!records}
\index{records!implicit conversions}

An expression of record type \chpl{D} can be implicitly converted to
another record type \chpl{C} if

\begin{itemize}
\item for each field in \chpl{C} there is a like-named field in \chpl{D},
      and
\item an implicit conversion is allowed from the type of the field in \chpl{D}
      to the type of the field in \chpl{C}.
\end{itemize}
%REVIEW: vass: the following silently loses part of the record value,
% which is different than what we aim for with numeric types.
% Consider disabling this.
Such a conversion removes any fields that are in \chpl{D} but not \chpl{C}.

The value produced by such a conversion is a record of type \chpl{C}.
The value of each field of this record is obtained by an implicit
conversion of the corresponding field in \chpl{D} to
that field's type in \chpl{C}.

%REVIEW: vass: did we agree that instead of a field in D,
% just a getter with that name would be sufficient?

\section{Differences between Classes and Records}
\label{Class_and_Record_Differences}
\index{records!differences with classes}

The key differences between records and classes are listed below.

\subsection{Declarations}
\label{Declaration_Differences}
\index{records!declarations!differences with classes}

Syntactically, class and record type declarations are identical, except that
they begin with the \chpl{class} and \chpl{record} keywords, respectively.
Also, a record type can only inherit from other record types.  Class inheritance
is not permitted.

\subsection{Storage Allocation}
\label{Storage_Allocation_Differences}
\index{classes!allocation}
\index{records!allocation}

For a variable of record type, storage necessary to contain the data fields
has a lifetime equivalent to the scope in which it is declared.  No two record
variables share the same data.  It is not necessary to call \chpl{new} to create
a record.

By contrast, a class variable contains only a reference to a
class instance.  A class instance is created through a call to its \chpl{new}
operator.  Storage for a class instance, including storage for
the data associated with the fields in the class, is allocated and reclaimed
separately from variables referencing that instance.  The same class instance
can be referenced by multiple class variables.

\subsection{Assignment}
\label{Assignment_Differences}
\index{classes!assignment}
\index{records!assignment}

Assignment to a class variable is performed by reference, whereas assignment to
a record is performed by value.  When a variable of class type is assigned to
another variable of class type, they both become names for the same object.  In
contrast, when a record variable is assigned to another record variable, then
contents of the source record are copied into the target record field-by-field.

When a variable of class type is assigned to a record, matching fields (matched
by name) are copied from the class instance into the corresponding record
fields.  Subsequent changes to the fields in the target record have no effect
upon the class instance.

Assignment of a record to a class variable is not permitted.

\subsection{Arguments}
\label{Argument_Differences}
\index{classes!arguments}
\index{records!arguments}

The semantics of argument passing is determined by the type of the formal
argument (as declared inside the function header).  An actual argument is of a
type compatible with the formal argument only if it is legal to assign the
actual to the formal.
%REVIEW: vass: how is this different from the general rule for argument passing?

%REVIEW: vass: only the implicit conversion rules are different.
% Argument passing is the same.
Specifically, if the formal argument is of class type, the actual argument must
be of that class type or of a type derived from that class type.  If the formal
argument is of a record type, then it is only necessary for the fields in the
actual argument to ``cover'' the fields in the formal argument type.

The receiver argument is passed by value for class methods but is
passed by reference for record methods. In both cases modifications to
the receiver fields are visible outside the method.

\subsection{Inheritance}
\label{Inheritance_Differences}
\index{records!inheritance}
\index{inheritance!records}
\index{records!inheritance}
\index{inheritance!records}

The difference between record inheritance and class
inheritance is that for records there is no dynamic dispatch.  The record type of
a variable is the exact type of that variable, i.e. a variable of a
base record type cannot store a derived record type.

Casting a derived record type to a base record type truncates all 
fields except those belonging to the base record type.  In the same way, only
those methods accessible to the base record type may be invoked using the result
of such a cast.

\subsection{Shadowing and Overriding}
\label{Base_Method_Differences}
\index{classes!shadowing}
\index{records!shadowing}
\index{classes!overriding}
\index{records!overriding}

Class variables have run-time types and (therefore) support dynamic dispatch.
Records are statically typed, so they do not have run-time types and they do not
support dynamic dispatch.

As a result, in record type hierarchies, shadowing and overriding are the same.  
Which field is accessed
and/or which method is invoked is determined statically by the declared type of
the record being referenced.

\subsection{No {\em nil} Value}
\index{nil@\chpl{nil}!not provided for records}

Records do not provide a counterpart of the \chpl{nil} value.  A variable of
record type is associated with storage throughout its lifetime, so \chpl{nil}
has no meaning with respect to records.

\subsection{The {\em delete} operator}
\label{Record_Delete_Illegal}
\index{records!delete illegal}
\index{delete!illegal for records}

Calling \chpl{delete} on a record is illegal.

%REVIEW: we could discuss this:
%An explicit call to \chpl{delete} with a record argument has no effect.  The
%compiler may treat this as a hint that the record should not be accessed later
%within its scope and diagnose that case.

\subsection{Default Comparison Operators}
\label{Comparison_Operator_Differences}
\index{classes!comparison}
\index{records!comparison}

For records, the compiler will supply default comparison operators if they are
not supplied by the user.  The compiler does not supply default comparison
operators for classes.

