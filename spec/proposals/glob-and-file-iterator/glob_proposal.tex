\documentclass{article}
\usepackage{listing}
\title{File and Glob iterator proposal}
\author{Tim Zakian and Brad Chamberlain}

%%====================================================================================================

% \usepackage{url}
% \usepackage{code}
% \usepackage{graphicx}
% \usepackage{enumerate}
% \usepackage{latexsym}
% \usepackage{amsmath}
% \usepackage{amssymb}
% \usepackage{amstext}

\usepackage{comment}
\usepackage{graphicx}
\usepackage{moresize}

% \usepackage{hyperref}
% \hypersetup{colorlinks,%
%             citecolor=black,%
%             filecolor=black,%
%             linkcolor=black,%
%             urlcolor=blue,%
%             pdftex}

\usepackage{url}
% \usepackage{multicol}
% \usepackage{subfigure}

%%====================================================================================================
%%  Listings package and configuration:

\usepackage{listings}
% \usepackage{listings-1.4/listings}
% \usepackage{listings-1.4/lstdoc}
% \usepackage{listings-1.4/lstlang1}
% \usepackage{listings-1.4/lstmisc}
% \input{listings-1.4/listings.sty}

\ExecuteOptions{letter}

\newenvironment{commentenv}[1]{\begin{list}{}{}\item[]{\sc [#1:}
}{{\rm {\sc End of comment.]}} \end{list}}

\long\gdef\comment#1#2{{\color{red} [\textsc{#1}: \textsf{#2}]}}
%\long\gdef\comment#1#2{}

% \newcommand{\code}[1]{\lstinline[basicstyle=\sffamily]{#1}}
\newcommand{\func}[1]{\lstinline[basicstyle=\sffamily]{#1()}}
\newcommand{\keyword}[1]{\emph{#1}}

\newcommand{\concept}[1]{{\sc #1}}

% ----------------------------------------

\lstdefinestyle{basic}{showstringspaces=false,columns=fullflexible,language=C++,escapechar=@,xleftmargin=1pc,%
%basicstyle=\small\bfseries\itshape,%
%keywordstyle=\underbar,
basicstyle=\small\sffamily,
commentstyle=\mdseries,
moredelim=**[is][\color{white}]{~}{~},
morekeywords={concept,model,require,where},
literate={->}{{$\rightarrow\;$}}1 {<-}{{$\leftarrow\;$}}1 {=>}{{$\Rightarrow\;$}}1,
}

\usepackage{color}
\definecolor{darkgreen}{rgb}{0,0.5,0}
\definecolor{darkred}{rgb}{0.5,0,0}

\lstloadlanguages{C++}
\lstnewenvironment{code}
    {\lstset{}%
      \csname lst@SetFirstLabel\endcsname}
    {\csname lst@SaveFirstLabel\endcsname}
    \lstset{
      language=C++,
%      basicstyle=\ssmall\ttfamily,
      basicstyle=\footnotesize\ttfamily,
      flexiblecolumns=true,
%      basicstyle=\ssmall\ttfamily,
      flexiblecolumns=false,
      numbers=left,
      basewidth={0.5em,0.45em},
      keywordstyle=\color{blue},
      commentstyle=\color{darkgreen},
      stringstyle=\ttfamily\color{darkred},
      escapechar=\@,
      xleftmargin=2mm,
      morekeywords={cilk_for,cilk_spawn,cilk_sync, ivar_payload_t,
      __cilkrts_ivar, ivar_payload_t,
      align, atomic,
      begin, bool, break, by,
      class, cobegin, coforall, complex, config,
      const, continue,
      delete, dmapped, do, domain,
      else, enum, extern, export,
        false, for, forall,
          if, imag, in, index,
          inline, inout, int,
          iter,
          label, lambda,
          let, local,
          locale,
          module,
          new,
          nil,
          on,
          opaque,
          otherwise,
          out,
          param,
          proc,
          range,
          real,
          record,
          reduce,
          ref,
          return,
          scan,
          select,
          serial,
          single,
          sparse,
          string,
          subdomain,
          sync,
          then,
          true,
          type,
          uint,
          union,
          use,
          var,
          when,
          where,
          while,
          yield,
          zip},
      %      literate={dotdotdot}{{$\ldots$}}3
        literate={...}{{$\dots$}}3
        %               {cilk_spawn}{\bf{\texttt cilk\_spawn}}8
        %               {cilk_spawn}{{cilkpawn}}10
        %
        % literate={+}{{$+$}}1 {/}{{$/$}}1 {*}{{$*$}}1 % {=}{{$=$}}1
        %          {>}{{$>$}}1 {<}{{$<$}}1 {\\}{{$\lambda$}}1
        %          {\\\\}{{\char`\\\char`\\}}1
        %          {->}{{$\rightarrow$}}2 {>=}{{$\geq$}}2 {<-}{{$\leftarrow$}}2
        %          {<=}{{$\leq$}}2 {=>}{{$\Rightarrow$}}2 
        %          {\ .}{{$\circ$}}2 {\ .\ }{{$\circ$}}2
        %          {>>}{{>>}}2 {>>=}{{>>=}}2 {=<<}{{=<<}}2
        %          {|}{{$\mid$}}1               
        %          {dotdotdot}{{$\ldots$}}3
    }


%\input{code.sty}

%%====================================================================================================


\makeatletter

\renewenvironment{thebibliography}[1]
{\section*{\refname
  \@mkboth{\MakeUppercase\refname}{\MakeUppercase\refname}}%
    \list{\@biblabel{\@arabic\c@enumiv}}%
    {\settowidth\labelwidth{\@biblabel{#1}}%
      \leftmargin\labelwidth
        \advance\leftmargin\labelsep
        \@openbib@code
        \usecounter{enumiv}%
        \let\p@enumiv\@empty
        \renewcommand\theenumiv{\@arabic\c@enumiv}}%
        \sloppy\clubpenalty4000\widowpenalty4000%
        \sfcode`\.\@m}
{\def\@noitemerr
  {\@latex@warning{Empty `thebibliography' environment}}%
  \endlist}

  \def\Box{{\ \vbox{\hrule\hbox{%                                            
    \vrule height1.3ex\hskip0.8ex\vrule}\hrule
  }}\par}

% \input{squeeze}



\if{0}
%%%%%%%%%%%%%%%%%%%%%%%%%%%%%%%%%%%%%%%%%%%%%%%%%%%%%%%
% A subfigure environment that can hold verbatim text %
%%%%%%%%%%%%%%%%%%%%%%%%%%%%%%%%%%%%%%%%%%%%%%%%%%%%%%%

\newbox\subfigbox % Create a box to hold the subfigure. 
\makeatletter 
\newenvironment{subfloat}% % Create the new environment. 
{\def\caption##1{\gdef\subcapsave{\relax##1}}% 
  \let\subcapsave=\@empty % Save the subcaption text. 
    \let\sf@oldlabel=\label 
    \def\label##1{\xdef\sublabsave{\noexpand\label{##1}}}% 
    \let\sublabsave\relax % Save the label key. 
    \setbox\subfigbox\hbox 
    \bgroup}% % Open the box... 
{\egroup % ... close the box and call \subfigure. 
  \let\label=\sf@oldlabel 
    \subfigure[\subcapsave]{\box\subfigbox}}% 
    \makeatother 
    \fi{}

    %%====================================================================================================
    %%   EXTRA DEFS FOR THIS PROPOSAL:


    %% ============================================================

    \newcommand{\nix}[1]{\textcolor{darkred}{#1}}
    \newcommand{\fixme}[1]{{\bf\textcolor{darkred}{#1}}}

    % % [2011.07.09] Hmm.  Having a problem with the code environment not being teletype.
    % \newenvironment{mycode}
    % {% This is the begin code
      %   \noindent
        %   \begin{code}\noindent\vspace{-1mm}
      %   \footnotesize
        % }
        % %  \begin{code}\noindent}
        % {% This is the end code
          %   \end{code} 
          % }

          % \newenvironment{inlinecode}
          % {
            %   \begin{center}
            %   \begin{minipage}{4in}
            %   \begin{mycode}
            % } 
            % {
              %   \end{mycode}
              %   \end{minipage}
              %   \end{center}
              %   \vspace{-1.5ex}
              % }


              % \newcommand{\myhref}[2]{\href{#1}{\underline{#2}}}
              \newcommand{\myhref}[2]{\href{#1}{\underbar{\smash{#2}}}}

              %% ================================================================================

              %% Proper nouns:

              \newcommand{\ccilk}{Concurrent Cilk}
              \newcommand{\eg}{\textit{e.g. }}
              \newcommand{\ie}{\textit{i.e. }}


              %% ================================================================================

\begin{document}
\maketitle
\lstMakeShortInline[]|

\noindent The goal of this proposal is to describe the interface of an
iterator that we are planning to implement as part of Chapel's
standard libraries, and to solicit comments on it.  The intention of
the iterator is to yield the names of files and/or directories
reachable from a given start directory (defaulting to the current
working directory).  This iterator is meant to model (to an extent)
the capabilities of things like |glob|, |wordexp|, |listdir|, |ls [-R]|, 
|find|, and the like.  The main design questions tend to be a
struggle between capability and simplicity.

%args
%----
%startdir
%pattern
%? types (can we avoid?)
%? skipdirs (intention: use to skip .svn directories, e.g. -- or could more
             %specifically have a skipSCMDirs concept)

%flags
%-----
%sorted=? (C glob defaults to true;
           %cl glob defaults to false;
                      %python: not sure)
%recursive=false
%expand=false
%dirs=false
%files=true
%dotfiles=false
%? symbolic links=false
%? chunked=false (intention would be to return an assoc array mapping dir
                  %names to file lists)
%? warnings/verbose (intention would be to optionally print out problems)
%??? append slash to dir names=true (would one ever want to do this?)
\section{User-facing interface}
The proposed user interface for this iterator is as follows:
\begin{lstlisting}
iter glob(pattern   : string = "*",
          startdir  : string = "",
          recursive : bool   = false,
          files     : bool   = true,
          dirs      : bool   = false,
          dotfiles  : bool   = false,
          sorted    : bool   = false,
          expand    : bool   = false
         ) : string
\end{lstlisting}
where
\begin{itemize}
\item |pattern| is a valid |glob| or |wordexp| glob pattern used to
  filter the filenames and/or directory names yielded by the iterator.
  Note that while the |pattern| affects what is yielded, we do not
  expect it to filter which subdirectories are traversed in a
  recursive crawl. More information on what exactly a valid pattern is
  can be found on the |man(3)| page for |glob| and |wordexp|.

\item |startdir| is the directory from which to start the search.
  Filenames and directory names yielded by the iterator will be
  relative to this string and include the string as a prefix.  The
  empty string |""| starts from the current working directory just
  like |"."| but has the effect of yielding names without |"./"|
  prefixed to them.

\item |recursive| indicates whether the iterator should recursively
  consider subdirectories or not.  There is an open issue about
  whether the iterator should take a |depth| argument in addition to
  (or in place of) |recursive| to limit the recursion at a particular
  subdirectory depth (see Section~\ref{s:open_issues}).

\item |files| says whether the iterator should yield filenames or not.

\item |dirs| says whether the iterator should yield directory names or not.

\item |dotfiles| tells whether the iterator should yield dotfiles or
  not.  It also indicates whether the iterator should recursively
  descend into dot directories.

\item |sorted| tells whether the iterator should yield its output in
  lexicographically sorted order.

\item |expand| says whether the iterator should expand environment
  variables, or run shell commands in |pattern| and |startdir| (a la
  |wordexp|). For more information on this capability, see the
  |man(3)| page for |wordexp|.

\end{itemize}

Additional flags have been discussed as possibilities. For a discussion
of these, see Section~\ref{s:open_issues}.

Note that while this is an argument-rich iterator, we expect that
typical uses will only need to supply a few arguments due to the heavy
use of default argument values and Chapel's support for keyword-based
argument passing.  For example, here are some sample use cases:

\begin{itemize}
%\item ``give me all files in \$PWD'' $\to$ |glob()|
\item ``give me all files in \$PWD and its descendents'' $\to$ |glob(recursive=true)|
\item ``give me all files and directories in \$PWD'' $\to$ |glob(dirs=true)|
%\item ``give me all directories in \$PWD'' $\to$ |glob(dirs=true, files=false)|
\item ``give me Python glob'' $\to$ |glob(pattern=...)|
\item ``give me C glob'' $\to$ |glob(pattern=..., sorted=true)|
\item ``give me C wordexp'' $\to$ |glob(pattern=..., expand=true)|
\end{itemize}

We anticipate having a parallel version of this iterator that supports
invocation with |forall| loops.  It is currently expected that
|sorted| will not be a valid option when using the parallel version.


\section{Open Issues}\label{s:open_issues}
The following are open issues that we are wrestling with, and for
which we are seeking opinions (though opinions on anything in this
proposal are fair game).

\begin{itemize}
\item Is there a better name for this iterator than |glob|?  On one
  hand, |glob| is short, sweet, and unambiguous; on the other hand,
  this iterator is also a lot like |ls [-R]|, and is not particularly
  glob-like if no argument is provided.

\item Should |sorted| be true by default?  This is arguably a
  productivity vs. performance issue (the implementation would have to
  sort a directory's local contents before yielding them or doing any
  recursion; though these are arguably going to be small/quick sorts).
  Ironically, it seems that C defaults to sorting while Python does
  not.

\item Should we add a |depth| argument to limit the depth of the
  recursion?  Or alternatively, replace |recursive| with a |depth|
  argument?  The main downside to the first proposal is that it uses
  two flags to indicate one logical thing; the main downside to the
  second proposal is that it makes the very common case of unbounded
  recursion uglier (e.g., requiring |max(int)| or a sentinel value
  like |-1| to be passed in rather than simply |recursive=true|).

\item Should we support an argument to control dropping the final
  slash when yielding directory names?  (i.e., we expect directory
  names to be yielded in the form |"subdir/"| by default).

\item Should we add a |symlinks| argument to say whether or not we
  should follow symbolic links? The main problems that we anticipate
  are (1) it adds another flag, and (2) without care, the iterator
  could get into a possible infinite loop if a symlink points to a
  directory that contains the directory we are currently in.

\item For recursive mode, should we support an argument indicating a
  subdirectory name to avoid recursively descending into?  The idea
  behind this being that if we wanted to avoid certain directory names
  (\eg |.svn| or |.git|), we could set this flag to the string (or
  strings?) we wanted to skip.  Alternatively, we could have a boolean
  indicating whether or not to skip over common SCM-based directory
  names.  On one hand, this feels like a reasonably fiddly option to
  support; on the other hand, Brad is always happy when tools bother
  to support it rather than trying to build the equivalent himself
  with more general features.

\item Should we support an argument for enabling warnings or a verbose
  mode?  The idea behind this flag would be to warn you about certain
  underlying problems. For instance, if we were using |glob| with
  |expand = true| and |pattern = "$CHPL_HOME/*"| and we had not set
  |$CHPL_HOME|, we would generate a warning that we were using an
  undefined environment variable in expansion rather than just
  expanding it to the empty string.

\item Is it reasonable for the parallel version of the iterator to not
  support sorting?  Implementing it would either require tricky
  synchronization within the iterator, or else gathering all the
  results and then sorting them (which is expensive and can trivially
  be expressed outside of the iterator).  Note that, in general,
  Chapel iterators that generate sorted results in their serial form
  (like |for i in 1..n|) are not necessarily sorted in their |forall|
  forms.

\item Should we support a multi-locale version of the parallel
  version?  If so, what scheduling policy should be used?  Should one
  be able to select between single- and multi-locale implementations
  via an argument?  What should the default be?

\item What should the iterator do by default in the event of special
  files like block special files, character special files, named
  pipes, or sockets?  Would we want to support additional arguments
  (or a bit vector) to control this behavior?

\end{itemize}

%If |recursive| is
%set to $0$ we do not recurse into subdirectories, if |recursive| is set to $-1$ we
%recurse into all subdirectories. Otherwise, we recurse to a depth of |recursive| num.
%Whether or not we should do this or use a |bool| instead is an open


\section{Examples}\label{s:examples}
\subsection{Example 1}
\begin{lstlisting}
// Print all files in the current directory
// think: ls -F | grep -v /  (though potentially unsorted)
for fl in glob() do
  writeln(fl);
\end{lstlisting}

\subsection{Example 2}
\begin{lstlisting}
// Print all files in the current directory in sorted order.
// think: ls -F | grep -v /
for fl in glob(sorted=true) do
  writeln(fl);
\end{lstlisting}

\subsection{Example 3}
\begin{lstlisting}
// Print all files and directories in the current directory
// think: ls
for fl in glob(dirs=true) do
  writeln(fl);
\end{lstlisting}

\subsection{Example 4}
\begin{lstlisting}
// Print all subdirectories of the current directory.
// Won't recurse (so we only get the children of of the current dir)
// think: ls -F | grep /
for dir in glob(dirs=true, files=false) do
  writeln(dir);
\end{lstlisting}

\subsection{Example 5}
\begin{lstlisting}
// Print all subdirectories of the current directory, recursively.
for dir in glob(dirs=true, files=false, recursive=true) do
  writeln(dir);
\end{lstlisting}

\subsection{Example 6}
\begin{lstlisting}
// Print all .c files in the current directory
// think: ls *.c
for fl in glob(pattern="*.c") do
  writeln(fl);
\end{lstlisting}

\subsection{Example 7}
\begin{lstlisting}
// Print all .c files in this directory or any of its subdirectories
// think: ls -R *.c
for fl in glob(pattern="*.c", recursive=true) do
  writeln(fl);
\end{lstlisting}

\subsection{Example 8}
\begin{lstlisting}
// Print all [a-p] directories and files from this directory down
for fl in glob(pattern="[a-p]", dirs=true, recursive=true) do
  writeln(fl);
\end{lstlisting}

\subsection{Example 9}
\begin{lstlisting}
// Print all [a-p] directories and files from this directory down.
// Print them in sorted order.
for fl in glob(pattern="[a-p]", dirs=true, recursive=true, sorted=true) do
  writeln(fl);
\end{lstlisting}

\subsection{Example 10}
\begin{lstlisting}
// Print all [a-p] directories and files from the $CHPL_HOME directory down.
// Print them in sorted order.
for fl in glob(pattern="[a-p]", startdir="$CHPL_HOME", dirs=true,
               recursive=true, sorted=true, expand=true) do
  writeln(fl);
\end{lstlisting}

\subsection{Example 11 (uses open issue)}
\begin{lstlisting}
// Print all [a-p] directories and files from $CHPL_HOME down.
// Ignore all .git directories and files. Print them in sorted order.
for fl in glob(pattern="[a-p]", dirs=true, startdir="$CHPL_HOME"
               recursive=true, sorted=true, expand=true, skipdirs=".git") do
  writeln(fl);
\end{lstlisting}

\subsection{Example 12 (uses open issue)}
\begin{lstlisting}
// Print all subdirectories of the current directory
// In this case we'd get all subdirectories up to depth 10
for dir in glob(dirs=true, files=false, depth=10) do
  writeln(dir);
\end{lstlisting}


%\subsection{Example 13}
%\begin{lstlisting}
%// Yield all [a-p] directories from this directory down. For each directory we
%// enter, return an array/domain of all files that were found in that directory
%for fl in glob(pattern="[a-p]", recursive=true, chunked=true) do
%// do something with the array of files handed back...
%\end{lstlisting}
\end{document}
