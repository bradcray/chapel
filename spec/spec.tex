\documentclass[10pt,oneside,titlepage]{spec}
\usepackage[T1]{fontenc}
\usepackage{amsmath}
\usepackage{amssymb}
\usepackage{color}
\usepackage{times}
%\usepackage{fullpage}
\usepackage{graphicx}
\usepackage{listings}
\usepackage{longtable}
\usepackage[nottoc]{tocbibind}
\usepackage{multirow}
%
% These are special environments for adding extra information about
% code snippets which can be later extracted and used to generate test
% codes for automated testing.
%
% During LaTeX compilation, the environments defined in this file throw
% away all text within the scope of the environment, with the
% exception of 'chapelprintoutput' which prints the output (and is
% also extracted for testing purposes).
%
% Usage:
%
% - chapelexample (REQUIRED) {f.chpl}
%   This marks the start of a test.  This environment requires a
%   single argument that is the name of the Chapel test program.  This
%   filename will appear in the spec.
%
% - chapelpre
%   Any Chapel code in this scope is put *before* the code in the
%   chapel|chapelcode scope.
%
% - chapelcode|chapel
%   This is the part of the code that is in the spec.
%
% - chapelnoprint
%   This is the part of the code that goes in the test with chapelcode
%   and chapel, but does not appear in the spec.
%   
% - chapelpost
%   Any Chapel code in this scope is put *after* the code in the
%   chapel|chapelcode scope.
%
% - chapelfuture
% - chapelcompopts
% - chapelexecopts
%   The lines in these scopes are put directly into the appropriate file.
%
% - chapeloutput|chapelprintoutput (REQUIRED)
%   These environment provide the test output (.good files).  There can be
%   multiple such environments, and the filename is specified by a LaTeX
%   style comment preceding the contents of the output.  The
%   'chapelprintoutput' scope is also outputted in the spec itself and
%   thus may contain LaTeX formatting (see GENERAL CAVEATS below)
%   
% - chapelwideoutput
%   Provides the test output for no-local tests, if that differs from the
%   normal test output.  The content of this environment is dumped into a
%   <test>.no-local.good file, along with a copy of the content of 
%   chplprintoutput.
%

%
% GENERAL CAVEATS:
%
% - Because the chapelprintoutput environment must used LaTeX
%   formatting, the script that extracts the tests must removed any
%   LaTeX specific formatting.
%
% - Using a backslash or other special LaTeX characters may also be
%   needed (e.g., \_ or \#) in the other environments for LaTex
%   parsing purposes.  Such characters are considered fragile and may
%   lead to unexpected results.
%

%
% Gobble up the text in this new box.  The text in each environment is
% dropped on the floor during LaTeX compilation.
%
\newsavebox{\teststuff}

%
% Any additional lines needed for the code snippet to run/compile
% (before and after the chapel code segment)
%
\newenvironment{chapelpre} {\begin{lrbox}{\teststuff}
\begin{minipage}{6in}}
{\end{minipage}\end{lrbox}}

\newenvironment{chapelnoprint} {\begin{lrbox}{\teststuff}
\begin{minipage}{6in}}
{\end{minipage}\end{lrbox}}

\newenvironment{chapelpost} {\begin{lrbox}{\teststuff}
\begin{minipage}{6in}}
{\end{minipage}\end{lrbox}}


%
% .future file
%
\newenvironment{chapelfuture} {\begin{lrbox}{\teststuff}
\begin{minipage}{6in}}
{\end{minipage}\end{lrbox}}

%
% .compopts file
%
\newenvironment{chapelcompopts} {\begin{lrbox}{\teststuff}
\begin{minipage}{6in}}
{\end{minipage}\end{lrbox}}

%
% .execopts file
%
\newenvironment{chapelexecopts} {\begin{lrbox}{\teststuff}
\begin{minipage}{6in}}
{\end{minipage}\end{lrbox}}


%
% .good file
% To get more than one file, use a LaTeX style comment to name the
% .good file
%
\newenvironment{chapeloutput} {\begin{lrbox}{\teststuff}
\begin{minipage}{6in}}
{\end{minipage}\end{lrbox}}

%
% .no-local.good file
% (The naming feature mentioned above does not yet work, so this
% environment is a Q&D way to get a .no-local.good file.)
%
\newenvironment{chapelwideoutput} {\begin{lrbox}{\teststuff}
\begin{minipage}{6in}}
{\end{minipage}\end{lrbox}}

%
% .prediff file
%
\newenvironment{chapelprediff} {\begin{lrbox}{\teststuff}
\begin{minipage}{6in}}
{\end{minipage}\end{lrbox}}

%
% .good file that is printed in the text of the Spec
% To get more than one file, use a LaTeX style comment to name the
% .good file
%
%\lstnewenvironment{chapelprintoutput} 
% (See chapel_listing.tex for the implementation.)

\lstdefinelanguage{chapel}
  {
    morekeywords={
      align, as, atomic,
      begin, bool, break, by,
      class, cobegin, coforall, complex, config, const, continue,
      delete, dmapped, do, domain,
      else, enum, except, export, extern,
      false, for, forall,
      if, imag, in, index, inline, inout, int, iter,
      label, lambda, let, local, locale,
      module,
      new, nil, noinit,
      on, only, opaque, otherwise, out,
      param, private, proc, public,
      range, real, record, reduce, ref, require, return,
      scan, select, serial, single, sparse, string, subdomain, sync,
      then, true, type,
      uint, union, use,
      var,
      when, where, while, with,
      yield,
      zip
    },
    sensitive=false,
    mathescape=true,
    morecomment=[l]{//},
    morecomment=[s]{/*}{*/},
    morestring=[b]",
}

\lstset{
    basicstyle=\footnotesize\ttfamily,
    keywordstyle=\bfseries,
    commentstyle=\em,
    showstringspaces=false,
    flexiblecolumns=false,
    numbers=left,
    numbersep=5pt,
    numberstyle=\tiny,
    numberblanklines=false,
    stepnumber=0,
    escapeinside={(*}{*)},
    language=chapel,
  }

%\newcommand{\chpl}[1]{\lstinline[language=chapel,basicstyle=\ttfamily,keywordstyle=\bfseries]!#1!}
\newcommand{\chpl}[1]{\lstinline[language=chapel,basicstyle=\small\ttfamily,keywordstyle=]!#1!}
\newcommand{\varname}[1]{\emph{#1}}
\newcommand{\typename}[1]{\emph{#1}}
\newcommand{\fnname}[1]{\chpl{#1}}

\lstnewenvironment{chapel}{\lstset{language=chapel,xleftmargin=2pc,stepnumber=0}}{}
\lstnewenvironment{invisible}{\lstset{language=chapel,xleftmargin=2pc,stepnumber=0,keywordstyle=\bfseries\color{white},basicstyle=\small\ttfamily\color{white}}}{}
\lstnewenvironment{chapel0}{\lstset{language=chapel,stepnumber=0}}{}

\lstnewenvironment{numberedchapel}{\lstset{language=chapel,xleftmargin=15pt,stepnumber=1}}{}

\lstnewenvironment{chapelcode}{\lstset{language=chapel,stepnumber=1}}{}

% Uses the same listing style as the {chapel} environment, but keyword
% formatting is turned off.  The argument is ignored in LaTeX
% but used to name the .good file during test extraction.
% The argument must be supplied but may be empty.
% If empty it defaults to null, which signals the test extractor to 
% autogenerate the .good file name as ``<test_name>.good''.
\lstnewenvironment{chapelprintoutput}[1]
  {\lstset{language=chapel,xleftmargin=2pc,stepnumber=0,keywordstyle=}}{}

\lstnewenvironment{commandline}{\lstset{keywordstyle=,xleftmargin=2pc}}{}

\lstnewenvironment{protohead}{\lstset{language=chapel,xleftmargin=0pc,belowskip=-10pt,stepnumber=0}}{}

\newenvironment{protobody}{\begin{description}\item[\quad\quad] }{\end{description}}

\input{syntax_listing}

%% High section numbers require different number widths
\usepackage[titles]{tocloft}
\setlength{\cftchapnumwidth}{1.3em} % Wide enough for a chapter number.
\setlength{\cftsecnumwidth}{3.2em}  % Same as cftsubsecnumwidth:
\setlength{\cftsubsecnumwidth}{3.2em} % Wide enough for three digits and two dots.
%\setlength{\cftsubsubsecnumwidth}{5.4em}
\setlength{\cftsecindent}{1.3em}    % cftchapnumwidth
\setlength{\cftsubsecindent}{1.3em} % cftchapnumwidth
\setlength{\cftsubsubsecindent}{4.5em} % cftchapnumwidth + cftsecnumwidth

% This can be re-enabled in order to aid in spec editing/reviewing.
%% bring in todonotes and make sure we have enough space in the margins. Also
%% note, even side margin is changed below since I couldn't figure out how to
%% force notes to the right side.
%\usepackage[paperwidth=9.5in, paperheight=11in]{geometry}
%\setlength{\marginparwidth}{1.8in}
%\usepackage[colorinlistoftodos, textwidth=1.8in, shadow]{todonotes}

\usepackage{ifpdf}
\ifpdf
\usepackage[pdftex,
            bookmarks,
            plainpages=false,
            breaklinks,
            pdftitle={Chapel Language Specification},
            pdfauthor={Cray Inc, 901 Fifth Avenue Suite 1000, Seattle, WA 98164},
            pdfsubject={Chapel High Productivity Language}
           ]{hyperref}
\else
\usepackage[ps2pdf]{hyperref}
\fi

\newcommand{\ie}{\emph{i.e.}}
\newcommand{\eg}{\emph{e.g.}}

\newenvironment{TODO} {
\begin{quote}
{\it TODO:}
}{
\end{quote}
}

\newenvironment{example}{
\begin{quote}
{\it Example}.
}{
\end{quote}
}

\newenvironment{chapelexample}[1]{
\begin{quote}
{\it Example (#1)}.
}{
\end{quote}
}

\newenvironment{note}{
\begin{quote}
{\it Implementors' note}.
}{
\end{quote}
}

\newenvironment{rationale}{
\begin{quote}
{\it Rationale}.
}{
\end{quote}
}

\newenvironment{openissue}{
\begin{quote}
{\it Open issue}.
}{
\end{quote}
}

\newenvironment{future}{
\begin{quote}
{\it Future}.
}{
\end{quote}
}

\newenvironment{craychapel}{
\begin{quote}
{\it Cray's Chapel Implementation}.
}{
\end{quote}
}

\newenvironment{suggestionbox}{
\begin{quote}
{\it Suggestions?}
}{
\end{quote}
}

\newcommand{\rsec}[1]
           {\S\ref{#1}}

% courtesy: http://www.iam.ubc.ca/~newbury/tex/page-set-up.html
\newcommand{\sekshun}[1]
           {
             \chapter{#1}
             \markboth{Chapel Language Specification}{#1}
           }

\oddsidemargin 0.0in
% not needed with onesided version \evensidemargin 0.5in
\textwidth 6.5in
\headheight 0.2in
\topmargin 0in
\headsep 0.3in
\textheight 8.5in

\makeindex
\title{Chapel Language Specification\\Version 0.986}

\author{Cray Inc\\
901 Fifth Avenue, Suite 1000\\
Seattle, WA 98164}

\date{September 20, 2018}

\setcounter{tocdepth}{3}

\begin{document}

\pagestyle{empty}
%\pagenumbering{alph}

\ifpdf
\pdfbookmark[1]{Title}{titlepage}
\fi
\maketitle

\setcounter{page}{2}
\null\vfill
\noindent
\begin{center}
\copyright 2019 Cray Inc.
\end{center}

\cleardoublepage
\include{tm}
\cleardoublepage

\pagestyle{myheadings}
\markboth{Chapel Language Specification}{Chapel Language Specification}
%\pagenumbering{roman}

\ifpdf
\pdfbookmark[1]{Table of Contents}{tablecontents}
\fi
\tableofcontents

\cleardoublepage

\pagestyle{myheadings}
%\pagenumbering{arabic}

\setlength{\parindent}{0in}
\setlength{\parskip}{4mm plus2mm minus1mm}

\input{Scope}
\cleardoublepage
\input{Notation}
\cleardoublepage
\sekshun{Organization}
\label{Organization}
\index{organization}

This specification is organized as follows:
\begin{itemize}

\item
Chapter~\ref{Scope}, Scope, describes the scope of this specification.

\item
Chapter~\ref{Notation}, Notation, introduces the notation that is used
throughout this specification.

\item
Chapter~\ref{Organization}, Organization, describes the contents of
each of the chapters within this specification.

\item
Chapter~\ref{Acknowledgments}, Acknowledgements, offers a note of
thanks to people and projects.

\item
Chapter~\ref{Language_Overview}, Language Overview, describes Chapel
at a high level.

\item
Chapter~\ref{Lexical_Structure}, Lexical Structure, describes the
lexical components of Chapel.

\item
Chapter~\ref{Types}, Types, describes the types in Chapel and defines
the primitive and enumerated types.

\item
Chapter~\ref{Variables}, Variables, describes variables and constants
in Chapel.

\item
Chapter~\ref{Conversions}, Conversions, describes the legal implicit
and explicit conversions allowed between values of different types.
Chapel does not allow for user-defined conversions.

\item
Chapter~\ref{Expressions}, Expressions, describes the non-parallel
expressions in Chapel.

\item
Chapter~\ref{Statements}, Statements, describes the non-parallel
statements in Chapel.

\item
Chapter~\ref{Modules}, Modules, describes modules in Chapel., Chapel
modules allow for name space management.

\item
Chapter~\ref{Functions}, Functions, describes functions and function
resolution in Chapel.

\item
Chapter~\ref{Tuples}, Tuples, describes tuples in Chapel.

\item
Chapter~\ref{Classes}, Classes, describes reference classes in Chapel.

\item
Chapter~\ref{Records}, Records, describes records or value classes in
Chapel.

\item
Chapter~\ref{Unions}, Unions, describes unions in Chapel.

\item
Chapter~\ref{Ranges}, Ranges, describes ranges in Chapel.

\item
Chapter~\ref{Domains}, Domains, describes domains in Chapel.  Chapel
domains are first-class index sets that support the description of
iteration spaces, array sizes and shapes, and sets of indices.

\item
Chapter~\ref{Arrays}, Arrays, describes arrays in Chapel.  Chapel arrays are
more general than in most languages including support for
multidimensional, sparse, associative, and unstructured arrays.

\item
Chapter~\ref{Iterators}, Iterators, describes iterator functions.

\item
Chapter~\ref{Generics}, Generics, describes Chapel's support for
generic functions and types.

\item
Chapter~\ref{Input_and_Output}, Input and Output, describes support
for input and output in Chapel, including file input and output..

\item
Chapter~\ref{Task_Parallelism_and_Synchronization}, Task Parallelism
and Synchronization, describes task-parallel expressions and statements
in Chapel as well as synchronization constructs, atomic variables, and
the atomic statement.

\item
Chapter~\ref{Data_Parallelism}, Data Parallelism, describes
data-parallel expressions and statements in Chapel including
reductions and scans, whole array assignment, and promotion.

\item
Chapter~\ref{Locales_Chapter}, Locales, describes constructs for managing
locality and executing Chapel programs on distributed-memory systems.

\item
Chapter~\ref{Domain_Maps}, Domain Maps, describes
Chapel's \emph{domain map} construct for defining the layout of
domains and arrays within a single locale and/or the distribution of
domains and arrays across multiple locales.

\item
Chapter~\ref{User_Defined_Reductions_and_Scans}, User-Defined
Reductions and Scans, describes how Chapel programmers can define
their own reduction and scan operators.

\item
  Chapter~\ref{Memory_Consistency_Model}, Memory Consistency Model,
  describes Chapel's rules for ordering the reads and writes performed
  by a program's tasks.

\item
Chapter~\ref{Interoperability} describes Chapel's interoperability
features for combining Chapel programs with code written in different
languages.

\item
Appendix~\ref{Syntax}, Collected Lexical and Syntax Productions,
contains the syntax productions listed throughout this specification
in both alphabetical and depth-first order.

\end{itemize}

\cleardoublepage
\sekshun{Acknowledgments}
\label{Acknowledgments}
\index{acknowledgments}

The following people have been actively involved in the
recent evolution of the Chapel language and its specification:
Kyle Brady,
Bradford Chamberlain,
Sung-Eun Choi,
Lydia Duncan,
Michael Ferguson,
Ben Harshbarger,
Tom Hildebrandt,
David Iten,
Vassily Litvinov,
Tom MacDonald,
Michael Noakes,
Elliot Ronaghan,
Greg Titus,
Thomas Van Doren,
and Tim Zakian

The following people have contributed to previous versions
of the language and its specification:
Robert Bocchino,
David Callahan,
Steven Deitz,
Roxana Diaconescu,
James Dinan,
Samuel Figueroa,
Shannon Hoffswell,
Mary Beth Hribar,
Mark James,
Mackale Joyner,
Jacob Nelson,
John Plevyak,
Lee Prokowich,
Albert Sidelnik,
Andy Stone,
Wayne Wong,
and Hans Zima.

We are also grateful to our many enthusiastic and vocal users for
helping us continually improve the quality of the Chapel language and
compiler.

Chapel is a derivative of a number of parallel and distributed
languages and takes ideas directly from them, especially the MTA
extensions of C, HPF, and ZPL.

Chapel also takes many serial programming ideas from many other
programming languages, especially C\#, C++, Java, Fortran, and Ada.

The preparation of this specification was made easier and the final
result greatly improved because of the good work that went in to the
creation of other language standards and specifications, in particular
the specifications of C\# and C.

\cleardoublepage
\sekshun{Language Overview}
\label{Language_Overview}
\index{overview}
\index{language overview}

Chapel is an emerging parallel programming language designed for
productive scalable computing. Chapel's primary goal is to make
parallel programming far more productive, from multicore desktops and
laptops to commodity clusters and the cloud to high-end
supercomputers. Chapel's design and development are being led by Cray
Inc. in collaboration with academia, computing centers, and industry.

Chapel is being developed in an open-source manner at GitHub under the
Apache v2.0 license and also makes use of other third-party
open-source packages under their own licenses. Chapel emerged from
Cray's entry in the DARPA-led High Productivity Computing Systems
program (HPCS). It is currently being hardened from that initial
prototype to more of a product-grade implementation.

This section provides a brief overview of the Chapel language by
discussing first the guiding principles behind the design of the
language and second how to get started with Chapel.

\section{Guiding Principles}
\label{Guiding_Principles}
\index{language principles}

The following four principles guided the design of Chapel:
\begin{enumerate}
\item General parallel programming
\item Locality-aware programming
\item Object-oriented programming
\item Generic programming
\end{enumerate}
The first two principles were motivated by a desire to support
general, performance-oriented parallel programming through high-level
abstractions.  The second two principles were motivated by a desire to
narrow the gulf between high-performance parallel programming
languages and mainstream programming and scripting languages.

\subsection{General Parallel Programming}
\label{General_Parallel_Programming}
\index{language principles!general parallel programming}

First and foremost, Chapel is designed to support general parallel
programming through the use of high-level language abstractions.
Chapel supports a \emph{global-view programming model} that raises the
level of abstraction in expressing both data and control flow as
compared to parallel programming models currently in use.
A global-view programming model is best defined in terms
of \emph{global-view data structures} and a \emph{global view of
control}.

\emph{Global-view data structures} are arrays and other data
aggregates whose sizes and indices are expressed globally even though
their implementations may distribute them across the \emph{locales} of
a parallel system.  A locale is an abstraction of a unit of uniform
memory access on a target architecture.  That is, within a locale all
threads exhibit similar access times to any specific memory address.
For example, a locale in a commodity cluster could be defined to be a
single core of a processor, a multicore processor, or an SMP node of
multiple processors.

Such a global view of data contrasts with most parallel languages
which tend to require users to partition distributed data aggregates
into per-processor chunks either manually or using language
abstractions.  As a simple example, consider creating a 0-based vector
with $n$ elements distributed between $p$ locales.  A language
that supports global-view data structures, as Chapel does, allows the user to
declare the array to contain $n$ elements and to refer to the array
using the indices $0 \ldots n-1$.  In contrast, most traditional
approaches require the user to declare the array as $p$ chunks of
$n/p$ elements each and to specify and manage inter-processor
communication and synchronization explicitly (and the details can be
messy if $p$ does not divide $n$ evenly).  Moreover, the chunks are
typically accessed using local indices on each processor
(\eg,~$0..n/p$), requiring the user to explicitly translate between
logical indices and those used by the implementation.

A \emph{global view of control} means that a user's program commences
execution with a single logical thread of control and then introduces
additional parallelism through the use of certain language concepts.
All parallelism in Chapel is implemented via multithreading, though
these threads are created via high-level language concepts and managed
by the compiler and runtime rather than through explicit
fork/join-style programming.  An impact of this approach is that
Chapel can express parallelism that is more general than the Single
Program, Multiple Data~(SPMD) model that today's most common parallel
programming approaches use.  Chapel's general support for parallelism does not
preclude users from coding in an SPMD style if they wish.

Supporting general parallel programming also means targeting a broad
range of parallel architectures.  Chapel is designed to target a wide
spectrum of HPC hardware including clusters of commodity processors
and SMPs; vector, multithreading, and multicore processors; custom
vendor architectures; distributed-memory, shared-memory, and 
shared address-space architectures; and networks of any topology.  Our
portability goal is to have any legal Chapel program run correctly on
all of these architectures, and for Chapel programs that express
parallelism in an architecturally-neutral way to perform reasonably on
all of them.  Naturally, Chapel programmers can tune their code to
more closely match a particular machine's characteristics.

\subsection{Locality-Aware Programming}
\label{Locality_Aware_Programming}
\index{language principles!locality-aware programming}

A second principle in Chapel is to allow the user to optionally and
incrementally specify where data and computation should be placed on
the physical machine.  Such control over program locality is essential
to achieve scalable performance on distributed-memory architectures.  Such control
contrasts with shared-memory programming models which present the user
with a simple flat memory model.  It also contrasts with SPMD-based
programming models in which such details are explicitly specified by
the programmer on a process-by-process basis via the multiple
cooperating program instances.

\subsection{Object-Oriented Programming}
\label{Object_Oriented_Programming}
\index{language principles!object-oriented programming}

A third principle in Chapel is support for object-oriented
programming.  Object-oriented programming has been instrumental in
raising productivity in the mainstream programming community due to
its encapsulation of related data and functions within a single software
component, its support for specialization and reuse, and its use as a
clean mechanism for defining and implementing interfaces.  Chapel
supports objects in order to make these benefits available in a
parallel language setting, and to provide a familiar coding paradigm for
members of the mainstream programming community.  Chapel supports
traditional reference-based classes as well as value classes that are
assigned and passed by value.

\subsection{Generic Programming}
\label{Generic_Programming}
\index{language principles!generic programming}

Chapel's fourth principle is support for generic programming and
polymorphism.  These features allow code to be written in a style that
is generic across types, making it applicable to variables of multiple
types, sizes, and precisions.  The goal of these features is to
support exploratory programming as in popular interpreted and
scripting languages, and to support code reuse by allowing algorithms
to be expressed without explicitly replicating them for each possible
type.  This flexibility at the source level is implemented by having
the compiler create versions of the code for each required type
signature rather than by relying on dynamic typing which would result
in unacceptable runtime overheads for the HPC community.

\section{Getting Started}
\label{Getting_Started}

A Chapel version of the standard ``hello, world'' computation is as
follows:
\vspace{0.5pc}
\begin{chapel}
writeln("hello, world");
\end{chapel}
This complete Chapel program contains a single line of code that makes
a call to the standard \chpl{writeln} function.

\index{modules}
\index{main@\chpl{main}}

In general, Chapel programs define code using one or more named
\emph{modules}, each of which supports top-level initialization code
that is invoked the first time the module is used.  Programs also
define a single entry point via a function named \chpl{main}.  To
facilitate exploratory programming, Chapel allows programmers to
define modules using files rather than an explicit module declaration
and to omit the program entry point when the program only has a single
user module.

Chapel code is stored in files with the extension \chpl{.chpl}.
Assuming the ``hello, world'' program is stored in a file
called \chpl{hello.chpl}, it would define a single user
module, \chpl{hello}, whose name is taken from the filename.  Since
the file defines a module, the top-level code in the file defines the
module's initialization code.  And since the program is composed of
the single \chpl{hello} module, the \chpl{main} function is omitted.
Thus, when the program is executed, the single \chpl{hello} module
will be initialized by executing its top-level code thus invoking the
call to the \chpl{writeln} function.  Modules are described in more
detail in~\rsec{Modules}.

To compile and run the ``hello world'' program, execute the following
commands at the system prompt:
\begin{commandline} 
> chpl hello.chpl
> ./a.out
\end{commandline}
The following output will be printed to the console:
\begin{commandline}
hello, world
\end{commandline}

\cleardoublepage
\sekshun{Lexical Structure}
\label{Lexical_Structure}
\index{lexical structure}

This section describes the lexical components of Chapel programs.
The purpose of lexical analysis is
to separate the raw input stream into a sequence of tokens suitable
for input to the parser.

\section{Comments}
\label{Comments}
\index{comments}
\index{lexical structure!comments}

Two forms of comments are supported.  All text following the
consecutive characters {\tt //} and before the end of the line is in a
comment.  All text following the consecutive characters {\tt /*} and
before the consecutive characters {\tt */} is in a comment.
A comment delimited by {\tt /*} and {\tt */} can be nested in
another comment delimited by {\tt /*} and {\tt */}

Comments, including the characters that delimit them, do not affect
the behavior of the program (except in delimiting tokens).  If the
delimiters that start the comments appear within a string literal,
they do not start a comment but rather are part of the string literal.

\begin{example}
The following program makes use of both forms of comment:
\begin{chapel}
/*
 *  main function
 */
proc main() {
  writeln("hello, world"); // output greeting with new line
}
\end{chapel}
\end{example}

\section{White Space}
\label{White_Space}
\index{white space}
\index{lexical structure!white space}

White-space characters are spaces, tabs, line feeds, and carriage
returns.  Along with comments, they delimit tokens, but are otherwise
ignored.

\section{Case Sensitivity}
\label{Case_Sensitivity}
\index{case sensitivity}
\index{lexical structure!case sensitivity}

Chapel is a case sensitive language.

\begin{example}
The following identifiers are considered
distinct: \chpl{chapel}, \chpl{Chapel}, and \chpl{CHAPEL}.
\end{example}

\section{Tokens}
\label{Tokens}
\index{lexical structure!tokens}

Tokens include identifiers, keywords, literals, operators, and
punctuation.

\subsection{Identifiers}
\label{Identifiers}
\index{identifiers}
\index{lexical structure!identifiers}

An identifier in Chapel is a sequence of characters that starts with a
lowercase or uppercase letter or an underscore and is optionally
followed by a sequence of lowercase or uppercase letters, digits,
underscores, and dollar-signs.  Identifiers are designated by the
following syntax:
\begin{syntax}
identifier:
  letter-or-underscore legal-identifier-chars[OPT]

legal-identifier-chars:
  legal-identifier-char legal-identifier-chars[OPT]

legal-identifier-char:
  letter-or-underscore
  digit
  `(*\texttt{\$}*)'

letter-or-underscore:
  letter
  `_'

letter: one of
  `A' `B' `C' `D' `E' `F' `G' `H' `I' `J' `K' `L' `M' `N' `O' `P' `Q' `R' `S' `T' `U' `V' `W' `X' `Y' `Z'
  `a' `b' `c' `d' `e' `f' `g' `h' `i' `j' `k' `l' `m' `n' `o' `p' `q' `r' `s' `t' `u' `v' `w' `x' `y' `z'

digit: one of
  `0' `1' `2' `3' `4' `5' `6' `7' `8' `9'
\end{syntax}

\begin{rationale}
Why include ``\$'' in the language?  The inclusion of the \$ character
is meant to assist programmers using sync and single variables by
supporting a convention (a \$ at the end of such variables) in order
to help write properly synchronized code.  It is felt that marking
such variables is useful since using such variables could result in
deadlocks.
\end{rationale}

\begin{example}
The following are legal
identifiers: \chpl{Cray1}, \chpl{syncvar$\mbox{\texttt{\$}}$},
\chpl{legalIdentifier}, and \chpl{legal_identifier}.
\end{example}

\subsection{Keywords}
\label{Keywords}
\index{keywords}
\index{lexical structure!keywords}

The following identifiers are reserved as keywords:

\begin{tabular}{llllll}
&
\begin{chapel}
_
align
atomic
begin
break
by
class
cobegin
coforall
config
const
continue
delete
\end{chapel} & \begin{chapel}
dmapped
do
domain
else
enum
export
extern
for
forall
if
in
index
\end{chapel} & \begin{chapel}
inline
inout
iter
label
let
local
module
new
nil
noinit
on
otherwise
\end{chapel} & \begin{chapel}
out
param
proc
record
reduce
ref
return
scan
select
serial
single
sparse
\end{chapel} & \begin{chapel}
subdomain
sync
then
type
union
use
var
when
where
while
yield
zip
\end{chapel} \\
\begin{invisible}
otherwise
\end{invisible} & \begin{invisible}
otherwise
\end{invisible} & \begin{invisible}
otherwise
\end{invisible} & \begin{invisible}
otherwise
\end{invisible} & \begin{invisible}
otherwise
\end{invisible} & \begin{invisible}
otherwise
\end{invisible}
\end{tabular}

The following identifiers are keywords reserved for future use:

\begin{tabular}{l}
\begin{chapel}
lambda
\end{chapel} \\
\begin{invisible}
otherwise
\end{invisible}
\end{tabular}

\subsection{Literals}
\label{Literals}
\label{Primitive_Type_Literals}
\index{lexical structure!literals}

\index{literals!primitive type}
Bool literals are designated by the following syntax:
\begin{syntax}
bool-literal: one of
  `true' $ $ $ $ `false'
\end{syntax}

Signed and unsigned integer literals are designated by the following
syntax:
\begin{syntax}
integer-literal:
  digits
  `0x' hexadecimal-digits
  `0X' hexadecimal-digits
  `0o' octal-digits
  `0O' octal-digits
  `0b' binary-digits
  `0B' binary-digits

digits:
  digit
  digit digits

hexadecimal-digits:
  hexadecimal-digit
  hexadecimal-digit hexadecimal-digits

hexadecimal-digit: one of
  `0' `1' `2' `3' `4' `5' `6' `7' `8' `9' `A' `B' `C' `D' `E' `F' `a' `b' `c' `d' `e' `f'

octal-digits:
  octal-digit
  octal-digit octal-digits

octal-digit: one of
  `0' `1' `2' `3' `4' `5' `6' `7'

binary-digits:
  binary-digit
  binary-digit binary-digits

binary-digit: one of
  `0' `1'
\end{syntax}

Integer literals in the range 0 to max(\chpl{int}), ~\rsec{Signed_and_Unsigned_Integral_Types},
have type \chpl{int} and the remaining literals have type \chpl{uint}.

\begin{rationale}
Why are there no suffixes on integral literals?  Suffixes, like those
in C, are not necessary.  Explicit
conversions can then be used to change the type of the literal to
another integer size.
\end{rationale}

Real literals are designated by the following syntax:
\begin{syntax}
real-literal:
  digits[OPT] . digits exponent-part[OPT]
  digits .[OPT] exponent-part
  `0x' hexadecimal-digits[OPT] . digits p-exponent-part[OPT]
  `0X' hexadecimal-digits[OPT] . digits p-exponent-part[OPT]
  `0x' hexadecimal-digits .[OPT] p-exponent-part
  `0X' hexadecimal-digits .[OPT] p-exponent-part

exponent-part:
  `e' sign[OPT] digits
  `E' sign[OPT] digits

p-exponent-part:
  `p' sign[OPT] digits
  `P' sign[OPT] digits


sign: one of
  + $ $ $ $ -
\end{syntax}

\begin{rationale}
Why can't a real literal end with '.'?  There is a lexical ambiguity
between real literals ending in '.' and the range operator '..' that
makes it difficult to parse.  For example, we want to
parse \chpl{1..10} as a range from 1 to 10 without concern
that \chpl{1.} is a real literal.
\end{rationale}

Hexadecimal real literals are supported with a hexadecimal integer and
fractional part. Because 'e' could be a hexadecimal character, the exponent for
these literals is instead marked with 'p' or 'P'. The exponent value follows
and is written in decimal.

The type of a real literal is \chpl{real}.  Explicit conversions are
necessary to change the size of the literal.

Imaginary literals are designated by the following syntax:

\begin{syntax}
imaginary-literal:
  real-literal `i'
  integer-literal `i'
\end{syntax}

The type of an imaginary literal is \chpl{imag}.  Explicit conversions
are necessary to change the size of the literal.

There are no complex literals.  Rather, a complex value can be
specified by adding or subtracting a real literal with an imaginary
literal.  Alternatively, a 2-tuple of integral or real expressions can
be cast to a complex such that the first component becomes the real
part and the second component becomes the imaginary part.
\begin{example}
The following expressions are identical: \chpl{1.0 + 2.0i}
and \chpl{(1.0, 2.0):complex}.
\end{example}

String literals are designated by the following syntax:
\begin{syntax}
string-literal:
  " double-quote-delimited-characters[OPT] "
  ' single-quote-delimited-characters[OPT] '

double-quote-delimited-characters:
  string-character double-quote-delimited-characters[OPT]
  ' double-quote-delimited-characters[OPT]

single-quote-delimited-characters:
  string-character single-quote-delimited-characters[OPT]
  " single-quote-delimited-characters[OPT]

string-character:
  `any character except the double quote, single quote, or new line'
  simple-escape-character
  hexadecimal-escape-character

simple-escape-character: one of
  `$\backslash\mbox{\bf '}\hspace{5pt}$' `$\backslash$"$\hspace{5pt}$' `$\backslash$?$\hspace{5pt}$' `$\backslash$a$\hspace{5pt}$' `$\backslash$b$\hspace{5pt}$' `$\backslash$f$\hspace{5pt}$' `$\backslash$n$\hspace{5pt}$' `$\backslash$r$\hspace{5pt}$' `$\backslash$t$\hspace{5pt}$' `$\backslash$v$\hspace{5pt}$'

hexadecimal-escape-character:
  `$\backslash$x' hexadecimal-digits
\end{syntax}

\subsection{Operators and Punctuation}
\label{Operators_and_Punctuation}
\index{lexical structure!operator}
\index{operators!lexical structure}
\index{lexical structure!punctuation}

The following operators and punctuation are defined in the syntax of
the language:
\begin{center}
\begin{tabular}{|l|l|}
\hline
{\bf symbols} & {\bf use} \\
\hline
\verb@=@ & assignment \\
\verb@+= -= *= /= **= %= &= |= ^= &&= ||= <<= >>=@ & compound assignment \\
\verb@<=>@ & swap \\
\verb@..@ & range specifier \\
\verb@by@ & range/domain stride specifier \\
\verb@#@ & range count operator \\
\verb@...@ & variable argument lists \\
\verb@&& || ! & | ^ ~ << >>@ & logical/bitwise operators \\
\verb@== != <= >= < >@ & relational operators \\
\verb@+ - * / % **@ & arithmetic operators \\
\verb@:@ & type specifier \\
\verb@;@ & statement separator \\
\verb@,@ & expression separator \\
\verb@.@ & member access \\
\verb@?@ & type query \\
\verb@" '@ & string delimiters \\
\hline
\end{tabular}
\end{center}

\subsection{Grouping Tokens}
\label{Grouping_Tokens}
\index{lexical structure!braces}
\index{lexical structure!parentheses}
\index{lexical structure!brackets}

The following braces are part of the Chapel language:
\begin{center}
\begin{tabular}{|l|l|}
\hline
{\bf braces} & {\bf use} \\
\hline
\verb@( )@ & parenthesization, function calls, and tuples \\
\verb@[ ]@ & array literals, array types, forall expressions, and function calls \\
\verb@{ }@ & domain literals, block statements \\
\hline
\end{tabular}
\end{center}

\cleardoublepage
\sekshun{Types}
\label{Types}
\index{types}

Chapel is a statically typed language with a rich set of types.  These
include a set of predefined primitive types, enumerated types,
structured types (classes, records, unions, tuples),
data parallel types (ranges, domains, arrays), and synchronization
types (sync, single, atomic).

The syntax of a type is as follows:

\begin{syntax}
type-specifier:
  primitive-type
  enum-type
  structured-type
  dataparallel-type
  synchronization-type
\end{syntax}

Programmers can define their own enumerated types, classes, records,
unions, and type aliases using type declaration statements:

\begin{syntax}
type-declaration-statement:
  enum-declaration-statement
  class-declaration-statement
  record-declaration-statement
  union-declaration-statement
  type-alias-declaration-statement
\end{syntax}

These statements are defined in Sections \rsec{Enumerated_Types},
\rsec{Class_Declarations}, \rsec{Record_Declarations},
\rsec{Union_Declarations}, and \rsec{Type_Aliases}, respectively.

\section{Primitive Types}
\label{Primitive_Types}
\index{types!primitive}

%REVIEW: vass: we should explain what separates primitive types from the rest.
The primitive types are: \chpl{void}, \chpl{bool},
\chpl{int}, \chpl{uint}, \chpl{real}, \chpl{imag}, \chpl{complex},
and \chpl{string}.  They are defined
in this section.

The primitive types are summarized by the following syntax:
\begin{syntax}
primitive-type:
  `void'
  `bool' primitive-type-parameter-part[OPT]
  `int' primitive-type-parameter-part[OPT]
  `uint' primitive-type-parameter-part[OPT]
  `real' primitive-type-parameter-part[OPT]
  `imag' primitive-type-parameter-part[OPT]
  `complex' primitive-type-parameter-part[OPT]
  `string'

primitive-type-parameter-part:
  ( integer-parameter-expression )

integer-parameter-expression:
  expression
\end{syntax}

If present, the parenthesized \sntx{integer-parameter-expression} must
evaluate to a compile-time constant of integer type.  See~\rsec{Compile-Time_Constants}

\begin{openissue}
There is an expectation of future support for larger bit width
primitive types depending on a platform's native support for those
types.
\end{openissue}

\subsection{The Void Type}
\label{The_Void_Type}
\index{void@\chpl{void}}
\index{types!void@\chpl{void}}

The \chpl{void} type is used to represent the lack of a value, for
example when a function has no arguments and/or no return type.  

There may be storage associated with a value of type \chpl{void}, in which
case its lifetime obeys the same rules as a value of type \chpl{int}.

\subsection{The Bool Type}
\label{The_Bool_Type}
\index{bool@\chpl{bool}}
\index{types!bool@\chpl{bool}}

Chapel defines a logical data type designated by the symbol
\chpl{bool} with the two predefined values \chpl{true} and
\chpl{false}.  This default boolean type is stored using an
implementation-defined number of bits.  A particular number of bits
can be specified using a parameter value following the \chpl{bool}
keyword, such as \chpl{bool(8)} to request an 8-bit boolean value.
Legal sizes are 8, 16, 32, and 64 bits.

%% The relational operators return values of \chpl{bool} type and the
%% logical operators operate on values of \chpl{bool} type.

Some statements require expressions of \chpl{bool} type and Chapel
supports a special conversion of values to \chpl{bool} type when used
in this context~(\rsec{Implicit_Statement_Bool_Conversions}).

\subsection{Signed and Unsigned Integral Types}
\label{Signed_and_Unsigned_Integral_Types}
\index{uint@\chpl{uint}}
\index{int@\chpl{int}}
\index{types!uint@\chpl{uint}}
\index{types!int@\chpl{int}}

The integral types can be parameterized by the number of bits used to
represent them.  Valid bit-sizes are 8, 16, 32, and 64.  The default signed integral type, \chpl{int}, and the
default unsigned integral type, \chpl{uint} correspond to 
\chpl{int(64)} and \chpl{uint(64)} respectively.

% BLC: I either don't understand this or don't believe it's true:
%
%, are treated as distinct types (for
%the purposes of type resolution) but behave like 

The integral types and their ranges are given in the following table:

\begin{center}
\begin{tabular}{|l|r|r|}
\hline
{\bf Type} & {\bf Minimum Value} & {\bf Maximum Value} \\
\hline
{\tt int(8)} & -128 & 127 \\
{\tt uint(8)} & 0 & 255 \\
{\tt int(16)} & -32768 & 32767 \\
{\tt uint(16)} & 0 & 65535 \\
{\tt int(32)} & -2147483648 & 2147483647 \\
{\tt uint(32)} & 0 & 4294967295 \\
{\tt int(64)}, {\tt int} & -9223372036854775808 & 9223372036854775807 \\
{\tt uint(64)}, {\tt uint} & 0 & 18446744073709551615 \\
\hline
\end{tabular}
\end{center}

The unary and binary operators that are pre-defined over the integral
types operate with 32- and 64-bit precision.  Using these operators on
integral types represented with fewer bits results in an implicit
conversion to the corresponding 32-bit types
according to the rules defined in~\rsec{Implicit_Conversions}.


\subsection{Real Types}
\label{Real_Types}
\index{real@\chpl{real}}
\index{types!real@\chpl{real}}

Like the integral types, the real types can be parameterized by the
number of bits used to represent them.  The default real
type, \chpl{real}, is 64 bits.  The real types that are supported are
machine-dependent, but usually include \chpl{real(32)} (single
precision) and \chpl{real(64)} (double precision) following the IEEE
754 standard.  

\subsection{Imaginary Types}
\label{Imaginary_Types}
\index{imaginary@\chpl{imaginary}}
\index{types!imaginary@\chpl{imaginary}}

The imaginary types can be parameterized by the number of bits used to
represent them.  The default imaginary type, \chpl{imag}, is 64 bits.
The imaginary types that are supported are machine-dependent, but
usually include \chpl{imag(32)} and \chpl{imag(64)}.

\begin{rationale}
The imaginary type is included to avoid numeric instabilities and
under-optimized code stemming from always converting real values to
complex values with a zero imaginary part.
\end{rationale}

\subsection{Complex Types}
\label{Complex_Types}
\index{complex@\chpl{complex}}
\index{types!complex@\chpl{complex}}

Like the integral and real types, the complex types can be
parameterized by the number of bits used to represent them.  A complex
number is composed of two real numbers so the number of bits used to
represent a complex is twice the number of bits used to represent the
real numbers.  The default complex type, \chpl{complex}, is 128 bits;
it consists of two 64-bit real numbers.  The complex types that are
supported are machine-dependent, but usually
include \chpl{complex(64)} and \chpl{complex(128)}.

The real and imaginary components can be accessed via the methods
\chpl{re} and \chpl{im}.  The type of these components is real.
The standard \chpl{Math} module provides some functions on
complex types. See
\\ %formatting
\mbox{$$ $$ $$ $$ $$} %indent
\url{http://chapel.cray.com/docs/latest/modules/standard/Math.html}

\begin{example}
Given a complex number \chpl{c} with the value \chpl{3.14+2.72i}, the
expressions \chpl{c.re} and \chpl{c.im} refer to \chpl{3.14}
and \chpl{2.72} respectively.
\end{example}

\subsection{The String Type}
\label{The_String_Type}
\index{string@\chpl{string}}
\index{types!string@\chpl{string}}

Strings are a primitive type designated by the symbol \chpl{string}
comprised of ASCII characters.  Their length is unbounded.
See~\rsec{Standard} for routines for manipulating strings.


\begin{openissue}
There is an expectation of future support for fixed-length strings.
\end{openissue}

\begin{openissue}
There is an expectation of future support for different character
sets, possibly including internationalization.
\end{openissue}


\section{Enumerated Types}
\label{Enumerated_Types}
\index{enumerated types}
\index{types!enumerated}

Enumerated types are declared with the following syntax:

\begin{syntax}
enum-declaration-statement:
  `enum' identifier { enum-constant-list }

enum-constant-list:
  enum-constant
  enum-constant , enum-constant-list[OPT]

enum-constant:
  identifier init-part[OPT]

init-part:
  = expression
\end{syntax}

The enumerated type can then be referenced by its name, as summarized
by the following syntax:

\begin{syntax}
enum-type:
  identifier
\end{syntax}

An enumerated type defines a set of named constants that can be
referred to via a member access on the enumerated type.
These constants are treated as parameters of integral type.  Each
enumerated type is a distinct type. If the \sntx{init-part} is
omitted, the \sntx{enum-constant} has an integral value one higher
than the previous \sntx{enum-constant} in the enum, with the first
having the value \chpl{1}.

\begin{chapelexample}{enum.chpl}
The code
\begin{chapel}
enum statesman { Aristotle, Roosevelt, Churchill, Kissinger }
\end{chapel}
defines an enumerated type with four constants.  The function
\begin{chapel}
proc quote(s: statesman) {
  select s {
    when statesman.Aristotle do
       writeln("All paid jobs absorb and degrade the mind.");
    when statesman.Roosevelt do
       writeln("Every reform movement has a lunatic fringe.");
    when statesman.Churchill do
       writeln("A joke is a very serious thing.");
    when statesman.Kissinger do
       { write("No one will ever win the battle of the sexes; ");
         writeln("there's too much fraternizing with the enemy."); }
  }
}
\end{chapel}
\begin{chapelnoprint}
for s in statesman.Aristotle..statesman.Kissinger do
  quote(s:statesman);
\end{chapelnoprint}
\begin{chapeloutput}
All paid jobs absorb and degrade the mind.
Every reform movement has a lunatic fringe.
A joke is a very serious thing.
No one will ever win the battle of the sexes; there's too much fraternizing with the enemy.
\end{chapeloutput}
outputs a quote from the given statesman.  Note that enumerated
constants must be prefixed by the enumerated type and a dot.
\end{chapelexample}


\clearpage
\section{Structured Types}
\label{Structured_Types}
\index{types!structured}

The structured types are summarized by the following syntax:

\begin{syntax}
structured-type:
  class-type
  record-type
  union-type
  tuple-type
\end{syntax}
% in README.firstClassFns: function-type

Classes are discussed in \rsec{Classes}.  Records are discussed
in \rsec{Records}.  Unions are discussed in \rsec{Unions}.  Tuples are
discussed in \rsec{Tuples}.

\subsection{Class Types}
\label{Types_Class_Types}

% TODO: the first sentence does not make strict sense.
% Overall, it does not feel like this paragraph has any value here.
The class type defines a type that contains variables and constants,
called fields, and functions, called methods.  Classes are defined
in~\rsec{Classes}.  The class type can also contain type aliases and
parameters.  Such a class is generic and is defined
in~\rsec{Generic_Types}.

\subsection{Record Types}
\label{Types_Record_Types}

The record type is similar to a class type; the primary difference is
that a record is a value rather than a reference.  Records are defined
in~\rsec{Records}.

\subsection{Union Types}
\label{Types_Union_Types}

The union type defines a type that contains one of a set of variables.
Like classes and records, unions may also define methods.  Unions are
defined in~\rsec{Unions}.

\subsection{Tuple Types}
\label{Types_Tuple_Types}

A tuple is a light-weight record that consists of one or more
anonymous fields.  If all the fields are of the same type, the tuple
is homogeneous.  Tuples are defined in~\rsec{Tuples}.

\clearpage
\section{Data Parallel Types}
\label{Data_Parallel_Types}
\index{types!dataparallel}

The data parallel types are summarized by the following syntax:

\begin{syntax}
dataparallel-type:
  range-type
  domain-type
  mapped-domain-type
  array-type
  index-type
\end{syntax}

Ranges and their index types are discussed in \rsec{Ranges}.
Domains and their index types are discussed in \rsec{Domains}.
Arrays are discussed in \rsec{Arrays}.

\subsection{Range Types}
\label{Types_Range_Types}

A range defines an integral sequence of some integral type.  Ranges
are defined in \rsec{Ranges}.

\subsection{Domain, Array, and Index Types}
\label{Domain_and_Array_Types}

A domain defines a set of indices. An array defines a set of
elements that correspond to the indices in its domain.
A domain's indices can be of any type.
Domains, arrays, and their index
types are defined in \rsec{Domains} and \rsec{Arrays}.

\section{Synchronization Types}
\label{Synchronization_Types}
\index{types!synchronization}

The synchronization types are summarized by the following syntax:

\begin{syntax}
synchronization-type:
  sync-type
  single-type
  atomic-type
\end{syntax}

Sync and single types are discussed in \rsec{Synchronization_Variables}.
The atomic type is discussed in \rsec{Atomic_Variables}.

\clearpage
\section{Type Aliases}
\label{Type_Aliases}
\index{types!aliases}

Type aliases are declared with the following syntax:
\begin{syntax}
type-alias-declaration-statement:
  privacy-specifier[OPT] `config'[OPT] `type' type-alias-declaration-list ;
  external-type-alias-declaration-statement

type-alias-declaration-list:
  type-alias-declaration
  type-alias-declaration , type-alias-declaration-list

type-alias-declaration:
  identifier = type-specifier
  identifier
\end{syntax}
A type alias is a symbol that aliases the type specified in the
\sntx{type-part}.  A use of a type alias has the same meaning as using
the type specified by \sntx{type-part} directly.

Type aliases defined at the module level are public by default.  The
optional \sntx{privacy-specifier} keywords are provided to specify or
change this behavior.  For more details on the visibility of symbols,
see ~\rsec{Visibility_Of_Symbols}.

If the keyword \chpl{config} precedes the keyword \chpl{type}, the
type alias is called a configuration type alias.  Configuration type
aliases can be set at compilation time via compilation flags or other
implementation-defined means.  The \sntx{type-specifier} in the
program is ignored if the type-alias is alternatively set.

If the keyword \chpl{extern} precedes the \chpl{type} keyword, the type alias is
external.  The declared type name is used by Chapel for type resolution, but no
type alias is generated by the backend.  See the chapter on interoperability
(\rsec{Interoperability}) for more information on external types.

The \sntx{type-part} is optional in the definition of a class or
record.  Such a type alias is called an unspecified type
alias. Classes and records that contain type aliases, specified or
unspecified, are generic~(\rsec{Type_Aliases_in_Generic_Types}).

\begin{openissue}
There is on going discussion on whether a type alias is a new
type or simply an alias.  The former should enable redefinition of
default values, identity elements, etc.
%hilde
% Would inheritance work?
\end{openissue}

\cleardoublepage
\sekshun{Variables}
\label{Variables}
\index{variables}

A variable is a symbol that represents memory.  Chapel is a
statically-typed, type-safe language so every variable has a type that
is known at compile-time and the compiler enforces that values
assigned to the variable can be stored in that variable as specified
by its type.

\section{Variable Declarations}
\label{Variable_Declarations}
\index{variables!declarations}
\index{declarations!variables}

Variables are declared with the following syntax:
\begin{syntax}
variable-declaration-statement:
  privacy-specifier[OPT] config-or-extern[OPT] variable-kind variable-declaration-list ;

config-or-extern: one of
  `config' $ $ $ $ `extern'

variable-kind:
  `param'
  `const'
  `var'
  `ref'
  `const ref'

variable-declaration-list:
  variable-declaration
  variable-declaration , variable-declaration-list

variable-declaration:
  identifier-list type-part[OPT] initialization-part
  identifier-list type-part no-initialization-part[OPT]
  array-alias-declaration

type-part:
  : type-specifier

initialization-part:
  = expression

no-initialization-part:
  = `noinit'

identifier-list:
  identifier
  identifier , identifier-list
  tuple-grouped-identifier-list
  tuple-grouped-identifier-list , identifier-list

tuple-grouped-identifier-list:
  ( identifier-list )
\end{syntax}
A \sntx{variable-declaration-statement} is used to define one or more
variables.  If the statement is a top-level module statement, the
variables are module level; otherwise they are local.  Module level variables are
discussed in~\rsec{Module_Level_Variables}.  Local variables are discussed
in~\rsec{Local_Variables}.

The optional \sntx{privacy-specifier} keywords indicate the visibility
of module level variables to outside modules.  By default, variables
are publicly visible.  More details on visibility can be found in
~\rsec{Visibility_Of_Symbols}.

The optional keyword \chpl{config} specifies that the variables are
configuration variables, described in
Section~\rsec{Configuration_Variables}.  The optional keyword \chpl{extern}
indicates that the variable is externally defined.  Its name and type are used
within the Chapel program for resolution, but no space is allocated for it and
no initialization code emitted.
See \rsec{Shared_Data} for further details.

The \sntx{variable-kind} specifies whether the variables are
parameters (\chpl{param}), constants (\chpl{const}),
ref variables (\chpl{ref}), or regular
variables (\chpl{var}).  Parameters are compile-time constants whereas
constants are runtime constants.  Both levels of constants are
discussed in~\rsec{Constants}.
Ref variables are discussed in \rsec{Ref_Variables}.

The \sntx{type-part} of a variable declaration specifies the type of
the variable.  It is optional if the \sntx{initialization-part} is
specified.  If the \sntx{type-part} is omitted, the type of the
variable is inferred using local type inference described
in~\rsec{Local_Type_Inference}.

The \sntx{initialization-part} of a variable declaration specifies an
initial expression to assign to the variable.  If
the \sntx{initialization-part} is omitted, the \sntx{type-part} must
be present, and the variable is initialized to the default value of
its type as described in~\rsec{Default_Values_For_Types}.

If the \sntx{no-initialization-part} is present, the variable
declaration does not initialize the variable to any value, as
described in~\rsec{Noinit_Capability}. The result of any read of an
uninitialized variable is undefined until that variable is written.

Multiple variables can be defined in the
same \sntx{variable-declaration-list}.  The semantics of declaring
multiple variables that share an \sntx{initialization-part}
and/or \sntx{type-part} is defined in~\rsec{Multiple_Variable_Declarations}.

Multiple variables can be grouped together using a tuple notation as
described in~\rsec{Variable_Declarations_in_a_Tuple}.

The \sntx{array-alias-declaration} is defined
in~\rsec{Array_Aliases}.

\subsection{Default Initialization}
\label{Default_Values_For_Types}
\index{default initialization!variables}
\index{variables!default initialization}
\index{variables!default values}

If a variable declaration has no initialization expression, a variable
is initialized to the default value of its type.  The default values
are as follows:
\begin{center}
\begin{tabular}{|l|l|}
\hline
{\bf Type} & {\bf Default Value} \\
\hline
{\tt bool(*)} & {\tt false} \\
{\tt int(*)} & {\tt 0} \\
{\tt uint(*)} & {\tt 0} \\
{\tt real(*)} & {\tt 0.0} \\
{\tt imag(*)} & {\tt 0.0i} \\
{\tt complex(*)} & {\tt 0.0 + 0.0i} \\
{\tt string} & {\tt ""} \\
enums & first enum constant \\
classes & {\tt nil} \\
records & default constructed record \\
ranges & {\tt 1..0} $ $ $ $ (empty sequence) \\
arrays & elements are default values \\
tuples & components are default values \\
sync/single & base default value and \emph{empty} status \\
atomic & base default value \\
\hline
\end{tabular}
\end{center}

\begin{openissue}
In the case that the first enumerator in an enumeration type is offset
from zero, as in
\begin{example}
enum foo \{ red = 0xff0000, green = 0xff00, blue = 0xff \} ;
\end{example}
the compiler has to look up the first named type to see what to use as
the default.  

An alternative would be to specify that the default
value is the enumerator whose underlying value is zero.  But that approach also
has issues, since the default value does not conform to any named enumerator.
\end{openissue}

\subsection{Deferred Initialization}
\label{Noinit_Capability}
\index{noinit}
\index{noinit!variables}

For performance purposes, a variable's declaration can specify that
the variable should not be default initialized by using
the \chpl{noinit} keyword in place of an initialization expression.
Since this variable should be written at a later point in order to be
read properly, it must be a regular variable (\chpl{var}).  It is
incompatible with declarations that require the variable to remain
unchanged throughout the program's lifetime, such as \chpl{const}
or \chpl{param}.  Additionally, its type must be specified at
declaration time.

The result of any read of this variable before it is written is
undefined; it exists and therefore can be accessed, but no guarantees
are made as to its contents.

\subsection{Local Type Inference}
\label{Local_Type_Inference}
\index{type inference}
\index{type inference!local}

If the type is omitted from a variable declaration, the type of the
variable is defined to be the type of the initialization expression.
With the exception of sync and single expressions, the declaration
\begin{chapel}
var v = e;
\end{chapel}
is equivalent to
\begin{chapel}
var v: e.type = e;
\end{chapel}
for an arbitrary expression \chpl{e}.  If \chpl{e} is of sync or
single type, the type of \chpl{v} is the base type of \chpl{e}.

\subsection{Multiple Variable Declarations}
\label{Multiple_Variable_Declarations}
\index{declarations!variables!multiple}
\index{variables!declarations!multiple}

All variables defined in the same \sntx{identifier-list} are defined
such that they have the same type and value, and so that the type and
initialization expression are evaluated only once.

\begin{chapelexample}{multiple.chpl}
In the declaration
\begin{chapel}
proc g() { writeln("side effect"); return "a string"; }
var a, b = 1.0, c, d:int, e, f = g();
\end{chapel}
\begin{chapelpost}
writeln((a,b,c,d,e,f));
\end{chapelpost}
variables \chpl{a} and \chpl{b} are of type \chpl{real} with
value \chpl{1.0}.  Variables \chpl{c} and \chpl{d} are of
type \chpl{int} and are initialized to the default value of \chpl{0}.
Variables \chpl{e} and \chpl{f} are of type \chpl{string} with
value \chpl{"a string"}.  The string \chpl{"side effect"} has been
written to the display once.  It is not evaluated twice.
\begin{chapeloutput}
side effect
(1.0, 1.0, 0, 0, a string, a string)
\end{chapeloutput}
\end{chapelexample}

The exact way that multiple variables are declared is defined as
follows:
\begin{itemize}
\item If the variables in the \sntx{identifier-list} are declared
with a type, but without an initialization expression as in
\begin{chapel}
var v1, v2, v3: t;
\end{chapel}
for an arbitrary type expression \chpl{t}, then the declarations are
rewritten so that the first variable is declared to be of
type \chpl{t} and each later variable is declared to be of the type of
the first variable as in
\begin{chapel}
var v1: t; var v2: v1.type; var v3: v1.type;
\end{chapel}

\item If the variables in the \sntx{identifier-list} are declared
without a type, but with an initialization expression as in
\begin{chapel}
var v1, v2, v3 = e;
\end{chapel}
for an arbitrary expression \chpl{e}, then the declarations are
rewritten so that the first variable is initialized by
expression \chpl{e} and each later variable is initialized by the
first variable as in
\begin{chapel}
var v1 = e; var v2 = v1; var v3 = v1;
\end{chapel}

\item If the variables in the \sntx{identifier-list} are declared
with both a type and an initialization expression as in
\begin{chapel}
var v1, v2, v3: t = e;
\end{chapel}
for an arbitrary type expression \chpl{t} and an arbitrary
expression \chpl{e}, then the declarations are rewritten so that the
first variable is declared to be of type \chpl{t} and initialized by
expression \chpl{e}, and each later variable is declared to be of the
type of the first variable and initialized by the result of calling
the function \chpl{readXX} on the first variable as in
\begin{chapel}
var v1: t = e; var v2: v1.type = readXX(v1); var v3: v1.type = readXX(v1);
\end{chapel}
where the function \chpl{readXX} is defined as follows:
\begin{chapel}
proc readXX(x: sync) return x.readXX();
proc readXX(x: single) return x.readXX();
proc readXX(x) return x;
\end{chapel}
Note that the use of the helper function \chpl{readXX()}
in this code fragment is solely for the purposes of illustration.
It is not actually a part of Chapel's semantics or implementation.
\end{itemize}

\begin{rationale}
This algorithm is complicated by the existence of \emph{sync}
and \emph{single} variables.  If these did not exist, we could rewrite
any multi-variable declaration such that later variables were simply
initialized by the first variable and the first variable was defined
as if it appeared alone in the \sntx{identifier-list}.  However,
both \emph{sync} and \emph{single} variables require careful handling
to avoid unintentional changes to their \emph{full}/\emph{empty}
state.
\end{rationale}

\section{Module Level Variables}
\label{Module_Level_Variables}
\index{variables!module level}

Variables declared in statements that are in a module but not in a
function or block within that module are module level variables.
Module level variables can be accessed anywhere within that module
after the declaration of that variable.  If they are public, they can
also be accessed in other modules that use that module.

\section{Local Variables}
\label{Local_Variables}
\index{variables!local}

Local variables are declared within block statements.  They can only
be accessed within the scope of that block statement (including all
inner nested block statements and functions).

A local variable only exists during the execution of code that lies
within that block statement.  This time is called the lifetime of the
variable.  When execution has finished within that block statement,
the local variable and the storage it represents is removed.
Variables of class type are the sole exception.  Constructors of class
types create storage that is not associated with any scope.  Such
storage can be reclaimed as described
in~\rsec{Dynamic_Memory_Management}.

\section{Constants}
\label{Constants}
\index{constants}

Constants are divided into two categories: parameters, specified with
the keyword \chpl{param}, are compile-time constants and constants,
specified with the keyword \chpl{const}, are runtime constants.

\subsection{Compile-Time Constants}
\label{Compile-Time_Constants}
\index{constants!compile-time}
\index{parameters}
\index{param@\chpl{param}}

A compile-time constant, or ``parameter'', must have a single value that is
known statically by the compiler.  Parameters are restricted to
primitive and enumerated types.

Parameters can be assigned expressions that are parameter expressions.
Parameter expressions are restricted to the following constructs:
\begin{itemize}
\item
 Literals of primitive or enumerated type.
\item
 Parenthesized parameter expressions.
\item
 Casts of parameter expressions to primitive or enumerated types.
\item
 Applications of the unary operators \verb@+@, \verb@-@, \verb@!@,
 and \verb@~@ on operands that are bool or integral parameter
 expressions.
\item
 Applications of the unary operators \verb@+@ and \verb@-@ on operands that are
 real, imaginary or complex parameter
 expressions.
\item
 Applications of the binary operators \verb@+@, \verb@-@, \verb@*@, \verb@/@, \verb@%@, \verb@**@, \verb@&&@, \verb@||@, \verb@&@, \verb@|@, \verb@^@, \verb@<<@, \verb@>>@, \verb@==@, \verb@!=@, \verb@<=@, \verb@>=@, \verb@<@, and \verb@>@ on operands that are bool or integral parameter expressions.
\item
 Applications of the binary
 operators \verb@+@, \verb@-@, \verb@*@, \verb@/@, \verb@**@, \verb@==@, \verb@!=@, \verb@<=@, \verb@>=@, \verb@<@,
 and \verb@>@ on operands that are real, imaginary or complex parameter expressions.
\item
 Applications of the string concatenation operator \verb@+@, string comparison operators \verb@==@, \verb@!=@, \verb@<=@, \verb@>=@, \verb@<@, \verb@>@, and the string length and ascii functions on parameter string expressions.
\item
 The conditional expression where the condition is a parameter and the
 then- and else-expressions are parameters.
\item
 Call expressions of parameter functions.  See~\rsec{Param_Return_Intent}.
\end{itemize}

\subsection{Runtime Constants}
\label{Runtime_Constants}
\index{constants!runtime}
\index{constants}
\index{const@\chpl{const}}

Runtime constants, or simply ``constants'',
do not have the restrictions that are associated with
parameters.  Constants can be of any type.  They require an initialization
expression and contain the value of that expression throughout their lifetime.

A variable of a class type that is a constant is a constant reference.
That is, the variable always
points to the object that it was initialized to reference.
However, the fields of that object are allowed to be modified.

\section{Configuration Variables}
\label{Configuration_Variables}
\index{variables!configuration}
\index{constants!configuration}
\index{config@\chpl{config}}

If the keyword \chpl{config} precedes the
keyword \chpl{var}, \chpl{const}, or \chpl{param}, the variable,
constant, or parameter is called a configuration variable,
configuration constant, or configuration parameter respectively.  Such
variables, constants, and parameters must be at the module level.

The initialization of these variables can be set via implementation
dependent means, such as command-line switches or environment
variables.  The initialization expression in the program is ignored if
the initialization is alternatively set.

\index{parameters!configuration}
Configuration parameters are set at compilation time via compilation
flags or other implementation-defined means.  The value passed via
these means can be an arbitrary Chapel expression as long as the
expression can be evaluated at compile-time.  If present, the value thus
supplied overrides the default value appearing in the Chapel code.

\begin{chapelexample}{config-param.chpl}
For example,
\begin{chapel}
config param rank = 2;
\end{chapel}
\begin{chapelnoprint}
writeln(rank);
\end{chapelnoprint}
\begin{chapeloutput}
2
\end{chapeloutput}
sets an integer parameter \chpl{rank} to \chpl{2}.
At compile-time, this default value of \chpl{rank} can be overridden
with another parameter expression, such as \chpl{3} or \chpl{2*n},
provided \chpl{n} itself is a parameter. The \chpl{rank}
configuration variable can be used to write rank-independent code.
\end{chapelexample}

\section{Ref Variables}
\label{Ref_Variables}
\index{variables!ref}
\index{ref@\chpl{ref}}

A \emph{ref} variable is a variable declared using the \chpl{ref} keyword.
A ref variable serves as an alias to another variable, field or array element.
The declaration of a ref variable must contain \sntx{initialization-part},
which specifies what is to be aliased and can be a variable
or any lvalue expression.

Access or update to a ref variable is equivalent to access or update
to the variable being aliased. For example, an update to a ref variable
is visible via the original variable, and visa versa.

If the expression being aliased is a runtime constant variable,
a formal argument with a \chpl{const ref} concrete intent
(\rsec{Concrete Intents}), or a call to a function with a \chpl{const ref}
return intent (\rsec{Const_Ref_Return_Intent}), the corresponding
ref variable must be declared as \chpl{const ref}.
Parameter constants and expressions cannot be aliased.

\begin{openissue}
The behavior of a \chpl{const ref} alias to a non-\chpl{const} variable
is an open issue. The options include disallowing such an alias,
disallowing changes to the variable while it can be accessed via
a \chpl{const ref} alias, making changes visible through the alias,
and making the behavior undefined.
\end{openissue}

\begin{chapelexample}{refVariables.chpl}
For example, the following code:

\begin{chapel}
var myInt = 51;
ref refInt = myInt;                   // alias of a local or global variable
myInt = 62;
writeln("refInt = ", refInt);
refInt = 73;
writeln("myInt = ", myInt);

var myArr: [1..3] int = 51;
proc arrayElement(i) ref  return myArr[i];
ref refToExpr = arrayElement(3);      // alias to lvalue returned by a function
myArr[3] = 62;
writeln("refToExpr = ", refToExpr);
refToExpr = 73;
writeln("myArr[3] = ", myArr[3]);

const constArr: [1..3] int = 51..53;
const ref myConstRef = constArr[2];   // would be an error without 'const'
writeln("myConstRef = ", myConstRef);
\end{chapel}

prints out:

\begin{chapelprintoutput}{}
refInt = 62
myInt = 73
refToExpr = 62
myArr[3] = 73
myConstRef = 52
\end{chapelprintoutput}
\end{chapelexample}

\cleardoublepage
\sekshun{Conversions}
\label{Conversions}
\index{conversions}

A \emph{conversion} converts an expression of one type to another type,
possibly changing its value.
\index{conversions!source type}
\index{conversions!target type}
We refer to these two types the \emph{source} and \emph{target} types.
Conversions can be either
implicit~(\rsec{Implicit_Conversions}) or
explicit~(\rsec{Explicit_Conversions}).


\section{Implicit Conversions}
\label{Implicit_Conversions}
\index{conversions!implicit}

An \emph{implicit conversion} is a conversion that occurs implicitly,
that is, not due to an explicit specification in the program.
Implicit conversions occur at the locations in the program listed below.
Each location determines the target type.
The source and target types of an implicit conversion must be allowed.
They determine whether and how the expression's value changes.

Implicit conversions are not applied when initializing \chpl{ref} or
\chpl{type} values or for actual arguments passed to \chpl{ref} or
\chpl{type} formal arguments.

\index{conversions!implicit!occurs at}
An implicit conversion occurs at each of the following program locations:

\begin{itemize}
\item In an assignment, the expression on the right-hand side of
      the assignment is converted to the type of the variable
      or another lvalue on the left-hand side of the assignment.

\item The actual argument of a function call or an operator is converted
      to the type of the corresponding formal argument, if the formal's
      intent is \chpl{param}, \chpl{in}, \chpl{const in}, or an abstract intent
      (\rsec{Abstract_Intents}) with the semantics of
      \chpl{in} or \chpl{const in}.

% MPF: This rule doesn't seem to be implemented right now,
%      but rather reflects ideal language design.
\item If the formal argument's intent is \chpl{out}, the formal argument
      is converted to the type of the corresponding actual argument
      upon function return.

\item The return or yield expression within a function without a \chpl{ref}
      return intent is converted to the return type of that function.

\item The condition of a conditional expression,
      conditional statement, while-do or do-while loop statement
      is converted to the boolean type~(\rsec{Implicit_Statement_Bool_Conversions}).
      A special rule defines the allowed source types and
      how the expression's value changes in this case.
\end{itemize}

\index{conversions!implicit!allowed types}
Implicit conversions \emph{are allowed} between
the following source and target types,
as defined in the referenced subsections:

\begin{itemize}
\item numeric and boolean types~(\rsec{Implicit_NumBool_Conversions}),
\item class types~(\rsec{Implicit_Class_Conversions}),
\item integral types in the special case when the expression's value
      is a compile-time constant~(\rsec{Implicit_Compile_Time_Constant_Conversions}), and
\item from an integral or class type to \chpl{bool}
      in certain cases~(\rsec{Implicit_Statement_Bool_Conversions}).
\end{itemize}

In addition,
an implicit conversion from a type to the same type is allowed for any type.
Such conversion does not change the value of the expression.

% TODO: If an implicit conversion is not allowed, it is an error.

Implicit conversion is not transitive. That is, if an implicit conversion
is allowed from type \chpl{T1} to \chpl{T2} and from \chpl{T2} to \chpl{T3},
that by itself does not allow an implicit conversion
from \chpl{T1} to \chpl{T3}.

\subsection{Implicit Numeric and Bool Conversions}
\label{Implicit_NumBool_Conversions}

\index{conversions!numeric}
\index{conversions!implicit!numeric}
Implicit conversions among numeric types are allowed when
all values representable in the source type can also be represented
in the target type, retaining their full precision.
%
%REVIEW: vass: I did not understand the point of the following,
% so I am commenting it out for now.
%When the implicit conversion is from an integral to a real type, source
%types are converted to type \chpl{int} before determining if the
%conversion is valid.
%
In addition, implicit conversions from
types \chpl{int(64)} and \chpl{uint(64)} to types \chpl{real(64)}
and \chpl{complex(128)} are allowed, even though they may result in a loss of
precision.

%REVIEW: hilde
% Unless we are supporting some legacy behavior, I would recommend removing this
% provision.  A loss of precision is a loss of precision, so I would favor
% consistent behavior that does not lead to surprising results.  EVERY ``if''
% costs money: which is to say that if a behavior can be described simply, it can
% be implemented simply.

\begin{rationale}
We allow these additional conversions because they are an important
convenience for application programmers. Therefore we are willing to
lose precision in these cases. The largest real and complex types
are chosen to retain precision as often as as possible.
\end{rationale}

\index{conversions!boolean}
\index{conversions!implicit!boolean}
Any boolean type can be implicitly converted to any other boolean type,
retaining the boolean value.
Any boolean type can be implicitly converted to any integral type
by representing \chpl{false} as 0 and \chpl{true} as 1,
except (if applicable)
a boolean cannot be converted to \chpl{int(1)}.
% Rationale: because 1 cannot be represented by \chpl{int(1)}.

\begin{rationale}
We disallow implicit conversion of a boolean to
a real, imaginary, or complex type because of the following.
We expect that the cases where such a conversion is needed
will more likely be unintended by the programmer.
Marking those cases as errors will draw the programmer's attention.
If such a conversion is actually desired, a cast \rsec{Explicit_Conversions}
can be inserted.
\end{rationale}

Legal implicit conversions with numeric and boolean types
may thus be tabulated as follows:

\begin{center}
\begin{tabular}{l|llllll}
& \multicolumn{6}{c}{Target Type} \\ [4pt]

Source Type  & bool($t$) & uint($t$) & int($t$) & real($t$) & imag($t$) & complex($t$) \\  [3pt]

\cline{1-7} \\

bool($s$)    & all $s,t$ & all $s,t$   & all $s$; $2 \le t$ & & & \\ [7pt]

uint($s$)    & & $s \le t$ & $s < t$   & $s \le mant(t)$   & & $s \le mant(t/2)$   \\ [7pt]

uint(64)     & &           &           & real(64)          & & complex(128)        \\ [7pt]

int($s$)     & &           & $s \le t$ & $s \le mant(t)+1$ & & $s \le mant(t/2)+1$ \\ [7pt]

int(64)      & &           &           & real(64)          & & complex(128)        \\ [7pt]

real($s$)    & & & & $s \le t$ &           & $s \le t/2$ \\ [7pt]

imag($s$)    & & & &           & $s \le t$ & $s \le t/2$ \\ [7pt]

complex($s$) & & & &           &           & $s \le t$   \\ [5pt]

\end{tabular}
\end{center}
Here, $mant(i)$ is the number of bits in the (unsigned) mantissa of
the $i$-bit floating-point type.\footnote{For the IEEE 754 format,
$mant(32)=24$ and $mant(64)=53$.}
%
Conversions for the default integral and real types (\chpl{uint},
\chpl{complex}, etc.) are the same as for their
explicitly-sized counterparts.

\subsection{Implicit Compile-Time Constant Conversions}
\label{Implicit_Compile_Time_Constant_Conversions}
\index{conversions!numeric!parameter}
\index{conversions!implicit!parameter}

The following implicit conversion of a parameter is allowed:
\begin{itemize}
\item A parameter of type \chpl{int(64)} can be implicitly converted
to \chpl{int(8)}, \chpl{int(16)}, \chpl{int(32)}, or any unsigned integral type if the
value of the parameter is within the range of the target type.
\end{itemize}

\subsection{Implicit Statement Bool Conversions}
\label{Implicit_Statement_Bool_Conversions}
\index{conversions!boolean!in a statement}
\index{conversions!implicit!boolean}

In the condition of an if-statement, while-loop, and do-while-loop,
the following implicit conversions to \chpl{bool} are supported:
\begin{itemize}
\item An expression of integral type is taken to be false if it is zero and is true otherwise.
\item An expression of a class type is taken to be false if it is nil and is true otherwise.
\end{itemize}

\section{Explicit Conversions}
\label{Explicit_Conversions}
\index{conversions!explicit}

Explicit conversions require a cast in the code.  Casts are defined
in~\rsec{Casts}.  Explicit conversions are supported between more
types than implicit conversions, but explicit conversions are not
supported between all types.

The explicit conversions are a superset of the implicit conversions.
In addition to the following definitions,
an explicit conversion from a type to the same type is allowed for any type.
Such conversion does not change the value of the expression.

\subsection{Explicit Numeric Conversions}
\label{Explicit_Numeric_Conversions}
\index{conversions!numeric}
\index{conversions!explicit!numeric}

Explicit conversions are allowed from any numeric type, boolean, or
string to any other numeric type, boolean, or string.  

% A valid \chpl{bool} value behaves like a single unsigned bit.  
When a \chpl{bool} is converted to a \chpl{bool}, \chpl{int}
or \chpl{uint} of equal or larger size, its value is zero-extended to fit the
new representation.  When a \chpl{bool} is converted to a
smaller \chpl{bool}, \chpl{int} or \chpl{uint}, its most significant
bits are truncated (as appropriate) to fit the new representation.

When a \chpl{int}, \chpl{uint}, or \chpl{real} is converted to a \chpl{bool}, the result is \chpl{false} if the number was equal to 0 and \chpl{true} otherwise.
% This has the odd effect that a bool stored in a signed one-bit bitfield would
% change sign without generating a conversion error.  But its subsequent
% conversion back to a bool would yield the original value.
% In regard to supporting bitfields: Be careful what you wish for.

% The source type determines whether a value is zero- or sign-extended.
When an \chpl{int} is converted to a larger \chpl{int} or \chpl{uint}, its value is
sign-extended to fit the new representation.  
When a \chpl{uint} is converted to a larger \chpl{int} or \chpl{uint}, its value
is zero-extended.
When an \chpl{int} or \chpl{uint} is converted to an \chpl{int} or \chpl{uint}
of the same size, its binary representation is unchanged.
When an \chpl{int} or \chpl{uint} is converted to a smaller \chpl{int}
or \chpl{uint}, its value is truncated to fit the new representation.

\begin{future}
There are several kinds of integer conversion which can result in a loss of
precision.  Currently, the conversions are performed as specified, and no error
is reported.  In the future, we intend to improve type checking, so the user can
be informed of potential precision loss at compile time, and actual precision
loss at run time.  Such cases include:
%
% An exception is thrown if the source value cannot be represented in the target type.
When an \chpl{int} is converted to a \chpl{uint} and the original value is
negative;
When a \chpl{uint} is converted to an \chpl{int} and the sign bit of the result
is true;
When an \chpl{int} is converted to a smaller \chpl{int} or \chpl{uint} and any
of the truncated bits differs from the original sign bit;
%
When a \chpl{uint} is converted to a smaller \chpl{int} or \chpl{uint} and any
of the truncated bits is true;
\end{future}

\begin{rationale}
For integer conversions, the default behavior of a program should be to produce
a run-time error if there is a loss of precision.  Thus, cast expressions not only
give rise to a value conversion at run time, but amount to an assertion
that the required precision is preserved.  Explicit conversion procedures would be
available in the run-time library so that one can perform explicit conversions
that result in a loss of precision but do not generate a run-time diagnostic.
\end{rationale}

When converting from a \chpl{real} type to a larger \chpl{real} type, the
represented value is preserved.  When converting from a \chpl{real} type to a
smaller \chpl{real} type, the closest representation in the target type is
chosen.\footnote{When converting to a smaller real type, a loss of precision is \emph{expected}.
Therefore, there is no reason to produce a run-time diagnostic.}

When converting to a \chpl{real} type from an integer type, integer types
smaller than \chpl{int} are first converted to \chpl{int}.  Then, the closest
representation of the converted value in the target type is chosen.  The exact
behavior of this conversion is implementation-defined.

When converting from \chpl{real($k$)} to \chpl{complex($2k$)}, the original
value is copied into the real part of the result, and the imaginary part of the
result is set to zero.  When converting from a \chpl{real($k$)} to
a \chpl{complex($\ell$)} such that $\ell > 2k$, the conversion is performed as
if the original value is first converted to \chpl{real($\ell/2$)} and then
to \chpl{$\ell$}.

The rules for converting from \chpl{imag} to \chpl{complex} are the same as for
converting from real, except that the imaginary part of the result is set using
the input value, and the real part of the result is set to zero.

\subsection{Explicit Tuple to Complex Conversion}
\label{Explicit_Tuple_to_Complex_Conversion}
\index{conversions!tuple to complex}
\index{conversions!explicit!tuple to complex}

A two-tuple of numerical values may be converted to a \chpl{complex} value.  If
the destination type is \chpl{complex(128)}, each member of the two-tuple must
be convertible to \chpl{real(64)}.  If the destination type
is \chpl{complex(64)}, each member of the two-tuple must be convertible
to \chpl{real(32)}.  The first member of the tuple becomes the real part of the
resulting complex value; the second member of the tuple becomes the imaginary
part of the resulting complex value.

\subsection{Explicit Enumeration Conversions}
\label{Explicit_Enumeration_Conversions}
\index{conversions!enumeration}
\index{conversions!explicit!enumeration}

Explicit conversions are allowed from any enumerated type to any
\chpl{string} and vice-versa, and include \chpl{param} conversions.
For enumerated types that are either concrete or semi-concrete
(\rsec{Enumerated_Types}), conversions are supported between the
enum's constants and any numeric type or \chpl{bool},
including \chpl{param} conversions.  For a semi-concrete enumerated
type, if a conversion is attempted involving a constant with no
underlying integer value, it will generate a compile-time error for
a \chpl{param} conversion or an execution-time error otherwise.

When the target type is an integer type, the value is first converted to its
underlying integer type and then to the target type, following the rules above
for converting between integers.

When the target type is a real, imaginary, or complex type, the value
is first converted to its underlying integer type and then to the
target type.

When the target type is \chpl{bool}, the value is first converted to its
underlying integer type.  If the result is zero, the value of the \chpl{bool}
is \chpl{false}; otherwise, it is \chpl{true}.

When the target type is \chpl{string}, the value becomes the name of the
enumerator.  % in the execution character set.

When the source type is \chpl{bool}, enumerators corresponding to the values 0
and 1 in the underlying integer type are selected, corresponding to input values
of \chpl{false} and \chpl{true}, respectively.

%REVIEW: hilde
% As with default values for variables of enumerated types, I am pushing for the
% simplest implementation -- in which the conversion does not actually change
% the stored value.  This means that it may be possible for an enumerated variable
% to assume a value that does not correspond to any of its enumerators.  Further
% encouragement to always supply a default clause in your switch statements!

When the source type is a real or integer type, the value is converted to the
target type's underlying integer type.  

The conversion from \chpl{complex} and \chpl{imag} types to an enumerated type is not
permitted.

When the source type is string, the enumerator whose name matches value of the input
string is selected.  If no such enumerator exists, a runtime error occurs.

\subsection{Explicit Class Conversions}
\label{Explicit_Class_Conversions}
\index{conversions!class}
\index{conversions!explicit!class}

An expression of static class type \chpl{C} can be explicitly
converted to a class type \chpl{D} provided that \chpl{C} is derived
from \chpl{D} or \chpl{D} is derived from \chpl{C}.

When at run time the source expression refers to an instance of
\chpl{D} or it subclass, its value is not changed.
Otherwise, or when the source expression is \chpl{nil},
the result of the conversion is \chpl{nil}.

\subsection{Explicit Record Conversions}
\label{Explicit_Record_Conversions}
\index{conversions!records}
\index{conversions!explicit!records}

An expression of record type \chpl{C} can be explicitly converted to
another record type \chpl{D} provided that \chpl{C} is derived
from \chpl{D}.  There are no explicit record conversions that are not
also implicit record conversions.

\subsection{Explicit Range Conversions}
\label{Explicit_Range_Conversions}
\index{conversions!range}
\index{conversions!explicit!range}

An expression of stridable range type can be explicitly converted
to an unstridable range type, changing the stride to 1 in the process.

\subsection{Explicit Domain Conversions}
\label{Explicit_Domain_Conversions}
\index{conversions!domain}
\index{conversions!explicit!domain}

An expression of stridable domain type can be explicitly converted
to an unstridable domain type, changing all strides to 1 in the process.

\subsection{Explicit Type to String Conversions}
\label{Explicit_Type_to_String_Conversions}
\index{conversions!type to string}
\index{conversions!explicit!type to string}

A type expression can be explicitly converted to a \chpl{string}. The resultant
\chpl{string} is the name of the type.

\begin{chapelexample}{explicit-type-to-string.chpl}
For example:
\begin{chapel}
var x: real(64) = 10.0;
writeln(x.type:string);
\end{chapel}
\begin{chapeloutput}
real(64)
\end{chapeloutput}
This program will print out the string \chpl{"real(64)"}.
\end{chapelexample}

\cleardoublepage
\sekshun{Expressions}
\label{Expressions}
\index{expressions}

Chapel provides the following expressions:

\begin{syntax}
expression:
  literal-expression
  nil-expression
  variable-expression
  enum-constant-expression
  call-expression
  type-expression
  iteratable-call-expression
  member-access-expression
  new-expression
  query-expression
  cast-expression
  lvalue-expression
  parenthesized-expression
  unary-expression
  binary-expression
  let-expression
  if-expression
  for-expression
  forall-expression
  reduce-expression
  scan-expression
  module-access-expression
  tuple-expression
  tuple-expand-expression
  locale-access-expression
  mapped-domain-expression
\end{syntax}
% in firstClassFns.rst: lambda-declaration-expression

Individual expressions are defined in the remainder of this chapter
and additionally as follows:

\begin{itemize}
\item forall, reduce, and scan \rsec{Data_Parallelism}
\item module access \rsec{Explicit_Naming}
\item tuple and tuple expand \rsec{Tuples}
\item locale access \rsec{Querying_the_Locale_of_a_Variable}
\item mapped domain \rsec{Domain_Maps}
\item initializer calls \rsec{Class_New}
\item \chpl{nil} \rsec{Class_nil_value}
\end{itemize}

\section{Literal Expressions}
\label{Literal_Expressions}
\index{literal expressions}
\index{expressions!literal}

A literal value for any of the predefined types is a literal expression.

Literal expressions are given by the following syntax:
\begin{syntax}
literal-expression:
  bool-literal
  integer-literal
  real-literal
  imaginary-literal
  string-literal
  range-literal
  domain-literal
  array-literal
\end{syntax}

Literal values for primitive types are described in
\rsec{Primitive_Type_Literals}.
Literal range values are described in \rsec{Range_Literals}.
Literal tuple values are described in \rsec{Tuple_Values}.
Literal values for domains are described in \rsec{Rectangular_Domain_Values}
and \rsec{Associative_Domain_Values}.
Literal values for arrays are described in  \rsec{Rectangular_Array_Literals}
and \rsec{Associative_Array_Literals}.


\section{Variable Expressions}
\label{Variable_Expressions}
\index{expressions!variable}

A use of a variable, constant, parameter, or formal argument, is
itself an expression.  The syntax of a variable expression is given
by:
\begin{syntax}
variable-expression:
  identifier
\end{syntax}

\section{Enumeration Constant Expression}
\label{Enumeration_Constant_Expression}
\index{expressions!enumeration constant}

A use of an enumeration constant is itself an expression.  Such a
constant must be preceded by the enumeration type name.  The syntax of
an enumeration constant expression is given by:
\begin{syntax}
enum-constant-expression:
  enum-type . identifier
\end{syntax}

For an example of using enumeration constants,
see~\rsec{Enumerated_Types}.

\section{Parenthesized Expressions}
\label{Parenthesized_Expressions}
\index{expressions!parenthesized}

A \sntx{parenthesized-expression} is an expression that is delimited
by parentheses as given by:
\begin{syntax}
parenthesized-expression:
  ( expression )
\end{syntax}
Such an expression evaluates to the expression.  The parentheses are
ignored and have only a syntactical effect.

\section{Call Expressions}
\label{Call_Expressions}
\index{function calls}
\index{expressions!call}

Functions and function calls are defined in~\rsec{Functions}.

\section{Indexing Expressions}
\label{Indexing_Expressions}
\index{indexing}
\index{expressions!indexing}

Indexing, for example into arrays, tuples, and domains,
has the same syntax as a call expression.
 
Indexing is performed by an implicit invocation of the \chpl{this} method
on the value being indexed,
passing the indices as the actual arguments.

\section{Member Access Expressions}
\label{Member_Access_Expressions}
\index{member access}
\index{expressions!member access}

Member access expressions provide access to a field or invoke a method
of an instance of a class, record, or union.
They are defined in \rsec{Class_Field_Accesses} and
\rsec{Class_Method_Calls}, respectively.

\begin{syntax}
member-access-expression:
  field-access-expression
  method-call-expression
\end{syntax}

\section{The Query Expression}
\label{The_Query_Expression}
\index{expressions!type query}
\index{? (type query)@\chpl{?} (type query)}
\index{operators!? (type query)@\chpl{?} (type query)}

A query expression is used to query a type or value within a formal
argument type expression.  The syntax of a query expression is given
by:
\begin{syntax}
query-expression:
  ? identifier[OPT]
\end{syntax}
Querying is restricted to querying the type of a formal argument, the
element type of a formal argument that is an array, the domain of a
formal argument that is an array, the size of a primitive type, or a
type or parameter field of a formal argument type.

The identifier can be omitted.  This is useful for ensuring the
genericity of a generic type that defines default values for all of
its generic fields when specifying a formal argument as discussed
in~\rsec{Formal_Arguments_of_Generic_Type}.

\begin{chapelexample}{query.chpl}
The following code defines a generic function where the type of the
first argument is queried and stored in the type alias \chpl{t} and
the domain of the second argument is queried and stored in the
variable \chpl{D}:
\begin{chapelnoprint}
{ // }
\end{chapelnoprint}
\begin{chapel}
proc foo(x: ?t, y: [?D] t) {
  for i in D do
    y[i] = x;
}
\end{chapel}
\begin{chapelnoprint}
// {
var x = 1.5;
var y: [1..4] x.type;
foo(x, y);
writeln(y);
}
\end{chapelnoprint}
This allows a generic specification of assigning a
particular value to all elements of an array.  The value and the
elements of the array are constrained to be the same type.  This
function can be rewritten without query expression as follows:
\begin{chapelnoprint}
{ // }
\end{chapelnoprint}
\begin{chapel}
proc foo(x, y: [] x.type) {
  for i in y.domain do
    y[i] = x;
}
\end{chapel}
\begin{chapelnoprint}
// {
var x = 1.5;
var y: [1..4] x.type;
foo(x, y);
writeln(y);
}
\end{chapelnoprint}
\begin{chapeloutput}
1.5 1.5 1.5 1.5
1.5 1.5 1.5 1.5
\end{chapeloutput}
\end{chapelexample}

There is an expectation that query expressions will be allowed in more
places in the future.

\section{Casts}
\label{Casts}
\index{casts}
\index{expressions!cast}
\index{: (cast)@\chpl{:} (cast)}
\index{operators!: (cast)@\chpl{:} (cast)}

A cast is specified with the following syntax:
\begin{syntax}
cast-expression:
  expression : type-expression
\end{syntax}
The expression is converted to the specified type.  A cast expression invokes
the corresponding explicit conversion~(\rsec{Explicit_Conversions}).  A
resolution error occurs if no such conversion exists.

\section{LValue Expressions}
\label{LValue_Expressions}
\index{lvalues}
\index{expressions!lvalue}

An {\em lvalue} is an expression that can be used on the left-hand
side of an assignment statement or on either side of a swap statement,
that can be passed to a formal argument of a function that
has \chpl{out}, \chpl{inout} or \chpl{ref} intent, or that can be returned by a
function with a \chpl{ref} return intent~(\rsec{Ref_Return_Intent}).  Valid
lvalue expressions include the following:
\begin{itemize}
\item
 Variable expressions.
\item
 Member access expressions.
\item
 Call expressions of functions with a \chpl{ref} return intent.
\item
 Indexing expressions.
\end{itemize}

LValue expressions are given by the following syntax:
\begin{syntax}
lvalue-expression:
  variable-expression
  member-access-expression
  call-expression
  parenthesized-expression
\end{syntax}
The syntax is less restrictive than the definition above.  For
example, not all \sntx{call-expression}s are lvalues.

\section{Precedence and Associativity}
\label{Operator_Precedence_and_Associativity}
\index{operators!precedence}
\index{operators!associativity}
\index{expressions!precedence}
\index{expressions!associativity}

The following table summarizes operator and expression precedence and
associativity.  Operators and expressions listed earlier have higher
precedence than those listed later.
\begin{center}
\begin{tabular}{|l|l|l|}
\hline
{\bf Operator} & {\bf Associativity} & {\bf Use} \\
\hline
\verb@.@ & \multirow{3}{*}{left} & member access \\
\verb@()@ & & function call or access \\
\verb@[]@ & & function call or access \\
\hline
\verb@new@ & right & initializer call \\
\hline
\verb@:@ & left & cast \\
\hline
\verb@**@ & right & exponentiation \\
\hline
\verb@reduce@ & \multirow{3}{*}{left} & reduction \\
\verb@scan@ & & scan \\
\verb@dmapped@ & & domain map application \\
\hline
\verb@!@ & \multirow{2}{*}{right} & logical negation \\
\verb@~@ & & bitwise negation \\
\hline
\verb@*@ & \multirow{3}{*}{left} & multiplication \\
\verb@/@ & & division \\
\verb@%@ & & modulus \\
\hline
unary \verb@+@ & \multirow{2}{*}{right} & positive identity \\
unary \verb@-@ & & negation \\
\hline
\verb@<<@ & \multirow{2}{*}{left} & left shift \\
\verb@>>@ & & right shift \\
\hline
\verb@&@ & left & bitwise/logical and \\
\hline
\verb@^@ & left & bitwise/logical xor \\
\hline
\verb@|@ & left & bitwise/logical or \\
\hline
\verb@+@ & \multirow{2}{*}{left} & addition \\
\verb@-@ & & subtraction \\
\hline
\verb@..@ & left & range initialization \\
\hline
\verb@<=@ & \multirow{4}{*}{left} & less-than-or-equal-to comparison \\
\verb@>=@ & & greater-than-or-equal-to comparison \\
\verb@<@ & & less-than comparison \\
\verb@>@ & & greater-than comparison \\
\hline
\verb@==@ & \multirow{2}{*}{left} & equal-to comparison \\
\verb@!=@ & & not-equal-to comparison \\
\hline
\verb@&&@ & left & short-circuiting logical and \\
\hline
\verb@||@ & left & short-circuiting logical or \\
\hline
\verb@in@ & left & forall expression \\
\hline
\verb@by@ & \multirow{3}{*}{left} & range/domain stride application \\
\verb@#@ & & range count application \\
\verb@align@ & & range alignment \\
\hline
\verb@if then else@ & \multirow{5}{*}{left} & conditional expression \\
\verb@forall do@ & & forall expression \\
\verb@[ ]@ & & forall expression \\
\verb@for do@ & & for expression \\
\verb@sync single atomic@ & & sync, single and atomic type \\
\hline
\verb@,@ & left & comma separated expressions \\
\hline
\end{tabular}
\end{center}

\begin{rationale}
In general, our operator precedence is based on that of the C family
of languages including C++, Java, Perl, and C\#.  We comment on a few
of the differences and unique factors here.

We find that there is tension between the relative precedence of
exponentiation, unary minus/plus, and casts.  The following three
expressions show our intuition for how these expressions should be
parenthesized.

\begin{center}
\begin{tabular}{lcl}
\chpl{-2**4} & wants & \chpl{-(2**4)} \\
\chpl{-2:uint} & wants & \chpl{(-2):uint} \\
\chpl{2:uint**4:uint} & wants & \chpl{(2:uint)**(4:uint)} \\
\end{tabular}
\end{center}

Trying to support all three of these cases results in a
circularity---exponentiation wants precedence over unary minus, unary
minus wants precedence over casts, and casts want precedence over
exponentiation.  We chose to break the circularity by making unary
minus have a lower precedence.  This means that for the second case
above:

\begin{center}
\begin{tabular}{lcl}
\chpl{-2:uint} & requires & \chpl{(-2):uint} \\
\end{tabular}
\end{center}

We also chose to depart from the C family of languages by making unary
plus/minus have lower precedence than binary multiplication, division,
and modulus as in Fortran.  We have found very few cases that
distinguish between these cases.  An interesting one is:

\begin{center}
\begin{tabular}{l}
\chpl{const minint = min(int(32));}\\
\chpl{...-minint/2...}
\end{tabular}
\end{center}

Intuitively, this should result in a positive value, yet C's
precedence rules results in a negative value due to asymmetry in
modern integer representations.  If we learn of cases that argue in
favor of the C approach, we would likely reverse this decision in
order to more closely match C.

We were tempted to diverge from the C precedence rules for the binary
bitwise operators to make them bind less tightly than comparisons.
This would allow us to interpret:

\begin{center}
\begin{tabular}{lcl}
\chpl{a | b == 0} & as & \chpl{(a | b) == 0} \\
\end{tabular}
\end{center}

However, given that no other popular modern language has made this
change, we felt it unwise to stray from the pack.  The typical
rationale for the C ordering is to allow these operators to be used as
non-short-circuiting logical operations.

In contrast to C, we give bitwise operations a higher precedence than binary
addition/subtraction and comparison operators.  This enables using the shift
operators as shorthand for multiplication/division by powers of 2, and also
makes it easier to extract and test a bitmapped field:

\begin{center}
\begin{tabular}{lcl}
\chpl{(x \& MASK) == MASK} & as & \chpl{x \& MASK == MASK} \\
\chpl{a + b * pow(2,y)} & as & \chpl{a * b << y} \\
\end{tabular}
\end{center}

One final area of note is the precedence of reductions.  Two common
cases tend to argue for making reductions very low or very high in the
precedence table:

\begin{center}
\begin{tabular}{lcl}
\chpl{max reduce A - min reduce A} & wants & \chpl{(max reduce A) - (min reduce A)} \\
\chpl{max reduce A * B} & wants & \chpl{max reduce (A * B)} \\
\end{tabular}
\end{center}

The first statement would require reductions to have a higher
precedence than the arithmetic operators while the second would
require them to be lower.  We opted to make reductions have high
precedence due to the argument that they tend to resemble unary
operators.  Thus, to support our intuition:

\begin{center}
\begin{tabular}{lcl}
\chpl{max reduce A * B} & requires & \chpl{max reduce (A * B)} \\
\end{tabular}
\end{center}

This choice also has the (arguably positive) effect of making the
unparenthesized version of this statement result in an aggregate value
if A and B are both aggregates---the reduction of A results in a
scalar which promotes when being multiplied by B, resulting in an
aggregate.  Our intuition is that users who forget the parentheses
will learn of their error at compilation time because the resulting
expression is not a scalar as expected.

\end{rationale}

\section{Operator Expressions}
\label{Binary_Expressions}
\label{Unary_Expressions}
\index{expressions!operator}

\index{operators!unary}
\index{expressions!unary operator}
The application of operators to expressions is itself an expression.
The syntax of a unary expression is given by:
\begin{syntax}
unary-expression:
  unary-operator expression

unary-operator: one of
  + $ $ $ $ - $ $ $ $ ~ $ $ $ $ !
\end{syntax}

\index{operators!binary}
\index{expressions!binary operator}
The syntax of a binary expression is given by:
\begin{syntax}
binary-expression:
  expression binary-operator expression

binary-operator: one of
  + $ $ $ $ - $ $ $ $ * $ $ $ $ / $ $ $ $ % $ $ $ $ ** $ $ $ $ & $ $ $ $ | $ $ $ $ ^ $ $ $ $ << $ $ $ $ >> $ $ $ $ && $ $ $ $ || $ $ $ $ == $ $ $ $ != $ $ $ $ <= $ $ $ $ >= $ $ $ $ < $ $ $ $ > $ $ $ $ `by' $ $ $ $ #
\end{syntax}

The operators are defined in subsequent sections.

\section{Arithmetic Operators}
\label{Arithmetic_Operators}
\index{operators!arithmetic}

This section describes the predefined arithmetic operators.  These
operators can be redefined over different types using operator
overloading~(\rsec{Function_Overloading}).

For each operator, implicit conversions are applied to the operands of
an operator such that they are compatible with one of the function
forms listed, those listed earlier in the list being given
preference.  If no compatible implicit conversions exist, then a
compile-time error occurs.  In these cases, an explicit cast is required.

\pagebreak
\subsection{Unary Plus Operators}
\label{Unary_Plus_Operators}
\index{+ (unary)@\chpl{+} (unary)}
\index{operators!+ (unary)@\chpl{+} (unary)}

The unary plus operators are predefined as follows:
\begin{chapel}
proc +(a: int(8)): int(8)
proc +(a: int(16)): int(16)
proc +(a: int(32)): int(32)
proc +(a: int(64)): int(64)

proc +(a: uint(8)): uint(8)
proc +(a: uint(16)): uint(16)
proc +(a: uint(32)): uint(32)
proc +(a: uint(64)): uint(64)

proc +(a: real(32)): real(32)
proc +(a: real(64)): real(64)

proc +(a: imag(32)): imag(32)
proc +(a: imag(64)): imag(64)

proc +(a: complex(64)): complex(64)
proc +(a: complex(128)): complex(128)
\end{chapel}
For each of these definitions, the result is the value of the operand.

\subsection{Unary Minus Operators}
\label{Unary_Minus_Operators}
\index{operators!negation}
\index{- (unary)@\chpl{-} (unary)}
\index{operators!- (unary)@\chpl{-} (unary)}

The unary minus operators are predefined as follows:
\begin{chapel}
proc -(a: int(8)): int(8)
proc -(a: int(16)): int(16)
proc -(a: int(32)): int(32)
proc -(a: int(64)): int(64)

proc -(a: real(32)): real(32)
proc -(a: real(64)): real(64)

proc -(a: imag(32)): imag(32)
proc -(a: imag(64)): imag(64)

proc -(a: complex(64)): complex(64)
proc -(a: complex(128)): complex(128)
\end{chapel}
For each of these definitions that return a value, the result is the
negation of the value of the operand.  For integral types, this
corresponds to subtracting the value from zero.  For real and
imaginary types, this corresponds to inverting the sign.  For complex
types, this corresponds to inverting the signs of both the real and
imaginary parts.

It is an error to try to negate a value of type \chpl{uint(64)}.  Note
that negating a value of type \chpl{uint(32)} first converts the type
to \chpl{int(64)} using an implicit conversion.

\pagebreak
\subsection{Addition Operators}
\label{Addition_Operators}
\index{operators!addition}
\index{+@\chpl{+}}
\index{operators!+@\chpl{+}}

The addition operators are predefined as follows:
\begin{chapel}
proc +(a: int(8), b: int(8)): int(8)
proc +(a: int(16), b: int(16)): int(16)
proc +(a: int(32), b: int(32)): int(32)
proc +(a: int(64), b: int(64)): int(64)

proc +(a: uint(8), b: uint(8)): uint(8)
proc +(a: uint(16), b: uint(16)): uint(16)
proc +(a: uint(32), b: uint(32)): uint(32)
proc +(a: uint(64), b: uint(64)): uint(64)

proc +(a: real(32), b: real(32)): real(32)
proc +(a: real(64), b: real(64)): real(64)

proc +(a: imag(32), b: imag(32)): imag(32)
proc +(a: imag(64), b: imag(64)): imag(64)

proc +(a: complex(64), b: complex(64)): complex(64)
proc +(a: complex(128), b: complex(128)): complex(128)

proc +(a: real(32), b: imag(32)): complex(64)
proc +(a: imag(32), b: real(32)): complex(64)
proc +(a: real(64), b: imag(64)): complex(128)
proc +(a: imag(64), b: real(64)): complex(128)

proc +(a: real(32), b: complex(64)): complex(64)
proc +(a: complex(64), b: real(32)): complex(64)
proc +(a: real(64), b: complex(128)): complex(128)
proc +(a: complex(128), b: real(64)): complex(128)

proc +(a: imag(32), b: complex(64)): complex(64)
proc +(a: complex(64), b: imag(32)): complex(64)
proc +(a: imag(64), b: complex(128)): complex(128)
proc +(a: complex(128), b: imag(64)): complex(128)
\end{chapel}
For each of these definitions that return a value, the result is the
sum of the two operands.

It is a compile-time error to add a value of type \chpl{uint(64)} and
a value of type \chpl{int(64)}.

Addition over a value of real type and a value of imaginary type
produces a value of complex type.  Addition of values of complex type
and either real or imaginary types also produces a value of complex
type.

\pagebreak
\subsection{Subtraction Operators}
\label{Subtraction_Operators}
\index{operators!subtraction}
\index{-@\chpl{-}}
\index{operators!-@\chpl{-}}

The subtraction operators are predefined as follows:
\begin{chapel}
proc -(a: int(8), b: int(8)): int(8)
proc -(a: int(16), b: int(16)): int(16)
proc -(a: int(32), b: int(32)): int(32)
proc -(a: int(64), b: int(64)): int(64)

proc -(a: uint(8), b: uint(8)): uint(8)
proc -(a: uint(16), b: uint(16)): uint(16)
proc -(a: uint(32), b: uint(32)): uint(32)
proc -(a: uint(64), b: uint(64)): uint(64)

proc -(a: real(32), b: real(32)): real(32)
proc -(a: real(64), b: real(64)): real(64)

proc -(a: imag(32), b: imag(32)): imag(32)
proc -(a: imag(64), b: imag(64)): imag(64)

proc -(a: complex(64), b: complex(64)): complex(64)
proc -(a: complex(128), b: complex(128)): complex(128)

proc -(a: real(32), b: imag(32)): complex(64)
proc -(a: imag(32), b: real(32)): complex(64)
proc -(a: real(64), b: imag(64)): complex(128)
proc -(a: imag(64), b: real(64)): complex(128)

proc -(a: real(32), b: complex(64)): complex(64)
proc -(a: complex(64), b: real(32)): complex(64)
proc -(a: real(64), b: complex(128)): complex(128)
proc -(a: complex(128), b: real(64)): complex(128)

proc -(a: imag(32), b: complex(64)): complex(64)
proc -(a: complex(64), b: imag(32)): complex(64)
proc -(a: imag(64), b: complex(128)): complex(128)
proc -(a: complex(128), b: imag(64)): complex(128)
\end{chapel}
For each of these definitions that return a value, the result is the
value obtained by subtracting the second operand from the first
operand.

It is a compile-time error to subtract a value of type \chpl{uint(64)}
from a value of type \chpl{int(64)}, and vice versa.

Subtraction of a value of real type from a value of imaginary type,
and vice versa, produces a value of complex type.  Subtraction of
values of complex type from either real or imaginary types, and vice
versa, also produces a value of complex type.

\pagebreak
\subsection{Multiplication Operators}
\label{Multiplication_Operators}
\index{operators!multiplication}
\index{operators!*@\chpl{*}}
\index{*@\chpl{*}}

The multiplication operators are predefined as follows:
\begin{chapel}
proc *(a: int(8), b: int(8)): int(8)
proc *(a: int(16), b: int(16)): int(16)
proc *(a: int(32), b: int(32)): int(32)
proc *(a: int(64), b: int(64)): int(64)

proc *(a: uint(8), b: uint(8)): uint(8)
proc *(a: uint(16), b: uint(16)): uint(16)
proc *(a: uint(32), b: uint(32)): uint(32)
proc *(a: uint(64), b: uint(64)): uint(64)

proc *(a: real(32), b: real(32)): real(32)
proc *(a: real(64), b: real(64)): real(64)

proc *(a: imag(32), b: imag(32)): real(32)
proc *(a: imag(64), b: imag(64)): real(64)

proc *(a: complex(64), b: complex(64)): complex(64)
proc *(a: complex(128), b: complex(128)): complex(128)

proc *(a: real(32), b: imag(32)): imag(32)
proc *(a: imag(32), b: real(32)): imag(32)
proc *(a: real(64), b: imag(64)): imag(64)
proc *(a: imag(64), b: real(64)): imag(64)

proc *(a: real(32), b: complex(64)): complex(64)
proc *(a: complex(64), b: real(32)): complex(64)
proc *(a: real(64), b: complex(128)): complex(128)
proc *(a: complex(128), b: real(64)): complex(128)

proc *(a: imag(32), b: complex(64)): complex(64)
proc *(a: complex(64), b: imag(32)): complex(64)
proc *(a: imag(64), b: complex(128)): complex(128)
proc *(a: complex(128), b: imag(64)): complex(128)
\end{chapel}
For each of these definitions that return a value, the result is the
product of the two operands.

It is a compile-time error to multiply a value of type \chpl{uint(64)} and
a value of type \chpl{int(64)}.

Multiplication of values of imaginary type produces a value of real
type.  Multiplication over a value of real type and a value of
imaginary type produces a value of imaginary type.  Multiplication of
values of complex type and either real or imaginary types produces a
value of complex type.

\pagebreak
\subsection{Division Operators}
\label{Division_Operators}
\index{operators!division}
\index{/@\chpl{/}}
\index{operators!/@\chpl{/}}

The division operators are predefined as follows:
\begin{chapel}
proc /(a: int(8), b: int(8)): int(8)
proc /(a: int(16), b: int(16)): int(16)
proc /(a: int(32), b: int(32)): int(32)
proc /(a: int(64), b: int(64)): int(64)

proc /(a: uint(8), b: uint(8)): uint(8)
proc /(a: uint(16), b: uint(16)): uint(16)
proc /(a: uint(32), b: uint(32)): uint(32)
proc /(a: uint(64), b: uint(64)): uint(64)

proc /(a: real(32), b: real(32)): real(32)
proc /(a: real(64), b: real(64)): real(64)

proc /(a: imag(32), b: imag(32)): real(32)
proc /(a: imag(64), b: imag(64)): real(64)

proc /(a: complex(64), b: complex(64)): complex(64)
proc /(a: complex(128), b: complex(128)): complex(128)

proc /(a: real(32), b: imag(32)): imag(32)
proc /(a: imag(32), b: real(32)): imag(32)
proc /(a: real(64), b: imag(64)): imag(64)
proc /(a: imag(64), b: real(64)): imag(64)

proc /(a: real(32), b: complex(64)): complex(64)
proc /(a: complex(64), b: real(32)): complex(64)
proc /(a: real(64), b: complex(128)): complex(128)
proc /(a: complex(128), b: real(64)): complex(128)

proc /(a: imag(32), b: complex(64)): complex(64)
proc /(a: complex(64), b: imag(32)): complex(64)
proc /(a: imag(64), b: complex(128)): complex(128)
proc /(a: complex(128), b: imag(64)): complex(128)
\end{chapel}
For each of these definitions that return a value, the result is the
quotient of the two operands.

It is a compile-time error to divide a value of type \chpl{uint(64)} by
a value of type \chpl{int(64)}, and vice versa.

Division of values of imaginary type produces a value of real type.
Division over a value of real type and a value of imaginary type
produces a value of imaginary type.  Division of values of complex
type and either real or imaginary types produces a value of complex
type.

When the operands are integers, the result (quotient) is also an integer.  If \chpl{b}
does not divide \chpl{a} exactly, then there are two candidate quotients $q1$ and $q2$
such that $b * q1$ and $b * q2$ are the two multiples of \chpl{b} closest to \chpl{a}.
The integer result $q$ is the candidate quotient which lies closest to zero.

\pagebreak
\subsection{Modulus Operators}
\label{Modulus_Operators}
\index{operators!modulus}
\index{\%@\chpl{\%}}
\index{operators!\%@\chpl{\%}}

The modulus operators are predefined as follows:
\begin{chapel}
proc %(a: int(8), b: int(8)): int(8)
proc %(a: int(16), b: int(16)): int(16)
proc %(a: int(32), b: int(32)): int(32)
proc %(a: int(64), b: int(64)): int(64)

proc %(a: uint(8), b: uint(8)): uint(8)
proc %(a: uint(16), b: uint(16)): uint(16)
proc %(a: uint(32), b: uint(32)): uint(32)
proc %(a: uint(64), b: uint(64)): uint(64)
\end{chapel}
For each of these definitions that return a value, the result is the
remainder when the first operand is divided by the second operand.

The sign of the result is the same as the sign of the dividend \chpl{a}, and the
magnitude of the result is always smaller than that of the divisor \chpl{b}.
For integer operands, the \chpl{\%} and \chpl{/} operators are related by the
following identity:
\begin{chapel}
var q = a / b;
var r = a % b;
writeln(q * b + r == a);    // true
\end{chapel}

It is a compile-time error to take the remainder of a value of
type \chpl{uint(64)} and a value of type \chpl{int(64)}, and vice
versa.

There is an expectation that the predefined modulus operators will be
extended to handle real, imaginary, and complex types in the future.

\subsection{Exponentiation Operators}
\label{Exponentiation_Operators}
\index{operators!exponentiation}
\index{**@\chpl{**}}
\index{operators!**@\chpl{**}}

The exponentiation operators are predefined as follows:
\begin{chapel}
proc **(a: int(8), b: int(8)): int(8)
proc **(a: int(16), b: int(16)): int(16)
proc **(a: int(32), b: int(32)): int(32)
proc **(a: int(64), b: int(64)): int(64)

proc **(a: uint(8), b: uint(8)): uint(8)
proc **(a: uint(16), b: uint(16)): uint(16)
proc **(a: uint(32), b: uint(32)): uint(32)
proc **(a: uint(64), b: uint(64)): uint(64)

proc **(a: real(32), b: real(32)): real(32)
proc **(a: real(64), b: real(64)): real(64)
\end{chapel}
For each of these definitions that return a value, the result is the
value of the first operand raised to the power of the second operand.

It is a compile-time error to take the exponent of a value of
type \chpl{uint(64)} by a value of type \chpl{int(64)}, and vice
versa.

There is an expectation that the predefined exponentiation operators
will be extended to handle imaginary and complex types in the future.

\section{Bitwise Operators}
\label{Bitwise_Operators}
\index{operators!bitwise}

This section describes the predefined bitwise operators.  These
operators can be redefined over different types using operator
overloading~(\rsec{Function_Overloading}).

\subsection{Bitwise Complement Operators}
\label{Bitwise_Complement_Operators}
\index{operators!bitwise!complement}
\index{\~@\chpl{\~}}
\index{operators!\~@\chpl{\~}}

The bitwise complement operators are predefined as follows:
\begin{chapel}
proc ~(a: int(8)): int(8)
proc ~(a: int(16)): int(16)
proc ~(a: int(32)): int(32)
proc ~(a: int(64)): int(64)

proc ~(a: uint(8)): uint(8)
proc ~(a: uint(16)): uint(16)
proc ~(a: uint(32)): uint(32)
proc ~(a: uint(64)): uint(64)
\end{chapel}
For each of these definitions, the result is the bitwise complement of
the operand.

\subsection{Bitwise And Operators}
\label{Bitwise_And_Operators}
\index{operators!bitwise!and}
\index{&@\chpl{&}}
\index{operators!&@\chpl{&}}

The bitwise and operators are predefined as follows:
\begin{chapel}
proc &(a: bool, b: bool): bool

proc &(a: int(?w), b: int(w)): int(w)
proc &(a: uint(?w), b: uint(w)): uint(w)

proc &(a: int(?w), b: uint(w)): uint(w)
proc &(a: uint(?w), b: int(w)): uint(w)
\end{chapel}
For each of these definitions, the result is
computed by applying the logical and operation to the bits of the
operands.

Chapel allows mixing signed and unsigned integers of the same size
when passing them as arguments to bitwise and.
In the mixed case the result is of the same size as the arguments
and is unsigned.
No run-time error is issued, even if the apparent sign changes as the
required conversions are performed.

\begin{rationale}
The mathematical meaning of integer arguments
is discarded when they are passed to bitwise operators.
Instead the arguments are treated simply as bit vectors.
The bit-vector meaning is preserved when converting
between signed and unsigned of the same size.
The choice of unsigned over signed as the result type in the mixed case
reflects the semantics of standard C.
\end{rationale}

\subsection{Bitwise Or Operators}
\label{Bitwise_Or_Operators}
\index{operators!bitwise!or}
\index{|@\chpl{|}}
\index{operators!|@\chpl{|}}

The bitwise or operators are predefined as follows:
\begin{chapel}
proc |(a: bool, b: bool): bool

proc |(a: int(?w), b: int(w)): int(w)
proc |(a: uint(?w), b: uint(w)): uint(w)

proc |(a: int(?w), b: uint(w)): uint(w)
proc |(a: uint(?w), b: int(w)): uint(w)
\end{chapel}

For each of these definitions, the result is
computed by applying the logical or operation to the bits of the
operands.
Chapel allows mixing signed and unsigned integers of the same size
when passing them as arguments to bitwise or.
No run-time error is issued, even if the apparent sign changes as the
required conversions are performed.

\begin{rationale}
The same as for bitwise and (\rsec{Bitwise_And_Operators}).
\end{rationale}

\subsection{Bitwise Xor Operators}
\label{Bitwise_Xor_Operators}
\index{operators!bitwise!exclusive or}
\index{^@\chpl{^}}
\index{operators!^@\chpl{^}}

The bitwise xor operators are predefined as follows:
\begin{chapel}
proc ^(a: bool, b: bool): bool

proc ^(a: int(?w), b: int(w)): int(w)
proc ^(a: uint(?w), b: uint(w)): uint(w)

proc ^(a: int(?w), b: uint(w)): uint(w)
proc ^(a: uint(?w), b: int(w)): uint(w)
\end{chapel}

For each of these definitions, the result is
computed by applying the XOR operation to the bits of the operands.
Chapel allows mixing signed and unsigned integers of the same size
when passing them as arguments to bitwise xor.
No run-time error is issued, even if the apparent sign changes as the required
conversions are performed.

\begin{rationale}
The same as for bitwise and (\rsec{Bitwise_And_Operators}).
\end{rationale}

\pagebreak
\section{Shift Operators}
\label{Shift_Operators}
\index{operators!shift}
\index{<<@\chpl{<<}}
\index{operators!<<@\chpl{<<}}
\index{>>@\chpl{>>}}
\index{operators!>>@\chpl{>>}}

This section describes the predefined shift operators.  These
operators can be redefined over different types using operator
overloading~(\rsec{Function_Overloading}).

The shift operators are predefined as follows:
\begin{chapel}
proc <<(a: int(8), b): int(8)
proc <<(a: int(16), b): int(16)
proc <<(a: int(32), b): int(32)
proc <<(a: int(64), b): int(64)

proc <<(a: uint(8), b): uint(8)
proc <<(a: uint(16), b): uint(16)
proc <<(a: uint(32), b): uint(32)
proc <<(a: uint(64), b): uint(64)

proc >>(a: int(8), b): int(8)
proc >>(a: int(16), b): int(16)
proc >>(a: int(32), b): int(32)
proc >>(a: int(64), b): int(64)

proc >>(a: uint(8), b): uint(8)
proc >>(a: uint(16), b): uint(16)
proc >>(a: uint(32), b): uint(32)
proc >>(a: uint(64), b): uint(64)
\end{chapel}
The type of the second actual argument must be any integral type.

The \chpl{<<} operator shifts the bits of \chpl{a} left by the
integer \chpl{b}.  The new low-order bits are set to zero.

The \chpl{>>} operator shifts the bits of \chpl{a} right by the
integer \chpl{b}.  When \chpl{a} is negative, the new high-order bits
are set to one; otherwise the new high-order bits are set to zero.

The value of \chpl{b} must be non-negative.

\section{Logical Operators}
\label{Logical_Operators}
\index{operators!logical}

This section describes the predefined logical operators.  These
operators can be redefined over different types using operator
overloading~(\rsec{Function_Overloading}).

\subsection{The Logical Negation Operator}
\label{Logical_Negation_Operators}
\index{operators!logical!not}
\index{\!@\chpl{!}}
\index{operators!\!@\chpl{!}}

The logical negation operator is predefined for booleans and integers
as follows:

\begin{chapel}
proc !(a: bool): bool
proc !(a: int(?w)): bool
proc !(a: uint(?w)): bool
\end{chapel}
For the boolean form, the result is the logical negation of the
operand.  For the integer forms, the result is true if the operand is
zero and false otherwise.

\subsection{The Logical And Operator}
\label{Logical_And_Operators}
\index{operators!logical!and}
\index{&&@\chpl{&&}}
\index{operators!&&@\chpl{&&}}

The logical and operator is predefined over bool type.  It returns
true if both operands evaluate to true; otherwise it returns false.
If the first operand evaluates to false, the second operand is not
evaluated and the result is false.
%% hilde sez: In the interest of supporting parallel execution, we should leave
%% unspecified whether the right operand is evaluated.
%% Where sufficient processing resources are available, it is faster on average
%% to evaluate both the left and right operands and perform the conjunction or
%% disjunction than to block until the value of the left operand is known and
%% only then commence to evaluate the right operand.

The logical and operator over expressions \chpl{a} and \chpl{b} given
by
\begin{chapel}
a && b
\end{chapel}
is evaluated as the expression
\begin{chapel}
if isTrue(a) then isTrue(b) else false
\end{chapel}

The function \chpl{isTrue} is predefined over bool type as follows:
\begin{chapel}
proc isTrue(a:bool) return a;
\end{chapel}
Overloading the logical and operator over other types is accomplished
by overloading the \chpl{isTrue} function over other types.

\subsection{The Logical Or Operator}
\label{Logical_Or_Operators}
\index{operators!logical!or}
\index{||@\chpl{||}}
\index{operators!||@\chpl{||}}


The logical or operator is predefined over bool type.  It returns
true if either operand evaluate to true; otherwise it returns false.
If the first operand evaluates to true, the second operand is not
evaluated and the result is true.

The logical or operator over expressions \chpl{a} and \chpl{b} given
by
\begin{chapel}
a || b
\end{chapel}
is evaluated as the expression
\begin{chapel}
if isTrue(a) then true else isTrue(b)
\end{chapel}

The function \chpl{isTrue} is predefined over bool type as described
in~\rsec{Logical_And_Operators}.  Overloading the logical or operator
over other types is accomplished by overloading the \chpl{isTrue}
function over other types.

\pagebreak
\section{Relational Operators}
\label{Relational_Operators}
\index{operators!relational}

This section describes the predefined relational operators.  These
operators can be redefined over different types using operator
overloading~(\rsec{Function_Overloading}).

\subsection{Ordered Comparison Operators}
\label{Ordered_Comparison_Operators}

\index{operators!less than}
\index{<@\chpl{<}}
\index{operators!<@\chpl{<}}
The ``less than'' comparison operators are predefined over numeric
types as follows:
\begin{chapel}
proc <(a: int(8), b: int(8)): bool
proc <(a: int(16), b: int(16)): bool
proc <(a: int(32), b: int(32)): bool
proc <(a: int(64), b: int(64)): bool

proc <(a: uint(8), b: uint(8)): bool
proc <(a: uint(16), b: uint(16)): bool
proc <(a: uint(32), b: uint(32)): bool
proc <(a: uint(64), b: uint(64)): bool

proc <(a: int(64), b: uint(64)): bool
proc <(a: uint(64), b: int(64)): bool

proc <(a: real(32), b: real(32)): bool
proc <(a: real(64), b: real(64)): bool

proc <(a: imag(32), b: imag(32)): bool
proc <(a: imag(64), b: imag(64)): bool
\end{chapel}
The result of \chpl{a < b} is true if \chpl{a} is less than \chpl{b};
otherwise the result is false.

\index{operators!greater than}
\index{>@\chpl{>}}
\index{operators!>@\chpl{>}}
The ``greater than'' comparison operators are predefined over numeric
types as follows:
\begin{chapel}
proc >(a: int(8), b: int(8)): bool
proc >(a: int(16), b: int(16)): bool
proc >(a: int(32), b: int(32)): bool
proc >(a: int(64), b: int(64)): bool

proc >(a: uint(8), b: uint(8)): bool
proc >(a: uint(16), b: uint(16)): bool
proc >(a: uint(32), b: uint(32)): bool
proc >(a: uint(64), b: uint(64)): bool

proc >(a: int(64), b: uint(64)): bool
proc >(a: uint(64), b: int(64)): bool

proc >(a: real(32), b: real(32)): bool
proc >(a: real(64), b: real(64)): bool

proc >(a: imag(32), b: imag(32)): bool
proc >(a: imag(64), b: imag(64)): bool
\end{chapel}
The result of \chpl{a > b} is true if \chpl{a} is greater
than \chpl{b}; otherwise the result is false.

\index{operators!less than or equal}
\index{<=@\chpl{<=}}
\index{operators!<=@\chpl{<=}}
The ``less than or equal to'' comparison operators are predefined over
numeric types as follows:
\begin{chapel}
proc <=(a: int(8), b: int(8)): bool
proc <=(a: int(16), b: int(16)): bool
proc <=(a: int(32), b: int(32)): bool
proc <=(a: int(64), b: int(64)): bool

proc <=(a: uint(8), b: uint(8)): bool
proc <=(a: uint(16), b: uint(16)): bool
proc <=(a: uint(32), b: uint(32)): bool
proc <=(a: uint(64), b: uint(64)): bool

proc <=(a: int(64), b: uint(64)): bool
proc <=(a: uint(64), b: int(64)): bool

proc <=(a: real(32), b: real(32)): bool
proc <=(a: real(64), b: real(64)): bool

proc <=(a: imag(32), b: imag(32)): bool
proc <=(a: imag(64), b: imag(64)): bool
\end{chapel}
The result of \chpl{a <= b} is true if \chpl{a} is less than or equal
to \chpl{b}; otherwise the result is false.

\index{operators!greater than or equal}
\index{>=@\chpl{>=}}
\index{operators!>=@\chpl{>=}}
The ``greater than or equal to'' comparison operators are predefined
over numeric types as follows:
\begin{chapel}
proc >=(a: int(8), b: int(8)): bool
proc >=(a: int(16), b: int(16)): bool
proc >=(a: int(32), b: int(32)): bool
proc >=(a: int(64), b: int(64)): bool

proc >=(a: uint(8), b: uint(8)): bool
proc >=(a: uint(16), b: uint(16)): bool
proc >=(a: uint(32), b: uint(32)): bool
proc >=(a: uint(64), b: uint(64)): bool

proc >=(a: int(64), b: uint(64)): bool
proc >=(a: uint(64), b: int(64)): bool

proc >=(a: real(32), b: real(32)): bool
proc >=(a: real(64), b: real(64)): bool

proc >=(a: imag(32), b: imag(32)): bool
proc >=(a: imag(64), b: imag(64)): bool
\end{chapel}
The result of \chpl{a >= b} is true if \chpl{a} is greater than or
equal to \chpl{b}; otherwise the result is false.

The ordered comparison operators are predefined over strings as follows:
\begin{chapel}
proc <(a: string, b: string): bool
proc >(a: string, b: string): bool
proc <=(a: string, b: string): bool
proc >=(a: string, b: string): bool
\end{chapel}
Comparisons between strings are defined based on the ordering of the
character set used to represent the string, which is applied
elementwise to the string's characters in order.


\subsection{Equality Comparison Operators}
\label{Equality_Comparison_Operators}
\index{operators!equality}
\index{==@\chpl{==}}
\index{operators!==@\chpl{==}}
\index{"!=@\chpl{"\"!=}}
\index{operators!"!=@\chpl{"\"!=}}

The equality comparison operators \chpl{==} and \chpl{\!=} are predefined over bool and the
numeric types as follows:
\begin{chapel}
proc ==(a: int(8), b: int(8)): bool
proc ==(a: int(16), b: int(16)): bool
proc ==(a: int(32), b: int(32)): bool
proc ==(a: int(64), b: int(64)): bool

proc ==(a: uint(8), b: uint(8)): bool
proc ==(a: uint(16), b: uint(16)): bool
proc ==(a: uint(32), b: uint(32)): bool
proc ==(a: uint(64), b: uint(64)): bool

proc ==(a: int(64), b: uint(64)): bool
proc ==(a: uint(64), b: int(64)): bool

proc ==(a: real(32), b: real(32)): bool
proc ==(a: real(64), b: real(64)): bool

proc ==(a: imag(32), b: imag(32)): bool
proc ==(a: imag(64), b: imag(64)): bool

proc ==(a: complex(64), b: complex(64)): bool
proc ==(a: complex(128), b: complex(128)): bool

proc !=(a: int(8), b: int(8)): bool
proc !=(a: int(16), b: int(16)): bool
proc !=(a: int(32), b: int(32)): bool
proc !=(a: int(64), b: int(64)): bool

proc !=(a: uint(8), b: uint(8)): bool
proc !=(a: uint(16), b: uint(16)): bool
proc !=(a: uint(32), b: uint(32)): bool
proc !=(a: uint(64), b: uint(64)): bool

proc !=(a: int(64), b: uint(64)): bool
proc !=(a: uint(64), b: int(64)): bool

proc !=(a: real(32), b: real(32)): bool
proc !=(a: real(64), b: real(64)): bool

proc !=(a: imag(32), b: imag(32)): bool
proc !=(a: imag(64), b: imag(64)): bool

proc !=(a: complex(64), b: complex(64)): bool
proc !=(a: complex(128), b: complex(128)): bool
\end{chapel}
The result of \chpl{a == b} is true if \chpl{a} and \chpl{b} contain
the same value; otherwise the result is false.  The result of \chpl{a
\!= b} is equivalent to \chpl{\!(a == b)}.

The equality comparison operators are predefined over classes as
follows:
\begin{chapel}
proc ==(a: object, b: object): bool
proc !=(a: object, b: object): bool
\end{chapel}
The result of \chpl{a == b} is true if \chpl{a} and \chpl{b} reference
the same storage location; otherwise the result is false.  The result
of \chpl{a \!= b} is equivalent to \chpl{\!(a == b)}.

Default equality comparison operators are generated for records if the
user does not define them.  These operators are described
in~\rsec{Record_Comparison_Operators}.

\index{== (string)@\chpl{==} (string)}
\index{operators!== (string)@\chpl{==} (string)}
\index{"!= (string)@\chpl{"\"!=} (string)}
\index{operators!"!= (string)@\chpl{"\"!=} (string)}
The equality comparison operators are predefined over strings as
follows:
\begin{chapel}
proc ==(a: string, b: string): bool
proc !=(a: string, b: string): bool
\end{chapel}
The result of \chpl{a == b} is true if the sequence of characters
in \chpl{a} matches exactly the sequence of characters in \chpl{b};
otherwise the result is false.  The result of \chpl{a \!= b} is
equivalent to \chpl{\!(a == b)}.

\section{Miscellaneous Operators}
\label{Miscellaneous_Operators}

This section describes several miscellaneous operators.  These
operators can be redefined over different types using operator
overloading~(\rsec{Function_Overloading}).

\subsection{The String Concatenation Operator}
\label{The_String_Concatenation_Operator}
\index{operators!string concatenation}
\index{operators!concatenation!string}
\index{operators!+ (string)@\chpl{+} (string)}

The string concatenation operator \chpl{+} is predefined over numeric, boolean,
and enumerated types with strings. It casts its operands to string type and
concatenates them together.

\begin{chapelexample}{string-concat.chpl}
The code
\begin{chapelnoprint}
var i:int = 3;
writeln(
\end{chapelnoprint}
\begin{chapel}
"result: "+i
\end{chapel}
\begin{chapelnoprint}
);
\end{chapelnoprint}
\begin{chapeloutput}
result: 3
\end{chapeloutput}
where \chpl{i} is an integer appends the string representation of \chpl{i} to the
string literal \chpl{"result: "}.  If \chpl{i} is \chpl{3}, then the resulting string
would be \chpl{"result: 3"}.
\begin{chapelnoprint}
\end{chapelnoprint}
\end{chapelexample}

\subsection{The By Operator}
\label{The_By_Operator}
\index{by@\chpl{by}}
\index{operators!by@\chpl{by}}

The operator \chpl{by} is predefined on ranges and rectangular domains.
It is described in~\rsec{By_Operator_For_Ranges} for ranges
and~\rsec{Domain_Striding} for domains.

\subsection{The Align Operator}
\label{The_Align_Operator}
\index{align@\chpl{align}}
\index{operators!align@\chpl{align}}

The operator \chpl{align} is predefined on ranges and rectangular domains.
It is described in~\rsec{Align_Operator_For_Ranges} for ranges
and~\rsec{Domain_Alignment} for domains.

\subsection{The Range Count Operator}
\label{The_Range_Count_Operator}
\index{operators!range!count}
\index{#@\chpl{#}}
\index{operators!#@\chpl{#}}

The operator \chpl{#} is predefined on ranges. It is described
in ~\rsec{Count_Operator}.

\section{Let Expressions}
\label{Let_Expressions}
\index{let@\chpl{let}}
\index{operators!let@\chpl{let}}

A let expression allows variables to be declared at the expression
level and used within that expression.  The syntax of a let expression
is given by:
\begin{syntax}
let-expression:
  `let' variable-declaration-list `in' expression
\end{syntax}
The scope of the variables is the let-expression.
\begin{chapelexample}{let.chpl}
Let expressions are useful for defining variables in the context of
an expression.  In the code
\begin{chapelnoprint}
  var a = 4;
  var b = 5;
  var l =
\end{chapelnoprint}
\begin{chapel}
  let x: real = a*b, y = x*x in 1/y
\end{chapel}
the value determined by \chpl{a*b} is computed and converted to type
real if it is not already a real.  The square of the real is then
stored in \chpl{y} and the result of the expression is the reciprocal
of that value.
\begin{chapelnoprint}
  ;
  writeln(l);
\end{chapelnoprint}
\begin{chapeloutput}
0.0025
\end{chapeloutput}
\end{chapelexample}

\section{Conditional Expressions}
\label{Conditional_Expressions}
\index{conditional expressions}
\index{expressions!conditional}
\index{expressions!if-then-else}
\index{if@\chpl{if}}
\index{then@\chpl{then}}
\index{else@\chpl{else}}

A conditional expression is given by the following syntax:
\begin{syntax}
if-expression:
  `if' expression `then' expression `else' expression
  `if' expression `then' expression
\end{syntax}
The conditional expression is evaluated in two steps.  First, the
expression following the \chpl{if} keyword is evaluated.  Then, if the
expression evaluated to true, the expression following the \chpl{then}
keyword is evaluated and taken to be the value of this expression.
Otherwise, the expression following the \chpl{else} keyword is
evaluated and taken to be the value of this expression.  In both
cases, the unselected expression is not evaluated.

The `else' clause can be omitted only when the conditional expression
is nested immediately inside a for or forall expression.  Such an expression
is used to filter predicates as described
in~\rsec{Filtering_Predicates_For} and~\rsec{Filtering_Predicates_Forall},
respectively.

\begin{chapelexample}{condexp.chpl}
This example shows how if-then-else can be used in a context in which an
expression is expected.  The code
\begin{chapel}
writehalf(8);
writehalf(21);
writehalf(1000);

proc writehalf(i: int) {
  var half = if (i % 2) then i/2 +1 else i/2;
  writeln("Half of ",i," is ",half); 
}
\end{chapel}
produces the output
\begin{chapelprintoutput}{}
Half of 8 is 4
Half of 21 is 11
Half of 1000 is 500
\end{chapelprintoutput}
\end{chapelexample}

\section{For Expressions}
\label{For_Expressions}
\index{for@\chpl{for}}
\index{expressions!for@\chpl{for}}

A for expression is given by the following syntax:
\begin{syntax}
for-expression:
  `for' index-var-declaration `in' iteratable-expression `do' expression
  `for' iteratable-expression `do' expression
\end{syntax}
A for expression is an iterator that executes a for loop (\rsec{The_For_Loop}),
evaluates the body expression on each iteration of the loop,
and yields each resulting value.

When a for expression is used to initialize a variable, such as
\begin{chapel}
var X = for iterableExpression() do computeValue();
\end{chapel}
the variable will be inferred to have an array type.
The array's domain is defined by the \sntx{iterable-expression}
following the same rules as for promotion, both in the regular
case \rsec{Promotion} and in the zipper case \rsec{Zipper_Promotion}.

\subsection{Filtering Predicates in For Expressions}
\label{Filtering_Predicates_For}
\index{for@\chpl{for}!filtering predicates}
\index{expressions!for@\chpl{for}!filtering predicates}

A conditional expression that is immediately enclosed in a for
expression and does not require an else clause filters the iterations
of the for expression.
The iterations for which the condition does not hold
are not reflected in the result of the for expression.

When a for expression with a filtering predicate is captured into
a variable, the resulting array has a 1-based one-dimensional domain.

\begin{chapelexample}{yieldPredicates.chpl}
The code
\begin{chapel}
var A = for i in 1..10 do if i % 3 != 0 then i;
\end{chapel}
\begin{chapelpost}
writeln(A);
\end{chapelpost}
\begin{chapeloutput}
1 2 4 5 7 8 10
\end{chapeloutput}
declares an array A that is initialized to the integers between
1 and 10 that are not divisible by 3.
\end{chapelexample}

\cleardoublepage
\sekshun{Statements}
\label{Statements}
\index{statement}

Chapel is an imperative language with statements that may have side
effects.  Statements allow for the sequencing of program execution.
Chapel provides the following statements:

\begin{syntax}
statement:
  block-statement
  expression-statement
  assignment-statement
  swap-statement
  conditional-statement
  select-statement
  while-do-statement
  do-while-statement
  for-statement
  label-statement
  break-statement
  continue-statement
  param-for-statement
  use-statement
  type-select-statement
  empty-statement
  return-statement
  yield-statement
  module-declaration-statement
  procedure-declaration-statement
  external-procedure-declaration-statement
  exported-procedure-declaration-statement
  iterator-declaration-statement
  method-declaration-statement
  type-declaration-statement
  variable-declaration-statement
  remote-variable-declaration-statement
  on-statement
  cobegin-statement
  coforall-statement
  begin-statement
  sync-statement
  serial-statement
  atomic-statement
  forall-statement
  delete-statement
\end{syntax}

Individual statements are defined in the remainder of this chapter
and additionally as follows:

\begin{itemize}
\item return \rsec{The_Return_Statement}
\item yield \rsec{The_Yield_Statement}
\item module declaration \rsec{Modules}
\item procedure declaration \rsec{Function_Definitions}
\item external procedure declaration \rsec{Calling_External_Functions}
\item exporting procedure declaration \rsec{Calling_Chapel_Functions}
\item iterator declaration \rsec{Iterator_Function_Definitions}
\item method declaration \rsec{Class_Methods}
\item type declaration \rsec{Types}
\item variable declaration \rsec{Variable_Declarations}
\item remote variable declaration ~\rsec{remote_variable_declarations}
\item \chpl{on} statement \rsec{On}
\item cobegin, coforall, begin, sync, serial and atomic statements
      \rsec{Task_Parallelism_and_Synchronization}
\item forall \rsec{Data_Parallelism}
\item delete \rsec{Class_Delete}
\end{itemize}

\section{Blocks}
\label{Blocks}
\index{block}

A block is a statement or a possibly empty list of statements that
form their own scope.  A block is given by
\begin{syntax}
block-statement:
  { statements[OPT] }

statements:
  statement
  statement statements
\end{syntax}

Variables defined within a block are local
variables~(\rsec{Local_Variables}).

The statements within a block are executed serially unless the block
is in a cobegin statement~(\rsec{Cobegin}).

\section{Expression Statements}
\label{Expression_Statements}
\index{expressions!statement}
\index{expression statement}
\index{statements!expression}
The expression statement evaluates an expression solely for side
effects. The syntax for an expression statement is given by
\begin{syntax}
expression-statement:
  variable-expression ;
  member-access-expression ;
  call-expression ;
  constructor-call-expression ;
  let-expression ; 
\end{syntax}

\section{Assignment Statements}
\label{Assignment_Statements}
\index{assignment}
\index{statements!assignment}

An assignment statement assigns the value of an expression to another
expression, for
example, a variable.  Assignment statements are given by

\index{= (see also assignment)@\chpl{=} (see also assignment)}
\index{+=@\chpl{+=}}
\index{-=@\chpl{-=}}
\index{*=@\chpl{*=}}
\index{/=@\chpl{/=}}
\index{\%=@\chpl{\%=}}
\index{**=@\chpl{**=}}
\index{&=@\chpl{&=}}
\index{|=@\chpl{|=}}
\index{^=@\chpl{^=}}
\index{||=@\chpl{||=}}
\index{&&=@\chpl{&&=}}
\index{<<=@\chpl{<<=}}
\index{>>=@\chpl{>>=}}
\index{operators!assignment}
\begin{syntax}
assignment-statement:
  lvalue-expression assignment-operator expression

assignment-operator: one of
   = $ $ $ $ += $ $ $ $ -= $ $ $ $ *= $ $ $ $ /= $ $ $ $ %= $ $ $ $ **= $ $ $ $ &= $ $ $ $ |= $ $ $ $ ^= $ $ $ $ &&= $ $ $ $ ||= $ $ $ $ <<= $ $ $ $ >>=
\end{syntax}

\index{operators!compound assignment}
\index{operators!assignment!compound}
\index{operators!simple assignment}
\index{operators!assignment!simple}
The assignment operators that contain a binary operator symbol as a prefix
are \emph{compound assignment} operators.  The remaining assignment
operator \chpl{=} is called \emph{simple assignment}.

The expression on the left-hand side of the assignment operator must
be a valid lvalue~(\rsec{LValue_Expressions}).  It is evaluated before the
expression on the right-hand side of the assignment operator, which
can be any expression.

When the left-hand side is of a numerical type, there is
an implicit conversion~(\rsec{Implicit_Conversions})
of the right-hand side expression
to the type of the left-hand side expression.  Additionally, for simple
assignment, if the left-hand side is of Boolean type, the right-hand side is
implicitly converted to the type of the left-hand side (i.e. a \chpl{bool(?w)}
with the same width \chpl{w}).

For simple assignment, the validity and semantics of assigning between
classes~(\rsec{Class_Assignment}), records~(\rsec{Record_Assignment}),
unions~(\rsec{Union_Assignment}), tuples~(\rsec{Tuple_Assignment}),
ranges~(\rsec{Range_Assignment}),
domains~(\rsec{Domain_Assignment}), and arrays~(\rsec{Array_Assignment})
are discussed in these later sections.

A compound assignment is
shorthand for applying the binary operator to the left- and
right-hand side expressions and then assigning the result
to the left-hand side expression.
For numerical types, the left-hand side expression is evaluated only once,
and there is an implicit conversion of the result of the binary operator
to the type of the left-hand side expression.  Thus, for
example, \chpl{x += y} is equivalent to \chpl{x = x + y} where the
expression \chpl{x} is evaluated once.

For all other compound assignments, Chapel provides a completely generic
catch-all implementation defined in the obvious way.  For example:

\begin{chapel}
inline proc +=(ref lhs, rhs) {
  lhs = lhs + rhs;
}
\end{chapel}

Thus, compound assignment can be used with operands of arbitrary types,
provided that the following provisions are met: If the type of the left-hand
argument of a compound assignment operator \chpl{op=} is $L$ and that of the
right-hand argument is $R$, then a definition for the corresponding binary
operator \chpl{op} exists, such that $L$ is coercible to the type of its
left-hand formal and $R$ is coercible to the type of its right-hand formal.
Further, the result of \chpl{op} must be coercible to $L$, and there must exist
a definition for simple assignment between objects of type $L$.

Both simple and compound assignment operators can be overloaded for different
types using operator overloading~(\rsec{Function_Overloading}).
In such an overload, the left-hand side expression should have
\chpl{ref} intent and be modified within the body of the function.  The return
type of the function should be \chpl{void}.

\section{The Swap Statement}
\label{The_Swap_Statement}
\index{swap!statement}
\index{statements!swap}
\index{swap!operator}
\index{operators!swap}

% TODO: if appropriate, define swap as a sequence of three assignments
The swap statement indicates to swap the values in the expressions
on either side of the swap operator.  Since both expressions are assigned
to, each must be a valid lvalue expression~(\rsec{LValue_Expressions}).

The swap operator can be overloaded for different types using operator
overloading~(\rsec{Function_Overloading}).
\begin{syntax}
swap-statement:
  lvalue-expression swap-operator lvalue-expression

swap-operator:
  <=>
\end{syntax}

To implement the swap operation, the compiler uses temporary variables
as necessary.

\begin{example}
When resolved to the default swap operator, the following swap statement
\begin{chapel}
var a, b: real;

a <=> b;
\end{chapel}
is semantically equivalent to:
\begin{chapel}
const t = b;
b = a;
a = t;
\end{chapel}
\end{example}

\section{The Conditional Statement}
\label{The_Conditional_Statement}
\index{statements!conditional}
\index{if@\chpl{if}}
\index{then@\chpl{then}}
\index{else@\chpl{else}}
\index{conditional statements}

The conditional statement allows execution to choose between two
statements based on the evaluation of an expression of \chpl{bool}
type. The syntax for a conditional statement is given by
\begin{syntax}
conditional-statement:
  `if' expression `then' statement else-part[OPT]
  `if' expression block-statement else-part[OPT]

else-part:
  `else' statement
\end{syntax}

A conditional statement evaluates an expression of bool type. If the
expression evaluates to true, the first statement in the conditional
statement is executed.  If the expression evaluates to false and the
optional else-clause exists, the statement following the
\chpl{else} keyword is executed.

If the expression is a parameter, the conditional statement is folded
by the compiler. If the expression evaluates to true, the first
statement replaces the conditional statement. If the expression
evaluates to false, the second statement, if it exists, replaces the
conditional statement; if the second statement does not exist, the
conditional statement is removed.

Each statement embedded in the {\em conditional-statement} has its own
scope whether or not an explicit block surrounds it.

\index{conditional statement!dangling else}
If the statement that immediately follows the optional \chpl{then}
keyword is a conditional statement and it is not in a block, the
else-clause is bound to the nearest preceding conditional statement
without an else-clause.
The statement in the else-clause can be a conditional statement, too.

\begin{chapelexample}{conditionals.chpl}
The following function prints \chpl{two} when \chpl{x} is \chpl{2}
and \chpl{B,four} when \chpl{x} is \chpl{4}.
\begin{chapel}
proc condtest(x:int) {
  if x > 3 then
    if x > 5 then
      write("A,");
    else
      write("B,");

  if x == 2 then
    writeln("two");
  else if x == 4 then
    writeln("four");
  else
    writeln("other");
}
\end{chapel}
\begin{chapelpost}
for i in 2..6 do condtest(i);
\end{chapelpost}
\begin{chapeloutput}
two
other
B,four
B,other
A,other
\end{chapeloutput}
\end{chapelexample}

\section{The Select Statement}
\label{The_Select_Statement}
\index{select@\chpl{select}}
\index{when@\chpl{when}}
\index{statements!select@\chpl{select}}
\index{statements!when@\chpl{when}}

The select statement is a multi-way variant of the conditional
statement.  The syntax is given by:
\begin{syntax}
select-statement:
  `select' expression { when-statements }

when-statements:
  when-statement
  when-statement when-statements

when-statement:
  `when' expression-list `do' statement
  `when' expression-list block-statement
  `otherwise' statement

expression-list:
  expression
  expression , expression-list
\end{syntax}
The expression that follows the keyword \chpl{select}, the select
expression, is compared with the list of expressions following the
keyword \chpl{when}, the case expressions, using the equality
operator \chpl{==}.  If the expressions cannot be compared with the
equality operator, a compile-time error is generated.  The first case
expression that contains an expression where that comparison
is \chpl{true} will be selected and control transferred to the
associated statement.  If the comparison is always \chpl{false}, the
statement associated with the keyword \chpl{otherwise}, if it exists,
will be selected and control transferred to it.  There may be at most
one \chpl{otherwise} statement and its location within the select
statement does not matter.

Each statement embedded in the {\em when-statement} has its own scope
whether or not an explicit block surrounds it.

\section{The While Do and Do While Loops}
\label{The_While_and_Do_While_Loops}
\index{while loops}
\index{while@\chpl{while}}
\index{statements!while@\chpl{while}}

There are two variants of the while loop in Chapel.  The syntax of the
while-do loop is given by:
\begin{syntax}
while-do-statement:
  `while' expression `do' statement
  `while' expression block-statement
\end{syntax}
The syntax of the do-while loop is given by:
\begin{syntax}
do-while-statement:
  `do' statement `while' expression ;
\end{syntax}
In both variants, the expression evaluates to a value of type \chpl{bool}
which determines when the loop terminates and control continues with
the statement following the loop.

The while-do loop is executed as follows:
\begin{enumerate}
\item The expression is evaluated.
\item If the expression evaluates to \chpl{false}, the statement is
  not executed and control continues to the statement following the
  loop.
\item If the expression evaluates to \chpl{true}, the statement is
  executed and control continues to step 1, evaluating the expression
  again.
\end{enumerate}

The do-while loop is executed as follows:
\begin{enumerate}
\item The statement is executed.
\item The expression is evaluated.
\item If the expression evaluates to \chpl{false}, control continues
  to the statement following the loop.
\item If the expression evaluates to \chpl{true}, control continues to
  step 1 and the the statement is executed again.
\end{enumerate}
In this second form of the loop, note that the statement is executed
unconditionally the first time.

\begin{chapelexample}{while.chpl}
The following example illustrates the difference between
the \sntx{do-while-statement} and the \sntx{while-do-statement}.  The body of
the do-while loop is always executed at least once, even if the loop conditional
is already false when it is entered.  The code
\begin{chapel}
var t = 11;

writeln("Scope of do while loop:");
do {
  t += 1;
  writeln(t);
} while (t <= 10);

t = 11;
writeln("Scope of while loop:");
while (t <= 10) {
  t += 1;
  writeln(t);
}
\end{chapel}
produces the output
\begin{chapelprintoutput}{}
Scope of do while loop:
12
Scope of while loop:
\end{chapelprintoutput}
\end{chapelexample}

Chapel do-while loops differ from those found in most other languages in
one important regard.  If the body of a do-while statement is a block
statement and new variables are defined within that block statement,
then the scope of those variables extends to cover the loop's
termination expression.
\begin{chapelexample}{do-while.chpl}
The following example demonstrates that the scope of the variable t 
includes the loop termination expression.
\begin{chapel}
var i = 0;
do {
  var t = i;
  i += 1;
  writeln(t);
} while (t != 5);
\end{chapel}
produces the output
\begin{chapelprintoutput}{}
0
1
2
3
4
5
\end{chapelprintoutput}
\end{chapelexample}


\section{The For Loop}
\label{The_For_Loop}
\index{for loops}
\index{for@\chpl{for}}
\index{statements!for@\chpl{for}}

The for loop iterates over ranges, domains, arrays, iterators, or any
class that implements an iterator named \chpl{these}.  The syntax of
the for loop is given by:
\begin{syntax}
for-statement:
  `for' index-var-declaration `in' iteratable-expression `do' statement
  `for' index-var-declaration `in' iteratable-expression block-statement
  `for' iteratable-expression `do' statement
  `for' iteratable-expression block-statement

index-var-declaration:
  identifier
  tuple-grouped-identifier-list

iteratable-expression:
  expression
  `zip' ( expression-list )
\end{syntax}

The \sntx{index-var-declaration} declares new variables for the scope
of the loop.  It may specify a new identifier or may specify multiple
identifiers grouped using a tuple notation in order to destructure the
values returned by the iterator expression, as described
in~\rsec{Indices_in_a_Tuple}.

The \sntx{index-var-declaration} is optional and may be omitted if the
indices do not need to be referenced in the loop.

If the iteratable-expression begins with the keyword \chpl{zip} followed
by a parenthesized expression-list, the listed expressions must support 
zipper iteration.

\subsection{Zipper Iteration}
\label{Zipper_Iteration}
\index{zipper iteration}
\index{iteration!zipper}

When multiple iterators are iterated over in a zipper context, on each
iteration, each expression is iterated over, the values are returned
by the iterators in a tuple and assigned to the index, and then
statement is executed.

The shape of each iterator, the rank and the extents in each
dimension, must be identical.

\begin{chapelexample}{zipper.chpl}
The output of
\begin{chapel}
for (i, j) in zip(1..3, 4..6) do
  write(i, " ", j, " ");
\end{chapel}
\begin{chapelpost}
writeln();
\end{chapelpost}
is
\begin{chapelprintoutput}{}
1 4 2 5 3 6 
\end{chapelprintoutput}
\end{chapelexample}

\subsection{Parameter For Loops}
\label{Parameter_For_Loops}
\index{statements!param for}
\index{for loops!parameters}
\index{for@\chpl{for}}
\index{param@\chpl{param}}

Parameter for loops are unrolled by the compiler so that the index
variable is a parameter rather than a variable.  The syntax for a
parameter for loop statement is given by:
\begin{syntax}
param-iteratable-expression:
  range-literal
  range-literal `by' integer-literal

param-for-statement:
  `for' `param' identifier `in' param-iteratable-expression `do' statement
  `for' `param' identifier `in' param-iteratable-expression block-statement
\end{syntax}
Parameter for loops are restricted to iteration over range literals
with an optional by expression where the bounds and stride must be
parameters.  The loop is then unrolled for each iteration.

\section{The Break, Continue and Label Statements}
\label{Label_Break_Continue}
\index{statements!jumps}
\index{label@\chpl{label}}
\index{break@\chpl{break}}
\index{continue@\chpl{continue}}
\index{statements!label@\chpl{label}}
\index{statements!break@\chpl{break}}
\index{statements!continue@\chpl{continue}}

The break- and continue-statements are used to alter the flow of control within a
loop construct.  A break-statement causes flow to exit the containing loop and
resume with the statement immediately following it.  A continue-statement causes
control to jump to the end of the body of the containing loop and resume
execution from there.  By default, break- and continue-statements exit
or skip the body of the immediately-containing loop construct.

The label-statement is used to name a specific loop so that \chpl{break}
and \chpl{continue} can exit or resume a less-nested loop.
Labels can only be attached to for-, while-do- and do-while-statements.
When a break statement has a label, execution continues with the first statement
following the loop statement with the matching label.  When a continue statement
has a label, execution continues at the end of the body of the loop with the
matching label.  If there is no containing loop construct with a matching label,
a compile-time error occurs.

The syntax for label, break, and continue statements is given by:
\begin{syntax}
break-statement:
  `break' identifier[OPT] ;

continue-statement:
  `continue' identifier[OPT] ;

label-statement:
  `label' identifier statement
\end{syntax}

Break-statements cannot be used to exit parallel loops.  

\begin{rationale}
Breaks are not permitted in parallel loops because the execution order
of the iterations of parallel loops is not defined.
\end{rationale}

\begin{future}
We expect to support a \emph{eureka} concept which would enable one or
more tasks to stop the execution of all current and future iterations
of the loop.
\end{future}

\begin{example}
In the following code, the index of the first element in each row of
\chpl{A} that is equal to \chpl{findVal} is printed.  Once a match is
found, the continue statement is executed causing the outer loop to
move to the next row.
\begin{chapel}
label outer for i in 1..n {
  for j in 1..n {
    if A[i, j] == findVal {
      writeln("index: ", (i, j), " matches.");
      continue outer;
    }
  }
}
\end{chapel}
\end{example}

\section{The Use Statement}
\label{The_Use_Statement}
\index{use@\chpl{use}}
\index{statements!use@\chpl{use}}
\index{modules!using}

The use statement makes the symbols defined by a module available
within the scope containing the use statement without requiring them
to be prefixed by the module's name.  The syntax of the use statement
is given by:

\begin{syntax}
use-statement:
  `use' module-name-list ;

module-name-list:
  module-name
  module-name , module-name-list

module-name:
  identifier
  module-name . module-name
\end{syntax}
The use statement makes symbols in each listed module's scope available
from the scope where the use statement occurs.

Symbols made available by a use statement are at an outer scope from those
defined directly in the scope where the use statement occurs, but at a
nearer scope than symbols defined in the scope containing the scope where
the use statement occurs.

If used modules themselves use other modules, symbols are scoped according
the depth of use statements followed to find them. It is an error for two
variables, types, or modules to be defined at the same depth.

\begin{openissue}
There is an expectation that this statement will be extended to allow
the programmer to restrict which symbols are 'used' as well as to
rename symbols that are 'used.'
\end{openissue}

\begin{chapelexample}{use.chpl}
  The following example illustrates how the use statement makes
  symbols declared in \chpl{M1}'s scope (like procedure
  \chpl{foo()}) visible within the scope of \chpl{M2}'s \chpl{main}
  function.  Without the \chpl{use} statement, the procedure call to
  \chpl{foo} could not be resolved since \chpl{M2} would not have
  access to symbols in \chpl{M1}.

  When executed, the program
\begin{chapel}
module M1 {
  proc foo() {
    writeln("In M1's foo.");
  }
}

module M2 {
  proc main() {
    use M1;

    writeln("In M2's main.");
    foo();
  }
}
\end{chapel}
prints out
\begin{chapelprintoutput}{}
In M2's main.
In M1's foo.
\end{chapelprintoutput}
\end{chapelexample}

\section{The Type Select Statement}
\label{The_Type_Select_Statement}
\index{statements!type select}
\index{type select statements}

A type select statement has two uses.  It can be used to determine the
type of a union, as discussed
in~\rsec{The_Type_Select_Statement_and_Unions}.  In its more general
form, it can be used to determine the types of one or more values
using the same mechanisms used to disambiguate function definitions.
It syntax is given by:
\begin{syntax}
type-select-statement:
  `type' `select' expression-list { type-when-statements }

type-when-statements:
  type-when-statement
  type-when-statement type-when-statements

type-when-statement:
  `when' type-list `do' statement
  `when' type-list block-statement
  `otherwise' statement

expression-list:
  expression
  expression , expression-list

type-list:
  type-specifier
  type-specifier , type-list
\end{syntax}

Let the expressions following the keyword \chpl{select} be called the 
\emph{select expressions}.  The number of select expressions must be equal to the
number of types following each of the \chpl{when} keywords.  Like the
select statement, one of the statements associated with a \chpl{when}
will be executed.  In this case, that statement is chosen by the
function resolution mechanism.  The select expressions are the actual
arguments, the types following the \chpl{when} keywords are the types
of the formal arguments for different anonymous functions.  The
function that would be selected by function resolution determines the
statement that is executed.  If none of the functions are chosen, the
the statement associated with the keyword \chpl{otherwise}, if it
exists, will be selected.

As with function resolution, this can result in an ambiguous
situation.  Unlike with function resolution, in the event of an
ambiguity, the first statement in the list of when statements is
chosen.

\begin{chapelexample}{typeselect.chpl}
The following example shows how the type select statement can be used to perform
a different action based on the static type of the operand.

The program
\begin{chapel}
var x = 32, y = 15.5; 
var z: int(8);
var coord = (0.0,0.0);
var yes: bool;

writetype(x);
writetype(y);
writetype(z);
writetype(coord);
writetype(yes);
writetype("no");
writetype(here);

proc writetype(x) {
  type select x {
    when int do writeln("Integer type");
    when uint do writeln("Unsigned integer type");
    when real do writeln("Real type");
    when complex do writeln("Complex type");
    when string do writeln("String type");
    when bool do writeln("Boolean type");
    when imag do writeln("Imag type");
    when locale do writeln("Locale type");
    when void do writeln("Void type");
    otherwise writeln("Non-primitive type");
  }
}
\end{chapel}
produces the output
\begin{chapelprintoutput}{}
Integer type
Real type
Integer type
Non-primitive type
Boolean type
String type
Locale type
\end{chapelprintoutput}
\end{chapelexample}

\section{The Empty Statement}
\label{The_Empty_Statement}
\index{statements!empty}

An empty statement has no effect.  The syntax of an empty statement is
given by
\begin{syntax}
empty-statement:
  ;
\end{syntax}

\cleardoublepage
\sekshun{Modules}
\label{Modules}
\index{modules}

Chapel supports modules to manage name spaces.  A program consists of
one or more modules.  Every symbol, including variables, functions,
and types, is associated with some module.

Module definitions are described in~\rsec{Module_Definitions}.  The
relation between files and modules is described
in~\rsec{Implicit_Modules}.  Nested modules are described
in~\rsec{Nested_Modules}.  The visibility of a module's symbols by
users of the module is described in~\rsec{Visibility_Of_Symbols}.  The execution
of a program and module initialization is described
in~\rsec{Program_Execution}.

\section{Module Definitions}
\label{Module_Definitions}
\index{module@\chpl{module}}
\index{modules!definitions}

A module is declared with the following syntax:
\begin{syntax}
module-declaration-statement:
  privacy-specifier[OPT] `module' module-identifier block-statement

privacy-specifier:
  `private'
  `public'

module-identifier:
  identifier
\end{syntax}

A module's name is specified after the \chpl{module} keyword.
The \sntx{block-statement} opens the module's scope.  Symbols defined
in this block statement are defined in the module's scope and are
called \emph{top-level module symbols}.  The visibility of a module is
defined by its \sntx{privacy-specifier}~(\rsec{Visibility_Of_A_Module}).

Module declaration statements must be top-level statements within a
module.  A module that is declared within another module is called a
nested module~(\rsec{Nested_Modules}).

\section{Files and Implicit Modules}
\label{Implicit_Modules}
\index{modules!and files}

Multiple modules can be defined in the same file and need not bear any
relation to the file in terms of their names.

\begin{chapelexample}{two-modules.chpl}
The following file contains two explicitly named modules
(\rsec{Explicit_Naming}), MX and MY.
\begin{chapel}
module MX {
  var x: string = "Module MX";
  proc printX() {
    writeln(x);
  }
}

module MY {
  var y: string = "Module MY";
  proc printY() {
    writeln(y);
  }
}
\end{chapel}
\begin{chapelpost}
MX.printX();
MY.printY();
\end{chapelpost}
\begin{chapeloutput}
Module MX
Module MY
\end{chapeloutput}
Module MX defines top-level module symbols x and printX, while MY
defines top-level module symbols y and printY.
\end{chapelexample}

For any file that contains top-level statements other than module
declarations, the file itself is treated as the module declaration.
In this case,
\index{implicit modules}
\index{modules!implicit}
the module is implicit and takes its name from the base filename.  In
particular, the module name is defined as the remaining string after
removing the \chpl{.chpl} suffix and any path specification from the
specified filename.  If the resulting name is not a legal Chapel
identifier, it cannot be referenced in a use statement.

\begin{chapelexample}{implicit.chpl}
The following file, named implicit.chpl, defines an implicitly named
module called implicit.
\begin{chapel}
var x: int = 0;
var y: int = 1;

proc printX() {
  writeln(x);
}
proc printY() {
  writeln(y);
}
\end{chapel}
\begin{chapelpost}
printX();
printY();
\end{chapelpost}
\begin{chapeloutput}
0
1
\end{chapeloutput}
Module implicit defines the top-level module symbols x, y, printX, and
printY.
\end{chapelexample}


\section{Nested Modules}
\label{Nested_Modules}
\index{modules!nested}

A nested module is a module that is defined within another module, the
outer module.  Nested modules automatically have access to all of the
symbols in the outer module.  However, the outer module needs to
explicitly use a nested module to have access to its symbols.

A nested module can be used without using the outer module by
explicitly naming the outer module in the use statement.
\begin{chapelexample}{nested-use.chpl}
The code
\begin{chapelpre}
module libsci {
  writeln("Initializing libsci");
  module blas {
    writeln("\\tInitializing blas");
  }
}
module testmain { // used to avoid warnings
}
\end{chapelpre}
\begin{chapel}
use libsci.blas;
\end{chapel}
\begin{chapeloutput}
Initializing libsci
	Initializing blas
\end{chapeloutput}
uses a module named \chpl{blas} that is nested inside a module
named \chpl{libsci}.
\end{chapelexample}

Files with both module declarations and top-level statements result in
nested modules.

\begin{chapelexample}{nested.chpl}
The following file, named nested.chpl, defines an
implicitly named module called nested, with nested modules
MX and MY.
\begin{chapel}
module MX {
  var x: int = 0;
}

module MY {
  var y: int = 0;
}

use MX, MY;

proc printX() {
  writeln(x);
}

proc printY() {
  writeln(y);
}
\end{chapel}
\begin{chapelpost}
printX();
printY();
\end{chapelpost}
\begin{chapeloutput}
0
0
\end{chapeloutput}
\end{chapelexample}


\section{Access of Module Contents}
\label{Access_Of_Module_Contents}
\index{modules!access}

A module's contents can be accessed by code outside of that module
depending on the visibility of the module
itself~(\rsec{Visibility_Of_A_Module}) and the visibility of each
individual symbol~(\rsec{Visibility_Of_Symbols}).  This can be done
via explicit naming~(\rsec{Explicit_Naming}) or the use
statement~(\rsec{Using_Modules}).

\subsection{Visibility Of A Module}
\label{Visibility_Of_A_Module}
\index{modules!access}

A top-level module is visible anywhere if the \sntx{privacy-specifier}
of its declaration is \chpl{public} or is omitted (i.e. by default).
A top-level module declared \chpl{private} is visible only within that
module.  The visibility of a nested module is subject to the rules
of~\rsec{Visibility_Of_Symbols}. There, the nested module is
considered a "symbol defined at the top level scope" of its outer
module.

\subsection{Visibility Of A Module's Symbols}
\label{Visibility_Of_Symbols}
\index{modules!access}

A symbol defined at the top level scope of a module is \emph{visible}
from outside the module when the \sntx{privacy-specifier} of its
definition is \chpl{public} or is omitted (i.e. by default). When a
symbol defined at the top level scope of a module is declared
\chpl{private}, it is not visible outside of that module. A
symbol's visibility inside its module is controlled by normal lexical
scoping and is not affected by its \sntx{privacy-specifier}.  A
module's visible symbols are accessible via explicit
naming~(\rsec{Explicit_Naming}) or the use
statement~(\rsec{Using_Modules}) only where the module's symbol is
visible~(\rsec{Visibility_Of_A_Module}).

\subsection{Explicit Naming}
\label{Explicit_Naming}
\index{modules!explicitly named}

All publicly visible top-level module symbols can be named explicitly
with the following syntax:
\begin{syntax}
module-access-expression:
  module-identifier-list . identifier

module-identifier-list:
  module-identifier
  module-identifier . module-identifier-list

\end{syntax}
This allows two variables that have the same name to be distinguished
based on the name of their module.  Using explicit module naming in a
function call restricts the set of candidate functions to those in the
specified module.

If code refers to symbols that are defined by multiple modules, the
compiler will issue an error.  Explicit naming can be used to
disambiguate the symbols in this case.

\begin{openissue}
It is currently unspecified whether the
first-named module is always at the outermost module level scope, or whether a
scope-search mechanism is used starting at the scope containing the
usage.
\end{openissue}

\begin{chapelexample}{ambiguity.chpl}
In the following example,
\begin{chapel}
module M1 {
  var x: int = 1;
  var y: int = -1;
  proc printX() {
    writeln("M1's x is: ", x);
  }
  proc printY() {
    writeln("M1's y is: ", y);
  }
}
 
module M2 {
  use M3;
  use M1;

  var x: int = 2;

  proc printX() {
    writeln("M2's x is: ", x);
  }

  proc main() {
    M1.x = 4;
    M1.printX();
    writeln(x);
    printX(); // This is not ambiguous
    printY(); // ERROR: This is ambiguous
  }
}

module M3 {
  var x: int = 3;
  var y: int = -3;
  proc printY() {
    writeln("M3's y is: ", y);
  }
}
\end{chapel}
\begin{chapeloutput}
ambiguity.chpl:22: In function 'main':
ambiguity.chpl:27: error: ambiguous call 'printY()'
ambiguity.chpl:34: note: candidates are: printY()
ambiguity.chpl:7: note:                 printY()
\end{chapeloutput}
The call to printX() is not ambiguous because M2's definition shadows
that of M1.  On the other hand, the call to printY() is ambiguous
because it is defined in both M1 and M3.  This will result in a
compiler error.
\end{chapelexample}

\subsection{Using Modules}
\label{Using_Modules}
\index{modules!using}

If a module is visible to where accessing its symbols is desirable,
then a use statement on that module may be employed.  Use statements
make a module's visible symbols available without requiring them to be
prefixed by the module's name.  The syntax of the use statement is
given by:

\begin{syntax}
use-statement:
  `use' module-name-list ;

module-name-list:
  module-name
  module-name , module-name-list

module-name:
  identifier
  module-name . module-name
\end{syntax}

Symbols made available by a use statement are at an outer scope from those
defined directly in the scope where the use statement occurs, but at a
nearer scope than symbols defined in the scope containing the scope where
the use statement occurs.

Use statements are transitive by default: if a module A used by
another module B also contains a use of a further module C, C's
symbols will also be considered visible within B, at a further scope
than the symbols of other modules immediately used by B.

% We would like to provide a way to specify that a use should not be
% transitive

It is an error for two public variables, types, or modules to be
defined at the same depth.

\begin{openissue}
There is an expectation that this statement will be extended to allow
the programmer to restrict which symbols are 'used' as well as to
rename symbols that are 'used.'
\end{openissue}

\begin{chapelexample}{use.chpl}
  The following example illustrates how the use statement makes
  symbols declared in \chpl{M1}'s scope (like procedure
  \chpl{foo()}) visible within the scope of \chpl{M2}'s \chpl{main}
  function.  Without the \chpl{use} statement, the procedure call to
  \chpl{foo} could not be resolved since \chpl{M2} would not have
  access to symbols in \chpl{M1}.

  When executed, the program
\begin{chapel}
module M1 {
  proc foo() {
    writeln("In M1's foo.");
  }
}

module M2 {
  proc main() {
    use M1;

    writeln("In M2's main.");
    foo();
  }
}
\end{chapel}
prints out
\begin{chapelprintoutput}{}
In M2's main.
In M1's foo.
\end{chapelprintoutput}
\end{chapelexample}



\subsection{Module Initialization}
\label{Module_Initialization}
\index{modules!initialization}

Module initialization occurs at program start-up.  All top-level
statements in a module other than function and type declarations are
executed during module initialization.

\begin{chapelexample}{init.chpl}
In the code,
\begin{chapelpre}
proc foo() {
    return 1;
}
\end{chapelpre}
\begin{chapel}
var x = foo();       // executed at module initialization
writeln("Hi!");      // executed at module initialization
proc sayGoodbye {
  writeln("Bye!");   // not executed at module initialization
}
\end{chapel}
\begin{chapeloutput}
Hi!
\end{chapeloutput}
The function foo() will be invoked and its result assigned to x.  Then
``Hi!'' will be printed.
\end{chapelexample}

Module initialization order is discussed
in~\rsec{Module_Initialization_Order}.


\section{Program Execution}
\label{Program_Execution}
\index{program execution}
\index{program initialization}

Chapel programs start by initializing all modules and then executing
the main function~(\rsec{The_main_Function}).

\subsection{The {\em main} Function}
\label{The_main_Function}

\index{main@\chpl{main}}
\index{functions!main@\chpl{main}}
The main function must be called \chpl{main} and must have zero
arguments.  It can be specified with or without parentheses.  In any
Chapel program, there is a single main function that defines the
program's entry point.  If a program defines multiple potential entry
points, the implementation may provide a compiler flag that
disambiguates between main functions in multiple modules.

\begin{craychapel}
In the Cray Chapel compiler implementation, the \emph{--
--main-module} flag can be used to specify the module from which the
main function definition will be used.
\end{craychapel}

\begin{chapelexample}{main-module.chpl}
Because it defines two \chpl{main} functions, the following code will yield an
error unless a main module is specified on the command line.
\begin{chapel}
module M1 {
  const x = 1;
  proc main() {
    writeln("M", x, "'s main");
  }
}
 
module M2 {
  use M1;

  const x = 2;
  proc main() {
    M1.main();
    writeln("M", x, "'s main");
  }
}
\end{chapel}
\begin{chapelcompopts}
--main-module M1 \# main\_module.M1.good
--main-module M2 \# main\_module.M2.good
\end{chapelcompopts}
If M1 is specified as the main module, the program will output:
\begin{chapelprintoutput}{main_module.M1.good}
M1's main
\end{chapelprintoutput}
If M2 is specified as the main module the program will output:
\begin{chapelprintoutput}{main_module.M2.good}
M1's main
M2's main
\end{chapelprintoutput}
Notice that main is treated like just another function if it is not in
the main module and can be called as such.
\end{chapelexample}

\index{exploratory programming}

%subsubsection{Programs with a Single Module}
%% \label{Programs_with_a_Single_Module}

To aid in exploratory programming, a default main function is
created if the program does not contain a user-defined main function.  The
default main function is equivalent to
\begin{chapel}
proc main() {}
\end{chapel}

\begin{chapelexample}{no-main.chpl}
The code
\begin{chapel}
writeln("hello, world");
\end{chapel}
\begin{chapeloutput}
hello, world
\end{chapeloutput}
is a legal and complete Chapel program.  The startup code for a Chapel program
first calls the module initialization code for the main module and then
calls \chpl{main()}.  This program's initialization function is the top-level
writeln() statement.  The module declaration is taken to be the entire file,
as described in~\rsec{Implicit_Modules}.
\end{chapelexample}


\subsection{Module Initialization Order}
\label{Module_Initialization_Order}
\index{modules!initialization order}

Module initialization is performed using the following algorithm.

Starting from the module that defines the main function, the modules named in
its use statements are visited depth-first and initialized in post-order.  If a
use statement names a module that has already been visited, it is not visited a
second time.  Thus, infinite recursion is avoided.

Modules used by a given module are visited in the order in which
they appear in the program text.  For nested modules, the
parent module and its uses are initialized before the nested module and its uses.

\begin{chapelexample}{init-order.chpl}
The code
\begin{chapel}
module M1 {
  use M2.M3;
  use M2;
  writeln("In M1's initializer");
  proc main() {
    writeln("In main");
  }
}

module M2 {
  use M4;
  writeln("In M2's initializer");
  module M3 {
    writeln("In M3's initializer");
  }
}

module M4 {
  writeln("In M4's initializer");
}
\end{chapel}
prints the following
\begin{chapelprintoutput}{}
In M4's initializer
In M2's initializer
In M3's initializer
In M1's initializer
In main
\end{chapelprintoutput}
M1, the main module, uses M2.M3 and then M2, thus M2.M3 must be
initialized.  Because M2.M3 is a nested module, M4 (which is used by
M2) must be initialized first.  M2 itself is initialized, followed by
M2.M3.  Finally M1 is initialized, and the main function is run.
\end{chapelexample}

\cleardoublepage
\sekshun{Procedures}
\label{Functions}
\index{functions}
\index{functions!call site}
\index{call site}
\index{functions!formal arguments}
\index{formal arguments}
\index{functions!actual arguments}
\index{actual arguments}

A \emph{function} is a code abstraction that can be invoked by a call
expression. Throughout this specification the term ``function''
is used in this programming-languages sense, rather than
in the mathematical sense.
% TODO: the above is here until we substitute a different word for 'function'
A function has zero or more \emph{formal arguments}, or simply
\emph{formals}. Upon a function call each formal is associated
with the corresponding \emph{actual argument}, or simply
\emph{actual}. Actual arguments are provided as part of the call
expression, or at the the \emph{call site}.
% TODO - move into a footnote?
Direct and indirect recursion is supported.

\index{functions!procedure}
\index{procedure}
\index{functions!operator}
\index{functions!method}
\index{operators!procedure}
A function can be a \emph{procedure}, which completes and returns to
the call site exactly once, returning no result, a single result, or
multiple results aggregated in a tuple. A function can also be an
iterator, which can generate, or \emph{yield}, multiple results (in
sequence and/or in parallel). A function (either a procedure or an
iterator) can be a \emph{method} if it is bound to a type (often a
class). An \emph{operator} in this chapter is a procedure
with a special name, which can be invoked using infix notation,
i.e., via a unary or binary expression.
This chapter defines procedures, but most of its contents
apply to iterators and methods as well.

Functions are presented as follows:
\begin{itemize}
\item procedures (this chapter)
\item operators \rsec{Function_Definitions}, \rsec{Binary_Expressions}
\item iterators \rsec{Iterators}
\item methods (when bound to a class) \rsec{Class_Methods}
\item function calls \rsec{Function_Calls}
\item various aspects of defining a procedure
      \rsec{Function_Definitions}--\rsec{Nested_Functions}
\item calling external functions from Chapel
      \rsec{Calling_External_Functions}
\item calling Chapel functions from external functions\rsec{Calling_Chapel_Functions}
\item determining the function to invoke for a given call site:
      function and operator overloading \rsec{Function_Overloading},
      function resolution \rsec{Function_Resolution}
\end{itemize}


\section{Function Calls}
\label{Function_Calls}
\index{function calls}
\index{calls!function}

The syntax to call a non-method function is given by:
\begin{syntax}
call-expression:
  lvalue-expression ( named-expression-list )
  lvalue-expression [ named-expression-list ]
  parenthesesless-function-identifier

named-expression-list:
  named-expression
  named-expression , named-expression-list

named-expression:
  expression
  identifier = expression

parenthesesless-function-identifier:
  identifier
\end{syntax}


A \sntx{call-expression} is resolved to a particular function
according to the algorithm for function resolution described
in~\rsec{Function_Resolution}.

Functions can be called using either parentheses or brackets.

\begin{rationale}
This provides an opportunity to blur the distinction between
an array access and a function call and thereby exploit a
possible space/time tradeoff.
\end{rationale}

Functions that are defined without parentheses must be called without
parentheses as defined by scope resolution.  Functions without
parentheses are discussed in~\rsec{Functions_without_Parentheses}.

A \sntx{named-expression} is an expression that may be optionally
named.  It provides
\index{actual arguments}
\index{functions!actual arguments}
an actual argument to the function being called.
The optional \sntx{identifier} refers to a named formal
argument described in~\rsec{Named_Arguments}.

Calls to methods are defined in Section~\rsec{Class_Method_Calls}.


\section{Procedure Definitions}
\label{Function_Definitions}
\index{functions!procedure definition}
\index{procedures!definition}

\index{proc@\chpl{proc}}
Procedures are defined with the following syntax:
\begin{syntax}
procedure-declaration-statement:
  privacy-specifier[OPT] procedure-kind[OPT] `proc' function-name argument-list[OPT] return-intent[OPT] return-type[OPT] where-clause[OPT]
    function-body

procedure-kind:
  `inline'
  `export'
  `extern'
  `override'

function-name:
  identifier
  operator-name

operator-name: one of
  + $ $ $ $ - $ $ $ $ * $ $ $ $ / $ $ $ $ % $ $ $ $ ** $ $ $ $ ! $ $ $ $ == $ $ $ $ != $ $ $ $ <= $ $ $ $ >= $ $ $ $ < $ $ $ $ > $ $ $ $ << $ $ $ $ >> $ $ $ $ & $ $ $ $ | $ $ $ $ ^ $ $ $ $ ~
  = $ $ $ $ += $ $ $ $ -= $ $ $ $ *= $ $ $ $ /= $ $ $ $ %= $ $ $ $ **= $ $ $ $ &= $ $ $ $ |= $ $ $ $ ^= $ $ $ $ <<= $ $ $ $ >>= $ $ $ $ <=> $ $ $ $ <~>

argument-list:
  ( formals[OPT] )

formals:
  formal
  formal , formals

formal:
  formal-intent[OPT] identifier formal-type[OPT] default-expression[OPT]
  formal-intent[OPT] identifier formal-type[OPT] variable-argument-expression
  formal-intent[OPT] tuple-grouped-identifier-list formal-type[OPT] default-expression[OPT]
  formal-intent[OPT] tuple-grouped-identifier-list formal-type[OPT] variable-argument-expression

formal-type:
  : type-expression
  : ? identifier[OPT]

default-expression:
  = expression

variable-argument-expression:
  ... expression
  ... ? identifier[OPT]
  ...

formal-intent:
  `const'
  `const in'
  `const ref'
  `in'
  `out'
  `inout'
  `ref'
  `param'
  `type'

return-intent:
  `const'
  `const ref'
  `ref'
  `param'
  `type'

return-type:
  : type-expression

where-clause:
  `where' expression

function-body:
  block-statement
  return-statement
\end{syntax}

%% This should be in the order that the sections appear in this
%% chapter (if they appear in this chapter).

Functions do not require parentheses if they have no arguments.  Such
functions are described in~\rsec{Functions_without_Parentheses}.

Formal arguments can be grouped together using a tuple notation as
described in~\rsec{Formal_Argument_Declarations_in_a_Tuple}.

Default expressions allow for the omission of actual arguments at the
call site, resulting in the implicit passing of a default value.
Default values are discussed in~\rsec{Default_Values}.

The intents \chpl{const}, \chpl{const in}, \chpl{const
ref}, \chpl{in}, \chpl{out}, \chpl{inout} and \chpl{ref} are discussed
in~\rsec{Argument_Intents}.  The intents \chpl{param} and \chpl{type} make a
function generic and are discussed in~\rsec{Generic_Functions}.  If
the formal argument's type is omitted, generic, or prefixed with a
question mark, the function is also generic and is discussed
in~\rsec{Generic_Functions}.

Functions can take a variable number of arguments.  Such functions are
discussed in~\rsec{Variable_Length_Argument_Lists}.

The \sntx{return-intent} can be used to indicate how the value is returned from
a function.  \sntx{return-intent} is described further in \rsec{Return_Intent}.
% TODO: define lvalue and rvalue

\begin{openissue}
Parameter and type procedures are supported. Parameter and type
iterators are currently not supported.
\end{openissue}

The \sntx{return-type} is optional and is discussed in~\rsec{Return_Types}.
A type function may not specify a return type.

The \sntx{where-clause} is optional and is discussed
in~\rsec{Where_Clauses}.

Function and operator overloading is supported in Chapel and is
discussed in~\rsec{Function_Overloading}.
Operator overloading is supported on the operators listed
above (see \sntx{operator-name}).

The optional \sntx{privacy-specifier} keywords indicate the visibility
of module level procedures to outside modules.  By default, procedures are
publicly visible.  More details on visibility can be found in
~\rsec{Visibility_Of_Symbols}.

The linkage specifier \chpl{inline} indicates that the function body must
be inlined at every call site.

\begin{rationale}
A Chapel compiler is permitted to inline any function if it determines
there is likely to be a performance benefit to do so.  Hence an error
must be reported if the compiler is unable to inline a procedure with
this specifier.  One example of a preventable inlining error is to
define a sequence of inlined calls that includes a cycle back to an
inlined procedure.
\end{rationale}

See the chapter on interoperability (\rsec{Interoperability})
for details on exported and imported functions.

\section{Functions without Parentheses}
\label{Functions_without_Parentheses}
\index{functions!without parentheses}

Functions do not require parentheses if they have empty argument
lists.  Functions declared without parentheses around empty argument
lists must be called without parentheses.

\begin{chapelexample}{function-no-parens.chpl}
Given the definitions
\begin{chapel}
proc foo { writeln("In foo"); }
proc bar() { writeln("In bar"); }
\end{chapel}
\begin{chapelpost}
foo;
bar();
\end{chapelpost}
\begin{chapeloutput}
In foo
In bar
\end{chapeloutput}
the procedure \chpl{foo} can be called by writing \chpl{foo} and the
procedure \chpl{bar} can be called by writing \chpl{bar()}.  It is an
error to use parentheses when calling \chpl{foo} or omit them
when calling \chpl{bar}.
\end{chapelexample}


\section{Formal Arguments}
\label{Formal_Arguments}
\index{formal arguments}
\index{functions!formal arguments}
\index{functions!arguments!formal}

A formal argument's intent~(\rsec{Argument_Intents}) specifies how the
actual argument is passed to the function.  If no intent is specified,
the default intent~(\rsec{The_Default_Intent}) is applied, resulting in
type-dependent behavior.

\subsection{Named Arguments}
\label{Named_Arguments}
\index{named arguments}
\index{functions!named arguments}
\index{functions!arguments!named}
\index{formal arguments!naming}

A formal argument can be named at the call site to explicitly map an
actual argument to a formal argument.

\begin{chapelexample}{named-args.chpl}
Running the code
\begin{chapel}
proc foo(x: int, y: int) { writeln(x); writeln(y); }

foo(x=2, y=3);
foo(y=3, x=2);
\end{chapel}
will produce the output
\begin{chapelprintoutput}{}
2
3
2
3
\end{chapelprintoutput}
named argument passing is used to map the actual arguments to the
formal arguments.  The two function calls are equivalent.
\end{chapelexample}

Named arguments are sometimes necessary to disambiguate calls or
ignore arguments with default values.  For a function that has many
arguments, it is sometimes good practice to name the arguments at the
call site for compiler-checked documentation.

\subsection{Default Values}
\label{Default_Values}
\index{default values}
\index{functions!default argument values}
\index{functions!arguments!defaults}
\index{formal arguments!defaults}

Default values can be specified for a formal argument by appending the
assignment operator and a default expression to the declaration of the
formal argument.  If the actual argument is omitted from the function
call, the default expression is evaluated when the function call is
made and the evaluated result is passed to the formal argument as if
it were passed from the call site. Note though that the default value
is evaluated in the same scope as the called function. Default value
expressions can refer to previous formal arguments or to variables
that are visible to the scope of the function definition.

\begin{chapelexample}{default-values.chpl}
The code
\begin{chapel}
proc foo(x: int = 5, y: int = 7) { writeln(x); writeln(y); }

foo();
foo(7);
foo(y=5);
\end{chapel}
writes out
\begin{chapelprintoutput}{}
5
7
7
7
5
5
\end{chapelprintoutput}
Default values are specified for the formal arguments \chpl{x}
and \chpl{y}.  The three calls to \chpl{foo} are equivalent to the
following three calls where the actual arguments are
explicit: \chpl{foo(5, 7)}, \chpl{foo(7, 7)}, and \chpl{foo(5, 5)}.
The example \chpl{foo(y=5)} shows how to use a named argument
for \chpl{y} in order to use the default value for \chpl{x} in the
case when \chpl{x} appears earlier than \chpl{y} in the formal
argument list.
\end{chapelexample}


\section{Argument Intents}
\label{Argument_Intents}
\index{intents}
\index{argument!intents}
\index{functions!arguments!intents}

Argument intents specify how an actual argument is passed to a
function where it is represented by the corresponding formal argument.

Argument intents are categorized as being either \emph{concrete}
or \emph{abstract}.  Concrete intents are those in which the semantics
of the intent keyword are independent of the argument's type.
Abstract intents are those in which the keyword (or lack thereof)
expresses a general intention that will ultimately be implemented via
one of the concrete intents.  The specific choice of concrete intent
depends on the argument's type and may be implementation-defined.
Abstract intents are provided to support productivity and code reuse.

\subsection{Concrete Intents}
\label{Concrete Intents}
\index{intents!concrete}

The concrete intents are \chpl{in}, \chpl{out}, \chpl{inout},
\chpl{ref}, \chpl{const in}, and \chpl{const ref}.

\subsubsection{The In Intent}
\label{The_In_Intent}
\index{in (intent)@\chpl{in} (intent)}
\index{intents!in@\chpl{in}}

When \chpl{in} is specified as the intent, the formal argument
represents a variable that is copy-initialized with the value of the
actual argument. For example, for integer arguments, the formal
argument will store a copy of the actual argument.
An implicit conversion occurs from the actual argument
to the type of the formal.  The
formal can be modified within the function, but such changes are local
to the function and not reflected back to the call site.


\subsubsection{The Out Intent}
\label{The_Out_Intent}
\index{out (intent)@\chpl{out} (intent)}
\index{intents!out@\chpl{out}}

When \chpl{out} is specified as the intent, the actual argument is
ignored when the call is made, but when the function returns, the
formal argument is copied back to the actual argument.
An implicit conversion occurs from the type of the formal
to the type of the actual.  The actual argument
must be a valid lvalue.  The formal
argument is initialized to its default value if one is supplied, or to
its type's default value otherwise.  The formal argument can be
modified within the function.


\subsubsection{The Inout Intent}
\label{The_Inout_Intent}
\index{inout (intent)@\chpl{inout} (intent)}
\index{intents!inout@\chpl{inout}}

When \chpl{inout} is specified as the intent, the actual argument is
copied into the formal argument as with the \chpl{in} intent and then
copied back out as with the \chpl{out} intent.  The actual argument
must be a valid lvalue.  The formal argument can be modified within
the function.
The type of the actual argument must be the same
as the type of the formal.


\subsubsection{The Ref Intent}
\label{The_Ref_Intent}
\index{ref (intent)@\chpl{ref} (intent)}
\index{intents!ref@\chpl{ref}}

When \chpl{ref} is specified as the intent, the actual argument is
passed by reference.  Any reads of, or modifications to, the formal
argument are performed directly on the corresponding actual argument
at the call site.  The actual argument must be a valid lvalue.
The type of the actual argument must be the same
as the type of the formal.

The \chpl{ref} intent differs from the \chpl{inout} intent in that
the \chpl{inout} intent requires copying from/to the actual argument on
the way in/out of the function, while \chpl{ref} allows direct
access to the actual argument through the formal argument without
copies.  Note that concurrent modifications to the \chpl{ref} actual argument by
other tasks may be visible within the function, subject to the memory
consistency model.


\subsubsection{The Const In Intent}
\label{The_Const_In_Intent}
\index{const in (intent)@\chpl{const in} (intent)}
\index{intents!const in@\chpl{const in}}

The \chpl{const in} intent is identical to the \chpl{in} intent,
except that modifications to the formal argument are prohibited within
the function.


\subsubsection{The Const Ref Intent}
\label{The_Const_Ref_Intent}
\index{const ref (intent)@\chpl{const ref} (intent)}
\index{intents!const ref@\chpl{const ref}}

The \chpl{const ref} intent is identical to the \chpl{ref} intent,
except that modifications to the formal argument are prohibited within
the dynamic scope of the function.  Note that concurrent tasks may
modify the actual argument while the function is executing and that
these modifications may be visible to reads of the formal argument
within the function's dynamic scope (subject to the memory consistency
model).

\subsubsection{Summary of Concrete Intents}
\label{Summary_of_Concrete_Intents}

The following table summarizes the differences between the concrete
intents:

\begin{center}
\begin{tabular}[c]{|l|c|c|c|c|c|c|}
\hline
                                       & \chpl{in} & \chpl{out} & \chpl{inout} & \chpl{ref} & \chpl{const in} & \chpl{const ref} \\
\hline
\hline
copied in on function call?      & yes & no  & yes & no  & yes & no  \\
copied out on function return?   & no  & yes & yes & no  & no  & no  \\
refers to actual argument?       & no  & no  & no  & yes & no  & yes \\
formal can be read?              & yes & yes & yes & yes & yes & yes \\
formal can be modified?          & yes & yes & yes & yes & no  & no  \\
local changes affect the actual? & no  & on return & on return & immediately & N/A & N/A \\
\hline
\end{tabular}
\end{center}


\subsection{Abstract Intents}
\label{Abstract_Intents}
\index{intents!abstract}

The abstract intents are \chpl{const} and the \emph{default intent}
(when no intent is specified).


\subsubsection{Abstract Intents Table}
\label{Abstract_Intents_Table}

The following table summarizes what these abstract intents mean for each type:

\begin{center}
\begin{tabular}[c]{|l|l|l|l|}
\hline
                 & meaning of          &  meaning of        & \\
  type           & \chpl{const} intent &  default intent    &  notes \\
\hline
\hline
  \chpl{bool}    & \chpl{const in}     & \chpl{const in}  & \\
  \chpl{int}     & \chpl{const in}     & \chpl{const in}  & \\
  \chpl{uint}    & \chpl{const in}     & \chpl{const in}  & \\
  \chpl{real}    & \chpl{const in}     & \chpl{const in}  & \\
  \chpl{imag}    & \chpl{const in}     & \chpl{const in}  & \\
  \chpl{complex} & \chpl{const in}     & \chpl{const in}  & \\
  \chpl{range}   & \chpl{const in}     & \chpl{const in}  & \\
\hline
  \chpl{owned class}     & \chpl{const in}     & \chpl{in}
   & see \hyperref[Default_Intent_for_owned_and_shared]{"Default Intent for owned and shared"} \\
  \chpl{shared class}    & \chpl{const in}     & \chpl{in}
   & see \hyperref[Default_Intent_for_owned_and_shared]{"Default Intent for owned and shared"} \\
  \chpl{borrowed class}  & \chpl{const in}     & \chpl{const in} & \\
  \chpl{unmanaged class} & \chpl{const in}     & \chpl{const in} & \\
\hline
  \chpl{atomic}  & \chpl{const ref}    & \chpl{ref} & \\
  \chpl{single}  & \chpl{const ref}    & \chpl{ref} & \\
  \chpl{sync}    & \chpl{const ref}    & \chpl{ref} & \\
\hline
  \chpl{string}  & \chpl{const ref}    & \chpl{const ref} & \\
  \chpl{record}  & \chpl{const ref}    & \chpl{const ref}
   & see \hyperref[Default_Intent_for_Arrays_and_Record_this]{"Default Intent for Arrays and Record this"} \\
  \chpl{union}   & \chpl{const ref}    & \chpl{const ref} & \\
  \chpl{dmap}    & \chpl{const ref}    & \chpl{const ref} & \\
  \chpl{domain}  & \chpl{const ref}    & \chpl{const ref} & \\
  array          & \chpl{const ref}    & \chpl{ref} / \chpl{const ref}
   & see \hyperref[Default_Intent_for_Arrays_and_Record_this]{"Default Intent for Arrays and Record this"} \\
\hline
\end{tabular}
\end{center}


\subsubsection{The Const Intent}
\label{The_Const_Intent}
\index{intents!const@\chpl{const}}

The \chpl{const} intent specifies the intention that the function will
not and cannot modify the formal argument within its dynamic scope.
Whether the actual argument will be passed by \chpl{const in} or
\chpl{const ref} intent depends on its type.  In general, small values,
such as scalar types, will be passed by \chpl{const in}; while larger
values, such as domains and arrays, will be passed by \chpl{const ref}
intent.  The \hyperref[Abstract_Intents_Table]{table} earlier in this
sub-section lists the meaning of the const intent for each type.


\subsubsection{The Default Intent}
\label{The_Default_Intent}
\index{intents!default}

When no intent is specified for a formal argument, the \emph{default
intent} is applied.  It is designed to take the most natural/least
surprising action for the argument, based on its type. The
\hyperref[Abstract_Intents_Table]{table}
earlier in this sub-section lists the meaning of the default
intent for each type.

Default argument passing for tuples generally matches the default
argument passing strategy that would be applied if each tuple element was
passed as a separate argument.

\begin{openissue}
How tuples should be handled under default intents is an open issue;
particularly for heterogeneous tuples whose components would fall into
separate categories in the table above.  One proposed approach is to
apply the default intent to each component of the tuple independently.
\end{openissue}

\subsubsection{Default Intent for Arrays and Record 'this'}
\label{Default_Intent_for_Arrays_and_Record_this}
\index{intents!array default}
\index{intents!record \chpl{this} default}

The default intent for arrays and for a \chpl{this} argument of record
type~(see \rsec{Method_receiver_and_this}) is \chpl{ref} or \chpl{const
ref}. It is \chpl{ref} if the formal argument is modified inside the
function, otherwise it is \chpl{const ref}.  Note that neither of these
cause an array or record to be copied by default.  The choice between
\chpl{ref} and \chpl{const ref} is similar to and interacts with return
intent overloads (see \rsec{Return_Intent_Overloads}).

\subsubsection{Default Intent for 'owned' and 'shared'}
\label{Default_Intent_for_owned_and_shared}
\index{intents!default for owned}
\index{intents!default for shared}

The default intent for \chpl{owned} and \chpl{shared} arguments is
generally \chpl{in}. There is an exception for generic formal arguments
that use default or \chpl{const} intent and have no type expression.
When a such a generic formal is instantiated for an \chpl{owned} or
\chpl{shared} actual argument, the instantiated formal will have the
relevant \chpl{borrowed} type. This exception does not apply to \chpl{type}
arguments or to the default initializer generated for a class type.

\begin{chapelexample}{owned-any-intent.chpl}
\begin{chapel}
proc defaultGeneric(arg) {
  writeln(arg.type:string);
}
class SomeClass { }
defaultGeneric(new owned SomeClass());
\end{chapel}
\begin{chapeloutput}
SomeClass
\end{chapeloutput}
Here, the program will output the type \chpl{borrowed SomeClass} rather
than \chpl{owned SomeClass}. It would output \chpl{owned SomeClass} if
the argument were declared with an \chpl{in} intent (e.g. \chpl{in arg}) or if
the argument included a type specifier (e.g. \chpl{arg: owned}).
\end{chapelexample}

\begin{openissue}
The reason for this rule is to make it clearer when ownership transfer will
happen. In particular, with this rule, ownership transfer can only occur
for arguments marked with an \chpl{in} intent or with an \chpl{owned} or
\chpl{shared} argument type.

This rule is likely change based upon further experience in using it.
\end{openissue}


\section{Variable Number of Arguments}
\label{Variable_Length_Argument_Lists}
\index{functions!variable number of arguments}
\index{functions!varargs}

Functions can be defined to take a variable number of arguments where
those arguments can have any intent or can be types.  A variable
number of parameters is not supported.  This allows the call site to
pass a different number of actual arguments.  There must be at least
one actual argument.

If the variable argument expression contains an identifier prepended by a
question mark, the number of actual arguments can vary, and the identifier
will be bound to an integer parameter value indicating the number of
arguments at a given call site.
If the variable argument expression contains an expression without
a question mark, that expression must evaluate to an integer parameter value
requiring the call site to pass that number of arguments to the
function.

Within the function, the formal argument that is marked with a
variable argument expression is a tuple of the actual
arguments.

\begin{chapelexample}{varargs.chpl}
The code
\begin{chapel}
proc mywriteln(x ...?k) {
  for param i in 1..k do
    writeln(x(i));
}
\end{chapel}
\begin{chapelpost}
mywriteln("hi", "there");
mywriteln(1, 2.0, 3, 4.0);
\end{chapelpost}
\begin{chapeloutput}
hi
there
1
2.0
3
4.0
\end{chapeloutput}
defines a generic procedure called \chpl{mywriteln} that takes a
variable number of arguments of any type and then writes them out on
separate lines.  The parameter for-loop~(\rsec{Parameter_For_Loops})
is unrolled by the compiler so that \chpl{i} is a parameter, rather
than a variable.  This needs to be a parameter for-loop because the
expression \chpl{x(i)} will have a different type on each iteration.
The type of \chpl{x} can be specified in the formal argument list to
ensure that the actuals all have the same type.
\end{chapelexample}

\begin{chapelexample}{varargs-with-type.chpl}
Either or both the number of variable arguments and their types can be
specified.  For example, a basic procedure to sum the values of three
integers can be written as
\begin{chapel}
proc sum(x: int...3) return x(1) + x(2) + x(3);
\end{chapel}
\begin{chapelpost}
writeln(sum(1, 2, 3));
writeln(sum(-1, -2, -3));
\end{chapelpost}
\begin{chapeloutput}
6
-6
\end{chapeloutput}
Specifying the type is useful if it is important that each argument
have the same type.  Specifying the number is useful in, for example,
defining a method on a class that is instantiated over a rank
parameter.
\end{chapelexample}

\begin{chapelexample}{varargs-returns-tuples.chpl}
The code
\begin{chapel}
proc tuple(x ...) return x;
\end{chapel}
\begin{chapelpost}
writeln(tuple(1));
writeln(tuple("hi", "there"));
writeln(tuple(tuple(1, 2), tuple(3, 4)));
\end{chapelpost}
\begin{chapeloutput}
(1)
(hi, there)
((1, 2), (3, 4))
\end{chapeloutput}
defines a generic procedure that is equivalent to building a tuple.
Therefore the expressions \chpl{tuple(1, 2)} and \chpl{(1,2)} are equivalent,
as are the expressions \chpl{tuple(1)} and \chpl{(1,)}.
\end{chapelexample}


\section{Return Intents}
\label{Return_Intent}
\index{functions!return intent}
\index{statements!return@\chpl{return}!return intent}

The \sntx{return-intent} specifies how the value is returned from a function,
and in what contexts that function is allowed to be used.  By default, or if
the \sntx{return-intent} is \chpl{const}, the function returns a value that
cannot be used as an lvalue.

% TODO: it's a little jarring that the above talks about returning
% a value when e.g. a type function does not return a value at all
% (since it returns a type).
% It would probably be better to consider type and param return
% functions "type functions" or "param functions" and leave this
% section to only discuss const/ref/const ref return intents.

\subsection{The Ref Return Intent}
\label{Ref_Return_Intent}
\index{functions!ref keyword and@\chpl{ref} keyword and}
\index{ref (return intent)@\chpl{ref} (return intent)}
\index{intents!ref return@\chpl{ref} return}
\index{functions!lvalues}

When using a \chpl{ref} return intent, the function call is an lvalue
(specifically, a call expression for a procedure and an iterator variable for
an iterator).

The \chpl{ref} return intent is specified by following the argument list with
the \chpl{ref} keyword.  The function must return or yield an lvalue.

\begin{chapelexample}{ref-return-intent.chpl}
The following code defines a procedure that can be interpreted as a
simple two-element array where the elements are actually module
level variables:
\begin{chapel}
var x, y = 0;

proc A(i: int) ref {
  if i < 0 || i > 1 then
    halt("array access out of bounds");
  if i == 0 then
    return x;
  else
    return y;
}
\end{chapel}
Calls to this procedure can be assigned to in order to write to the ``elements''
of the array as in
\begin{chapel}
A(0) = 1;
A(1) = 2;
\end{chapel}
It can be called as an expression to access the ``elements'' as in
\begin{chapel}
writeln(A(0) + A(1));
\end{chapel}
This code outputs the number \chpl{3}.

\begin{chapeloutput}
3
\end{chapeloutput}
\end{chapelexample}


\subsection{The Const Ref Return Intent}
\label{Const_Ref_Return_Intent}
\index{functions!const ref keyword and@\chpl{const ref} keyword and}
\index{const ref (return intent)@\chpl{const ref} (return intent)}
\index{intents!const ref return@\chpl{const ref} return}

The \chpl{const ref} return intent is also available. It is a restricted
form of the \chpl{ref} return intent. Calls to functions marked with
the \chpl{const ref} return intent are not lvalue expressions.

\subsection{Return Intent Overloads}
\label{Return_Intent_Overloads}
\index{functions!return intent overloads}

In some situations, it is useful to choose the function called based upon
how the returned value is used.  In particular, suppose that there are
two functions that have the same formal arguments and differ only in
their return intent. One might expect such a situation to result in an
error indicating that it is ambiguous which function is called. However,
the Chapel language includes a special rule for determining which
function to call when the candidate functions are otherwise ambiguous
except for their return intent.  This rule enables data structures such
as sparse arrays.

See \ref{Choosing_Return_Intent_Overload} for a detailed description of
how return intent overloads are chosen based upon calling context.

\begin{chapelexample}{ref-return-intent-pair.chpl}

\index{functions!return intent overloads}
Return intent overload can be used to ensure, for
example, that the second element in the pseudo-array is only assigned
a value if the first argument is positive.  The following is an
example:
\begin{chapel}
var x, y = 0;

proc doA(param setter, i: int) ref {
  if i < 0 || i > 1 then
    halt("array access out of bounds");

  if setter && i == 1 && x <= 0 then
    halt("cannot assign value to A(1) if A(0) <= 0");

  if i == 0 then
    return x;
  else
    return y;
}
proc A(i: int) ref {
  return doA(true, i);
}
proc A(i: int) {
  return doA(false, i);
}

A(0) = 0;
A(1) = 1;

\end{chapel}
\begin{chapeloutput}
ref-return-intent-pair.chpl:8: error: halt reached - cannot assign value to A(1) if A(0) <= 0
\end{chapeloutput}
\end{chapelexample}


\subsection{The Param Return Intent}
\label{Param_Return_Intent}
\index{functions!as parameters}
\index{functions!parameter function}
\index{parameter function}

A \emph{parameter function}, or a \emph{param function}, is a function that
returns a parameter expression.  It is specified by following the function's
argument list by the keyword \chpl{param}.  It is often, but not necessarily,
generic.

It is a compile-time error if a parameter function does not return a
parameter expression.  The result of a parameter function is computed
during compilation and substituted for the call expression.

\begin{chapelexample}{param-functions.chpl}
In the code
\begin{chapel}
proc sumOfSquares(param a: int, param b: int) param
  return a**2 + b**2;

var x: sumOfSquares(2, 3)*int;
\end{chapel}
\begin{chapelpost}
writeln(x);
\end{chapelpost}
\begin{chapeloutput}
(0, 0, 0, 0, 0, 0, 0, 0, 0, 0, 0, 0, 0)
\end{chapeloutput}
\chpl{sumOfSquares} is a parameter procedure that takes
two parameters as arguments.  Calls to this procedure can be used in
places where a parameter expression is required.  In this example, the
call is used in the declaration of a homogeneous tuple and so is
required to be a parameter.
\end{chapelexample}

Parameter functions may not contain control flow that is not resolved
at compile-time.  This includes loops other than the parameter for
loop~\rsec{Parameter_For_Loops} and conditionals with a conditional
expressions that is not a parameter.


\subsection{The Type Return Intent}
\label{Type_Return_Intent}
\index{functions!as types}
\index{functions!type functions}

A \emph{type function} is a function that returns a type, not a value.  It is
specified by following the function's argument list by the keyword \chpl{type},
without the subsequent return type.  It is often, but not necessarily, generic.

It is a compile-time error if a type function does not return a type.
The result of a type function is computed during compilation.

As with parameter functions, type functions may not contain control
flow that is not resolved at compile-time.  This includes loops other
than the parameter for loop~\rsec{Parameter_For_Loops} and
conditionals with a conditional expression that is not a parameter.

\begin{chapelexample}{type-functions.chpl}
In the code
\begin{chapel}
proc myType(x) type {
  if numBits(x.type) <= 32 then return int(32);
  else return int(64);
}
\end{chapel}
\begin{chapelpost}
var a = 4: int(32),
    b = 4.0;
var at: myType(a),
    bt: myType(b);
writeln((numBits(at.type), numBits(bt.type)));
\end{chapelpost}
\begin{chapeloutput}
(32, 64)
\end{chapeloutput}
\chpl{myType} is a type procedure that takes a single
argument \chpl{x} and returns \chpl{int(32)} if the number of bits used to
represent \chpl{x} is less than or equal to 32, otherwise it
returns \chpl{int(64)}.  \chpl{numBits} is a param
procedure defined in the standard Types module.
\end{chapelexample}


\section{The Return Statement}
\label{The_Return_Statement}
\index{return@\chpl{return} (see also statements, return)}
\index{statements!return@\chpl{return}}

The return statement can only appear in a function.  It causes control
to exit that function, returning it to the point at which that function
was called.

A procedure can return a value by executing a return statement
that includes an expression. If it does, that expression's value
becomes the value of the invoking call expression.

A return statement in a procedure of a non-\chpl{void} return type
(\rsec{Return_Types}) must include an expression.
A return statement in a procedure of a \chpl{void} return type
or in an iterator must not include an expression.
A return statement of a variable procedure must contain an lvalue expression.

% TODO: currently our implementation does not require that a user return
% statement be executed for a procedure of a non-void return type.
% If execution falls through to the end, the default value of the
% return type is returned.
% Do we want to formalize that or to disallow that?

The syntax of the return statement is given by
\begin{syntax}
return-statement:
  `return' expression[OPT] ;
\end{syntax}

\begin{chapelexample}{return.chpl}
The following code defines a procedure that returns the sum of three
integers:
\begin{chapel}
proc sum(i1: int, i2: int, i3: int)
  return i1 + i2 + i3;
\end{chapel}
\begin{chapelpost}
writeln(sum(1, 2, 3));
\end{chapelpost}
\begin{chapeloutput}
6
\end{chapeloutput}
\end{chapelexample}


\section{Return Types}
\label{Return_Types}
\index{statements!return@\chpl{return}!return type}
\index{functions!return types}

Every procedure has a return type. The return type is either
specified explicitly via \sntx{return-type} in the procedure
declaration, or is inferred implicitly.

% TODO: both subsections below state that the return type
% must match the type of each returned expression for variable procedures
% and that an implicit conversion occurs for non-variable procedures.
% Factor that out up here.

\subsection{Explicit Return Types}
\label{Explicit_Return_Types}
\index{explicit return type}
\index{functions!return types!explicit}

If a return type is specified and is not \chpl{void},
each return statement of the procedure must include an expression.
For a non-\chpl{ref} return intent, an implicit conversion occurs
from each return expression to the specified return type.
For a \chpl{ref} return intent~(\rsec{Ref_Return_Intent}), the return
type must match the type returned in all of the return statements
exactly, when checked after generic instantiation and parameter folding
(if applicable).

\subsection{Implicit Return Types}
\label{Implicit_Return_Types}
\index{type inference!of return types}
\index{implicit return type}
\index{functions!return types!implicit}

If a return type is not specified, it is inferred from the return statements.
It is illegal for a procedure to have a return statement with an expression
and a return statement without an expression.
For procedures without any return statements, or when none of the
return statements include an expression, the return type is \chpl{void}.

Otherwise, the types of the expressions in all of the procedure's
return statements are considered.
If a function has a \chpl{ref} return intent (\rsec{Ref_Return_Intent}), they
all must be the same exact type, which becomes the inferred return type.
Otherwise, there must exist exactly one type such that an implicit conversion
is allowed between every other type and that type, and that type becomes the
inferred return type.
If the above requirements are not satisfied, it is an error.

\section{Where Clauses}
\label{Where_Clauses}
\index{where@\chpl{where}}
\index{functions!where@\chpl{where}}

The list of function candidates can be constrained by {\em where clauses}.  A
where clause is specified in the definition of a
function~(\rsec{Function_Definitions}).  The expression in the where clause
must be a boolean parameter expression that evaluates to either \chpl{true} or
\chpl{false}. If it evaluates to \chpl{false}, the function is rejected and
thus is not a possible candidate for function resolution.

\begin{chapelexample}{whereClause.chpl}
Given two overloaded function definitions
\begin{chapel}
proc foo(x) where x.type == int { writeln("int"); }
proc foo(x) where x.type == real { writeln("real"); }
\end{chapel}
\begin{chapelpost}
foo(3);
foo(3.14);
\end{chapelpost}
\begin{chapeloutput}
int
real
\end{chapeloutput}
the call foo(3) resolves to the first definition because the where clause on
the second function evaluates to false.
\end{chapelexample}

\section{Nested Functions}
\label{Nested_Functions}
\index{functions!nested}
\index{nested function}

A function defined in another function is called a nested function.
Nesting of functions may be done to arbitrary degrees, i.e., a
function can be nested in a nested function.

Nested functions are only visible to function calls within the lexical scope
in which they are defined.

Nested functions may refer to variables defined in the function(s) in
which they are nested.


\section{Function and Operator Overloading}
\label{Function_Overloading}
\index{overloading}
\index{overloading functions (see also functions, overloading)}
\index{overloading operators (see also operators, overloading)}
\index{functions!overloading}
\index{operators!overloading}

Functions that have the same name but different argument lists are
called overloaded functions.  Function calls to overloaded functions
are resolved according to the function resolution algorithm in~\rsec{Function_Resolution}.

Operator overloading is achieved by defining a function with a name
specified by that operator.  The operators that may be overloaded are
listed in the following table:

\begin{center}
\begin{tabular}{|l|l|}
\hline
{\bf arity} & {\bf operators} \\
\hline
unary & \verb@+ - ! ~@ \\
binary & \verb@+ - * / % ** == <= >= < > << >> & | ^ by@ \\
& \verb@= += -= *= /= %= **= &= |= ^= <<= >>= <=> <~>@ \\
\hline
\end{tabular}
\end{center}

The arity and precedence of the operator must be maintained when it is
overloaded.  Operator resolution follows the same algorithm as
function resolution.


\section{Function Resolution}
\label{Function_Resolution}
\index{functions!resolution}

\emph{Function resolution} is the algorithm that determines
which function to invoke for a given call expression.
Function resolution is defined as follows.
\begin{itemize}
\item
Identify the set of visible functions for the function call.  A
\emph{visible function} is any function that satisfies the criteria
in~\rsec{Determining_Visible_Functions}.  If no visible function can
be found, the compiler will issue an error stating that the call
cannot be resolved.
\item
From the set of visible functions for the function call, determine the
set of candidate functions for the function call.  A \emph{candidate
function} is any function that satisfies the criteria
in~\rsec{Determining_Candidate_Functions}.  If no candidate function
can be found, the compiler will issue an error stating that the call
cannot be resolved.  If exactly one candidate function is found, this
is determined to be the function.
\item
From the set of candidate functions, determine the set of most specific
functions. In most cases, there is one most specific function, but there
can be several if they differ only in return intent. The set of most
specific functions is the set of functions that are not \emph{more
specific} than each other but that are \emph{more specific} than every
other candidate function. The \emph{more specific} relationship is
defined in ~\rsec{Determining_More_Specific_Functions}.
\item
From the set of most specific functions, the compiler determines a best
function for each return intent as described in
~\rsec{Determining_Best_Functions}. If there is more than one
best function for a given return intent, the compiler will issue
an error stating that the call is ambiguous. Otherwise, it will choose
which function to call based on the calling context as described
in~\rsec{Choosing_Return_Intent_Overload}.
\end{itemize}

\subsection{Determining Visible Functions}
\label{Determining_Visible_Functions}
\index{functions!visible}

Given a function call, a function is determined to be a \emph{visible
function} if the name of the function is the same as the name of the
function call and the function is defined in the same scope as the
function call or a lexical outer scope of the function call, or if the
function is publicly declared in a module that is used from the same
scope as the function call or a lexical outer scope of the function
call.  Function visibility in generic functions is discussed
in~\rsec{Function_Visibility_in_Generic_Functions}.

\subsection{Determining Candidate Functions}
\label{Determining_Candidate_Functions}
\index{functions!candidates}

Given a function call, a function is determined to be
a \emph{candidate function} if there is a \emph{valid mapping} from
the function call to the function and each actual argument is mapped
to a formal argument that is a \emph{legal argument mapping}.

\subsubsection{Valid Mapping}
\label{Valid_Mapping}
\index{functions!resolution!valid mapping}

The following algorithm determines a valid mapping from a function
call to a function if one exists:
\begin{itemize}
\item
Each actual argument that is passed by name is matched to the formal
argument with that name.  If there is no formal argument with that
name, there is no valid mapping.
\item
The remaining actual arguments are mapped in order to the remaining
formal arguments in order.  If there are more actual arguments then
formal arguments, there is no valid mapping.  If any formal argument
that is not mapped to by an actual argument does not have a default
value, there is no valid mapping.
\item
The valid mapping is the mapping of actual arguments to formal
arguments plus default values to formal arguments that are not mapped
to by actual arguments.
\end{itemize}

\subsubsection{Legal Argument Mapping}
\label{Legal_Argument_Mapping}
\index{functions!resolution!legal argument mapping}

An actual argument of type $T_A$ can be mapped to a formal argument of
type $T_F$ if any of the following conditions hold:
\begin{itemize}
\item $T_A$ and $T_F$ are the same type.
\item There is an implicit conversion from $T_A$ to $T_F$.
\item $T_A$ is derived from $T_F$.
\item $T_A$ is scalar promotable to $T_F$.
\end{itemize}

\subsection{Determining More Specific Functions}
\label{Determining_More_Specific_Functions}
\index{functions!most specific}
\index{functions!resolution!most specific}

Given two functions $F_1$ and $F_2$, the more specific function is
determined by the first of the following steps that applies:

\begin{itemize}
\item If $F_1$ does not require promotion and $F_2$ does require promotion, then $F_1$ is more specific.
\item If $F_2$ does not require promotion and $F_1$ does require promotion, then $F_2$ is more specific.
\item
If at least one of the legal argument mappings to $F_1$ is a {\em more
specific argument mapping} than the corresponding legal argument
mapping to $F_2$ and none of the legal argument mappings to $F_2$ is a
more specific argument mapping than the corresponding legal argument
mapping to $F_1$, then $F_1$ is more specific.

\item
If at least one of the legal argument mappings to $F_2$ is a {\em more
specific argument mapping} than the corresponding legal argument
mapping to $F_1$ and none of the legal argument mappings to $F_1$ is a
more specific argument mapping than the corresponding legal argument
mapping to $F_2$, then $F_2$ is more specific.


\item If $F_1$ shadows $F_2$, then $F_1$ is more specific.
\item If $F_2$ shadows $F_1$, then $F_2$ is more specific.

\item If at least one of the legal argument mappings to $F_1$ is {\em
weak preferred} and none of the legal argument mappings to $F_2$ are {\em weak
preferred}, then $F_1$ is more specific.

\item If at least one of the legal argument mappings to $F_2$ is {\em
weak preferred} and none of the legal argument mappings to $F_1$ are {\em weak
preferred}, then $F_2$ is more specific.

\item If at least one of the legal argument mappings to $F_1$ is {\em
weaker preferred} and none of the legal argument mappings to $F_2$ are
{\em weaker preferred}, then $F_1$ is more specific.

\item If at least one of the legal argument mappings to $F_2$ is {\em
weaker preferred} and none of the legal argument mappings to $F_1$ are
{\em weaker preferred}, then $F_2$ is more specific.

\item If at least one of the legal argument mappings to $F_1$ is {\em
weakest preferred} and none of the legal argument mappings to $F_2$ are
{\em weakest preferred}, then $F_1$ is more specific.

\item If at least one of the legal argument mappings to $F_2$ is {\em
weakest preferred} and none of the legal argument mappings to $F_1$ are
{\em weakest preferred}, then $F_2$ is more specific.

\item Otherwise neither function is more specific.
\end{itemize}

Given an argument mapping, $M_1$, from an actual argument, $A$, of
type $T_A$ to a formal argument, $F1$, of type $T_{F1}$ and an
argument mapping, $M_2$, from the same actual argument to a formal
argument, $F2$, of type $T_{F2}$, the level of preference for one of
these argument mappings is determined by the first of the following steps
that applies:
\begin{itemize}
\item
 If $T_{F1}$ and $T_{F2}$ are the same type, $F1$ is an instantiated
 parameter, and $F2$ is not an instantiated parameter, $M_1$ is more
 specific.
\item
 If $T_{F1}$ and $T_{F2}$ are the same type, $F2$ is an instantiated
 parameter, and $F1$ is not an instantiated parameter, $M_2$ is more
 specific.
\item
 If $M_1$ does not require scalar promotion and $M_2$ requires scalar
 promotion, $M_1$ is more specific.
\item
 If $M_1$ requires scalar promotion and $M_2$ does not require scalar
 promotion, $M_2$ is more specific.
\item
 If $T_{F1}$ and $T_{F2}$ are the same type, $F1$ is generic, and $F2$
 is not generic, $M_1$ is more specific.
\item
 If $T_{F1}$ and $T_{F2}$ are the same type, $F2$ is generic, and $F1$
 is not generic, $M_2$ is more specific.
\item
 If $F1$ is not generic over all types and $F2$ is generic over all
 types, $M_1$ is more specific.
\item
 If $F1$ is generic over all types and $F2$ is not generic over all
 types, $M_2$ is more specific.
\item
 If $F1$ and $F2$ are both generic, and $F1$ is partially concrete but
 $F2$ is not, then $M_1$ is more specific.
\item
 If $F1$ and $F2$ are both generic, and $F2$ is partially concrete but
 $F1$ is not, then $M_2$ is more specific.
\item
  If $F1$ is a \chpl{param} argument but $F2$ is not, then $M_1$ is weak
  preferred.
\item
  If $F2$ is a \chpl{param} argument but $F1$ is not, then $M_2$ is weak
  preferred.
\item
  If $A$ is not a \chpl{param} argument with a default size and $F2$
  requires a narrowing conversion but $F1$ does not, then $M_1$ is weak
  preferred.
\item
  If $A$ is not a \chpl{param} argument with a default size and $F1$
  requires a narrowing conversion but $F2$ does not, then $M_2$ is weak
  preferred.

\item
 If $T_A$ and $T_{F1}$ are the same type and $T_A$ and $T_{F2}$ are
 not the same type, $M_1$ is more specific.
\item
 If $T_A$ and $T_{F1}$ are not the same type and $T_A$ and $T_{F2}$
 are the same type, $M_2$ is more specific.

\item
If $A$ uses a scalar promotion type equal to $T_{F1}$ but different
from $T_{F2}$, then $M_1$ will be preferred as follows:

\begin{itemize}
  \item if $A$ is a \chpl{param} argument with a default size, then $M_1$
    is weakest preferred
  \item if $A$ is a \chpl{param} argument with non-default size, then $M_1$
    is weaker preferred
  \item otherwise, $M_1$ is more specific
\end{itemize}

\item
If $A$ uses a scalar promotion type equal to $T_{F2}$ but different
from $T_{F1}$, then $M_2$ will be preferred as follows:

\begin{itemize}
  \item if $A$ is a \chpl{param} argument with a default size, then $M_2$
    is weakest preferred
  \item if $A$ is a \chpl{param} argument with non-default size, then $M_2$
    is weaker preferred
  \item otherwise, $M_2$ is more specific
\end{itemize}

\item
If $T_A$ or its scalar promotion type prefers conversion to $T_{F1}$
over conversion to $T_{F2}$, then $M_1$ is preferred. If $A$ is a
\chpl{param} argument with a default size, then $M_1$ is weakest
preferred. Otherwise, $M_1$ is weaker preferred.

Type conversion preferences are as follows:
\begin{itemize}
  \item
    Prefer converting a numeric argument to a numeric argument of
    a different width but the same category
    over converting to another type. Categories are
    \begin{itemize}
      \item
        bool
      \item
        enum
      \item
        int or uint
      \item
        real
      \item
        imag
      \item
        complex
    \end{itemize}

  \item Prefer an enum or bool cast to int over uint
  \item Prefer an enum or bool cast to a default-sized int or uint over another
    size of int or uint
  \item Prefer an enum, bool, int, or uint cast to a default-sized real
    over another size of real or complex
  \item Prefer an enum, bool, int, or uint cast to a default-sized
    complex over another size of complex
  \item Prefer real/imag cast to the complex with that component size (ie
    total width of twice the real/imag) over another size of complex

\end{itemize}

\item
If $T_A$ or its scalar promotion type prefers conversion to $T_{F2}$
over conversion to $T_{F1}$, then $M_2$ is preferred. If $A$ is a
\chpl{param} argument with a default size, then $M_2$ is weakest
preferred. Otherwise, $M_2$ is weaker preferred.

\item
 If $T_{F1}$ is derived from $T_{F2}$, then $M_1$ is more specific.
\item
 If $T_{F2}$ is derived from $T_{F1}$, then $M_2$ is more specific.
\item
 If there is an implicit conversion from $T_{F1}$ to $T_{F2}$, then
 $M_1$ is more specific.
\item
 If there is an implicit conversion from $T_{F2}$ to $T_{F1}$, then
 $M_2$ is more specific.
\item
 If $T_{F1}$ is any \chpl{int} type and $T_{F2}$ is any \chpl{uint}
 type, $M_1$ is more specific.
\item
 If $T_{F2}$ is any \chpl{int} type and $T_{F1}$ is any \chpl{uint}
 type, $M_2$ is more specific.
\item
 Otherwise neither mapping is more specific.
\end{itemize}

\subsection{Determining Best Functions}
\label{Determining_Best_Functions}

Given the set of most specific functions for a given return intent,
only the following function(s) are selected as best functions:
\begin{itemize}
\item all functions, if none of them contain a \chpl{where} clause;
\item only those functions that have a \chpl{where} clause, otherwise.
\end{itemize}

\subsection{Choosing Return Intent Overloads Based on Calling Context}
\label{Choosing_Return_Intent_Overload}
\index{functions!return intent overloads}

See also \ref{Return_Intent_Overloads}.

The compiler can choose between overloads differing in return intent
when:

\begin{itemize}

\item there are zero or one best functions for each of \chpl{ref},
\chpl{const ref}, \chpl{const}, or the default (blank) return intent

\item at least two of the above return intents have a best function.

\end{itemize}

In that case, the compiler is able to choose between \chpl{ref} return,
\chpl{const ref} return, and value return functions based upon the
context of the call. The compiler chooses between these return intent
overloads as follows:

If present, a \chpl{ref} return version will be chosen when:

\begin{itemize}

\item the call appears on the left-hand side of a variable initialization
or assignment statement

\item the call is passed to another function as a formal argument with
\chpl{out}, \chpl{inout}, or \chpl{ref} intent

\item the call is captured into a \chpl{ref} variable

\item the call is returned from a function with \chpl{ref} return intent

\end{itemize}

Otherwise, the \chpl{const ref} return or value return version will be
chosen. If only one of these is in the set of most specific functions, it
will be chosen. If both are present in the set, the choice will be made
as follows:

The \chpl{const ref} version will be chosen when:

\begin{itemize}

\item the call is passed to another function as a formal argument with
\chpl{const ref} intent

\item the call is captured into a \chpl{const ref} variable

\item the call is returned from a function with \chpl{const ref} return intent

\end{itemize}

Otherwise, the value version will be chosen.

\cleardoublepage
\sekshun{Methods}
\label{Methods}
\index{methods}
\index{methods!primary}
\index{methods!secondary}
\index{primary methods}
\index{secondary methods}

A \emph{method} is a procedure or iterator that is associated with an
expression known as the \emph{receiver}.

Methods are declared with the following syntax:
\begin{syntax}
method-declaration-statement:
  linkage-specifier[OPT] proc-or-iter this-intent[OPT] type-binding[OPT] function-name argument-list[OPT] 
    return-intent[OPT] return-type[OPT] where-clause[OPT] function-body

proc-or-iter:
  `proc'
  `iter'

this-intent:
  `param'
  `type'
  `ref'
  `const ref'
  `const'

type-binding:
  identifier .
  `(' expr `)' .

\end{syntax}

% Note: at the time `const ref` and `const` this intents were
% added, we didn't see any conceptual reason not to support
% `const in` and `in` this intents; it's just not a pressing concern.

Methods defined within the lexical scope of a class, record, or union
are referred to as \emph{primary methods}.  For such methods,
the \sntx{type-binding} is omitted and is taken to be the
innermost class, record, or union in which the method is defined.

Methods defined outside of such scopes are known as \emph{secondary
methods} and must have a \sntx{type-binding} (otherwise, they would
simply be standalone functions rather than methods).  Note that
secondary methods can be defined not only for classes, records, and
unions, but also for any other type (e.g., integers, reals, strings).

Secondary methods can be declared with a type expression instead of a
type identifier. In particular, if the \sntx{type-binding} is a
parenthesized expression, the compiler will evaluate that expression to
find the receiver type for the method. In that case, the method applies
only to receivers of that type. See also
\rsec{Creating_General_and_Specialized_Versions_of_a_Function}.

Method calls are described in \rsec{Method_Calls}.

The use of \sntx{this-intent} is described in \rsec{Method_receiver_and_this}.

\section{Method Calls}
\label{Method_Calls}
\index{method calls}
\index{methods!calling}
\index{methods!receiver}

A method is invoked with a method call, which is similar to a non-method
call expression, but it can include a receiver clause. The receiver
clause syntactically identifies a single argument by
putting it before the method name. That argument is the method receiver.
When calling a method from another method, or from within a class or
record declaration, the receiver clause can be omitted.

\begin{syntax}
method-call-expression:
  receiver-clause[OPT] expression ( named-expression-list )
  receiver-clause[OPT] expression [ named-expression-list ]
  receiver-clause[OPT] parenthesesless-function-identifier
\end{syntax}

The receiver-clause (or its absence) specifies the method's receiver
\rsec{Method_receiver_and_this}.

\begin{chapelexample}{defineMethod.chpl}
A method to output information about an instance of the \chpl{Actor}
class can be defined as follows:
\begin{chapelpre}
class Actor {
  var name: string;
  var age: uint;
}
var anActor = new Actor(name="Tommy", age=27);
writeln(anActor);
\end{chapelpre}
\begin{chapel}
proc Actor.print() {
  writeln("Actor ", name, " is ", age, " years old");
}
\end{chapel}
\begin{chapelpost}
anActor.print();
delete anActor;
\end{chapelpost}
\begin{chapeloutput}
{name = Tommy, age = 27}
Actor Tommy is 27 years old
\end{chapeloutput}
This method can be called on an instance of the \chpl{Actor}
class, \chpl{anActor}, with the call expression \chpl{anActor.print()}.
\end{chapelexample}

The actual arguments supplied in the method call are bound to the
formal arguments in the method declaration following the rules specified for
procedures (\rsec{Functions}). The exception is the receiver
\rsec{Method_receiver_and_this}.

\section{The Method Receiver and the {\em this} Argument}
\label{Method_receiver_and_this}
\index{methods!receiver}
\index{this@\chpl{this}}
\index{classes!this@\chpl{this}}
\index{receiver}
\index{type methods}
\index{instance methods}
\index{methods!type}
\index{methods!instance}

A method's \emph{receiver} is an implicit formal argument
named \chpl{this} representing the expression on which the method is
invoked.  The receiver's actual argument is specified by the
\sntx{receiver-clause} of a method-call-expression as specified
in \rsec{Method_Calls}.



% TODO: specify how the receiver affects the choice of the method.

\begin{chapelexample}{implicitThis.chpl}
Let class \chpl{C}, method \chpl{foo}, and function \chpl{bar} be
defined as
\begin{chapel}
class C {
  proc foo() {
    bar(this);
  }
}
proc bar(c: C) { writeln(c); }
\end{chapel}
\begin{chapelpost}
var c1: C = new C();
c1.foo();
delete c1;
\end{chapelpost}
\begin{chapeloutput}
{}
\end{chapeloutput}
Then given an instance of \chpl{C} called \chpl{c1}, the method
call \chpl{c1.foo()} results in a call to \chpl{bar} where the
argument is \chpl{c1}.  Within primary method \chpl{C.foo()}, the
(implicit) receiver formal has static type \chpl{C} and is referred to
as \chpl{this}.
\end{chapelexample}

Methods whose receivers are objects are called \emph{instance
methods}.  Methods may also be defined to have \chpl{type}
receivers---these are known as \emph{type methods}.

The optional \sntx{this-intent} is used to specify type methods, to
constrain a receiver argument to be a \chpl{param}, or to specify how
the receiver argument should be passed to the method.

When no \sntx{this-intent} is used, a default this intent applies. For
methods on classes and other primitive types, the default this intent is
the same as the default intent for that type.
For record methods, the intent for the receiver formal argument is \chpl{ref}
or \chpl{const ref}, depending on whether the formal argument is modified
inside of the method. Programmers wishing to be explicit about whether or
not record methods modify the receiver can explicitly use the \chpl{ref}
or \chpl{const ref} \sntx{this-intent}.

A method whose \sntx{this-intent} is \chpl{type} defines a \emph{type
method}.  It can only be called on the type itself rather than on an
instance of the type.  When \sntx{this-intent} is \chpl{param}, it
specifies that the function can only be applied to param objects of
the given type binding.

\begin{chapelexample}{paramTypeThisIntent.chpl}
In the following code, the \chpl{isOdd} method is defined with
a \sntx{this-intent} of \chpl{param}, permitting it to be called on
params only.  The \chpl{size} method is defined with
a \sntx{this-intent} of \chpl{type}, requiring it to be called on
the \chpl{int} type itself, not on integer values.
\begin{chapel}
proc param int.isOdd() param {
  return this & 0x1 == 0x1;
}

proc type int.size() param {
  return 64;
}

param three = 3;
var seven = 7;

writeln(42.isOdd());          // prints false
writeln(three.isOdd());       // prints true
writeln((42+three).isOdd());  // prints true
// writeln(seven.isOdd());    // illegal since 'seven' is not a param

writeln(int.size());          // prints 64
// writeln(42.size());        // illegal since 'size()' is a type method
\end{chapel}
\begin{chapeloutput}
false
true
true
64
\end{chapeloutput}
\end{chapelexample}

Type methods can also be iterators.

\begin{chapelexample}{typeMethodIter.chpl}
In the following code, the class \chpl{C} defines a type method
iterator which can be invoked on the type itself:
\begin{chapel}
class C {
  var x: int;
  var y: string;

  iter type myIter() {
    yield 3;
    yield 5;
    yield 7;
    yield 11;
  }
}

for i in C.myIter() do
  writeln(i);
\end{chapel}
\begin{chapeloutput}
3
5
7
11
\end{chapeloutput}
\end{chapelexample}

When \sntx{this-intent} is \chpl{ref}, the receiver argument will be
passed by reference, allowing modifications to \chpl{this}.  If
\sntx{this-intent} is \chpl{const ref}, the receiver argument is passed
by reference but it cannot be modified inside the method. The
\sntx{this-intent} can also describe an abstract intent as follows. If it is
\chpl{const}, the receiver argument will be passed with \chpl{const}
intent. If it is left out entirely, the receiver will be passed with
a default intent. For records, that default intent is \chpl{ref} if
\chpl{this} is modified within the function and \chpl{const ref}
otherwise.  For other types, the default \chpl{this} intent matches the
default argument intent described in \rsec{The_Default_Intent}.

\begin{chapelexample}{refThisIntent.chpl}
In the following code, the \chpl{doubleMe} function is defined with a
\sntx{this-intent} of \chpl{ref}, allowing variables of type \chpl{int} to
double themselves.
\begin{chapel}
proc ref int.doubleMe() { this *= 2; }
\end{chapel}
\begin{chapelpost}
var x: int = 2;
x.doubleMe();
writeln(x);
\end{chapelpost}
\begin{chapeloutput}
4
\end{chapeloutput}
Given a variable \chpl{x = 2}, a call to \chpl{x.doubleMe()} will set \chpl{x}
to \chpl{4}.
\end{chapelexample}

\section{The {\em this} Method}
\label{The_this_Method}
\index{methods!indexing}
\index{this@\chpl{this}}
\index{methods!this@\chpl{this}}

A procedure method declared with the name \chpl{this} allows the receiver to be
``indexed'' similarly to how an array is indexed.  Indexing (as with
\chpl{A[1]}) has the semantics of calling a method
named \chpl{this}.  There is no other way to call a method
called \chpl{this}.  The \chpl{this} method must be declared with
parentheses even if the argument list is empty.

\begin{chapelexample}{thisMethod.chpl}
In the following code, the \chpl{this} method is used to create a
class that acts like a simple array that contains three integers
indexed by 1, 2, and 3.
\begin{chapel}
class ThreeArray {
  var x1, x2, x3: int;
  proc this(i: int) ref {
    select i {
      when 1 do return x1;
      when 2 do return x2;
      when 3 do return x3;
    }
    halt("ThreeArray index out of bounds: ", i);
  }
}
\end{chapel}
\begin{chapelpost}
var ta = new ThreeArray();
ta(1) = 1;
ta(2) = 2;
ta(3) = 3;
for i in 1..3 do
  writeln(ta(i));
ta(4) = 4;
\end{chapelpost}
\begin{chapeloutput}
1
2
3
thisMethod.chpl:9: error: halt reached - ThreeArray index out of bounds: 4
\end{chapeloutput}
\end{chapelexample}

\section{The {\em these} Method}
\label{The_these_Method}
\index{methods!iterating}
\index{these@\chpl{these}}
\index{methods!these@\chpl{these}}

An iterator method declared with the name \chpl{these} allows the
receiver to be ``iterated over'' similarly to how a domain or array
supports iteration. When a value supporting a \chpl{these} method
is used as the the \sntx{iteratable-expression} of a loop, the
loop proceeds in a manner controlled by the \chpl{these} iterator.

\begin{chapelexample}{theseIterator.chpl}
In the following code, the \chpl{these} method is used to create a
class that acts like a simple array that can be iterated over and
contains three integers.
\begin{chapel}
class ThreeArray {
  var x1, x2, x3: int;
  iter these() ref {
    yield x1;
    yield x2;
    yield x3;
  }
}
\end{chapel}
\begin{chapelpost}
var ta = new ThreeArray();
for (i, j) in zip(ta, 1..) do
  i = j;

for i in ta do
  writeln(i);
delete ta;
\end{chapelpost}
\begin{chapeloutput}
1
2
3
\end{chapeloutput}

\end{chapelexample}

An iterator type method with the name \chpl{these} supports iteration
over the class type itself.

\begin{chapelexample}{typeMethodIterThese.chpl}
In the following code, the class \chpl{C} defines a type method
iterator named \chpl{these}, supporting direct iteration over the type:
\begin{chapel}
class C {
  var x: int;
  var y: string;

  iter type these() {
    yield 1;
    yield 2;
    yield 4;
    yield 8;
  }
}

for i in C do
  writeln(i);
\end{chapel}
\begin{chapeloutput}
1
2
4
8
\end{chapeloutput}
\end{chapelexample}

\cleardoublepage
\sekshun{Tuples}
\label{Tuples}
\index{tuples}

A tuple is an ordered set of components that allows for the
specification of a light-weight collection of values.  As the examples
in this chapter illustrate, tuples are a boon to the Chapel
programmer.  In addition to making it easy to return multiple values
from a function, tuples help to support multidimensional indices, to
group arguments to functions, and to specify mathematical concepts.

\section{Tuple Types}
\label{Tuple_Types}
\index{tuples!types}
\index{types!tuple}

A tuple type is defined by a fixed number (a compile-time constant) of
component types.  It can be specified by a parenthesized,
comma-separated list of types.  The number of types in the list
defines the size of the tuple; the types themselves specify the
component types.

The syntax of a tuple type is given by:
\begin{syntax}
tuple-type:
  ( type-specifier , type-list )

type-list:
  type-specifier
  type-specifier , type-list
\end{syntax}

\index{tuples!homogeneous}
\index{types!* (tuples)@\chpl{*} tuples}
A homogeneous tuple is a special-case of a general tuple where the
types of the components are identical.  Homogeneous tuples have fewer
restrictions for how they can be indexed~(\rsec{Tuple_Indexing}).
Homogeneous tuple types can be defined using the above syntax, or they
can be defined as a product of an integral parameter (a compile-time
constant integer) and a type.  This latter specification is
implemented by overloading \chpl{*} with the following prototype:
\begin{chapel}
proc *(param p: int, type t) type
\end{chapel}

\begin{rationale}
Homogeneous tuples require the size to be specified as a parameter
(a compile-time constant).  This avoids any overhead associated with
storing the runtime size in the tuple.  It also avoids the question as
to whether a non-parameter size should be part of the type of the
tuple.  If a programmer requires a non-parameter value to define a
data structure, an array may be a better choice.
\end{rationale}

\begin{chapelexample}{homogenous.chpl}
The statement
\begin{chapel}
var x1: (string, real),
    x2: (int, int, int),
    x3: 3*int;
\end{chapel}
defines three variables.  Variable \chpl{x1} is a 2-tuple with
component types \chpl{string} and \chpl{real}.  Variables \chpl{x2}
and \chpl{x3} are homogeneous 3-tuples with component type \chpl{int}.
The types of \chpl{x2} and \chpl{x3} are identical even though they
are specified in different ways.
\begin{chapelpost}
writeln((x1,x2,x3));
\end{chapelpost}
\begin{chapeloutput}
((, 0.0), (0, 0, 0), (0, 0, 0))
\end{chapeloutput}
\end{chapelexample}

Note that if a single type is delimited by parentheses, the
parentheses only impact precedence.  Thus \chpl{(int)} is equivalent
to \chpl{int}.  Nevertheless, tuple types with a single component type
are legal and useful.  One way to specify a 1-tuple is to use the
overloaded \chpl{*} operator since every 1-tuple is trivially a
homogeneous tuple.

\begin{rationale}
Like parentheses around expressions, parentheses around types are
necessary for grouping in order to avoid the default precedence of the
grammar.  Thus it is not the case that we would always want to create
a tuple.  The type \chpl{3*(3*int)} specifies a 3-tuple of 3-tuples of
integers rather than a 3-tuple of 1-tuples of 3-tuples of integers.
The type \chpl{3*3*int}, on the other hand, specifies a 9-tuple of
integers.
\end{rationale}

\section{Tuple Values}
\label{Tuple_Values}
\index{tuples!values}
\index{values!tuple}

A value of a tuple type attaches a value to each component type.
Tuple values can be specified by a parenthesized, comma-separated list
of expressions.  The number of expressions in the list defines the
size of the tuple; the types of these expressions specify the
component types of the tuple.

The syntax of a tuple expression is given by:
\begin{syntax}
tuple-expression:
  ( tuple-component , )
  ( tuple-component , tuple-component-list )

tuple-component:
  expression
  `_'

tuple-component-list:
  tuple-component
  tuple-component , tuple-component-list
\end{syntax}

An underscore can be used to omit components when splitting
a tuple (see \ref{Assignments_in_a_Tuple}).

\begin{chapelexample}{values.chpl}
The statement
\begin{chapel}
var x1: (string, real) = ("hello", 3.14),
    x2: (int, int, int) = (1, 2, 3),
    x3: 3*int = (4, 5, 6);
\end{chapel}
defines three tuple variables.  Variable \chpl{x1} is a 2-tuple with
component types \chpl{string} and \chpl{real}.  It is initialized such
that the first component is \chpl{"hello"} and the second
component is \chpl{3.14}.  Variables \chpl{x2} and \chpl{x3} are
homogeneous 3-tuples with component type \chpl{int}.  Their
initialization expressions specify 3-tuples of integers.
\begin{chapelpost}
writeln((x1,x2,x3));
\end{chapelpost}
\begin{chapeloutput}
((hello, 3.14), (1, 2, 3), (4, 5, 6))
\end{chapeloutput}
\end{chapelexample}

Note that if a single expression is delimited by parentheses, the
parentheses only impact precedence.  Thus \chpl{(1)} is equivalent
to \chpl{1}.  To specify a 1-tuple, use the form with the trailing
comma \chpl{(1,)}.

\begin{chapelexample}{onetuple.chpl}
The statement
\begin{chapel}
var x: 1*int = (7,);
\end{chapel}
creates a 1-tuple of integers storing the value 7.
\begin{chapelpost}
writeln(x);
\end{chapelpost}
\begin{chapeloutput}
(7)
\end{chapeloutput}
\end{chapelexample}

Tuple expressions are evaluated similarly to function calls where the
arguments are all generic with no explicit intent.  So a tuple
expression containing an array does not copy the array.  

When a tuple is passed as an argument to a function, it is passed as
if it is a record type containing fields of the same type and in
the same order as in the tuple.

\section{Tuple Indexing}
\label{Tuple_Indexing}
\index{tuples!indexing}

A tuple component may be accessed by an integral parameter (a compile-time
constant) as if the tuple were an array.  Indexing is 1-based, so the
first component in the tuple is accessed by the index \chpl{1}, and so
forth.

\begin{chapelexample}{access.chpl}
The loop
\begin{chapel}
var myTuple = (1, 2.0, "three");
for param i in 1..3 do
  writeln(myTuple(i));
\end{chapel}
uses a param loop to output the components of a tuple.
\begin{chapelpost}
\end{chapelpost}
\begin{chapeloutput}
1
2.0
three
\end{chapeloutput}
\end{chapelexample}

Homogeneous tuples may be accessed by integral values that are not
necessarily compile-time constants.

\begin{chapelexample}{access-homogeneous.chpl}
The loop
\begin{chapel}
var myHTuple = (1, 2, 3);
for i in 1..3 do
  writeln(myHTuple(i));
\end{chapel}
uses a serial loop to output the components of a homogeneous tuple.
Since the index is not a compile-time constant, this would result in
an error were tuple not homogeneous.
\begin{chapelpost}
\end{chapelpost}
\begin{chapeloutput}
1
2
3
\end{chapeloutput}
\end{chapelexample}

\begin{rationale}
Non-homogeneous tuples can only be accessed by compile-time constants
since the type of an expression must be statically known.
\end{rationale}

\section{Iteration over Tuples}
\label{Iteration_over_Tuples}
\index{tuples!iteration}
\index{iteration!tuple}

% FYI: Similar to text regarding array iteration.  Slightly less
% similar for domain iteration.
Only homogenous tuples support iteration via
standard \chpl{for}, \chpl{forall} and \chpl{coforall} loops.  These loops
iterate over all of the tuple's elements.  A loop of the form:

% This is difficult to capture in a test program
\begin{chapel}
[for|forall|coforall] e in t do
  ...e...
\end{chapel}

where t is a homogenous tuple of size \chpl{n}, is semantically
equivalent to:

% This is difficult to capture in a test program
\begin{chapel}
[for|forall|coforall] i in 1..n do
  ...t(i)...
\end{chapel}

The iterator variable for an tuple iteration is a either a const value
or a reference to the tuple element type, following default intent
semantics.

\section{Tuple Assignment}
\label{Tuple_Assignment}
\index{assignment!tuple}
\index{tuples!assignment}

In tuple assignment, the components of the tuple on the left-hand side
of the assignment operator are each assigned the components of the
tuple on the right-hand side of the assignment.  These assignments
occur in component order (component one followed by component two,
etc.).

\section{Tuple Destructuring}
\label{Tuple_Destructuring}
\index{tuples!destructuring}

Tuples can be split into their components in the following ways:
\begin{itemize}
\item In assignment where multiple expression on the left-hand side of
the assignment operator are grouped using tuple notation.
\item In variable declarations where multiple variables in a
declaration are grouped using tuple notation.
\item In for, forall, and coforall loops (statements and expressions)
where multiple indices in a loop are grouped using tuple notation.
\item In function calls where multiple formal arguments in a function
declaration are grouped using tuple notation.
\item In an expression context that accepts a comma-separated list of
expressions where a tuple expression is expanded in place using the
tuple expansion expression.
\end{itemize}

\subsection{Splitting a Tuple with Assignment}
\label{Assignments_in_a_Tuple}
\index{tuples!assignments grouped as}

When multiple expression on the left-hand side of an assignment
operator are grouped using tuple notation, the tuple on the right-hand
side is split into its components.  The number of grouped expressions
must be equal to the size of the tuple on the right-hand side.  In
addition to the usual assignment evaluation order of left to right,
the assignment is evaluated in component order.

\begin{chapelexample}{splitting.chpl}
The code
\begin{chapel}
var a, b, c: int;
(a, (b, c)) = (1, (2, 3));
\end{chapel}
defines three integer variables \chpl{a}, \chpl{b}, and \chpl{c}.  The
second line then splits the tuple \chpl{(1, (2, 3))} such that \chpl{1}
is assigned to \chpl{a}, \chpl{2} is assigned to \chpl{b},
and \chpl{3} is assigned to \chpl{c}.
\begin{chapelpost}
writeln((a, b, c));
\end{chapelpost}
\begin{chapeloutput}
(1, 2, 3)
\end{chapeloutput}
\end{chapelexample}

\begin{chapelexample}{aliasing.chpl}
The code
\begin{chapel}
var A = [i in 1..4] i;
writeln(A);
(A(1..2), A(3..4)) = (A(3..4), A(1..2));
writeln(A);
\end{chapel}
creates a non-distributed, one-dimensional array containing the four
integers from \chpl{1} to \chpl{4}.  Line 2 outputs \chpl{1 2 3 4}.
Line 3 does what appears to be a swap of array slices.  However,
because the tuple is created with array aliases (like a function
call), the assignment to the second component uses the values just
overwritten in the assignment to the first component.  Line 4
outputs \chpl{3 4 3 4}.
\begin{chapelpost}
\end{chapelpost}
\begin{chapeloutput}
1 2 3 4
3 4 3 4
\end{chapeloutput}
\end{chapelexample}

\index{tuples!omitting components}
When splitting a tuple with assignment, the underscore token can
be used to omit storing some of the components.  In this case, the
full expression on the right-hand side of the assignment operator is
evaluated, but the omitted values will not be assigned to anything.

\begin{chapelexample}{omit-component.chpl}
The code
\begin{chapel}
proc f()
  return (1, 2);

var x: int;
(x,_) = f();
\end{chapel}
defines a function that returns a 2-tuple, declares an integer
variable \chpl{x}, calls the function, assigns the first component in
the returned tuple to \chpl{x}, and ignores the second component in
the returned tuple.  The value of \chpl{x} becomes \chpl{1}.
\begin{chapelpost}
writeln(x);
\end{chapelpost}
\begin{chapeloutput}
1
\end{chapeloutput}
\end{chapelexample}

\subsection{Splitting a Tuple in a Declaration}
\label{Variable_Declarations_in_a_Tuple}
\index{tuples!variable declarations grouped as}

When multiple variables in a declaration are grouped using tuple
notation, the tuple initialization expression is
split into its type and/or value components.  The number of grouped variables must be
equal to the size of the tuple initialization
expression.  The variables are initialized in component order.

The syntax of grouped variable declarations is defined
in~\rsec{Variable_Declarations}.

\begin{chapelexample}{decl.chpl}
The code
\begin{chapel}
var (a, (b, c)) = (1, (2, 3));
\end{chapel}
defines three integer variables \chpl{a}, \chpl{b}, and \chpl{c}.  It
splits the tuple \chpl{(1, (2, 3))} such that \chpl{1}
initializes \chpl{a}, \chpl{2} initializes \chpl{b}, and \chpl{3}
initializes \chpl{c}.
\begin{chapelpost}
writeln((a, b, c));
\end{chapelpost}
\begin{chapeloutput}
(1, 2, 3)
\end{chapeloutput}
\end{chapelexample}

Grouping variable declarations using tuple notation allows a 1-tuple
to be destructured by enclosing a single variable declaration in
parentheses.
\begin{chapelexample}{onetuple-destruct.chpl}
The code
\begin{chapel}
var (a) = (1, );
\end{chapel}
initialize the new variable \chpl{a} to 1.
\begin{chapelpost}
writeln(a);
\end{chapelpost}
\begin{chapeloutput}
1
\end{chapeloutput}
\end{chapelexample}

\index{tuples!omitting components}
When splitting a tuple into multiple variable declarations, the
underscore token may be used to omit components of the tuple rather
than declaring a new variable for them.  In this case, no variables
are defined for the omitted components.

\begin{chapelexample}{omit-component-decl.chpl}
The code
\begin{chapel}
proc f()
  return (1, 2);

var (x,_) = f();
\end{chapel}
defines a function that returns a 2-tuple, calls the function,
declares and initializes variable \chpl{x} to the first component in
the returned tuple, and ignores the second component in the returned
tuple.  The value of \chpl{x} is initialized to \chpl{1}.
\begin{chapelpost}
writeln(x);
\end{chapelpost}
\begin{chapeloutput}
1
\end{chapeloutput}
\end{chapelexample}

\subsection{Splitting a Tuple into Multiple Indices of a Loop}
\label{Indices_in_a_Tuple}
\index{tuples!indices grouped as}

When multiple indices in a loop are grouped using tuple notation, the tuple
returned by the iterator (\rsec{Iterators}) is split across the index tuple's components.  The
number of indices in the index tuple must equal the size of the tuple
returned by the iterator.

\begin{chapelexample}{indices.chpl}
The code
\begin{chapel}
iter bar() {
  yield (1, 1);
  yield (2, 2);
}

for (i,j) in bar() do
  writeln(i+j);
\end{chapel}
defines a simple iterator that yields two 2-tuples before completing.
The for-loop uses a tuple notation to group two indices that take
their values from the iterator.
\begin{chapelpost}
\end{chapelpost}
\begin{chapeloutput}
2
4
\end{chapeloutput}
\end{chapelexample}

\index{tuples!omitting components}
When a tuple is split across an index tuple, indices in the index
tuple (left-hand side) may be omitted.  In this case, no indices are
defined for the omitted components.

However even when indices are omitted, the iterator is
evaluated as if an index were defined.  Execution proceeds as if the
omitted indices are present but invisible.  This means that the loop body
controlled by the iterator may be executed multiple times with the
same set of (visible) indices.

\subsection{Splitting a Tuple into Multiple Formal Arguments in a Function Call}
\label{Formal_Argument_Declarations_in_a_Tuple}
\index{tuples!formal arguments grouped as}

When multiple formal arguments in a function declaration are grouped
using tuple notation, the actual expression is split into its
components during a function call.  The number of grouped formal
arguments must be equal to the size of the actual tuple expression.
The actual arguments are passed in component order to the formal
arguments.

The syntax of grouped formal arguments is defined
in~\rsec{Function_Definitions}.

\begin{chapelexample}{formals.chpl}
The function
\begin{chapel}
proc f(x: int, (y, z): (int, int)) {
  // body
}
\end{chapel}
is defined to take an integer value and a 2-tuple of integer values.
The 2-tuple is split when the function is called into two formals.  A
call may look like the following:
\begin{chapel}
f(1, (2, 3));
\end{chapel}
\begin{chapelpost}
\end{chapelpost}
\begin{chapeloutput}
\end{chapeloutput}
\end{chapelexample}

An implicit \chpl{where} clause is created when arguments are grouped using
tuple notation, to ensure that the function is called with an actual
tuple of the correct size.  Arguments grouped in tuples may be
nested arbitrarily.  Functions with arguments grouped into tuples may not be
called using named-argument passing on the tuple-grouped arguments.
In addition, tuple-grouped arguments may not be specified individually
with types or default values (only in aggregate).  They may not be
specified with any qualifier appearing before the group of arguments
(or individual arguments) such as \chpl{inout} or \chpl{type}.  They
may not be followed by \chpl{...} to indicate that there are a
variable number of them.

\begin{chapelexample}{implicit-where.chpl}
The function \chpl{f} defined as
\begin{chapel}
proc f((x, (y, z))) {
  writeln((x, y, z));
}
\end{chapel}
is equivalent to the function \chpl{g} defined as
\begin{chapel}
proc g(t) where isTuple(t) && t.size == 2 && isTuple(t(2)) && t(2).size == 2 {
  writeln((t(1), t(2)(1), t(2)(2)));
}
\end{chapel}
except without the definition of the argument name \chpl{t}.
\begin{chapelpost}
f((1, (2, 3)));
g((1, (2, 3)));
\end{chapelpost}
\begin{chapeloutput}
(1, 2, 3)
(1, 2, 3)
\end{chapeloutput}
\end{chapelexample}

Grouping formal arguments using tuple notation allows a 1-tuple to be
destructured by enclosing a single formal argument in parentheses.
\begin{chapelexample}{grouping-Formals.chpl}
The empty function
\begin{chapel}
proc f((x)) { }
\end{chapel}
accepts a 1-tuple actual with any component type.
\begin{chapelpost}
f((1, ));
var y: 1*real;
f(y);
\end{chapelpost}
\begin{chapeloutput}
\end{chapeloutput}
\end{chapelexample}

\index{tuples!omitting components}
When splitting a tuple into multiple formal arguments, the arguments
that are grouped using the tuple notation may be omitted.  In this
case, no names are associated with the omitted components.  The
call is evaluated as if an argument were defined.
%TODO: hilde
% Example required.

\subsection{Splitting a Tuple via Tuple Expansion}
\label{Tuple_Expansion}
\index{... (tuple expansion)@\chpl{...} (tuple expansion)}
\index{tuples!expanding in place}

Tuples can be expanded in place using the following syntax:
\begin{syntax}
tuple-expand-expression:
  ( ... expression )
\end{syntax}
In this expression, the tuple defined by \sntx{expression} is expanded
in place to represent its components.  This can only be used in a
context where a comma-separated list of components is valid.

\begin{chapelexample}{expansion.chpl}
Given two 2-tuples
\begin{chapel}
var x1 = (1, 2.0), x2 = ("three", "four");
\end{chapel}
the following statement
\begin{chapel}
var x3 = ((...x1), (...x2));
\end{chapel}
creates the 4-tuple \chpl{x3} with the value \chpl{(1, 2.0, "three",
"four")}.
\begin{chapelpost}
writeln(x3);
\end{chapelpost}
\begin{chapeloutput}
(1, 2.0, three, four)
\end{chapeloutput}
\end{chapelexample}

\begin{chapelexample}{expansion-2.chpl}
The following code defines two functions, a function \chpl{first} that
returns the first component of a tuple and a function \chpl{rest} that
returns a tuple containing all of the components of a tuple except for
the first:
\begin{chapel}
proc first(t) where isTuple(t) {
  return t(1);
}
proc rest(t) where isTuple(t) {
  proc helper(first, rest...)
    return rest;
  return helper((...t));
}
\end{chapel}
\begin{chapelpost}
writeln(first((1, 2, 3)));
writeln(rest((1, 2, 3)));
\end{chapelpost}
\begin{chapeloutput}
1
(2, 3)
\end{chapeloutput}
\end{chapelexample}

\section{Tuple Operators}
\label{Tuple_Operators}
\index{tuples!operators}

\subsection{Unary Operators}
\label{Tuple_Unary_Operators}
\index{operators!tuple!unary}

The unary operators \chpl{\+}, \chpl{\-}, \chpl{\~}, and \chpl{\!} are
overloaded on tuples by applying the operator to each argument component
and returning the results as a new tuple.

The size of the result tuple is the same as the size of the
argument tuple. The type of each result component is the result
type of the operator when applied to the corresponding argument component.

The type of every element of the operand tuple must have a
well-defined operator matching the unary operator being applied.  That
is, if the element type is a user-defined type, it must supply an
overloaded definition for the unary operator being used.  Otherwise, a
compile-time error will be issued.

\subsection{Binary Operators}
\label{Tuple_Binary_Operators}
\index{operators!tuple!binary}
%\index{\+@\chpl{\+}}
%\index{\-@\chpl{\-}}
%\index{\*@\chpl{\*}}
%\index{\/@\chpl{\/}}
%\index{\%@\chpl{\%}}
%\index{\*\*@\chpl{\*\*}}
%\index{\&@\chpl{\&}}
%\index{\|@\chpl{\|}}
%\index{\^@\chpl{\^}}
%\index{\<\<@\chpl{\<\<}}
%\index{\>\>@\chpl{\>\>}}

The binary operators \chpl{\+}, \chpl{\-}, \chpl{\*}, \chpl{\/}, \chpl{\%},
\chpl{\*\*}, \chpl{\&}, \chpl{\|}, \chpl{\^}, \chpl{\<\<}, and \chpl{\>\>}
are overloaded on tuples by applying them to pairs of the respective
argument components and returning the results as a new tuple.  The
sizes of the two argument tuples must be the same.  These operators
are also defined for homogenous tuples and scalar values of matching
type.

The size of the result tuple is the same as the argument tuple(s).
The type of each result component is the result type of the operator
when applied to the corresponding pair of the argument components.

When a tuple binary operator is used, the same operator must be
well-defined for successive pairs of operands in the two tuples.
Otherwise, the operation is illegal and a compile-time error will
result.

\begin{chapelexample}{binary-ops.chpl}
The code
\begin{chapel}
var x = (1, 1, 1) + (2, 2.0, "2");
\end{chapel}
creates a 3-tuple of an int, a real and a string with the value \chpl{(3, 3.0, "12")}.
\begin{chapelpost}
writeln(x);
\end{chapelpost}
\begin{chapeloutput}
(3, 3.0, 12)
\end{chapeloutput}
\end{chapelexample}


\subsection{Relational Operators}
\label{Tuple_Relational_Operators}
\index{operators!tuple!relational}
%\index{\>@\chpl{\>}}
%\index{\>\=@\chpl{\>\=}}
%\index{\<@\chpl{\<}}
%\index{\<\=@\chpl{\<\=}}
%\index{\=\=@\chpl{\=\=}}
%\index{\!\=@\chpl{\!\=}}

% (\rsec{Relational_Operators})

The relational operators \chpl{\>}, \chpl{\>\=}, \chpl{\<}, \chpl{\<\=},
\chpl{\=\=}, and \chpl{\!\=} are defined over tuples of matching size.
They return a single boolean value indicating whether the two
arguments satisfy the corresponding relation.

The operators \chpl{\>}, \chpl{\>\=}, \chpl{\<}, and \chpl{\<\=}
check the corresponding lexicographical order
based on pair-wise comparisons between the argument tuples' components.
%based on comparisons between pairs of the respective
%components of the arguments.
The operators \chpl{\=\=} and \chpl{\!\=} check whether
the two arguments are pair-wise equal or not.
The relational operators on tuples may be short-circuiting, i.e.
they may execute only the pair-wise comparisons that are necessary
to determine the result.

However, just as for other binary tuple operators, the corresponding operation
must be well-defined on each successive pair of operand types in the two operand
tuples.  Otherwise, a compile-time error will result.

\begin{chapelexample}{relational-ops.chpl}
The code
\begin{chapel}
var x = (1, 1, 0) > (1, 0, 1);
\end{chapel}
creates a variable initialized to \chpl{true}.  After comparing the
first components and determining they are equal, the second components
are compared to determine that the first tuple is greater than the
second tuple.
\begin{chapelpost}
writeln(x);
\end{chapelpost}
\begin{chapeloutput}
true
\end{chapeloutput}
\end{chapelexample}

\section{Predefined Functions and Methods on Tuples}
\label{Predefined_Functions_and_Methods_on_Tuples}
\index{tuples!predefined functions}
\index{predefined functions!tuples}
\index{functions!tuples!predefined}

\begin{protohead}
proc isHomogeneousTuple(t: $Tuple$) param
\end{protohead}
\begin{protobody}
Returns true if \chpl{t} is a homogeneous tuple; otherwise false.
\end{protobody}

\index{tuples!isTuple@\chpl{isTuple}}
\index{predefined functions!isTuple@\chpl{isTuple}}
\begin{protohead}
proc isTuple(t: $Tuple$) param
\end{protohead}
\begin{protobody}
Returns true if \chpl{t} is a tuple; otherwise false.
\end{protobody}

\index{tuples!isTupleType@\chpl{isTupleType}}
\index{predefined functions!isTupleType@\chpl{isTupleType}}
\begin{protohead}
proc isTupleType(type t) param
\end{protohead}
\begin{protobody}
Returns true if \chpl{t} is a tuple of types; otherwise false.
\end{protobody}

\index{tuples!max@\chpl{max}}
\index{predefined functions!max@\chpl{max}}
\begin{protohead}
proc max(type t) where isTupleType(t)
\end{protohead}
\begin{protobody}
Returns a tuple of type \chpl{t} with each component set to the maximum
value that can be stored in its position.
\end{protobody}

\index{tuples!min@\chpl{min}}
\index{predefined functions!min@\chpl{min}}
\begin{protohead}
proc min(type t) where isTupleType(t)
\end{protohead}
\begin{protobody}
Returns a tuple of type \chpl{t} with each component set to the minimum
value that can be stored in its position.
\end{protobody}

\index{tuples!size@\chpl{size}}
\index{predefined functions!size@\chpl{size}}
\begin{protohead}
proc $Tuple$.size param
\end{protohead}
\begin{protobody}
Returns the size of the tuple.
\end{protobody}

\cleardoublepage
\sekshun{Classes}
\label{Classes}
\index{classes}

\index{classes!instances}
\index{objects}
Classes are data structures with associated state and functions. Storage for
a class instance, or object, is allocated independently of the scope of
the variable that refers to it.
An object is created by calling a class constructor
(\rsec{Class_Constructors}), which allocates storage, initializes it,
and returns a reference to the newly-created object.
Storage can be reclaimed by deleting the object (\rsec{Class_Delete}).

A class declaration (\rsec{Class_Declarations}) generates a class
type (\rsec{Class_Types}).  A variable of a class type can refer to an
instance of that class or any of its derived classes.

A class is generic if it has generic fields. A class can also
be generic if it inherits from a generic class. Generic classes and fields
are discussed in~\rsec{Generic_Types}.

\section{Class Declarations}
\label{Class_Declarations}
\index{classes!declarations}
\index{class@\chpl{class}}

A class is defined with the following syntax:
\begin{syntax}
class-declaration-statement:
  simple-class-declaration-statement
  external-class-declaration-statement

simple-class-declaration-statement:
  `class' identifier class-inherit-list[OPT] { class-statement-list[OPT] }

class-inherit-list:
  : class-type-list

class-type-list:
  class-type
  class-type , class-type-list

class-statement-list:
  class-statement
  class-statement class-statement-list

class-statement:
  variable-declaration-statement
  method-declaration-statement
  type-declaration-statement
  empty-statement
\end{syntax}

A \sntx{class-declaration-statement} defines a new type symbol
specified by the identifier.  Classes inherit data and functionality
from other classes %and/or records
if the \sntx{inherit-type-list} is specified.
Inheritance is described in~\rsec{Inheritance}.

\begin{openissue}
Classes that inherit from records are an area for future work.
\end{openissue}

The body of a class declaration consists of a sequence of statements
where each of the statements either defines a variable (called a
field), a procedure or iterator (called a method), or a type alias.  In addition, empty
statements are allowed in class declarations, and they have no effect.

If a class declaration contains a type alias or a parameter field, or it contains a variable or
constant without a specified type and without an initialization
expression, then it declares a generic class type.  Generic classes are described
in~\rsec{Generic_Types}.

If the \chpl{extern} keyword appears before the \chpl{class} keyword, then an
external class type is declared.  An external class type declaration must not
contain a \sntx{class-inherit-list}.  An external class is used within Chapel
for type and field resolution, but no corresponding backend definition is
generated.  It is presumed that the definition of an external class is supplied
by a library or the execution environment.  See the chapter on interoperability
(\rsec{Interoperability}) for more information on external classes.
% External class inheritance is not currently supported.

\begin{future}
Privacy controls for classes and records are currently not specified,
as discussion is needed regarding its impact on inheritance, for
instance.
\end{future}

\subsection{Class Types}
\label{Class_Types}
\index{classes!types}
\index{class type}

A class type is given simply by the class name for non-generic classes.
Generic classes must be instantiated to serve as a fully-specified
type, for example to declare a variable.  This is done with
type constructors, which are defined in Section~\ref{Type_Constructors}.

\begin{syntax}
class-type:
  identifier
  identifier ( named-expression-list )
\end{syntax}

A class type, including a generic class type that is not
fully specified, may appear in the inheritance lists
of other class declarations.

\subsection{Class Values}
\label{Class_Values}
\index{classes!values}
\index{class value}

A class value is either a reference to an instance of a class
or \chpl{nil} (\rsec{Class_nil_value}). Class instances can be created
using the \chpl{new} operator (\rsec{Class_New}) and deleted using
the \chpl{delete} operator (\rsec{Class_Delete}).

For a given class type, a legal value of that type is a reference to
an instance of either that class or a class inheriting, directly or
indirectly, from that class.
\chpl{nil} is a legal value of any class type.

The default value of a class type is \chpl{nil}.

\begin{chapelexample}{declaration.chpl}
\begin{chapel}
class C { }
var c : C;      // c has the class type C, initialized with the value nil.
c = new C();    // Now c refers to an object of type C.
var c2 = c;     // The type of c2 is also C.
                // c2 refers to the same object as c.
class D : C {}  // Class D is derived from C.
c = new D();    // Now c refers to an object of type D.
\end{chapel}
\begin{chapelpost}
delete c;   // This deletes the new D.
delete c2;  // This deletes the new C.
\end{chapelpost}
\begin{chapeloutput}
\end{chapeloutput}
When the variable \chpl{c} is declared, it initially has the value
of \chpl{nil}.  The next statement assigned to it an instance of the
class \chpl{C}.  The declaration of variable \chpl{c2} shows that these steps can
be combined.  The type of \chpl{c2} is also \chpl{C}, determined implicitly from
the the initialization expression.  Finally, an object of type \chpl{D} is created and
assigned to \chpl{c}.  The object previously referenced by \chpl{c} is no longer
referenced anywhere. It could be reclaimed by the garbage collector.
\end{chapelexample}

\subsection{Class Fields}
\label{Class_Fields}
\index{classes!fields}
\index{fields!class}

A variable declaration within a class declaration defines
a \emph{field} within that class.
Each class instance consists of one variable per each
\chpl{var} or \chpl{const} field in the class.

\begin{chapelexample}{defineActor.chpl}
The code
\begin{chapelpre}
config param cleanUp = false;
\end{chapelpre}
\begin{chapel}
class Actor {
  var name: string;
  var age: uint;
}
\end{chapel}
\begin{chapeloutput}
\end{chapeloutput}
defines a new class type called \chpl{Actor} that has two fields: the
string field \chpl{name} and the unsigned integer field \chpl{age}.
\end{chapelexample}

Field access is described in \rsec{Class_Field_Accesses}.

\begin{future}
\chpl{ref} fields, which are fields corresponding to variable declarations
with \chpl{ref} or \chpl{const ref} keywords, are an area of future work.
\end{future}

\subsection{Class Methods}
\label{Class_Methods}
\index{classes!methods}
\index{methods!classes}
\index{methods!primary}
\index{methods!secondary}
\index{primary methods}
\index{secondary methods}

A \emph{method} is a procedure or iterator that is associated with a
type known as the \emph{receiver}.  Methods on classes are referred
to as to as \emph{class methods}.  Methods may be defined on other
types as well.

Methods are declared with the following syntax:
\begin{syntax}
method-declaration-statement:
  linkage-specifier[OPT] proc-or-iter this-intent[OPT] type-binding[OPT] function-name argument-list[OPT] 
    return-intent[OPT] return-type[OPT] where-clause[OPT] function-body

proc-or-iter:
  `proc'
  `iter'

this-intent:
  `param'
  `ref'
  `type'

type-binding:
  identifier .
  `(' expr `)' .

\end{syntax}
Methods defined within the lexical scope of a class, record, or union
are referred to as \emph{primary methods}.  For such methods,
the \sntx{type-binding} is omitted and is taken to be the
innermost class, record, or union in which the method is defined.
Methods defined outside of such scopes are known as \emph{secondary
methods} and must have a \sntx{type-binding} (otherwise, they would
simply be standalone functions rather than methods).  Note that
secondary methods can be defined not only for classes, records, and
unions, but also for any other type (e.g., integers, reals, strings).

Secondary methods can be declared with a type expression instead of a
type identifier. In particular, if the \sntx{type-binding} is a
parenthesized expression, the compiler will evaluate that expression to
find the receiver type for the method. In that case, the method applies
only to receivers of that type. See also
\rsec{Creating_General_and_Specialized_Versions_of_a_Function}.

Method calls are described in \rsec{Class_Method_Calls}.

The use of \sntx{this-intent} is described in \rsec{The_em_this_Reference}.

\subsection{Nested Classes}
\label{Nested_Classes}
\index{classes!nested classes}
\index{nested classes}

A class or record defined within another class is a nested class (or record).

Nested classes or records can refer to fields and methods in the outer class (or
record) implicitly, or explicitly by means of an \chpl{outer} reference.

A nested class (or record) can be referenced only within
its immediately enclosing class (or record).

\section{Inheritance}
\label{Inheritance}
\index{inheritance}
\index{classes!inheritance}
\index{derived class}
\index{classes!derived}

A \emph{derived} class can inherit from one or more other classes by
listing those classes in the derived class declaration.
%REVIEW: vass: do we want to allow multiple base classes in the spec?
%REVIEW: vass: need to define "base" class and add it to index, if useful
When inheriting from multiple base classes, only one of the base classes
may contain fields.  The other classes can only define methods.  Note
that a class can still be derived from a class that contains fields
which is itself derived from a class that contains fields.  

%REVIEW: vass: the below ("tree") does not match the above ("multiple base classes")
These restrictions on inheritance induce a class hierarchy which has the form of
a tree.  A variable referring to an instance of class \chpl{C} can be
cast to any type that is an ancestor of \chpl{C}.  Note that casts to more- and
less-derived classes are both permitted.

It is possible for a class to inherit from a generic class. Suppose for
example that a class \chpl{C} inherits from class \chpl{ParentC}. In this
situation, \chpl{C} will have type constructor arguments based upon
generic fields in the \chpl{ParentC} as described in
~\ref{Type_Constructors}. Furthermore, a fully specified \chpl{C} will be
a subclass of a corresponding fully specified \chpl{ParentC}.

\begin{future}
A derived class may also incorporate any number of records by listing them
in the derived class declaration.
As with record inheritance, this has the effect of injecting the record's fields and
methods into the new class type.  Record inheritance does not induce a
well-defined class hierarchy.  See~\rsec{Record_Inheritance} for details.
\end{future}

\subsection{The object Class}
\label{The_object_Class}
\index{object@\chpl{object}}
\index{classes!object@\chpl{object}}

All classes are derived from the \chpl{object} class, either directly or
indirectly.  If no class name appears in the inheritance list, the class derives
implicitly from \chpl{object}.  Otherwise, a class derives from \chpl{object}
indirectly through the class or classes it inherits.  A variable of type \chpl{object}
can hold a reference to an object of any class type. 

\subsection{Accessing Base Class Fields}
\label{Accessing_Base_Class_Fields}
\index{classes!base!field access}
\index{classes!field access!base class}

A derived class contains data associated with the fields in its base
classes.  The fields can be accessed in the same way that they are
accessed in their base class unless a getter method is
overridden in the derived class, as discussed
in~\rsec{Overriding_Base_Class_Methods}.

\subsection{Derived Class Constructors}
\label{Derived_Class_Constructors}
\index{constructors!derived class}
\index{classes!derived!constructors}

The default initializer of a derived class automatically calls the default
initializer of each of its base classes %and records.  
The same is not true for constructors:
To initialize inherited fields to anything other than its default-initialized
value, a constructor defined in a derived class must either call base class
constructors or manipulate those base-class fields directly.

\begin{openissue}
The syntax for calling a base-class constructor from a derived-class constructor
has not yet been defined.

There is an expectation that a more standard way
of chaining constructor calls will be supported.
\end{openissue}

\subsection{Shadowing Base Class Fields}
\label{Shadowing_Base_Class_Fields}
\index{shadowing!base class fields}
A field in the derived class can be declared with the same name as a
field in the base class.  Such a field shadows the field in the base
class in that it is always referenced when it is accessed in the
context of the derived class.  

\begin{openissue}
There is an expectation that there will
be a way to reference the field in the base class but this is not
defined at this time.
\end{openissue}

\subsection{Overriding Base Class Methods}
\label{Overriding_Base_Class_Methods}
\index{dynamic dispatch}
\index{methods!base class!overriding}

If a method in a derived class is declared with a
signature identical to that of a method in a base class, then it is said to override the
base class's method.  Such a method is a candidate for dynamic
dispatch in the event that a variable that has the base class type
references an object that has the derived class type.

The identical signature requires that the names, types, and order of
the formal arguments be identical. The return type of the overriding
method must be the same as the return type of the base class's method,
or must be a subclass of the base class method's return type.

Methods without parentheses are not candidates for dynamic dispatch.
\begin{rationale}
Methods without parentheses are primarily used for field accessors.  
A default is created if none is specified.  The field accessor
should not dispatch dynamically since that would make it
impossible to access a base field within a base method should that
field be shadowed by a subclass.
\end{rationale}

\subsection{Inheriting from Multiple Classes}
\label{Inheriting_from_Multiple_Classes}
\index{multiple inheritance}
\index{inheritance!multiple}
\index{classes!inheritance!multiple}

%REVIEW: vass: need to decide as a group whether we want this in the spec
A class can be derived from multiple base classes provided that only
one of the base classes contains fields either directly or from base
classes that it is derived from.  The methods defined by the other
base classes can be overridden.  This provides functionality similar to the C\#
concept of interfaces.

\begin{openissue}
It is an open question whether the language will support \chpl{interface}
declarations and multiple inheritance. This is currently under study
at the University of Colorado (Boulder).
\end{openissue}

\subsection{The {\em nil} Value}
\label{Class_nil_value}
\index{classes!nil}
\index{nil@\chpl{nil}}

Chapel provides \chpl{nil} to indicate the absence of a reference to
any object.  \chpl{nil} can be assigned to a variable of any class
type.  Invoking a class method or accessing a field of the \chpl{nil}
value results in a run-time error.

\begin{syntax}
nil-expression:
  `nil'
\end{syntax}

\subsection{Default Initialization}
\label{Default_Initialization}
\index{classes!initialization}
\index{initialization!classes}
\index{classes!initialization!default}
\index{classes!default initialization}
\index{default initialization!classes}
\index{initialization!classes!default}

%REVIEW: vass:this does not correspond to the current implementation:
% * when invoking the compiler-generated constructor, default initialization
%   does not happen;
% * in many cases (?), fields' initialization expressions are not invoked.
When an instance of a class (an object) is created, it is brought to a
known and legal state first, before it can be accessed or operated upon.
This is done through default initialization.

An object is default-initialized by initializing all of its fields in
the order of the field declarations within the class. Fields inherited
from a superclass are initialized before fields declared in current class.

If a field in the class is declared with an initialization expression, that
expression is used to initialize the field.  Otherwise, the field is
initialized to the default value of its type
(\rsec{Default_Values_For_Types}).

\section{Class Constructors}
\label{Class_Constructors}
\label{Class_New}
\index{classes!constructors}
\index{new!classes}
\index{new@\chpl{new}}
\index{classes!new@\chpl{new}}
\index{constructors}

Class instances are created by invoking class constructors.
A class constructor is a method with the same name as the class.
It is invoked by the \chpl{new} operator, where the
class name and constructor arguments are preceded with the
\chpl{new} keyword.

When the constructor is called, memory is allocated to store
a class instance, the instance undergoes default initialization, and
then the constructor method is invoked on this newly-created
instance.

If the program declares a class constructor method,
it is a user-defined constructor.  
If the program declares no constructors for a class,
a compiler-generated constructor for that class is created automatically.

\subsection{User-Defined Constructors}
\label{User_Defined_Constructors}
\index{classes!constructors!user-defined}
\index{user-defined constructors}
\index{constructors!user-defined}

A user-defined constructor is a constructor method explicitly declared
in the program.  A constructor declaration has the same
syntax as a method declaration, except that the name of the function matches
the name of the class, and there is no return type specifier.

A constructor for a given class is called by placing the \chpl{new} operator
in front of the class name.  Any constructor arguments follow the class name in a
parenthesized list.

\begin{syntax}
constructor-call-expression:
  `new' class-name ( argument-list )

class-name:
  identifier
\end{syntax}

When a constructor is called, the usual function resolution mechanism
(\rsec{Function_Resolution}) is applied to determine which
user-defined constructor to invoke.

\begin{chapelexample}{simpleConstructors.chpl}
The following example shows a class with two constructors:
\begin{chapel}
class MessagePoint {
  var x, y: real;
  var message: string;

  proc MessagePoint(x: real, y: real) {
    this.x = x;
    this.y = y;
    this.message = "a point";
  }

  proc MessagePoint(message: string) {
    this.x = 0;
    this.y = 0;
    this.message = message;
  }
}  // class MessagePoint

// create two objects
var mp1 = new MessagePoint(1.0,2.0);
var mp2 = new MessagePoint("point mp2");
\end{chapel}
\begin{chapelpost}
writeln(mp1);
writeln(mp2);
delete mp1;
delete mp2;
\end{chapelpost}
\begin{chapeloutput}
{x = 1.0, y = 2.0, message = a point}
{x = 0.0, y = 0.0, message = point mp2}
\end{chapeloutput}
The first constructor lets the user specify the initial coordinates
and the second constructor lets the user specify the initial message
when creating a MessagePoint.
\end{chapelexample}

Constructors for generic classes (\rsec{Generic_Types}) handle certain
arguments differently and may need to satisfy additional
requirements. See Section~\ref{Generic_User_Constructors} for details.

\subsection{The Compiler-Generated Constructor}
\label{The_Compiler_Generated_Constructor}
\index{classes!constructors!compiler-generated}
\index{constructors!compiler-generated}
\index{compiler-generated constructors}

A compiler-generated constructor for a class is created automatically
if there are no constructors for that class in the program.
The compiler-generated constructor has one argument for every field in the class,
each of which has a default value equal to the field's initializer (if present) or default value of the field's type (if not).
The list of fields (and hence arguments) includes fields inherited from superclasses, type aliases
and parameter fields, if any.
The order of the arguments in the argument list matches the order of the field declarations
within the class, with the arguments for a superclass's fields occurring
before the arguments for the fields declared in current class.

Generic fields are discussed in Section~\rsec{Generic_Compiler_Generated_Constructors}.

When invoked, the compiler-generated constructor initializes each field in the class to the
value of the corresponding actual argument.  

%
% BLC: I'm not comfortable enough with this to put it into the release;
% I've made minor mods to the above to present what I hope is a sufficiently
% consistent story.  We're going to need to revisit this whole section upon
% finishing constructors anyway
%
%% In contrast to an actual function
%% call (including calls to user-defined constructors), the compiler-generated
%% constructor only initializes fields for which arguments are supplied.  All other
%% fields retain their default-initialized values.  In this sense the
%% compiler-generated constructor is actually a family of constructors, one for
%% each permissible combination of named and unnamed arguments.\footnote{This makes
%% it clear why the compiler-generated constructor is visible only if no
%% user-defined constructors are present.}

%REVIEW: vass
% I am unhappy with having to present this as "not an actual function call".
% Background: two ways for presenting constructors have been used.
% Previously:
% - compiler-generated constructor is always there
% - it is implicitly invoked at the beginning of a user-defined constructor
% - there is no default initialization
% Currently
% - compiler-generated constructor exists only if no user-defined constructors
% - default initialization always happens before a constructor is invoked
% - compiler-generated constructor initializes only the fields for which
%   its invocation provides actual arguments
%   (hence it is magic, unlike a function call)
% 
% The motivation for this refactoring is:
% - give the reader a firm understanding that an object is zeroed-out
%   before they can do anything with it, and
% - eliminate compiler-generated constructor when there is a user-defined one
% This leads us to the "unlike a function call" semantics here -
% to avoid duplicate invocation of fields' default initializers
% (which otherwise would happen once during the object's default initialization
% and the second time upon invoking the compiler-generated constructor).
% But overall I do not feel that this refactoring is a win, especially
% because we now need to define the magic (and I am not sure presently
% it is defined formally enough).

\begin{chapelexample}{defaultConstructor.chpl}
Given the class
\begin{chapel}
class C {
  var x: int;
  var y: real = 3.14;
  var z: string = "Hello, World!";
}
\end{chapel}
\begin{chapelpost}
var c1 = new C();
var c2 = new C(2);
var c3 = new C(z="");
var c4 = new C(2, z="");
var c5 = new C(0, 0.0, "");
writeln((c1, c2, c3, c4, c5));
delete c1;
delete c2;
delete c3;
delete c4;
delete c5;
\end{chapelpost}
\begin{chapeloutput}
({x = 0, y = 3.14, z = Hello, World!}, {x = 2, y = 3.14, z = Hello, World!}, {x = 0, y = 3.14, z = }, {x = 2, y = 3.14, z = }, {x = 0, y = 0.0, z = })
\end{chapeloutput}
there are no user-defined constructors for \chpl{C}, so \chpl{new} operators
will invoke \chpl{C}'s compiler-generated constructor. The \chpl{x} argument
of the compiler-generated constructor has the default value \chpl{0}.
The \chpl{y} and \chpl{z} arguments have the default values \chpl{3.14} and
\chpl{"Hello, World\!"}, respectively.

\chpl{C} instances can be created by calling the compiler-generated constructor as follows:
\begin{itemize}
\item The call \chpl{new C()} is equivalent to \chpl{C(0,3.14,"Hello, World\!")}.
\item The call \chpl{new C(2)} is equivalent to \chpl{C(2,3.14,"Hello, World\!")}.
\item The call \chpl{new C(z="")} is equivalent to \chpl{C(0,3.14,"")}.
\item The call \chpl{new C(2, z="")} is equivalent to \chpl{C(2,3.14,"")}.
\item The call \chpl{new C(0,0.0,"")} specifies the initial values for all fields explicitly.
\end{itemize}
\end{chapelexample}

\section{Field Accesses}
\label{Class_Field_Accesses}
\index{classes!field access}
\index{field access!class}

The field in a class is accessed via a field access expression.

\begin{syntax}
field-access-expression:
  receiver-clause[OPT] identifier

receiver-clause:
  expression .
\end{syntax}

\index{classes!receiver}
\index{receiver!class}
% TODO: these are the rules to determine the receiver - move them
% to a separate section so they can be uniformly referenced from everywhere.
The receiver-clause specifies the \emph{receiver}, which is the class
instance whose field is being accessed.
The receiver clause can be omitted when the field access is within a method.
In this case the receiver is the method's receiver \rsec{The_em_this_Reference}.
The receiver clause can also be omitted when the field access is within
a class declaration. In this case the receiver is the instance
being implicitly defined or referenced.

The identifier in the field access expression indicates which field is accessed.

% TODO: rephrase all this in terms of the getter methods

A field can
be modified via an assignment statement where the left-hand side of
the assignment is a field access expression.
Accessing a parameter field returns a parameter.

\begin{chapelexample}{useActor1.chpl}
Given a variable \chpl{anActor} of type \chpl{Actor} as defined above,
the code
\begin{chapelpre}
use defineActor;
var anActor = new Actor(name="Tommy", age=26);
\end{chapelpre}
\begin{chapel}
var s: string = anActor.name;
anActor.age = 27;
\end{chapel}
\begin{chapelpost}
writeln((s, anActor));
if (cleanUp) then delete anActor;
\end{chapelpost}
\begin{chapelcompopts}
-scleanUp=true
\end{chapelcompopts}
\begin{chapeloutput}
(Tommy, {name = Tommy, age = 27})
\end{chapeloutput}
reads the field \chpl{name} and assigns the value to the variable
\chpl{s}, and assigns the field \chpl{age} in the object
\chpl{anActor} the value \chpl{27}.
\end{chapelexample}

\subsection{Variable Getter Methods}
\label{Getter_Methods}
\index{classes!getter method}
\index{getter method!class}
\index{methods!class!getter}

All field accesses are performed via getters.  A getter is a method without
parentheses with the same name as the field. It is defined in the field's class
and has a \chpl{ref} return intent (\rsec{Ref_Return_Intent}).  If the program
does not define it, the default getter, which simply returns the field, is
provided.

\begin{chapelexample}{getterSetter.chpl}
In the code
\begin{chapel}
class C {
  var setCount: int;
  var x: int;
  proc x ref {
    setCount += 1;
    return x;
  }
  proc x {
    return x;
  }

}
\end{chapel}
\begin{chapelpost}
var c = new C();
c.x = 1;
writeln(c.x);
c.x = 2;
writeln(c.x);
c.x = 3;
writeln(c.x);
writeln(c.setCount);
delete c;
\end{chapelpost}
\begin{chapeloutput}
1
2
3
3
\end{chapeloutput}
an explicit variable getter method is defined for field \chpl{x}.  It
returns the field \chpl{x} and increments another field that records
the number of times x was assigned a value.
\end{chapelexample}

\section{Class Method Calls}
\label{Class_Method_Calls}
\index{classes!method calls}
\index{methods!calling}

A method is invoked with a method call, which is similar to a non-method
call expression.

\begin{syntax}
method-call-expression:
  receiver-clause[OPT] expression ( named-expression-list )
  receiver-clause[OPT] expression [ named-expression-list ]
  receiver-clause[OPT] parenthesesless-function-identifier
\end{syntax}

The receiver-clause (or its absence) specifies the method's receiver
\rsec{The_em_this_Reference} in the same way it does for field accesses
\rsec{Class_Field_Accesses}.

\begin{chapelexample}{defineMethod.chpl}
A method to output information about an instance of the \chpl{Actor}
class can be defined as follows:
\begin{chapelpre}
use useActor1;
\end{chapelpre}
\begin{chapel}
proc Actor.print() {
  writeln("Actor ", name, " is ", age, " years old");
}
\end{chapel}
\begin{chapelpost}
anActor.print();
delete anActor;
\end{chapelpost}
\begin{chapeloutput}
(Tommy, {name = Tommy, age = 27})
Actor Tommy is 27 years old
\end{chapeloutput}
This method can be called on an instance of the \chpl{Actor}
class, \chpl{anActor}, with the call expression \chpl{anActor.print()}.
\end{chapelexample}

The actual arguments supplied in the method call are bound to the
formal arguments in the method declaration following the rules specified for
procedures (\rsec{Functions}). The exception is the receiver
\rsec{The_em_this_Reference}.

\subsection{The Method Receiver and the {\em this} Argument}
\label{The_em_this_Reference}
\index{classes!receiver}
\index{this@\chpl{this}}
\index{classes!this@\chpl{this}}
\index{receiver}
\index{type methods}
\index{instance methods}
\index{methods!type}
\index{methods!instance}

A method's \emph{receiver} is an implicit formal argument
named \chpl{this} representing the expression on which the method is
invoked.  The receiver's actual argument is specified by the
\sntx{receiver-clause} of a method-call-expression as specified
in \rsec{Class_Field_Accesses}.  



% TODO: specify how the receiver affects the choice of the method.

\begin{chapelexample}{implicitThis.chpl}
Let class \chpl{C}, method \chpl{foo}, and function \chpl{bar} be
defined as
\begin{chapel}
class C {
  proc foo() {
    bar(this);
  }
}
proc bar(c: C) { writeln(c); }
\end{chapel}
\begin{chapelpost}
var c1: C = new C();
c1.foo();
delete c1;
\end{chapelpost}
\begin{chapeloutput}
{}
\end{chapeloutput}
Then given an instance of \chpl{C} called \chpl{c1}, the method
call \chpl{c1.foo()} results in a call to \chpl{bar} where the
argument is \chpl{c1}.  Within primary method \chpl{C.foo()}, the
(implicit) receiver formal has static type \chpl{C} and is referred to
as \chpl{this}.
\end{chapelexample}

Methods whose receivers are objects are called \emph{instance
methods}.  Methods may also be defined to have \chpl{type}
receivers---these are known as \emph{type methods}.

The optional \sntx{this-intent} is used to specify type methods, to
constrain a receiver argument to be a \chpl{param}, or to specify how
the receiver argument should be passed to the method.

A method whose \sntx{this-intent} is \chpl{type} defines a \emph{type
method}.  It can only be called on the type itself rather than on an
instance of the type.  When \sntx{this-intent} is \chpl{param}, it
specifies that the function can only be applied to param objects of
the given type binding.

\begin{chapelexample}{paramTypeThisIntent.chpl}
In the following code, the \chpl{isOdd} method is defined with
a \sntx{this-intent} of \chpl{param}, permitting it to be called on
params only.  The \chpl{size} method is defined with
a \sntx{this-intent} of \chpl{type}, requiring it to be called on
the \chpl{int} type itself, not on integer values.
\begin{chapel}
proc param int.isOdd() param {
  return this & 0x1 == 0x1;
}

proc type int.size() param {
  return 64;
}

param three = 3;
var seven = 7;

writeln(42.isOdd());          // prints false
writeln(three.isOdd());       // prints true
writeln((42+three).isOdd());  // prints true
// writeln(seven.isOdd());    // illegal since 'seven' is not a param

writeln(int.size());          // prints 64
// writeln(42.size());        // illegal since 'size()' is a type method
\end{chapel}
\begin{chapeloutput}
false
true
true
64
\end{chapeloutput}
\end{chapelexample}

\pagebreak
Type methods can also be iterators.

\begin{chapelexample}{typeMethodIter.chpl}
In the following code, the class \chpl{C} defines a type method
iterator which can be invoked on the type itself:
\begin{chapel}
class C {
  var x: int;
  var y: string;

  iter type myIter() {
    yield 3;
    yield 5;
    yield 7;
    yield 11;
  }
}

for i in C.myIter() do
  writeln(i);
\end{chapel}
\begin{chapeloutput}
3
5
7
11
\end{chapeloutput}
\end{chapelexample}

When \sntx{this-intent} is \chpl{ref}, the receiver argument will be
passed by reference, allowing modifications to \chpl{this}.  If
no \sntx{this-intent} is specified, the receiver will be passed with
the default intent as specified in \rsec{The_Default_Intent}.

\begin{chapelexample}{refThisIntent.chpl}
In the following code, the \chpl{doubleMe} function is defined with a
\sntx{this-intent} of \chpl{ref}, allowing variables of type \chpl{int} to
double themselves.
\begin{chapel}
proc ref int.doubleMe() { this *= 2; }
\end{chapel}
\begin{chapelpost}
var x: int = 2;
x.doubleMe();
writeln(x);
\end{chapelpost}
\begin{chapeloutput}
4
\end{chapeloutput}
Given a variable \chpl{x = 2}, a call to \chpl{x.doubleMe()} will set \chpl{x}
to \chpl{4}.
\end{chapelexample}

\section{The {\em this} Method}
\label{The_em_this_Method}
\index{classes!indexing}
\index{this@\chpl{this}}
\index{classes!this@\chpl{this}}

A procedure method declared with the name \chpl{this} allows a class to be
``indexed'' similarly to how an array is indexed.  Indexing into a
class instance has the semantics of calling a method
named \chpl{this}.  There is no other way to call a method
called \chpl{this}.  The \chpl{this} method must be declared with
parentheses even if the argument list is empty.

\begin{chapelexample}{thisMethod.chpl}
In the following code, the \chpl{this} method is used to create a
class that acts like a simple array that contains three integers
indexed by 1, 2, and 3.
\begin{chapel}
class ThreeArray {
  var x1, x2, x3: int;
  proc this(i: int) ref {
    select i {
      when 1 do return x1;
      when 2 do return x2;
      when 3 do return x3;
    }
    halt("ThreeArray index out of bounds: ", i);
  }
}
\end{chapel}
\begin{chapelpost}
var ta = new ThreeArray();
ta(1) = 1;
ta(2) = 2;
ta(3) = 3;
for i in 1..3 do
  writeln(ta(i));
ta(4) = 4;
\end{chapelpost}
\begin{chapeloutput}
1
2
3
thisMethod.chpl:9: error: halt reached - ThreeArray index out of bounds: 4
\end{chapeloutput}
\end{chapelexample}

\section{The {\em these} Method}
\label{The_these_Method}
\index{classes!iterating over}
\index{these@\chpl{these}}
\index{classes!these@\chpl{these}}

An iterator method declared with the name \chpl{these} allows a class object to be
``iterated over'' similarly to how a domain or array supports iteration.
Using a class in the context of a loop where
an \sntx{iteratable-expression} is expected has the semantics of calling
a method on the class named \chpl{these}.

\begin{chapelexample}{theseIterator.chpl}
In the following code, the \chpl{these} method is used to create a
class that acts like a simple array that can be iterated over and
contains three integers.
\begin{chapel}
class ThreeArray {
  var x1, x2, x3: int;
  iter these() ref {
    yield x1;
    yield x2;
    yield x3;
  }
}
\end{chapel}
\begin{chapelpost}
var ta = new ThreeArray();
for (i, j) in zip(ta, 1..) do
  i = j;

for i in ta do
  writeln(i);
delete ta;
\end{chapelpost}
\begin{chapeloutput}
1
2
3
\end{chapeloutput}

\end{chapelexample}

An iterator type method with the name \chpl{these} supports iteration
over the class type itself.

\begin{chapelexample}{typeMethodIterThese.chpl}
In the following code, the class \chpl{C} defines a type method
iterator named \chpl{these}, supporting direct iteration over the type:
\begin{chapel}
class C {
  var x: int;
  var y: string;

  iter type these() {
    yield 1;
    yield 2;
    yield 4;
    yield 8;
  }
}

for i in C do
  writeln(i);
\end{chapel}
\begin{chapeloutput}
1
2
4
8
\end{chapeloutput}
\end{chapelexample}

\section{Common Operations}

\subsection{Class Assignment}
\label{Class_Assignment}
\index{classes!assignment}
\index{assignment!class}

Classes are assigned by reference.  After an assignment from one
variable of a class type to another, both variables reference the same
class instance.

\subsection{Implicit Class Conversions}
\label{Implicit_Class_Conversions}
\index{conversions!class}
\index{conversions!implicit!class}
\index{classes!implicit conversion}

An implicit conversion from class type \chpl{D} to
another class type \chpl{C} is allowed when \chpl{D} is a subclass
of \chpl{C}.
The value \chpl{nil} can be implicitly converted to any class type.
These conversions do not change the value.


%TODO: Move memory management explanation up, closer to class constructors.
% Perhaps make the memory management part of the introduction, and then let the
% description of deinitializers appear naturally at the same indentation level as
% constructors.
\section{Dynamic Memory Management}
\label{Dynamic_Memory_Management}
\label{Class_Delete}
\index{memory management}
\index{classes!delete}
\index{delete!classes}

Memory associated with class instances can be reclaimed with the \chpl{delete}
statement:

\begin{syntax}
delete-statement:
  `delete' expression ;
\end{syntax}

where the expression is a reference to the instance that will be reclaimed.
The expression may evaluate to \chpl{nil}, in which case the \chpl{delete}
statement has no effect.  If an object is referenced after it has
been deleted, the behavior is undefined.

\begin{chapelexample}{delete.chpl}
The following example allocates a new object \chpl{c} of class type \chpl{C}
and then deletes it.
\begin{chapelpre}
class C {
  var i,j,k: int;
}
\end{chapelpre}
\begin{chapel}
var c : C = nil;
delete c;        // Does nothing: c is nil.

c = new C();     // Creates a new object.
delete c;        // Deletes that object.

// The following statements reference an object after it has been deleted, so
// the behavior of each is "undefined":
// writeln(c.i); // May read from freed memory.
// c.i = 3;      // May overwrite freed memory.
// delete c;     // May confuse some allocators.
\end{chapel}
\begin{chapelpost}
\end{chapelpost}
\begin{chapelexecopts}
--memLeaksByType
\end{chapelexecopts}
\begin{chapeloutput}

====================
Leaked Memory Report
==============================================================
Number of leaked allocations
           Total leaked memory (bytes)
                      Description of allocation
==============================================================
==============================================================
\end{chapeloutput}
\end{chapelexample}

\begin{openissue}
Chapel was originally specified without a \chpl{delete} keyword.  The intention
was that Chapel would be implemented with a distributed-memory garbage
collector.  This is a research challenge.  In order to focus elsewhere, the
design has been scaled back.  There is an expectation that Chapel will
eventually support an optional distributed-memory garbage collector as well as
a region-based memory management scheme similar to that used in the Titanium
language.  Support of \chpl{delete} will likely continue even as these optional
features become supported.
\end{openissue}


\subsection{Class Deinitializer}
\label{Class_Deinitializer}
\index{classes!deinitializer}
\index{deinitializer!classes}

A class author may specify additional actions to be performed before a class object is
reclaimed, by defining a class deinitializer.  A class deinitializer is a method
named \chpl{deinit}.  A class deinitializer takes no arguments (aside from
the implicit \chpl{this} argument).  If defined, the deinitializer is called each time
a \chpl{delete} statement is invoked with a valid instance of that class type.  The
deinitializer is not called if the argument of \chpl{delete} evaluates to \chpl{nil}.

\begin{chapelexample}{classDeinitializer.chpl}
\begin{chapel}
class C {
  var i,j,k: int;
  proc deinit() { writeln("Bye, bye."); }
}

var c : C = nil;
delete c;        // Does nothing: c is nil.

c = new C();     // Creates a new object.
delete c;        // Deletes that object: Writes out "Bye, bye." 
                 // and reclaims the memory that was held by c.
\end{chapel}
\begin{chapeloutput}
Bye, bye.
Bye, bye.
\end{chapeloutput}
\end{chapelexample}

\cleardoublepage
\sekshun{Records}
\label{Records}
\index{records}

A record is a data structure that is similar to a class but instead has value
semantics, similar to primitive types.  Value semantics mean that assignment, argument passing and function
return values are by default all done by copying.  Value semantics also imply that a
variable of record type is associated with only one piece of storage and has
only one type throughout its lifetime.  Storage is allocated for a variable of
record type when the variable declaration is executed, and the record variable
is also initialized at that time.

A record declaration creates a record type~\rsec{Record_Declarations}.  A
variable of record type contains all and only the fields defined by that type
(\rsec{Record_Types}).  Value semantics imply that the type of a record variable
is known at compile time (i.e. it is statically typed).  

Records can be created through an explicit call to a record
constructor, which allocates storage, initializes
it and returns it.  
A record is also created upon a variable declaration of a record type.

A record type is generic if it contains generic fields.  Generic record types
are discussed in detail in~\rsec{Generic_Types}.

\section{Record Declarations}
\label{Record_Declarations}
\index{records!declarations}
\index{declarations!records}
\index{record@\chpl{record}}

A record is defined with the following syntax:
\begin{syntax}
record-declaration-statement:
  simple-record-declaration-statement
  external-record-declaration-statement

simple-record-declaration-statement:
  `record' identifier record-inherit-list[OPT] { record-statement-list }

record-inherit-list:
  : record-type-list

record-type-list:
  record-type
  record-type , record-type-list

record-statement-list:
  record-statement
  record-statement record-statement-list

record-statement:
  variable-declaration-statement
  method-declaration-statement
  type-declaration-statement
  empty-statement
\end{syntax}

A \sntx{record-declaration-statement} defines a new type symbol specified by the
identifier.  A record inherits data and methods from other records
if the \sntx{record-inherit-list} is specified.

\begin{future}
Allowing a record to inherit from more than one record is future work.
\end{future}

\begin{rationale}
We do not allow records to inherit from classes because of the following.

Inheritance implies that the derived type can be cast to one of its base types.
If the base type is a record type, casting to the base type has the effect of
removing all of the data fields and all of functions that are not associated
with the base type.  Thereafter, the record variable has the base record type,
in both compile-time and run-time interpretations.

If the base type were a class type, the result of the cast would have the static
type of the base class while its run-time type was a record type.  Since a
record's type is supposed to be determined at compile time, this is a bit
incongruous with the definition of a record.  Moreover, space would have to be
allocated in this special case, to store the record's run-time type.
\end{rationale}

As in a class declarations, the body of a record declaration can contain
variable, iterator and method declarations as well as nested type declarations.

If a record declaration contains a type alias or parameter field, or it contains
a variable or constant without a specified type and without an initialization
expression, then it declares a generic record type.  Generic record types are
described in~\rsec{Generic_Types}.

If the \chpl{extern} keyword appears before the \chpl{record} keyword, then an
external record type is declared.  An external record type declaration must not
contain a \sntx{record-inherit-list}.  An external record is used within Chapel
for type and field resolution, but no corresponding backend definition is
generated.  It is presumed that the definition of an external record is supplied
by a library or the execution environment.  See the chapter on interoperability
(\rsec{Interoperability}) for more information on external records.
% External record inheritance is not currently supported.

\subsection{Record Types}
\label{Record_Types}
\index{records!record types}
\index{records!types}
\index{types!records}

A record type specifier simply names a record type, using
the following syntax:
\begin{syntax}
record-type:
  identifier
  identifier ( named-expression-list )
\end{syntax}
A record type specifier may appear anywhere a type specifier is permitted.

For non-generic records, the record name by itself is sufficient to specify the
type.  Generic records must be instantiated to serve as a fully-specified
type, for example to declare a variable.  This is done with
type constructors, which are defined in Section~\ref{Type_Constructors}.

\subsection{Record Fields}
\label{Record_Fields}
\index{records!fields}
\index{fields!records}

Variable declarations within a record type declaration define fields within that
record type.  The presence of at least one parameter field causes the record
type to become generic.  Variable fields define the storage associated with a
record.

\begin{chapelexample}{defineActorRecord.chpl}
The code
\begin{chapel}
record ActorRecord {
  var name: string;
  var age: uint;
}
\end{chapel}
\begin{chapeloutput}
\end{chapeloutput}
defines a new record type called \chpl{ActorRecord} that has two fields: the
string field \chpl{name} and the unsigned integer field \chpl{age}.  The data
contained by a record of this type is exactly the same as that contained by
an instance of the \chpl{Actor} class defined in the preceding chapter~\rsec{Class_Fields}.
\end{chapelexample}

\subsection{Record Methods}
\label{Record_Methods}
\index{records!methods}
\index{methods!records}

A record method is a function or iterator that is bound to a record.  Unlike
functions that take a record as an argument, record methods access the record by
reference, so that persistent field updates are possible.

The syntax for record method declarations is identical to that for class method
declarations (\rsec{Class_Methods}).

\subsection{Nested Record Types}
\label{Nested_Record_Types}
\index{nested records}
\index{records!nested}

Record type declarations may be nested within other class, record and union
declarations.  Methods defined in a nested record type may access fields
declared in the containing aggregate type either implicitly, or explicitly by
means of an \chpl{outer} reference.

\section{Record Inheritance}
\label{Record_Inheritance}
\index{records!inheritance}
\index{inheritance!records}

A \emph{derived} record type is a type that inherits from other record types.  For each named
base record type, inheritance effectively inserts all of its fields and methods
into the new record type.

Since record types are resolved statically, there is no type hierarchy implied
by record inheritance.  It is merely a shorthand for including a list of fields
in the record (or class) type being defined.  Record inheritance can be useful
for grouping data common to several or class or record types.

\begin{future}
From the definition of record inheritance, it is apparent that a record of a
derived type can be cast legally to any of its base record types.  But given
their semantics, records can also be legally cast to types with which they have
no inheritance relationship.  Thus, records do not induce a well-defined type
hierarchy.
\end{future}

\begin{chapelexample}{recordInheritance.chpl}
\begin{chapel}
record Center { var x, y: real; }
record Circle : Center {
  var radius: real;
}
record Ellipse : Center {
  var major, minor: real;
}
\end{chapel}
\begin{chapeloutput}
\end{chapeloutput}
The record \chpl{Center} is defined and used as a shorthand in defining
the \chpl{Circle} and \chpl{Ellipse} records.  The \chpl{Circle} record contains
three \chpl{real} fields named \chpl{x}, \chpl{y} and \chpl{radius}.  The
\chpl{Ellipse} record contains four \chpl{real} fields named \chpl{x}, \chpl{y},
\chpl{major} and \chpl{minor}.
\end{chapelexample}

The syntax and semantics for accessing methods (including getter methods and
hence fields) in a base
record type is the same as for accessing fields in a base class (\rsec{Accessing_Base_Class_Fields}).

\subsection{Shadowing Base Record Fields}
\label{Shadowing_Base_Record_Fields}
\index{records!fields!shadowing}

A field in the derived record can be declared with the same name as a
field in a base record.  Such a field shadows the field in the base
record, meaning that the field by the same name in the base record is not
directly accessible.

\begin{openissue}
A syntax for accessing shadowed fields has not yet been specified.
\end{openissue}

\subsection{Overriding Base Record Methods}
\label{Overriding_Base_Record_Methods}
\index{records!base method!overriding}

\begin{future}
If a method in a derived record is declared with a signature identical to that
of a method in a base record, then it is said to override the
base record's method.  Since records do not support dynamic dispatch, method
overriding is the same as method shadowing: When referenced via the derived
record type, the derived type's version of the method is called; when referenced
via the base record type, the base record type's version of the method is called.

The identical signature requires that the names, types, and order of
the formal arguments be identical. The return type of the overriding
method must be the same as the return type of the base record's method,
or must be a subrecord of the base record method's return type.
\end{future}

\section{Record Variable Declarations}
\label{Record_Variable_Declarations}
\index{records!variable declarations}
\index{variables!records}

A record variable declaration is a variable declaration using a record type.
When a variable of record type is declared, storage is allocated sufficient to
store all of the fields defined in that record type.  

In the context of a class or record or union declaration, the fields are
allocated within the object as if they had been declared individually.  In this
sense, records provide a way to group related fields within a containing class
or record type. 

In the context of a function body, a record variable declaration
causes storage to be allocated sufficient to store all of the fields in that
record type.  The record variable is initialized through a call to its
default initializer.  The default initializer for a record is defined in the
same way as the default initializer for a class (\rsec{Default_Initialization}).

\subsection{Storage Allocation}
\label{Record_Storage}
\index{records!allocation}

Storage for a record variable directly contains the data associated
with the fields in the record, in the same manner as variables
of primitive types directly contain the primitive values.
Record storage is reclaimed when the record variable goes out of scope.
No additional storage for a record is allocated or reclaimed.
Field data of one variable's record is not shared with data
of another variable's record.

\subsection{Record Initialization}
\label{Record_Initialization}
\index{records!initialization}
\index{initialization!record}

A variable of a record type declared without an initialization expression
is initialized through a call to the record's default initializer, passing no arguments.
The default initializer for a record is defined in the same way as the default
initializer for a class (\rsec{Default_Initialization}).

If the new record type is derived from other record types, the
default initializer for each base record will be called in lexical order before default
initializer for the record itself.

To construct a record as an expression,
i.e. without binding it to a variable, the \chpl{new} operator is
required.  In this case, storage is allocated and reclaimed as for a record
variable declaration (\rsec{Record_Storage}), except that the temporary record
goes out of scope at the end of the enclosing expression.

To initialize a record variable with a non-default value, it can be assigned
the value of a constructor call expression.  The constructors for a record are
defined in the same way as those for a class (\rsec{Class_Constructors}).

\begin{rationale}
The \chpl{new} keyword disambiguates types from values. This is needed because of the close
relationship between constructors and type specifiers for classes and
records.
\end{rationale}

\begin{chapelexample}{recordCreation.chpl}
The program
\begin{chapel}
record TimeStamp {
  var time: string = "1/1/1011";
}

var timestampDefault: TimeStamp;                  // use the default for 'time'
var timestampCustom = new TimeStamp("2/2/2022");  // ... or a different one
writeln(timestampDefault);
writeln(timestampCustom);

var idCounter = 0;
record UniqueID {
  var id: int;
  proc UniqueID() { idCounter += 1; id = idCounter; }
}

writeln(new UniqueID());  // create and use a record value without a variable
writeln(new UniqueID());
\end{chapel}
produces the output
\begin{chapelprintoutput}{}
(time = 1/1/1011)
(time = 2/2/2022)
(id = 1)
(id = 2)
\end{chapelprintoutput}
The variable \chpl{timestampDefault} is initialized with \chpl{TimeStamp}'s
default initializer. The expression \chpl{new TimeStamp} creates a record that
is assigned to \chpl{timestampCustom}.  It effectively
initializes \chpl{timestampCustom} via a call to the constructor with desired
arguments. The records created with \chpl{new UniqueID()} are discarded after
they are used.
\end{chapelexample}

As with classes, the user can provide his own constructors
(\rsec{User_Defined_Constructors}).  If any user-defined constructors are
supplied, the default initializer cannot be called directly.  

\section{Record Arguments}
\label{Record_Arguments}
\index{records!arguments}
\index{arguments!records}

When records are copied into or out of a function's formal argument,
the copy is performed consistently with the semantics described for
record assignment (\rsec{Record_Assignment}).

\begin{chapelexample}{paramPassing.chpl}
The program
\begin{chapel}
record MyColor {
  var color: int;
}
proc printMyColor(in mc: MyColor) {
  writeln("my color is ", mc.color);
  mc.color = 6;   // does not affect the caller's record
}
var mc1: MyColor;        // 'color' defaults to 0
var mc2: MyColor = mc1;  // mc1's value is copied into mc2
mc1.color = 3;           // mc1's value is modified
printMyColor(mc2);       // mc2 is not affected by assignment to mc1
printMyColor(mc2);       // ... or by assignment in printMyColor()

proc modifyMyColor(inout mc: MyColor, newcolor: int) {
  mc.color = newcolor;
}
modifyMyColor(mc2, 7);   // mc2 is affected because of the 'inout' intent
printMyColor(mc2);
\end{chapel}
produces
\begin{chapelprintoutput}{}
my color is 0
my color is 0
my color is 7
\end{chapelprintoutput}
The assignment to \chpl{mc1.color} affects only the record stored
in \chpl{mc1}. The record in \chpl{mc2} is not affected by
the assignment to \chpl{mc1} or by the assignment in \chpl{printMyColor}.
\chpl{mc2} is affected by the assignment in \chpl{modifyMyColor}
because the intent \chpl{inout} is used.
\end{chapelexample}

\section{Record Field Access}
\label{Record_Field_Access}
\index{records!field access}
\index{field access}

A record field is accessed the same way as a class field
(\rsec{Class_Field_Accesses}).  When a field access is used as an
rvalue, the value of that field is returned.  When it is used as
an lvalue, the value of the record field is updated.

Member access expressions that access parameter fields
produce a parameter.

\subsection{Field Getter Methods}
\label{Field_Getter_Methods}
\index{records!getters}

As in classes, field accesses are performed via getter methods
(\rsec{Getter_Methods}).  By default, these methods simply return a reference to
the specified field (so they can be written as well as read).  The user may
redefine these as needed.

\section{Record Method Calls}
\label{Record_Method_Access}
\index{records!method calls}
\index{method calls}

A record method may be invoked the same way as a class method
(\rsec{Class_Method_Calls}).  Unlike class methods, record methods are
resolved at compile time.  

\subsection{The Method Receiver and the {\em this} Argument}
\label{The_this_Reference}
\index{this@\chpl{this}}
\index{records!receiver}
\index{receiver}

The \emph{receiver} of a record method is similar to and is determined in the
same way as the receiver of a class method (\rsec{The_em_this_Reference}).
The type of the receiver is the record in which the method is defined.
The receiver formal argument can be referred to within the method
using the identifier \chpl{this}.

The difference from a class method is that the receiver actual argument,
which must be a record value, is passed to the record method by reference,
rather than by copying. Therefore updates to the receiver made in the
method, if any, are visible outside the method.

\section{The {\em this} Method}
\index{records!indexing}

As with classes, records can be supplied with a \chpl{this} method.  This method
defines the behavior of the indexing operator \chpl{[]}.

\section{The {\em these} Method}
\index{records!iterating}

A \chpl{these} method can be defined for records as well as classes (\rsec{The_these_Method}).  It
provides an iterator which iterates over the contents of the record in a
user-defined manner.

\section{Common Operations}

\subsection{Record Assignment}
\label{Record_Assignment}
\index{records!assignment}

A variable of record type may be updated by assignment.  The compiler provides
a default assignment operator for each record type \chpl{R} having the signature
\begin{example}
\begin{chapel}
proc =(ref lhs:R, rhs:R) : void ;
\end{chapel}
\end{example}
\noindent
In it, the value of each field of the record on the right-hand side is assigned
to the corresponding field of the record on the left-hand side.

The compiler-provided assignment operator may be overridden.

The following example demonstrates record assignment.
\begin{chapelexample}{assignment.chpl}
\begin{chapel}
record R {
  var i: int;
  var x: real;
  proc print() { writeln("i = ", this.i, ", x = ", this.x); }
}
var A: R;
A.i = 3;
A.print();	// "i = 3, x = 0.0"

var C: R;
A = C;
A.print();	// "i = 0, x = 0.0"

C.x = 3.14;
A.print();	// "i = 0, x = 0.0"
\end{chapel}
\begin{chapeloutput}
i = 3, x = 0.0
i = 0, x = 0.0
i = 0, x = 0.0
\end{chapeloutput}
Prior to the first call to \chpl{R.print}, the record \chpl{A} is created and
initialized to all zeroes.  Then, its \chpl{i} field is set to \chpl{3}.
For the second call to \chpl{R.print}, the record \chpl{C} is created assigned
to \chpl{A}.  Since \chpl{C} is default-initialized to all zeroes, those zero
values overwrite both values in \chpl{A}.

The next clause demonstrates that \chpl{A} and \chpl{C} are distinct entities,
rather than two references to the same object.  Assigning \chpl{3.14}
to \chpl{C.x} does not affect the \chpl{x} field in \chpl{A}.
\end{chapelexample}

%REVIEW: vass: need to define "reference assignment"
% and ideally remove the reference to C++.
% Also, this seems not specific to records and so should be moved
% to the "assignment statement" section.
\begin{openissue}
Whether reference assignment is to be supported is an open question.
If so, it would work like reference assignment in C++ -- basically creating an
alias for the RHS.
References can be used to reduce the length of dereference expression, and also
improve performance -- especially if that expression is used repeatedly.
\end{openissue}

\subsection{Default Comparison Operators}
\label{Record_Comparison_Operators}
\index{records!equality}
\index{records!inequality}
\index{records!==@\chpl{==}}
\index{records!"!=@\chpl{"\"!=}}
\index{== (record)@\chpl{==} (record)}
\index{"!= (record)@\chpl{"\"!=} (record)}
Default functions to overload \chpl{==} and \chpl{\!=} are defined for
records if none are explicitly defined.
The default implementation of \chpl{==} applies \chpl{==} to each
field of the two argument records and reduces the result with
the \chpl{&&} operator.  The default implementation of \chpl{\!=}
applies \chpl{\!=} to each field of the two argument records and
reduces the result with the \chpl{||} operator.

\subsection{Implicit Record Conversions}
\label{Implicit_Record_Conversions}
\index{conversions!records}
\index{conversions!implicit!records}
\index{records!implicit conversions}

An expression of record type \chpl{D} can be implicitly converted to
another record type \chpl{C} if

\begin{itemize}
\item for each field in \chpl{C} there is a like-named field in \chpl{D},
      and
\item an implicit conversion is allowed from the type of the field in \chpl{D}
      to the type of the field in \chpl{C}.
\end{itemize}
%REVIEW: vass: the following silently loses part of the record value,
% which is different than what we aim for with numeric types.
% Consider disabling this.
Such a conversion removes any fields that are in \chpl{D} but not \chpl{C}.

The value produced by such a conversion is a record of type \chpl{C}.
The value of each field of this record is obtained by an implicit
conversion of the corresponding field in \chpl{D} to
that field's type in \chpl{C}.

%REVIEW: vass: did we agree that instead of a field in D,
% just a getter with that name would be sufficient?

\section{Differences between Classes and Records}
\label{Class_and_Record_Differences}
\index{records!differences with classes}

The key differences between records and classes are listed below.

\subsection{Declarations}
\label{Declaration_Differences}
\index{records!declarations!differences with classes}

Syntactically, class and record type declarations are identical, except that
they begin with the \chpl{class} and \chpl{record} keywords, respectively.
Also, a record type can only inherit from other record types.  Class inheritance
is not permitted.

\subsection{Storage Allocation}
\label{Storage_Allocation_Differences}
\index{classes!allocation}
\index{records!allocation}

For a variable of record type, storage necessary to contain the data fields
has a lifetime equivalent to the scope in which it is declared.  No two record
variables share the same data.  It is not necessary to call \chpl{new} to create
a record.

By contrast, a class variable contains only a reference to a
class instance.  A class instance is created through a call to its \chpl{new}
operator.  Storage for a class instance, including storage for
the data associated with the fields in the class, is allocated and reclaimed
separately from variables referencing that instance.  The same class instance
can be referenced by multiple class variables.

\subsection{Assignment}
\label{Assignment_Differences}
\index{classes!assignment}
\index{records!assignment}

Assignment to a class variable is performed by reference, whereas assignment to
a record is performed by value.  When a variable of class type is assigned to
another variable of class type, they both become names for the same object.  In
contrast, when a record variable is assigned to another record variable, then
contents of the source record are copied into the target record field-by-field.

When a variable of class type is assigned to a record, matching fields (matched
by name) are copied from the class instance into the corresponding record
fields.  Subsequent changes to the fields in the target record have no effect
upon the class instance.

Assignment of a record to a class variable is not permitted.

\subsection{Arguments}
\label{Argument_Differences}
\index{classes!arguments}
\index{records!arguments}

The semantics of argument passing is determined by the type of the formal
argument (as declared inside the function header).  An actual argument is of a
type compatible with the formal argument only if it is legal to assign the
actual to the formal.
%REVIEW: vass: how is this different from the general rule for argument passing?

%REVIEW: vass: only the implicit conversion rules are different.
% Argument passing is the same.
Specifically, if the formal argument is of class type, the actual argument must
be of that class type or of a type derived from that class type.  If the formal
argument is of a record type, then it is only necessary for the fields in the
actual argument to ``cover'' the fields in the formal argument type.

The receiver argument is passed by value for class methods but is
passed by reference for record methods. In both cases modifications to
the receiver fields are visible outside the method.

\subsection{Inheritance}
\label{Inheritance_Differences}
\index{records!inheritance}
\index{inheritance!records}
\index{records!inheritance}
\index{inheritance!records}

The difference between record inheritance and class
inheritance is that for records there is no dynamic dispatch.  The record type of
a variable is the exact type of that variable, i.e. a variable of a
base record type cannot store a derived record type.

Casting a derived record type to a base record type truncates all 
fields except those belonging to the base record type.  In the same way, only
those methods accessible to the base record type may be invoked using the result
of such a cast.

\subsection{Shadowing and Overriding}
\label{Base_Method_Differences}
\index{classes!shadowing}
\index{records!shadowing}
\index{classes!overriding}
\index{records!overriding}

Class variables have run-time types and (therefore) support dynamic dispatch.
Records are statically typed, so they do not have run-time types and they do not
support dynamic dispatch.

As a result, in record type hierarchies, shadowing and overriding are the same.  
Which field is accessed
and/or which method is invoked is determined statically by the declared type of
the record being referenced.

\subsection{No {\em nil} Value}
\index{nil@\chpl{nil}!not provided for records}

Records do not provide a counterpart of the \chpl{nil} value.  A variable of
record type is associated with storage throughout its lifetime, so \chpl{nil}
has no meaning with respect to records.

\subsection{The {\em delete} operator}
\label{Record_Delete_Illegal}
\index{records!delete illegal}
\index{delete!illegal for records}

Calling \chpl{delete} on a record is illegal.

%REVIEW: we could discuss this:
%An explicit call to \chpl{delete} with a record argument has no effect.  The
%compiler may treat this as a hint that the record should not be accessed later
%within its scope and diagnose that case.

\subsection{Default Comparison Operators}
\label{Comparison_Operator_Differences}
\index{classes!comparison}
\index{records!comparison}

For records, the compiler will supply default comparison operators if they are
not supplied by the user.  The compiler does not supply default comparison
operators for classes.


\cleardoublepage
\sekshun{Unions}
\label{Unions}
\index{unions}

Unions have the semantics of records, however, only one field in the
union can contain data at any particular point in the program's
execution.  Unions are safe so that an access to a field that does not
contain data is a runtime error.  When a union is constructed, it is
in an unset state so that no field contains data.

\section{Union Types}
\label{Union_Types}
\index{types!unions}
\index{union types}

The syntax of a union type is summarized as follows:
\begin{syntax}
union-type:
  identifier
\end{syntax}
The union type is specified by the name of the union type.  This
simplification from class and record types is possible because generic
unions are not supported.

\section{Union Declarations}
\label{Union_Declarations}
\index{union@\chpl{union}}
\index{declarations!union@\chpl{union}}

A union is defined with the following syntax:
\begin{syntax}
union-declaration-statement:
  `extern'[OPT] `union' identifier { union-statement-list }

union-statement-list:
  union-statement
  union-statement union-statement-list

union-statement:
  type-declaration-statement
  procedure-declaration-statement
  iterator-declaration-statement
  variable-declaration-statement
  empty-statement
\end{syntax}

If the \chpl{extern} keyword appears before the \chpl{union} keyword, then an
external union type is declared.  An external union is used within Chapel
for type and field resolution, but no corresponding backend definition is
generated.  It is presumed that the definition of an external union type is supplied
by a library or the execution environment.

\subsection{Union Fields}
\label{Union_Fields}
\index{unions!fields}

Union fields are accessed in the same way that record fields are
accessed.  It is a runtime error to access a field that is not
currently set.

Union fields should not be specified with initialization expressions.

\section{Union Assignment}
\label{Union_Assignment}
\index{unions!assignment}

Union assignment is by value.  The field set by the union on the
right-hand side of the assignment is assigned to the union on the
left-hand side of the assignment and this same field is marked as set.

\cleardoublepage
\sekshun{Ranges}
\label{Ranges}
\index{ranges}

A \emph{range} is a first-class, constant-space representation of a
regular sequence of integer indices.
Ranges support iteration over the sequences they represent
and are the basis for defining domains (\rsec{Domains}).

Ranges are presented as follows:
\begin{itemize}
\item definition of the key range concepts \rsec{Range_Concepts}
\item range types \rsec{Range_Types}
\item range values \rsec{Range_Values}
\item range assignment \rsec{Range_Assignment}
\item operators on ranges \rsec{Range_Operators}
\item predefined functions on ranges \rsec{Predefined_Range_Functions}
\end{itemize}


\section{Range Concepts}
\label{Range_Concepts}
\index{ranges!concepts}

\index{ranges!represented sequence}
\index{ranges!sequence}
A range has four primary properties. Together they define the sequence
of indices that the range represents, or the \emph{represented sequence},
as follows.

\begin{itemize}

\index{ranges!low bound}
\item The \emph{low bound} is either an integer or -$\infty$.

\index{ranges!high bound}
\item The \emph{high bound} is either an integer or +$\infty$.
  The low and high bounds determine the span of the represented sequence.
%
  Chapel does not represent $\infty$ explicitly. Instead, infinite
  bound(s) are represented implicitly in the range's type
  (\rsec{Range_Types}).
  When the low and/or high bound is $\infty$, the represented sequence
  is unbounded in the corresponding direction(s).

\index{ranges!stride}
\item The \emph{stride} is a non-zero integer.
  It defines the distance between any two adjacent members of the
  represented sequence.
  The sign of the stride indicates the direction of the sequence:
  \begin{itemize}
  \item[$\bullet$] $stride > 0$ indicates an increasing sequence,
  \item[$\bullet$] $stride < 0$ indicates a decreasing sequence.
  \end{itemize}

\index{ranges!alignment}
\index{ranges!alignment!ambiguous}
\item The \emph{alignment} is either an integer or is \emph{ambiguous}.
  It defines how the represented sequence's members are aligned relative to 0.
%REVIEW: vass: please put a better definition of "alignment" above
% if you can make one.
% (BTW see also the definition of "aligned integer" below.)
  For a range with a stride other than 1 or -1, 
  ambiguous alignment means that the represented sequence is undefined.
  In such a case, certain operations discussed later result in an error.

\begin{openissue}
We consider disallowing ambiguous alignment for ranges whose both bounds
are integers (not $\infty$), in order to enable more efficient
implementation.
\end{openissue}

\end{itemize}

More formally, the represented sequence for the range
$(low, high, stride, alignmt)$
contains all indices $ix$ such that:

\begin{tabular}{ll}

$low \leq ix \leq high$ and $ix \equiv alignmt \pmod{|stride|}$ &
 if $alignmt$ is not ambiguous \\
$low \leq ix \leq high$ &
 if $stride = 1$ or $stride = -1$ \\
the represented sequence is undefined &
 otherwise

\end{tabular}

\index{ranges!represented sequence!increasing}
\index{ranges!represented sequence!decreasing}
The sequence, if defined, is increasing if $stride > 0$ and decreasing if $stride < 0$.

\index{ranges!empty}
If the represented sequence is defined but
there are no indices satisfying the applicable equation(s) above,
the range and its represented sequence are \emph{empty}.

\index{ranges!aligned integer}
We will say that an integer $ix$ is \emph{aligned}
w.r.t. the range $(low, high, stride, alignmt)$
if:
\begin{itemize}
\item $alignmt$ is not ambiguous and $ix \equiv alignmt \pmod{|stride|}$, or
\item $stride$ is 1 or -1.
\end{itemize}
\noindent Furthermore, $\infty$ is never aligned.

Ranges have the following additional properties.
\begin{itemize}

\index{ranges!alignment!ambiguous}
\item A range is \emph{ambiguously aligned} if
  \begin{itemize}
  \item its alignment is ambiguous, and
  \item its stride is neither 1 nor -1.
  \end{itemize}

\index{ranges!first index}
\item The \emph{first index} is the first member of the represented sequence.

  A range \emph{has no} first index when the first member is undefined,
  that is, in the following cases:
  \begin{itemize}
  \item the range is ambiguously aligned,
  \item the represented sequence is empty,
  \item the represented sequence is increasing and the low bound is -$\infty$,
  \item the represented sequence is decreasing and the high bound is +$\infty$.
  \end{itemize}

\index{ranges!last index}
\item The \emph{last index} is the last member of the represented sequence.

  A range \emph{has no} last index when the last member is undefined,
  that is, in the following cases:
  \begin{itemize}
  \item it is ambiguously aligned,
  \item the represented sequence is empty,
  \item the represented sequence is increasing and the high bound is +$\infty$,
  \item the represented sequence is decreasing and the low bound is -$\infty$.
  \end{itemize}

\index{ranges!aligned low bound}
\item The \emph{aligned low bound} is the smallest integer that is
  greater than or equal to the low bound and is aligned w.r.t. the range,
  if such an integer exists.
% For now we are allowing an empty range to have an aligned low bound.

  The aligned low bound equals the smallest member of the represented
  sequence, when both exist.

\index{ranges!aligned high bound}
\item The \emph{aligned high bound} is the largest integer that is
  less than or equal to the high bound and is aligned w.r.t. the range,
  if such an integer exists.
% For now we are allowing an empty range to have an aligned high bound.

  The aligned high bound equals the largest member of the represented
  sequence, when both exist.

% Can re-introduce this if needed, but need to discuss the definition.
% The intention is that a range is "naturally aligned" IFF
% its alignment does not need to be printed out.
%
%A range is \emph{naturally aligned} if it has a positive stride and its
%alignment coincides with its low bound, or it has a negative stride and its
%alignment coincides with its high bound.
%Specifically, a range with \chpl{low}, \chpl{high}, \chpl{stride}
%and \chpl{alignment} values of $l$, $h$, $s$ and $a$ (respectively) is naturally aligned if:
%\begin{itemize}
%\item It is \chpl{bounded}, its stride is positive, and $a = l ({\rm modulo} s)$,
%\item It is \chpl{bounded}, its stride is negative, and $a = h ({\rm modulo} s)$,
%\item It is \chpl{boundedLow} and $a = l ({\rm modulo} s)$,
%\item It is \chpl{boundedHigh} and $a = h ({\rm modulo} s)$ or
%\item It is $boundedNone$ and $a = 0$.
%\end{itemize}
%\noindent The \chpl{isNaturallyAligned} predicate returns \chpl{true} when the operand
%range is naturally aligned.  When a range is formatted for output, the alignment
%is printed only if the range is not naturally aligned.
%\index{ranges!alignment!natural}

\index{ranges!iterable}
\item The range is \emph{iterable}, that is, it is legal to iterate over it,
  if is has the first index.

\end{itemize}

%REVIEW: any other properties to include in the above?


\section{Range Types}
\label{Range_Types}
\index{ranges!types}
\index{types!range}

The type of a range is characterized by three parameters:
\begin{itemize}

\index{ranges!idxType}
\index{ranges!stride type}
\item \chpl{idxType} is the type of the indices of the range's
  represented sequence. However, when the range's low and/or high
  bound is $\infty$, the represented sequence also contains indices
  that are not representable by \chpl{idxType}.

  \chpl{idxType} must be an integral type and is \chpl{int} by default.
%
  The range's low bound and high bound (when they are not $\infty$)
  and alignment are of the type \chpl{idxType}. The range's stride
  is of the signed integer type that has the same bit size as \chpl{idxType}.

\index{ranges!boundedType}
\item \chpl{boundedType} indicates which of the range's bounds are not $\infty$.
  \chpl{boundedType} is an enumeration constant of the type
  \chpl{BoundedRangeType}. It is discussed further below.

\index{ranges!stridable}
\item \chpl{stridable} is a boolean that determines whether the range's stride
  can take on values other than 1.
  \chpl{stridable} is \chpl{false} by default.
  A range is called \emph{stridable}
  if its type's \chpl{stridable} is \chpl{true}.

\end{itemize}

\chpl{boundedType} is one of the constants of the following type:

\begin{chapel}
enum BoundedRangeType { bounded, boundedLow, boundedHigh, boundedNone };
\end{chapel}

The value of \chpl{boundedType} determines which bounds of the range
are integers (making the range ``bounded'', as opposed to infinite,
in the corresponding direction(s))
as follows:

\begin{itemize}

\index{ranges!bounded}
\item \chpl{bounded}:
  both bounds are integers.

\index{ranges!boundedLow}
\item \chpl{boundedLow}:
  the low bound is an integer (the high bound is +$\infty$).

\index{ranges!boundedHigh}
\item \chpl{boundedHigh}:
  the high bound is an integer (the low bound is -$\infty$).

\index{ranges!boundedNone}
\item \chpl{boundedNone}:
  neither bound is an integer (both bounds are $\infty$).

\end{itemize}

\noindent \chpl{boundedType} is \chpl{BoundedRangeType.bounded} by default.

The parameters \chpl{idxType}, \chpl{boundedType} and \chpl{stridable}
affect all values of the corresponding range type.
For example, the range's low bound is -$\infty$ if and only if
the \chpl{boundedType} of that range's type is either \chpl{boundedHigh}
or \chpl{boundedNone}.

\begin{rationale}
Providing \chpl{boundedType} and \chpl{stridable} in a range's type
allows the compiler to identify the more common cases
where the range is \chpl{bounded} and/or its stride is 1.
The compiler can also detect user and library code that is
specialized to these cases.
As a result, the compiler has the opportunity to optimize these
cases and the specialized code more aggressively.
\end{rationale}

%REVIEW: vass: we need to come up with a common way to define
% generic types prior to the generics chapter.

% gotta specify syntax for 'range-type' because it's referenced in Types.tex
A range type has the following syntax:
\begin{syntax}
range-type:
  `range' ( named-expression-list )
\end{syntax}

That is, a range type is obtained as if by invoking the range type constructor
(\rsec{Type_Constructors}) that has the following header:

\begin{chapel}
  proc range(type idxType = int,
             param boundedType = BoundedRangeType.bounded,
             param stridable = false) type
\end{chapel}

As a special case, the keyword \chpl{range} without a parenthesized
argument list refers to the range type with the default values
of all its parameters, i.e.,
\chpl{range(int, BoundedRangeType.bounded, false)}.

\begin{chapelexample}{rangeVariable.chpl}
The following declaration declares a variable \chpl{r}
that can represent ranges of 32-bit integers,
with both high and low bounds specified,
and the ability to have a stride other than 1.
\begin{chapel}
var r: range(int(32), BoundedRangeType.bounded, stridable=true);
\end{chapel}
\begin{chapelpost}
writeln(r);
var i32: int(32) = 3;
r = i32..13 by 3 align 1;
writeln(r);
\end{chapelpost}
\begin{chapeloutput}
1..0
3..13 by 3 align 1
\end{chapeloutput}
\end{chapelexample}


\section{Range Values}
\label{Range_Values}
\index{ranges!values}

A range value consists of the range's four primary properties
(\rsec{Range_Concepts}):
low bound, high bound, stride and alignment.

\subsection{Range Literals}
\label{Range_Literals}
\index{ranges!literals} 

Range literals are specified with the following syntax.

\begin{syntax}
range-literal:
  expression .. expression
  expression ..
  .. expression
  ..
\end{syntax}

The expressions to the left and to the right of \chpl{..}, when given,
are called the low bound and the high bound expression, respectively.

The type of a range literal is a range with the following parameters:

\begin{itemize}

\item \chpl{idxType} is determined as follows:
  \begin{itemize}

  \item If both the low bound and the high bound expressions are given
        and have the same integral type, then \chpl{idxType} is that type.

%REVIEW: not precise. E.g. it does not allow one bound to be converted
% to another bound's type. Also, should be the most specific such type.
  \item If both the low bound and the high bound expressions are given
        and an implicit conversion is allowed from each expression's type
        to the same integral type, then \chpl{idxType} is that
        integral type.

%REVIEW: should be the most specific integral type.
  \item If only one bound expression is given and it has an integral type
        or an implicit conversion is allowed from that expression's type
        to an integral type, then \chpl{idxType} is that integral type.

  \item If neither bound expression is given, then \chpl{idxType} is
        \chpl{int}.

  \item Otherwise, the range literal is not legal.
  \end{itemize}

\item \chpl{boundedType} is a value of the type \chpl{BoundedRangeType}
that is determined as follows:
  \begin{itemize}

  \item \chpl{bounded}, if both the low bound and the high bound expressions
         are given,

  \item \chpl{boundedLow}, if only the high bound expression is given,

  \item \chpl{boundedHigh}, if only the low bound expression is given,

  \item \chpl{boundedNone}, if neither bound expression is given.
  \end{itemize}

\item \chpl{stridable} is \chpl{false}.

\end{itemize}

The value of a range literal is as follows:

\begin{itemize}

\item The low bound is given by the low bound expression,
if present, and is -$\infty$ otherwise.

\item The high bound is given by the upper bound expression,
if present, and is +$\infty$ otherwise.

\item The stride is 1.

\item The alignment is ambiguous.

\end{itemize}


\subsection{Default Values}
\label{Range_Default_Values}
\index{ranges!default values}

The default value for a range type depends on the type's
\chpl{boundedType} parameter as follows:

\begin{itemize}

\item \chpl{1..0} (an empty range) if \chpl{boundedType} is \chpl{bounded}

\item \chpl{1..} if \chpl{boundedType} is \chpl{boundedLow}

\item \chpl{..0} if \chpl{boundedType} is \chpl{boundedHigh}

\item \chpl{..} if \chpl{boundedType} is \chpl{boundedNone}

\end{itemize}

\begin{rationale}
We use 0 and 1 to represent an empty range because these values
are available for any \chpl{idxType}.

We have not found the natural choice of the default value for
\chpl{boundedLow} and \chpl{boundedHigh} ranges.
The values indicated above are distinguished by the following property.
Slicing the default value for a \chpl{boundedLow} range with
the default value for a \chpl{boundedHigh} range (or visa versa)
produces an empty range, matching the default value for a
\chpl{bounded} range
\end{rationale}


\section{Common Operations}
\label{Ranges_Common_Operations}
\index{ranges!operations}

All operations on a range return a new range rather than modifying the existing one.  This
supports a coding style in which all ranges are \emph{immutable} (i.e. declared
as \chpl{const}).  

\begin{rationale}

The intention is to provide ranges as immutable objects.

Immutable objects may be cached without
creating coherence concerns.  They are also inherently thread-safe.  In terms of
implementation, immutable objects are created in a consistent state and stay that way:
Outside of constructors, internal consistency checks can be dispensed with.

These are the same arguments as were used to justify making strings immutable in Java and C\#.

\end{rationale}

%REVIEW: vass: this should really be implicit conversion
% and apply uniformly, like e.g. numeric implicit conversions.
%If needed, define also explicit range conversions.
\subsection{Range Assignment}
\label{Range_Assignment}
\index{ranges!assignment}

Assigning one range to another results in the target range
copying the low and high bounds, stride, and alignment
from the source range.

Range assignment is legal when:
\begin{itemize}
\item An implicit conversion is allowed from \chpl{idxType} of the source range
       to \chpl{idxType} of the destination range type,
\item the two range types have the same \chpl{boundedType}, and
\item either the destination range is stridable or the source range's
      stride is 1.
\end{itemize}

\subsection{Range Comparisons}
\label{Range_Comparisons}
\index{ranges!comparisons}

Ranges can be compared using equality and inequality.

\begin{protohead}
proc ==(r1: range(?), r2: range(?)): bool
\end{protohead}
\begin{protobody}
Returns \chpl{true} if the two ranges have the same represented sequence
or the same four primary properties,
and \chpl{false} otherwise.
%REVIEW: present examples of equal ranges whose low/high bounds differ
% or one is ambiguously aligned (with low>high) and the other is not.
\end{protobody}

\subsection{Iterating over Ranges}
\label{Iterating_over_Ranges}
\index{ranges!iteration}

A range can be used as an iterator expression in a loop. This is legal
only if the range is iterable. In this case the loop iterates over the
members of the range's represented sequence, in the order defined by
the sequence. If the range is empty, no iterations are executed.

\begin{craychapel}
An attempt to iterate over a range causes an error if adding stride to
the range's last index overflows its index type, i.e. if the sum is
greater than the index type's maximum value, or smaller than its
minimum value.
\end{craychapel}

\subsubsection{Iterating over Unbounded Ranges in Zippered Iterations}
\label{Iterating_over_Unbounded_Ranges_in_Zippered_Iterations}
\index{ranges!iteration!zippered}

When a range with the first index but without the last index is used
in a zippered iteration (~\rsec{Zipper_Iteration}),
it generates as many indices as needed
to match the other iterator(s).

\begin{chapelexample}{zipWithUnbounded.chpl}
The code
\begin{chapel}
for i in zip(1..5, 3..) do
  write(i, "; ");
\end{chapel}
\begin{chapelpost}
writeln();
\end{chapelpost}
produces the output 
\begin{chapelprintoutput}{}
(1, 3); (2, 4); (3, 5); (4, 6); (5, 7); 
\end{chapelprintoutput}
\end{chapelexample}

\subsection{Range Promotion of Scalar Functions}
\label{Range_Promotion_of_Scalar_Functions}
\index{ranges!promotion}
\index{promotion!range}

Range values may be passed to a scalar function argument whose type
matches the range's index type.  This results in a promotion of the
scalar function as described in~\rsec{Promotion}.

\begin{chapelexample}{rangePromotion.chpl}
Given a function \chpl{addOne(x:int)} that accepts \chpl{int} values and
a range \chpl{1..10}, the function \chpl{addOne()} can be called with
\chpl{1..10} as its actual argument which will result in the function
being invoked for each value in the range.

\begin{chapel}
proc addOne(x:int) {
  return x + 1;
}
var A:[1..10] int;
A = addOne(1..10);
\end{chapel}
\begin{chapelpost}
writeln(A);
\end{chapelpost}
\begin{chapeloutput}
2 3 4 5 6 7 8 9 10 11
\end{chapeloutput}

\end{chapelexample}

The last statement is equivalent to:
\begin{chapel}
forall (a,i) in zip(A,1..10) do
  a = addOne(i);
\end{chapel}



%REVIEW: perhaps define implicit/explicit conversions as well
% under "Common Operations"?


\section{Range Operators}
\label{Range_Operators}
\index{ranges!operators}

The following operators can be applied to range
expressions and are described in this section: stride (\chpl{by}),
alignment (\chpl{align}), count (\chpl{\#}) and slicing (\chpl{\(\)}
or \chpl{\[\]}).
Chapel also defines a set of functions that operate on ranges.
They are described in \rsec{Predefined_Range_Functions}.

\begin{syntax}
range-expression:
  expression
  strided-range-expression
  counted-range-expression
  aligned-range-expression
  sliced-range-expression
\end{syntax}

\subsection{By Operator}
\label{By_Operator_For_Ranges}
\index{ranges!strided}
\index{by@\chpl{by}!on ranges}
\index{operators!by (range)@\chpl{by} (range)}
\index{ranges!by operator@\chpl{by} operator}

The \chpl{by} operator selects a subsequence of the range's represented sequence,
optionally reversing its direction.
The operator takes two arguments, a base range and an integral step.
It produces a new range whose represented sequence contains
each $|$step$|$-th element of the base range's represented sequence.
The operator reverses the direction of the represented sequence if step$<$0.
If the resulting sequence is increasing,
it starts at the base range's aligned low bound, if it exists.
If the resulting sequence is decreasing,
it starts at the base range's aligned high bound, if it exists.
%
Otherwise, the base range's alignment is used to determine
which members of the represented sequence to retain.
%
If the base range's represented sequence is undefined,
the resulting sequence is undefined, too.

The syntax of the \chpl{by} operator is:
\begin{syntax}
strided-range-expression:
  range-expression `by' step-expression

step-expression:
  expression
\end{syntax}

The type of the step must be a signed or unsigned integer of the same
bit size as the base range's \chpl{idxType}, or an implicit conversion must be allowed
to that type from the step's type.
It is an error for the step to be zero.

\begin{future}
We may consider allowing the step to be of any integer type,
for maximum flexibility.
\end{future}

The type of the result of the \chpl{by} operator is the type of the
base range, but with the \chpl{stridable} parameter set to \chpl{true}.

Formally, the result of the \chpl{by} operator is a range with the following
primary properties:

\begin{itemize}

\item The low and upper bounds are the same as those of the base range.

\item The stride is the product of the base range's stride
      and the step, cast to the base range's stride type before multiplying.

\item The alignment is:

  \begin{itemize}
 
  \item the aligned low bound of the base range, if such exists
        and the stride is positive;

  \item the aligned high bound of the base range, if such exists
        and the stride is negative;

  \item the same as that of the base range, otherwise.

  \end{itemize}

\end{itemize}

\begin{chapelexample}{rangeByOperator.chpl}
In the following declarations, range \chpl{r1} represents the odd integers
between 1 and 20. Range \chpl{r2} strides \chpl{r1} by two and represents
every other odd integer between 1 and 20: 1, 5, 9, 13 and 17.
\begin{chapel}
var r1 = 1..20 by 2;
var r2 = r1 by 2;
\end{chapel}
\begin{chapelpost}
writeln(r1);
writeln(r2);
\end{chapelpost}
\begin{chapeloutput}
1..20 by 2
1..20 by 4
\end{chapeloutput}
\end{chapelexample}

%REVIEW we were not happy with this rationale at 7/19/2011 spec review
\begin{rationale}
{\it Why isn't the high bound specified first if the stride is
negative?}  The reason for this choice is that the \chpl{by} operator
is binary, not ternary.  Given a range \chpl{R} initialized
to \chpl{1..3}, we want \chpl{R by -1} to contain the ordered sequence
$3,2,1$.  But then \chpl{R by -1} would be different from \chpl{3..1
by -1} even though it should be identical by substituting the value in
R into the expression.
\end{rationale}


\subsection{Align Operator}
\label{Align_Operator_For_Ranges}
\index{ranges!align}
\index{align@\chpl{align}!on ranges}
\index{operators!align (range)@\chpl{align} (range)}
\index{ranges!align operator@\chpl{align} operator}

The \chpl{align} operator can be applied to any range, and creates a new range
with the given alignment.  

The syntax for the \chpl{align} operator is:
\begin{syntax}
aligned-range-expression:
  range-expression `align' expression
\end{syntax}
\noindent The type of the resulting range expression is the same as that of the
range appearing as the left operand.  An implicit conversion from
the type of the right operand to the index type of the operand range
must be allowed.
The resulting range has the
same low and high bounds and stride as the source range. The
alignment equals the \chpl{align} operator's right operand
and therefore is not ambiguous. 

\begin{chapelexample}{alignedStride.chpl}
\begin{chapel}
var r1 = 0 .. 10 by 3 align 0;
for i in r1 do
  write(" ", i);			// Produces "0 3 6 9".
writeln();

var r2 = 0 .. 10 by 3 align 1;
for i in r2 do
  write(" ", i);			// Produces "1 4 7 10".
writeln();
\end{chapel}
\begin{chapeloutput}
 0 3 6 9
 1 4 7 10
\end{chapeloutput}
\end{chapelexample}

When the stride is negative, the same indices are printed in reverse:
\begin{chapelexample}{alignedNegStride.chpl}
\begin{chapel}
var r3 = 0 .. 10 by -3 align 0;
for i in r3 do
  write(" ", i);			// Produces "9 6 3 0".
writeln();

var r4 = 0 .. 10 by -3 align 1;
for i in r4 do
  write(" ", i);			// Produces "10 7 4 1".
writeln();
\end{chapel}
\begin{chapeloutput}
 9 6 3 0
 10 7 4 1
\end{chapeloutput}
\end{chapelexample}

To create a range aligned relative to its \chpl{first} index, use
the \chpl{offset} method (\rsec{Range_Offset_Method}).


\subsection{Count Operator}
\label{Count_Operator}
\index{ranges!count operator}
\index{ranges!#@\chpl{#}}
\index{operators!# (range)@\chpl{#} (range)}

The \chpl{#} operator takes a range and an integral count and creates a new
range containing the specified number of indices.  The low or high bound of the
left operand is preserved, and the other bound adjusted to provide the specified
number of indices.  If the count is positive, indices are taken from the start
of the range; if the count is negative, indices are taken from the end of the
range.  The count must be less than or equal to the \chpl{length} of the range.

\begin{syntax}
counted-range-expression:
  range-expression # expression
\end{syntax}

The type of the count expression must be a signed or unsigned integer
of the same bit size as the base range's \chpl{idxType}, or an
implicit conversion must be allowed to that type from the count's
type.

The type of the result of the \chpl{#} operator is the type of the
range argument.

Depending on the sign of the count and the stride, the high or low bound is
unchanged and the other bound is adjusted so that it is $c * stride - 1$ units
away.  Specifically:
\begin{itemize}
\item If the count times the stride is positive, the low bound is preserved
and the high bound is adjusted to be one less than the low bound plus that
product.
\item If the count times the stride is negative, the high bound is preserved
and the low bound is adjusted to be one greater than the high bound plus that
product.
\end{itemize}

\begin{rationale}
Following the principle of preserving as much information from the original
range as possible, we must still choose the other bound so that
exactly \emph{count} indices lie within the range.  Making the two bounds lie
$count * stride - 1$ apart will achieve this, regardless of the current
alignment of the range.

This choice also has the nice symmetry that the alignment can be adjusted
without knowing the bounds of the original range, and the same number of indices
will be produced:
\begin{chapel}
r # 4 align 0   // Contains four indices.
r # 4 align 1   // Contains four indices.
r # 4 align 2   // Contains four indices.
r # 4 align 3   // Contains four indices.
\end{chapel}
\end{rationale}

It is an error to apply the count operator with a positive count to a range that
has no first index.  It is also an error to apply the count operator
with a negative count to a range that has no last index.
It is an error to apply the count operator to a range that is ambiguously aligned.
%REVIEW: vass: does not have to be an error -
% just produce a range whose alignment is ambiguous.

\begin{chapelexample}{rangeCountOperator.chpl}
The following declarations result in equivalent ranges.
\begin{chapel}
var r1 = 1..10 by -2 # -3;
var r2 = ..6 by -2 # 3;
var r3 = -6..6 by -2 # 3;
var r4 = 1..#6 by -2;
\end{chapel}
\begin{chapelpost}
writeln(r1 == r2 \&\& r2 == r3 \&\& r3 == r4);
writeln((r1, r2, r3, r4));
\end{chapelpost}
\begin{chapeloutput}
true
(1..6 by -2, 1..6 by -2, 1..6 by -2, 1..6 by -2)
\end{chapeloutput}
Each of these ranges represents the ordered set of three indices: 6, 4, 2.
\end{chapelexample}

\subsection{Arithmetic Operators}
\label{Range_Arithmetic}
\index{ranges!arithmetic operators}
\index{operators!arithmetic!range}

The following arithmetic operators are defined on ranges and integral
types:

\begin{chapel}
proc +(r: range, s: integral): range
proc +(s: integral, r: range): range
proc -(r: range, s: integral): range
\end{chapel}

The \chpl{+} and \chpl{-} operators apply the scalar via the operator
to the range's low and high bounds, producing a shifted version of the
range.  If the operand range is unbounded above or below, the missing bounds
are ignored.
The index type of the resulting range is the type of the value
that would result from an addition between the scalar value and a value
with the range's index type.  The bounded and stridable parameters for
the result range are the same as for the input range.

The stride of the resulting range is the same as the stride of the
original. The alignment of the resulting range is shifted by the same amount as
the high and low bounds.  It is permissible to apply the shift operators to a
range that is ambiguously aligned.  In that case, the resulting range is also
ambiguously aligned.

\begin{chapelexample}{rangeAdd.chpl}
The following code creates a bounded, non-stridable range \chpl{r}
which has an index type of \chpl{int} representing the indices ${0, 1, 2, 3}$.  
It then uses the \chpl{+} operator to create a second range \chpl{r2}
representing the indices ${1, 2, 3, 4}$.  The \chpl{r2} range is bounded,
non-stridable, and is represented by indices of type \chpl{int}.
% REVIEW: bradc: More interesting example?
\begin{chapel}
var r = 0..3;
var r2 = r + 1;    // 1..4
\end{chapel}
\begin{chapelpost}
writeln((r, r2));
\end{chapelpost}
\begin{chapeloutput}
(0..3, 1..4)
\end{chapeloutput}
\end{chapelexample}


\subsection{Range Slicing}
\label{Range_Slicing}
\index{ranges!slicing}

Ranges can be \emph{sliced} using other ranges to create new
sub-ranges.  The resulting range represents the intersection between
the two ranges' represented sequences.  The stride and alignment of the resulting range are adjusted as
needed to make this true.
\chpl{idxType} and the sign of the stride of the result are determined
by the first operand.

Range slicing is specified by the syntax:
\begin{syntax}
sliced-range-expression:
  range-expression ( range-expression )
  range-expression [ range-expression ]
\end{syntax}

If either of the operand ranges is ambiguously aligned, then the resulting range
is also ambiguously aligned.  In this case, the result is valid only if the
strides of the operand ranges are relatively prime.  Otherwise, an error is
generated at run time.

\begin{rationale}
If the strides of the two operand ranges are relatively prime, then they are
guaranteed to have some elements in their intersection, regardless whether their
relative alignment can be determined.  In that case, the bounds and stride in the resulting
range are valid with respect to the given inputs.
The alignment can be supplied later to create a valid range.

If the strides are not relatively prime, then the result of the slicing
operation would be completely ambiguous.  The only reasonable action for the
implementation is to generate an error.
\end{rationale}

If the resulting sequence cannot be expressed as a range of the
original type, the slice expression evaluates to the empty
range \chpl{1..0}. This can happen, for example, when the operands
represent all odd and all even numbers, or when the first operand is
an unbounded range with unsigned \chpl{idxType} and the second operand
represents only negative numbers.

\begin{chapelexample}{rangeSlicing.chpl}
In the following example, \chpl{r} represents the integers from 1 to
20 inclusive.  Ranges \chpl{r2} and \chpl{r3} are defined using range
slices and represent the indices from 3 to 20 and the odd integers
between 1 and 20 respectively. Range \chpl{r4} represents the odd
integers between 1 and 20 that are also divisible by 3.
%REVIEW: bradc: Give <l,h,s,a> values?
\begin{chapel}
var r = 1..20;
var r2 = r[3..];
var r3 = r[1.. by 2];
var r4 = r3[0.. by 3];
\end{chapel}
\begin{chapelpost}
writeln((r, r2, r3, r4));
\end{chapelpost}
\begin{chapeloutput}
(1..20, 3..20, 1..20 by 2, 1..20 by 6 align 3)
\end{chapeloutput}
\end{chapelexample}


\section{Predefined Functions on Ranges}
\label{Predefined_Range_Functions}
\index{ranges!predefined functions}

\subsection{Range Type Parameters}
\label{Range_Type_Accessors}
\index{ranges!type accessors}

\index{ranges!boundedType@\chpl{boundedType}}
\index{predefined functions!boundedType@\chpl{boundedType}}
\begin{protohead}
proc $range$.boundedType : BoundedRangeType
\end{protohead}
\begin{protobody}
Returns the \chpl{boundedType} parameter of the range's type.
\end{protobody}

\index{ranges!idxType@\chpl{idxType}}
\index{predefined functions!idxType@\chpl{idxType}}
\begin{protohead}
proc $range$.idxType : type
\end{protohead}
\begin{protobody}
Returns the \chpl{idxType} parameter of the range's type.
\end{protobody}

\index{ranges!stridable@\chpl{stridable}}
\index{predefined functions!stridable@\chpl{stridable}}
\begin{protohead}
proc $range$.stridable : bool
\end{protohead}
\begin{protobody}
Returns the \chpl{stridable} parameter of the range's type.
\end{protobody}

\subsection{Range Properties}
\label{Range_Properties}
\index{ranges!properties}

Most of the methods in this subsection report on
the range properties defined in \rsec{Range_Concepts}.
A range's represented sequence can be examined, for example,
by iterating over the range in a for loop \rsec{The_For_Loop}.

\begin{openissue}
The behavior of the methods that report properties that may be
undefined, $\infty$, or ambiguous, may change.
\end{openissue}

\index{ranges!aligned@\chpl{aligned}}
\index{predefined functions!aligned@\chpl{aligned}}
\begin{protohead}
proc $range$.aligned : bool
\end{protohead}
%REVIEW: better name?
\begin{protobody}
Reports whether the range's alignment is \emph{not} ambiguous.
\end{protobody}

\index{ranges!alignedHigh@\chpl{alignedHigh}}
\index{predefined functions!alignedHigh@\chpl{alignedHigh}}
\begin{protohead}
proc $range$.alignedHigh : idxType
\end{protohead}
\begin{protobody}
Returns the range's aligned high bound.
If the aligned high bound is undefined (does not exist),
the behavior is undefined.
\end{protobody}
\begin{chapelexample}{alignedHigh.chpl}
The following code:
\begin{chapel}
var r = 0..20 by 3;
writeln(r.alignedHigh);
\end{chapel}
produces the output
\begin{chapelprintoutput}{}
18
\end{chapelprintoutput}
\end{chapelexample}

\index{ranges!alignedLow@\chpl{alignedLow}}
\index{predefined functions!alignedLow@\chpl{alignedLow}}
\begin{protohead}
proc $range$.alignedLow : idxType
\end{protohead}
\begin{protobody}
Returns the range's aligned low bound.
If the aligned low bound is undefined (does not exist),
the behavior is undefined.
\end{protobody}

\index{ranges!alignment@\chpl{alignment}}
\index{predefined functions!alignment@\chpl{alignment}}
\begin{protohead}
proc $range$.alignment : idxType
\end{protohead}
\begin{protobody}
Returns the range's alignment.
If the alignment is ambiguous, the behavior is undefined.
See also \chpl{aligned}.
\end{protobody}

\index{ranges!first@\chpl{first}}
\index{predefined functions!first@\chpl{first}}
\begin{protohead}
proc $range$.first : idxType
\end{protohead}
\begin{protobody}
Returns the range's first index.
If the range has no first index, the behavior is undefined.
See also \chpl{hasFirst}.
\end{protobody}

\index{ranges!hasFirst@\chpl{hasFirst}}
\index{predefined functions!hasFirst@\chpl{hasFirst}}
\begin{protohead}
proc $range$.hasFirst(): bool
\end{protohead}
\begin{protobody}
Reports whether the range has the first index.
\end{protobody}

\index{ranges!hasHighBound@\chpl{hasHighBound}}
\index{predefined functions!hasHighBound@\chpl{hasHighBound}}
\begin{protohead}
proc $range$.hasHighBound() param: bool
\end{protohead}
\begin{protobody}
Reports whether the range's high bound is \emph{not} +$\infty$.
\end{protobody}

\index{ranges!hasLast@\chpl{hasLast}}
\index{predefined functions!hasLast@\chpl{hasLast}}
\begin{protohead}
proc $range$.hasLast(): bool
\end{protohead}
\begin{protobody}
Reports whether the range has the last index.
\end{protobody}

\index{ranges!hasLowBound@\chpl{hasLowBound}}
\index{predefined functions!hasLowBound@\chpl{hasLowBound}}
\begin{protohead}
proc $range$.hasLowBound() param: bool
\end{protohead}
\begin{protobody}
Reports whether the range's low bound is \emph{not} -$\infty$.
\end{protobody}

\index{ranges!high@\chpl{high}}
\index{predefined functions!high@\chpl{high}}
\begin{protohead}
proc $range$.high : idxType
\end{protohead}
\begin{protobody}
Returns the range's high bound.
If the high bound is +$\infty$, the behavior is undefined.
See also \chpl{hasHighBound}.
\end{protobody}

% Can re-introduce this if needed, but need to define natural alignment first.
%\begin{protohead}
%proc $range$.isNaturallyAligned(): bool
%\end{protohead}
%\begin{protobody}
%Reports whether the range is naturally aligned.
%\end{protobody}

\index{ranges!isAmbiguous@\chpl{isAmbiguous}}
\index{predefined functions!isAmbiguous@\chpl{isAmbiguous}}
\begin{protohead}
proc $range$.isAmbiguous(): bool
\end{protohead}
\begin{protobody}
Reports whether the range is ambiguously aligned.
\end{protobody}

\index{ranges!last@\chpl{last}}
\index{predefined functions!last@\chpl{last}}
\begin{protohead}
proc $range$.last : idxType
\end{protohead}
\begin{protobody}
Returns the range's last index.
If the range has no last index, the behavior is undefined.
See also \chpl{hasLast}.
\end{protobody}

\index{ranges!length@\chpl{length}}
\index{predefined functions!length@\chpl{length}}
\begin{protohead}
proc $range$.length : idxType
\end{protohead}
\begin{protobody}
Returns the number of indices in the range's represented sequence.
If the represented sequence is infinite or is undefined,
an error is generated.
\end{protobody}

\index{ranges!low@\chpl{low}}
\index{predefined functions!low@\chpl{low}}
\begin{protohead}
proc $range$.low : idxType
\end{protohead}
\begin{protobody}
Returns the range's low bound.
If the low bound is -$\infty$, the behavior is undefined.
See also \chpl{hasLowBound}.
\end{protobody}

\index{ranges!size@\chpl{size}}
\index{predefined functions!size@\chpl{size}}
\begin{protohead}
proc $range$.size : idxType
\end{protohead}
\begin{protobody}
Same as $range$.length.
\end{protobody}

\index{ranges!stride@\chpl{stride}}
\index{predefined functions!stride@\chpl{stride}}
\begin{protohead}
proc $range$.stride : int(numBits(idxType))
\end{protohead}
\begin{protobody}
Returns the range's stride. This will never return 0.
If the range is not stridable, this will always return 1.
\end{protobody}

\subsection{Other Queries}
\label{Range_Queries}
\index{ranges!other queries}

\index{ranges!boundsCheck@\chpl{boundsCheck}}
\index{predefined functions!boundsCheck@\chpl{boundsCheck}}
\begin{protohead}
proc $range$.boundsCheck(r2: range(?)): bool
\end{protohead}
\begin{protobody}
Returns \chpl{false} if either range is ambiguously aligned.
Returns \chpl{true} if range \chpl{r2} lies entirely within this range
and \chpl{false} otherwise.  
\end{protobody}

\index{ranges!ident@\chpl{ident}}
\index{predefined functions!ident@\chpl{ident}}
\begin{protohead}
proc ident(r1: range(?), r2: range(?)): bool
\end{protohead}
\begin{protobody}
Returns \chpl{true} if the two ranges are the same in every respect: i.e. the
two ranges have the same \chpl{idxType},
\chpl{boundedType}, \chpl{stridable}, \chpl{low}, \chpl{high}, \chpl{stride} and
\chpl{alignment} values.
%REVIEW: bradc: Might be nice if only the field properties (and not the type params) were
%compared?  Types could be compared separately using '=='?
\end{protobody}

\index{ranges!indexOrder@\chpl{indexOrder}}
\index{predefined functions!indexOrder@\chpl{indexOrder}}
\begin{protohead}
proc $range$.indexOrder(i: idxType): idxType
\end{protohead}
\begin{protobody}
If \chpl{i} is a member of the range's represented sequence, returns an integer giving
the ordinal index of \chpl{i} within the sequence using 0-based indexing.
Otherwise, returns \chpl{(-1):idxType}.
It is an error to invoke \chpl{indexOrder} if the represented sequence
is not defined or the range does not have the first index.
\end{protobody}

\begin{example}
The following calls show the order of index 4 in each of the given
ranges:
\begin{chapel}
(0..10).indexOrder(4) == 4
(1..10).indexOrder(4) == 3
(3..5).indexOrder(4) == 1
(0..10 by 2).indexOrder(4) == 2
(3..5 by 2).indexOrder(4) == -1
\end{chapel}
\end{example}

\index{ranges!member@\chpl{member}}
\index{predefined functions!member@\chpl{member}}
\begin{protohead}
proc $range$.member(i: idxType): bool
\end{protohead}
\begin{protobody}
Returns \chpl{true} if the range's represented sequence
contains \chpl{i}, \chpl{false} otherwise.
It is an error to invoke \chpl{member} if the represented sequence
is not defined.
\end{protobody}

\index{ranges!member@\chpl{member}}
\index{predefined functions!member@\chpl{member}}
\begin{protohead}
proc $range$.member(other: range): bool
\end{protohead}
\begin{protobody}
Reports whether \chpl{other} is a subrange of the receiver. That is,
if the represented sequences of the receiver and \chpl{other}
are defined and the receiver's sequence contains all members of the
\chpl{other}'s sequence.
%REVIEW: bradc: How different than boundsCheck?
\end{protobody}

\subsection{Range Transformations}
\label{Range_Transformations}
\index{ranges!transformations}

\index{ranges!alignHigh@\chpl{alignHigh}}
\index{predefined functions!alignHigh (range)@\chpl{alignHigh} (range)}
\begin{protohead}
proc $range$.alignHigh()
\end{protohead}
\begin{protobody}
Sets the high bound of this range to its aligned high bound, if it is defined.
Generates an error otherwise.
\end{protobody}

\index{ranges!alignLow@\chpl{alignLow}}
\index{predefined functions!alignLow (range)@\chpl{alignLow} (range)}
\begin{protohead}
proc $range$.alignLow()
\end{protohead}
\begin{protobody}
Sets the low bound of this range to its aligned low bound, if it is defined.
Generates an error otherwise.
\end{protobody}

\index{ranges!expand@\chpl{expand}}
\index{predefined functions!expand (range)@\chpl{expand}}
\begin{protohead}
proc $range$.expand(i: idxType)
\end{protohead}
\begin{protobody}
Returns a new range whose bounds are extended by $i$ units on each end.  If $i <
0$ then the resulting range is contracted by its absolute value.  In symbols,
given that the operand range is represented by the tuple $(l,h,s,a)$, the result
is $(l-i,h+i,s,a)$.  The stride and alignment of the original range are preserved.
If the operand range is ambiguously aligned, then so is the resulting range.
\end{protobody}

\index{ranges!exterior@\chpl{exterior}}
\index{predefined functions!exterior (range)@\chpl{exterior} (range)}
\begin{protohead}
proc $range$.exterior(i: idxType)
\end{protohead}
\begin{protobody}
Returns a new range containing the indices just outside the low or high bound of
the range (low if $i < 0$ and high otherwise).  The stride and alignment of the
original range are preserved.  Let the operand range
be denoted by the tuple $(l,h,s,a)$.  Then:
\begin{itemize}
\item[] if $i < 0$, the result is $(l+i,l-1,s,a)$,
\item[] if $i > 0$, the result is $(h+1,h+i,s,a)$, and
\item[] if $i = 0$, the result is $(l,h,s,a)$.
\end{itemize}
If the operand range is ambiguously aligned, then so is the resulting range.
\end{protobody}

\index{ranges!interior@\chpl{interior}}
\index{predefined functions!interior (range)@\chpl{interior} (range)}
\begin{protohead}
proc $range$.interior(i: idxType)
\end{protohead}
\begin{protobody}
Returns a new range containing the indices just inside the low or high bound of
the range (low if $i < 0$ and high otherwise).  The stride and alignment of the
original range are preserved.  Let the operand range
be denoted by the tuple $(l,h,s,a)$.  Then:
\begin{itemize}
\item[] if $i < 0$, the result is $(l,l-(i-1),s,a)$,
\item[] if $i > 0$, the result is $(h-(i-1),h,s,a)$, and
\item[] if $i = 0$, the result is $(l,h,s,a)$.
\end{itemize}
This differs from the behavior of the count operator, in that \chpl{interior()}
preserves the alignment, and it uses the low and high bounds rather
than \chpl{first} and \chpl{last} to establish the bounds of the resulting range.
If the operand range is ambiguously aligned, then so is the resulting range.
\end{protobody}

\index{ranges!offset@\chpl{offset}}
\index{predefined functions!offset (range)@\chpl{offset} (range)}
\begin{protohead}
proc $range$.offset(n: idxType)
\end{protohead}
\label{Range_Offset_Method}
\begin{protobody}
Returns a new range whose alignment is this range's first index plus
\chpl{n}. The new alignment, therefore, is not ambiguous.
If the range has no first index, a run-time error is generated.
\end{protobody}

\index{ranges!translate@\chpl{translate}}
\index{predefined functions!translate (range)@\chpl{translate} (range)}
\begin{protohead}
proc $range$.translate(i: integral)
\end{protohead}
\begin{protobody}
Returns a new range with its \chpl{low}, \chpl{high} and \chpl{alignment} values
adjusted by $i$.  The \chpl{stride} value is preserved.
If the range's alignment is ambiguous, the behavior is undefined.
%REVIEW: vass: can define this to produce ambiguous alignment in such cases.
\end{protobody}


\cleardoublepage
\sekshun{Domains}
\label{Domains}
\index{domains}

A \emph{domain} is a first-class representation of an index set.
Domains are used to specify iteration spaces, to define the size and
shape of arrays (\rsec{Arrays}), and to specify aggregate operations
like slicing.
A domain can specify a single- or multi-dimensional
rectangular iteration space or represent a set of indices of
a given type.  Domains can also represent a subset of another domain's index set,
using either a dense or sparse representation.
A domain's
indices may potentially be distributed across multiple locales as
described in~\rsec{Domain_Maps}, thus supporting global-view data
structures.

In the next subsection, we introduce the key characteristics of
domains.  In~\rsec{Base_Domain_Types_and_Values}, we discuss the types
and values that can be associated with a base domain.
In~\rsec{Simple_Subdomain_Types_and_Values}, we discuss the types and
values of simple subdomains that can be created from those base
domains.  In~\rsec{Sparse_Subdomain_Types_and_Values}, we discuss the
types and values of sparse subdomains.  The remaining sections
describe the important manipulations that can be performed with
domains, as well as the predefined operators and functions defined for
domains.

\section{Domain Overview}
\index{domains!kinds}

There are three \emph{kinds} of domain, distinguished by their subset
dependencies: \emph{base domains}, \emph{subdomains} and \emph{sparse
subdomains}.  A base domain describes an index set spanning one or more
dimensions.  A subdomain creates an index set that is a subset of the indices in
a base domain or another subdomain.  Sparse subdomains are subdomains which can
represent sparse index subsets efficiently.  Simple subdomains are subdomains
that are not sparse.  These relationships can be represented as follows:

\begin{syntax}
domain-type:
  base-domain-type
  simple-subdomain-type
  sparse-subdomain-type
\end{syntax}

Domains can be further classified according to whether they are \emph{regular}
or \emph{irregular}.  A regular domain represents a rectangular iteration
space and can have a compact representation whose size is independent
of the number of indices. Rectangular domains, with the exception of
sparse subdomains, are regular.

An irregular domain can store an arbitrary set of indices of an arbitrary but
homogeneous index type.  Irregular domains typically require space proportional
to the number of indices being represented.  All \emph{associative} domain types
and their subdomains (including sparse subdomains) are irregular.  Sparse
subdomains of regular domains are also irregular.

An index set can be either \emph{ordered} or \emph{unordered} depending on
whether its members have a well-defined order relationship.  All regular
domains are ordered.  All associative domains are
unordered.

The type of a domain describes how a domain is represented and the operations
that can be performed upon it, while its value is the set of indices it represents.
In addition to storing a value, each domain variable has an identity that
distinguishes it from other domains that may have the same type and
value.  This identity is used to define the domain's relationship
with subdomains, index types~(\rsec{Index_Types}),
and arrays~(\rsec{Association_of_Arrays_to_Domains}).

The runtime representation of a domain is controlled by its domain map.
Domain maps are presented in \rsec{Domain_Maps}.


\section{Base Domain Types and Values}
\label{Base_Domain_Types_and_Values}
\index{domains!types and values}

Base domain types can be classified as regular or irregular.  Dense and
strided rectangular domains are regular domains.
Irregular base domain types include all of the associative domain types.

\begin{syntax}
base-domain-type:
  rectangular-domain-type
  associative-domain-type
\end{syntax}

These base domain types are discussed in turn in the following
subsections.

\subsection{Rectangular Domains}
\index{rectangular domains (see also domains, rectangular)}
\index{domains!rectangular}

Rectangular domains describe multidimensional rectangular index sets.  They are
characterized by a tensor product of ranges and represent indices that are
tuples of an integral type.  Because their index sets can be represented using
ranges, regular domain values typically require only $O(1)$ space.

\subsubsection{Rectangular Domain Types}
\index{domains!rectangular!types}
\index{types!rectangular domains}

Rectangular domain types are parameterized by three things:
\begin{itemize}
\item \chpl{rank} a positive \chpl{int} value indicating the number
of dimensions that the domain represents;
\item \chpl{idxType} a type member representing the index type for
each dimension; and
% BLC: we should potentially rename idxType to idxDimType to make it
% more consistent with the irregular case
\item \chpl{stridable} a \chpl{bool} parameter indicating whether
any of the domain's dimensions will be characterized by a strided
range.
\end{itemize}
If \chpl{rank} is $1$, the index type represented by a rectangular
domain is \chpl{idxType}.  Otherwise, the index type is the homogeneous
tuple type \chpl{rank*idxType}.
If unspecified, \chpl{idxType} defaults
to \chpl{int} and \chpl{stridable} defaults to \chpl{false}.

\begin{openissue}
We may represent a rectangular domain's index type as rank*idxType even if rank is 1.  This
would eliminate a lot of code currently used to support the special (rank == 1) case.
\end{openissue}

The syntax of a rectangular domain type is summarized as follows:
\begin{syntax}
rectangular-domain-type:
  `domain' ( named-expression-list )
\end{syntax}

\noindent where \sntx{named-expression-list} permits the values of
\chpl{rank}, \chpl{idxType}, and \chpl{stridable} to be specified
using standard type signature.

\begin{chapelexample}{typeFunctionDomain.chpl}
The following declarations both create an uninitialized rectangular domain with three dimensions, with \chpl{int} indices:
\begin{chapel}
var D1 : domain(rank=3, idxType=int, stridable=false);
var D2 : domain(3);
\end{chapel}
\begin{chapelpost}
writeln(D1);
writeln(D2);
\end{chapelpost}
\begin{chapeloutput}
{1..0, 1..0, 1..0}
{1..0, 1..0, 1..0}
\end{chapeloutput}
\end{chapelexample}

\subsubsection{Rectangular Domain Values}
\label{Rectangular_Domain_Values}
\index{domains!values!rectangular}
\index{domains!rectangular!values}

Each dimension of a rectangular domain is a range of
type \chpl{range(idxType,} \chpl{BoundedRangeType.bounded,} \chpl{
stridable)}.  The index set for a rank~1 domain is the set of indices
described by its singleton range.  The index set for a rank~$n$
domain is the set of all \chpl{n*idxType} tuples described by the
tensor product of its ranges.  When expanded (as by an iterator), rectangular domain indices are ordered
according to the lexicographic order of their values.  That is, the index with
the highest rank is listed first and changes most slowly.\footnote{This is also
known as row-major ordering.}

%REVIEW: vass: we have not settled on the lexicographic order of the values.
% That needs to be reflected here.

%REVIEW: vass: rephrase the futures below to (a) be more formal
% and (b) motivate why we are considering them (or add some contents)?

\begin{future}
Domains defined using unbounded ranges may be supported.
\end{future}

\index{domains!rectangular!literals}

Literal rectangular domain values are represented by a comma-separated
list of range expressions of matching \chpl{idxType} enclosed in
curly braces:

%% \begin{future}
%% Will we support domains with heterogeneous index types?
%% \end{future}

\begin{syntax}
rectangular-domain-literal:
  { range-expression-list }

range-expression-list:
  range-expression
  range-expression, range-expression-list
\end{syntax}

\noindent The type of a rectangular domain literal is defined as follows:

\begin{itemize}
\item \chpl{rank} = the number of range expressions in the literal;
\item \chpl{idxType} = the type of the range expressions;
\item \chpl{stridable} = \chpl{true} if any of the range expressions
are stridable, otherwise \chpl{false}.
\end{itemize}
\noindent If the index types in the ranges differ and all of them can be
promoted to the same type, then that type is used as the \chpl{idxType}.
Otherwise, the domain literal is invalid.

\begin{example}
The expression \chpl{\{1..5, 1..5\}} defines a rectangular domain with
type \chpl{domain(rank=2,} \chpl{ idxType=int,} \chpl{ stridable=false)}.
It is a $5 \times 5$ domain with the indices:
\begin{equation}
(1, 1), (1, 2), \ldots, (1, 5), (2, 1), \ldots (5, 5).
\end{equation}
\end{example}

A domain expression may contain bounds which are evaluated at runtime.
\begin{example}
In the code
\begin{chapel}
var D: domain(2) = {1..n, 1..n};
\end{chapel}

\chpl{D} is defined as a two-dimensional, nonstridable rectangular
domain with an index type of \chpl{2*int} and is initialized to
contain the set of indices $(i,j)$ for all $i$ and $j$ such that
$i \in {1, 2, \ldots, n}$ and $j \in {1, 2, \ldots, n}$.
\end{example}

\index{domains!rectangular!default value}

The default value of a domain type is the \chpl{rank} default range
values for type:
\begin{quote}
\chpl{range(idxType, BoundedRangeType.bounded, stridable)}
\end{quote}

\begin{chapelexample}{rectangularDomain.chpl}
The following creates a two-dimensional rectangular domain and then uses this to
declare an array.  The array indices are iterated over using the domain's
\chpl{dim()} method, and each element is filled with
some value.  Then the array is printed out.

Thus, the code
\begin{chapel}
var D : domain(2) = {1..2, 1..7};
var A : [D] int;
for i in D.dim(1) do
  for j in D.dim(2) do
    A[i,j] = 7 * i**2 + j;
writeln(A);
\end{chapel}
produces
\begin{chapelprintoutput}{}
8 9 10 11 12 13 14
29 30 31 32 33 34 35
\end{chapelprintoutput}
\end{chapelexample}

\subsection{Associative Domains}
\index{associative domains (see also domains, associative)}

Associative domains represent an arbitrary set of indices
of a given type and can be used to describe sets or to create
dictionary-style arrays (hash tables).
The type of indices of an associative domain, or its \chpl{idxType},
can be any primitive type except \chpl{void} or any class type.

\subsubsection{Associative Domain Types}

\label{Associative_Domain_Types}
\index{types!associative domains}
\index{domains!associative}

An associative domain type is parameterized by \chpl{idxType}, the
type of the indices that it stores.  The syntax is as follows:

\begin{syntax}
associative-domain-type:
  `domain' ( associative-index-type )
  `domain' ( `opaque' )

associative-index-type:
  type-specifier
\end{syntax}

\index{types!opaque domains}
\index{opaque domains!types}
\index{domains!opaque}
The three expansions of \sntx{associative-domain-type} correspond to the three
kinds of associative domain listed below.
\begin{enumerate} 
\item In general, \sntx{associative-index-type} determines \chpl{idxType}
of the associative domain type.
\item Opaque domains are a special case, indicated by the type \chpl{opaque}.
Anonymous values of the type \chpl{opaque} are used as index values
in this case.
% TODO: need to define and explain the 'opaque' type elsewhere.
\end{enumerate}

When an associative domain is used as the index set of an array, the relation
between the indices and the array elements can be thought of as a map between
the values of the index set and the elements stored in the array.
Opaque domains can be used to build unstructured arrays that are similar to
pointer-based data structures in conventional languages.

\subsubsection{Associative Domain Values}
\label{Associative_Domain_Values}
\index{domains!values!associative}
\index{domains!associative!values}

An associative domain's value is simply the set of all index values
that the domain describes.  The iteration order over the indices of
an associative domain is undefined.

\index{domains!associative!literals}
\index{domains!associative!initialization}

Specification of an associative domain literal value follows a similar syntax as
rectangular domain literal values.  What differentiates the two are the types 
of expressions specified in the comma separated list.  Use of values of a 
type other than ranges will result in the construction of an associative domain.  

\begin{syntax}
associative-domain-literal:
   { associative-expression-list }

associative-expression-list:
   non-range-expression
   non-range-expression, associative-expression-list

non-range-expression:
   expression
\end{syntax}

It is required that the types of the values used in constructing an associative
domain literal value be of the same type.  If the types of the indices does not
match a compiler error will be issued.

\begin{future}
Due to implementation of == over arrays it is currently not possible to use
arrays as indices within an associative domain. 
\end{future}

\begin{openissue}
Assignment of an associative domain literal results in a warning message
being printed alerting the user that whole-domain assignment has been
serialized. This results from the resize operation over associative arrays not
being parsafe. 
\end{openissue}

\begin{chapelexample}{associativeDomain.chpl}
The following example illustrates construction of an associative domain
containing string indices "bar" and "foo".  Note that due to internal hashing 
of indices the order in which the values of the associative domain are iterated
is not the same as their specification order.

This code
\begin{chapel}
var D : domain(string) = {"bar", "foo"};
writeln(D);
\end{chapel}
\begin{chapelcompopts}
--no-warnings
\end{chapelcompopts}
produces the output
\begin{chapelprintoutput}{}
{foo, bar}
\end{chapelprintoutput}
\end{chapelexample}

\index{domains!associative!default values}

If uninitialized, the default value of an associative domain is the
empty index set.

Indices can be added to or removed from an associative domain
as described in \rsec{Adding_and_Removing_Domain_Indices}.


\section{Simple Subdomain Types and Values}
\label{Simple_Subdomain_Types_and_Values}
\index{subdomains}
\index{subdomains!simple}

A subdomain is a domain whose indices are guaranteed to be a subset of
those described by another domain known as its \emph{parent domain}.
A subdomain has the same type as its parent domain, and by default
it inherits the domain map of its parent domain.  All domain types
support subdomains.

Simple subdomains are subdomains which are not sparse.  Sparse
subdomains are discussed in the following section
(\rsec{Sparse_Subdomain_Types_and_Values}).  A simple subdomain
inherits its representation (regular or irregular) from its base
domain (or base subdomain).  A sparse subdomain is always irregular,
even if its base domain is regular.

In all other respects, the two kinds of subdomain behave identically.  In this
specification, ``subdomain'' refers to both simple and sparse subdomains, unless
it is specifically distinguished as one or the other.

\begin{rationale}
Subdomains are provided in Chapel for a number of reasons: to
facilitate the ability of the compiler or a reader to reason about the
inter-relationship of distinct domain variables; to support the
author's ability to omit redundant domain mapping specifications; to
support the compiler's ability to reason about the relative alignment
of multiple domains; and to improve the compiler's ability to prove
away bounds checks for array accesses.
\end{rationale}

\subsection{Simple Subdomain Types}
\label{Simple_Subdomain_Types}
\index{subdomains!simple!types}
\index{subdomains!types!simple}
\index{types!subdomains!simple}

A simple subdomain type is specified using the following syntax:
\begin{syntax}
simple-subdomain-type:
  `subdomain' ( domain-expression )
\end{syntax}

This declares that \sntx{domain-expression} is the parent domain of
this subdomain type.  A simple subdomain specifies a subdomain
with the same underlying representation as its base domain.  

\begin{openissue}

An open semantic issue for subdomains is when a subdomain's subset
property should be re-verified once its parent domain is reassigned
and whether this should be done aggressively or lazily.

\end{openissue}

\subsection{Simple Subdomain Values}
\index{subdomains!simple!values}
\index{values!subdomains!simple}

The value of a simple subdomain is the set of all index values
that the subdomain describes.

\index{subdomains!simple!default values}

The default value of a simple subdomain type is the same as the default value
of its parent's type
(\rsec{Rectangular_Domain_Values}, \rsec{Associative_Domain_Values}).

A simple subdomain variable can be initialized or assigned to
with a tuple of values of the parent's \chpl{idxType}.
Indices can also be added to or removed from a simple subdomain
as described in \rsec{Adding_and_Removing_Domain_Indices}.
It is an error to attempt to add an index to a subdomain that is not also
a member of the parent domain.


\section{Sparse Subdomain Types and Values}
\label{Sparse_Subdomain_Types_and_Values}
\index{domains!sparse}
\index{subdomains!sparse}

\begin{syntax}
sparse-subdomain-type:
  `sparse' `subdomain'[OPT] ( domain-expression )
\end{syntax}

This declaration creates a sparse subdomain.
 \emph{Sparse subdomains} are irregular domains that describe an
arbitrary subset of a domain, even if the parent domain is a regular
domain.  Sparse subdomains are useful in Chapel for
defining \emph{sparse arrays} in which a single element value (usually ``zero'')
 occurs
frequently enough that it is worthwhile to avoid storing it
redundantly.  The set difference between a sparse subdomain's index set
and that of parent domain is the set of indices for which the
sparse array will store this replicated value.
%%NB: This is a nice mathematical definition, but do we really want to torque
%%the reader's brain with the notion of redundant values?
%REVIEW:hilde -- I would suggest "uninteresting values" or values that can be omitted.
See~\rsec{Sparse_Arrays} for details about sparse arrays.

\subsection{Sparse Subdomain Types}
\index{domains!sparse!types}
\index{subdomains!sparse!types}
\index{types!domains!sparse}
\index{types!subdomains!sparse}

Each root domain type has a unique corresponding sparse subdomain
type.  Sparse subdomains whose parent domains are also sparse
subdomains share the same type.

\subsection{Sparse Subdomain Values}
\label{Sparse_Domain_Values}
\index{domains!sparse!values}
\index{subdomains!sparse!values}
\index{values!domains!sparse}
\index{values!subdomains!sparse}

A sparse subdomain's value is simply the set of all index values that
the domain describes.  If the parent domain defines an iteration order
over its indices, the sparse subdomain inherits that order.

\index{sparse domains!literals!lack thereof}
\index{sparse domains!initialization}
\index{domains!sparse!initialization}
\index{initialization!sparse domains}
There is no literal syntax for a sparse subdomain.  However, a variable of a
sparse subdomain type can be initialized using a tuple of values
of the parent domain's index type.

\index{sparse domains!default value}
\index{domains!sparse!default value}
The default value for a sparse subdomain value is the empty set.

\begin{example}
The following code declares a two-dimensional dense domain \chpl{D},
followed by a two dimensional sparse subdomain of \chpl{D}
named \chpl{SpsD}.  Since \chpl{SpsD} is uninitialized, it will
initially describe an empty set of indices from \chpl{D}.
\begin{chapel}
const D: domain(2) = {1..n, 1..n};
var SpsD: sparse subdomain(D);
\end{chapel}
\end{example}

\section{Domain Index Types}
\label{Index_Types}
\index{domains!index types}

Each domain value has a corresponding compiler-provided \emph{index
type} which can be used to represent values belonging to that domain's
index set.  Index types are described using the following syntax:

\begin{syntax}
index-type:
  `index' ( domain-expression )
\end{syntax}

A variable with a given index type is constrained to take on only values
available within the domain on which it is defined.  This restriction allows the
compiler to prove away the bound checking that code safety considerations might
otherwise require.  Due to the subset relationship between a base domain and its
subdomains, a variable of an index type defined with respect to a subdomain is
also necessarily a valid index into the base domain.

Since an index types are known to be legal for a given domain, it may
also afford the opportunity to represent that index using an optimized
format that doesn't simply store the index variable's value.  This fact could be
used to support accelerated access to arrays declared over that domain.  For
example, iteration over an index type could be implemented using memory pointers
and strides, rather than explicitly calculating the offset of each index
within the domain.

These potential optimizations may make it less expensive to
index into arrays using index type variables of their domains or
subdomains.

In addition, since an index type is associated with a specific domain or subdomain, it
carries more semantic weight than a generic index.  For example, one could
iterate over a rectangular domain with integer bounds using an \chpl{int(n)} as
the index variable.  However, it would be more precise to use a variable of the
domain's index type.

\begin{openissue}

An open issue for index types is what the semantics should be for an
index type value that is live across a modification to its domain's
index set---particularly one that shrinks the index set.  Our
hypothesis is that most stored indices will either have short
lifespans or belong to constant or monotonically growing domains.  But
these semantics need to be defined nevertheless.

\end{openissue}

\section{Iteration Over Domains}
\label{Iteration_over_Domains}
\index{domains!iteration}
\index{iteration!domain}

All domains support iteration via standard \chpl{for}, \chpl{forall}, and \chpl{coforall}
loops.  These loops iterate over all of the indices that the domain
describes.  If the domain defines an iteration order of its indices,
then the indices are visited in that order.  

The type of the iterator variable for an iteration over a
domain named \chpl{D} is that domain's index type, \chpl{index(D)}.


\section{Domains as Arguments}
\label{Domain_Arguments}
\index{domains!as arguments}
\index{argument passing!domains}

This section describes the semantics of passing domains as arguments
to functions.

\subsection{Formal Arguments of Domain Type}

When a domain value is passed to a formal argument of compatible
domain type by default intent, it is passed by reference in order to
preserve the domain's identity.

\subsection{Domain Promotion of Scalar Functions}
\label{Domain_Promotion_of_Scalar_Functions}
\index{domains!promotion}
\index{promotion!domain}

Domain values may be passed to a scalar function argument whose type
matches the domain's index type.  This results in a promotion of the
scalar function as defined in~\rsec{Promotion}.

\begin{example}
Given a function \chpl{foo()} that accepts real floating point values
and an associative domain \chpl{D} of
type \chpl{domain(real)}, \chpl{foo} can be called with \chpl{D} as
its actual argument which will result in the function being invoked
for each value in the index set of \chpl{D}.
\end{example}

\begin{example}
Given an array \chpl{A} with element type \chpl{int} declared over a
one-dimensional domain \chpl{D} with \chpl{idxType} \chpl{int}, the
array elements can be assigned their corresponding index values by
writing:
\begin{chapel}
A = D;
\end{chapel}
This is equivalent to:
\begin{chapel}
forall (a,i) in zip(A,D) do
  a = i;
\end{chapel}
\end{example}


\section{Domain Operations}

Chapel supplies predefined operators and functions that can be used to manipulate
domains.  Unless otherwise noted, these operations are applicable to a domain of
any type, whether a base domain or a subdomain.

\subsection{Domain Assignment}
\label{Domain_Assignment}
\index{domains!assignment}
\index{assignment!domain}

All domain types support domain assignment.  

\begin{syntax}
domain-expression:
  domain-literal
  domain-name
  domain-assignment-expression
  domain-striding-expression
  domain-alignment-expression
  domain-slice-expression

domain-literal:
  rectangular-domain-literal
  associative-domain-literal

domain-assignment-expression:
  domain-name = domain-expression

domain-name:
  identifier
\end{syntax}

Domain assignment is by
value and causes the target domain variable to take on the index set
of the right-hand side expression.  In practice, the right-hand side
expression is often another domain value; a tuple of ranges (for
regular domains); or a tuple of indices or a loop that enumerates
indices (for irregular domains).  If the domain variable being
assigned was used to declare arrays, these arrays are reallocated as
discussed in~\rsec{Association_of_Arrays_to_Domains}.

It is an error to assign a stridable domain to an unstridable domain
without an explicit conversion.

\begin{example}
The following three assignments show ways of assigning indices to a
sparse domain, \chpl{SpsD}.  The first assigns the domain two index
values, \chpl{(1,1)} and \chpl{(n,n)}.  The second assigns the domain
all of the indices along the diagonal from
\chpl{(1,1)}$\ldots$\chpl{(n,n)}.  The third invokes an iterator that
is written to \chpl{yield} indices read from a file named
``inds.dat''.  Each of these assignments has the effect of replacing
the previous index set with a completely new set of values.
\begin{chapel}
SpsD = ((1,1), (n,n));
SpsD = [i in 1..n] (i,i);
SpsD = readIndicesFromFile("inds.dat");
\end{chapel}
\end{example}

\subsection{Domain Striding}
\label{Domain_Striding}
\index{domains!striding}
\index{by@\chpl{by}!on rectangular domains}
\index{operators!by (domain)@\chpl{by} (domain)}

The \chpl{by} operator can be applied to a rectangular domain value in
order to create a strided rectangular domain value.  The right-hand
operand to the \chpl{by} operator can either be an integral value or
an integral tuple whose size matches the domain's rank.

\begin{syntax}
domain-striding-expression:
  domain-expression `by' expression
\end{syntax}

The type of the resulting domain is the same as the original domain
but with \chpl{stridable} set to true.  In the case of an integer
stride value, the value of the resulting domain is computed by
applying the integer value to each range in the value using the
\chpl{by} operator.  In the case of a tuple stride value, the resulting
domain's value is computed by applying each tuple component to the
corresponding range using the \chpl{by} operator.


\subsection{Domain Alignment}
\label{Domain_Alignment}
\index{domains!alignment}
\index{align@\chpl{align}!on rectangular domains}
\index{operators!align (domain)@\chpl{align} (domain)}

The \chpl{align} operator can be applied to a rectangular domain value in
order to change the alignment of a rectangular domain value.  The right-hand
operand to the \chpl{align} operator can either be an integral value or
an integral tuple whose size matches the domain's rank.

\begin{syntax}
domain-alignment-expression:
  domain-expression `align' expression
\end{syntax}

The type of the resulting domain is the same as the original domain
but with \chpl{stridable} set to true.
In the case of an integer alignment value, the value of the resulting
domain is computed by applying the integer value to each range in the
value using the \chpl{align} operator.  In the case of a tuple
alignment value, the resulting domain's value is computed by applying
each tuple component to the corresponding range using the \chpl{align}
operator.


\subsection{Domain Slicing}
\label{Domain_Slicing}
\index{slicing!domains}
\index{domains!slicing}

Slicing is the application of an index set to a domain.
It can be written using either parentheses or square brackets.
The index set can be defined with either a domain or a list of ranges.

\begin{syntax}
domain-slice-expression:
  domain-expression [ slicing-index-set ]
  domain-expression ( slicing-index-set )

slicing-index-set:
  domain-expression
  range-expression-list
\end{syntax}

The result of slicing, or a \emph{slice}, is a new domain value
that represents the intersection of
the index set of the domain being sliced and
the index set being applied.
The type and domain map of the slice match the domain being sliced.

Slicing can also be performed on an array,
resulting in aliasing a subset of the array's elements
(\rsec{Array_Slicing}).

\subsubsection{Domain-based Slicing}
\index{domain-based slicing}
\index{slicing!domain-based}

If the brackets or parentheses contain a domain value,
its index set is applied for slicing.

\begin{openissue}
Can we say that it is an alias in the case of sparse/associative?
% If so, need to reconcile getting an "alias" with getting
% a "new domain value", as claimed earlier.
\end{openissue}

\subsubsection{Range-based Slicing}
\label{Range_Based_Slicing}
\index{slicing!range-based}
\index{range-based slicing}
When slicing rectangular domains or arrays, the brackets or parentheses
can contain a list of \chpl{rank} ranges.  These ranges can either be bounded
or unbounded.
%
%REVIEW: vass: no, they don't inherit. We should either drop
% the following sentence or rephrase it to make it correct.
When unbounded, they inherit their bounds from the
domain or array being sliced.
%
The Cartesian product of the ranges' index sets is applied for slicing.

\begin{example}
The following code declares a two dimensional rectangular
domain \chpl{D}, and then a number of subdomains of \chpl{D} by
slicing into \chpl{D} using bounded and unbounded ranges.
The \chpl{InnerD} domain describes the inner indices of
D, \chpl{Col2OfD} describes the 2nd column of
\chpl{D}, and \chpl{AllButLastRow} describes all of \chpl{D} except
for the last row.

\begin{chapel}
const D: domain(2) = {1..n, 1..n},
      InnerD = D[2..n-1, 2..n-1],
      Col2OfD = D[.., 2..2],
      AllButLastRow = D[..n-1, ..];
\end{chapel}
\end{example}

\subsubsection{Rank-Change Slicing}
\label{Rank_Change_Slicing}
\index{slicing!rank-change}
\index{rank-change slicing}

For multidimensional rectangular domains and arrays, substituting
integral values for one or more of the ranges in a range-based slice
will result in a domain or array of lower rank.

The result of a rank-change slice on an array is an alias to a subset
of the array's elements as described
in~\rsec{Rectangular_Array_Slicing}.

The result of rank-change slice on a domain is a subdomain of the
domain being sliced.  The resulting
subdomain's type will be the same as the original domain, but with
a \chpl{rank} equal to the number of dimensions that were sliced by
ranges rather than integers.


\subsection{Count Operator}
\label{Count_Operator_Domains}
\index{domains!count operator}
\index{domains!#@\chpl{#}}
\index{# (domain)@\chpl{#} (domain)}
\index{operators!# (domain)@\chpl{#} (domain)}
The \chpl{#} operator can be applied to dense rectangular domains with
a tuple argument whose size matches the rank of the domain (or
optionally an integer in the case of a 1D domain).  The operator is
equivalent to applying the \chpl{#} operator to the component ranges
of the domain and then using them to slice the domain as in
Section~\ref{Range_Based_Slicing}.


\subsection{Adding and Removing Domain Indices}
\label{Adding_and_Removing_Domain_Indices}
\index{domains!adding indices}
\index{domains!removing indices}

All irregular domain types support the ability to incrementally add
and remove indices from their index sets.  This can either be done
using \chpl{add(i:idxType)} and \chpl{remove(i:idxType)} methods on a
domain variable or by using the \chpl{+=} and \chpl{-=} assignment
operators.  It is legal to add the same index to an irregular domain's
index set twice, but illegal to remove an index that does not belong
to the domain's index set.

\begin{openissue}
These remove semantics seem dangerous in a parallel context; maybe
add flags to both the method versions of the call that say whether
they should balk or not?  Or add exceptions...
\end{openissue}

As with normal domain assignments, arrays declared in terms of a
domain being modified in this way will be reallocated as discussed
in~\rsec{Association_of_Arrays_to_Domains}.

% TODO: describe operations on opaque domains, esp. domain.create().
% Note that domain.add() (should) work as defined above on
% opaque domains, taking values of the opaque type for its argument.


\section{Predefined Methods on Domains}
\index{domains!predefined functions}

This section gives a brief description of the library functions provided for
Domains.  These are categorized by the type of domain to which they apply: all,
regular or irregular.  Within each subsection, entries are listed in
alphabetical order.

\subsection{Methods on All Domain Types}
\index{domains!methods!common}
\index{domains!common methods}

The methods in this subsection can be applied to any domain.

\index{domains!clear@\chpl{clear}}
\index{predefined functions!clear@\chpl{clear}}
\begin{protohead}
proc $Domain$.clear()
\end{protohead}
\begin{protobody}
Resets this domain's index set to the empty set.
\end{protobody}

\begin{chapelexample}{clearAssociativeDomain}
This function provides a way to produce an empty associative domain.

When run, the code
\begin{chapel}
enum Counter { one, two, three };
var D : domain ( Counter ) = {Counter.one, Counter.two};
writeln("D has ", D.numIndices, " indices.");
D.clear();
writeln("D has ", D.numIndices, " indices.");
\end{chapel}
prints out
\begin{chapelprintoutput}{}
D has 2 indices.
D has 0 indices.
\end{chapelprintoutput}
\end{chapelexample}

\index{domains!dist@\chpl{dist}}
\index{predefined functions!dist@\chpl{dist}}
\begin{protohead}
proc $Domain$.dist : dmap
\end{protohead}
\begin{protobody}
Returns the domain map that implements this domain
\end{protobody}

\begin{chapelexample}{getDomainMap}
In the code
\begin{chapel}
use BlockDist;
proc foo(d : domain) where d.dist : Block {
  writeln("Block-distributed domain");
}
proc foo(d : domain) {
  writeln("Unknown distribution");
}
var D = {1..10} dmapped Block({1..10});
foo(D);
\end{chapel}
\chpl{dist} is used in a where-clause to determine the type of the argument's
distribution. The output is:
\begin{chapelprintoutput}{}
Block-distributed domain
\end{chapelprintoutput}
\end{chapelexample}

\index{domains!idxType@\chpl{idxType}}
\index{predefined functions!idxType@\chpl{idxType}}
\begin{protohead}
proc $Domain$.idxType type
\end{protohead}
\begin{protobody}
Returns the domain type's \chpl{idxType}.
This function is not available on opaque domains.
\end{protobody}

\begin{protohead}
proc $Domain$.indexOrder(i: index($Domain$)): idxType
\end{protohead}
\begin{protobody}
If \chpl{i} is a member of the domain, returns the ordinal value of
\chpl{i} using a total ordering of the domain's indices using 0-based
indexing.  Otherwise, it returns \chpl{(-1):idxType}.  For rectangular
domains, this ordering will be based on a row-major ordering of the
indices; for other domains, the ordering may be
implementation-defined and unstable as indices are added and
removed from the domain.
\end{protobody}

\index{domains!isIrregularDom@\chpl{isIrregularDom}}
\index{predefined functions!isIrregularDom@\chpl{isIrregularDom}}
\begin{protohead}
proc isIrregularDom(d: domain) param
\end{protohead}
\begin{protobody}
Returns a param \chpl{true} if the given domain is irregular, false otherwise.
\end{protobody}

\index{domains!isOpaqueDom@\chpl{isOpaqueDom}}
\index{predefined functions!isOpaqueDom@\chpl{isOpaqueDom}}
\begin{protohead}
proc isOpaqueDom(d: domain) param
\end{protohead}
\begin{protobody}
Returns a param \chpl{true} if the given domain is opaque, false otherwise.
\end{protobody}

\index{domains!isRectangularDom@\chpl{isRectangularDom}}
\index{predefined functions!isRectangularDom@\chpl{isRectangularDom}}
\begin{protohead}
proc isRectangularDom(d: domain) param
\end{protohead}
\begin{protobody}
Returns a param \chpl{true} if the given domain is rectangular, false otherwise.
\end{protobody}

\index{domains!isSparseDom@\chpl{isSparseDom}}
\index{predefined functions!isSparseDom@\chpl{isSparseDom}}
\begin{protohead}
proc isSparseDom(d: domain) param
\end{protohead}
\begin{protobody}
Returns a param \chpl{true} if the given domain is sparse, false otherwise.
\end{protobody}

\index{domains!member@\chpl{member}}
\index{predefined functions!member (domain)@\chpl{member} (domain)}
\begin{protohead}
proc $Domain$.member(i)
\end{protohead}
\begin{protobody}
Returns true if the given index \chpl{i} is a member of this domain's index set,
and false otherwise.
\end{protobody}

\begin{openissue}
We would like to call the type of i above idxType, but it's not true
for rectangular domains.  That observation provides some motivation to normalize
the behavior.
\end{openissue}

%REVIEW: vass: need to define 'capType' or replace with something that is defined
\index{domains!numIndices@\chpl{numIndices}}
\index{predefined functions!numIndices (domain)@\chpl{numIndices} (domain)}
\begin{protohead}
proc $Domain$.numIndices: capType
\end{protohead}
\begin{protobody}
Returns the number of indices in the domain as a value of the capacity type.
\end{protobody}

\subsection{Methods on Regular Domains}
\index{domains!methods!regular}
\index{regular domains!methods}

The methods described in this subsection can be applied to regular domains only.

\index{domains!dim@\chpl{dim}}
\index{predefined functions!dim (domain)@\chpl{dim} (domain)}
\begin{protohead}
proc $Domain$.dim(d: int): range
\end{protohead}
\begin{protobody}
Returns the range of indices described by dimension \chpl{d} of the
domain.
\end{protobody}

\begin{example}
In the code
\begin{chapel}
for i in D.dim(1) do
  for j in D.dim(2) do
    writeln(A(i,j));
\end{chapel}
domain \chpl{D} is iterated over by two nested loops.  The first
dimension of \chpl{D} is iterated over in the outer loop.  The second
dimension is iterated over in the inner loop.
\end{example}

\index{domains!dims@\chpl{dims}}
\index{predefined functions!dims (domain)@\chpl{dims} (domain)}
\begin{protohead}
proc $Domain$.dims(): rank*range
\end{protohead}
\begin{protobody}
Returns a tuple of ranges describing the dimensions of the domain.
\end{protobody}

% BLC: ``integral'' isn't really correct in the two 1D cases below,
% however, we don't really seem to have a user-level name for the
% per-dimension index type in the language that I can see.

\index{domains!expand@\chpl{expand}}
\index{predefined functions!expand (domain)@\chpl{expand} (domain)}
\begin{protohead}
proc $Domain$.expand(off: integral): domain
proc $Domain$.expand(off: rank*integral): domain
\end{protohead}
\begin{protobody}
Returns a new domain that is the current domain expanded in
dimension \chpl{d} if \chpl{off} or \chpl{off(d)} is positive or
contracted in dimension \chpl{d} if \chpl{off} or \chpl{off(d)} is
negative.
\end{protobody}

\index{domains!exterior@\chpl{exterior}}
\index{predefined functions!exterior (domain)@\chpl{exterior} (domain)}
\begin{protohead}
proc $Domain$.exterior(off: integral): domain
proc $Domain$.exterior(off: rank*integral): domain
\end{protohead}
\begin{protobody}
Returns a new domain that is the exterior portion of the current
domain with \chpl{off} or \chpl{off(d)} indices for each
dimension \chpl{d}.  If \chpl{off} or \chpl{off(d)} is negative,
compute the exterior from the low bound of the dimension; if positive,
compute the exterior from the high bound.
\end{protobody}

\index{domains!high@\chpl{high}}
\index{predefined functions!high (domain)@\chpl{high} (domain)}
\begin{protohead}
proc $Domain$.high: index($Domain$)
\end{protohead}
\begin{protobody}
Returns the high index of the domain as a value of the domain's index
type.
\end{protobody}

\index{domains!interior@\chpl{interior}}
\index{predefined functions!interior (domain)@\chpl{interior} (domain)}
\begin{protohead}
proc $Domain$.interior(off: integral): domain
proc $Domain$.interior(off: rank*integral): domain
\end{protohead}
\begin{protobody}
Returns a new domain that is the interior portion of the current
domain with \chpl{off} or \chpl{off(d)} indices for each
dimension \chpl{d}.  If \chpl{off} or \chpl{off(d)} is negative,
compute the interior from the low bound of the dimension; if positive,
compute the interior from the high bound.
\end{protobody}

\index{domains!low@\chpl{low}}
\index{predefined functions!low (domain)@\chpl{low} (domain)}
\begin{protohead}
proc $Domain$.low: index($Domain$)
\end{protohead}
\begin{protobody}
Returns the low index of the domain as a value of the domain's index
type.
\end{protobody}

\index{domains!rank@\chpl{rank}}
\index{predefined functions!rank (domain)@\chpl{rank} (domain)}
\begin{protohead}
proc $Domain$.rank param : int
\end{protohead}
\begin{protobody}
Returns the rank of the domain.
\end{protobody}

\index{domains!size@\chpl{size}}
\index{predefined functions!size (domain)@\chpl{size} (domain)}
\begin{protohead}
proc $Domain$.size: capType
\end{protohead}
\begin{protobody}
Same as $Domain$.numIndices.
\end{protobody}

\index{domains!stridable@\chpl{stridable}}
\index{predefined functions!stridable (domain)@\chpl{stridable} (domain)}
\begin{protohead}
proc $Domain$.stridable param : bool
\end{protohead}
\begin{protobody}
Returns whether or not the domain is stridable.
\end{protobody}

\index{domains!stride@\chpl{stride}}
\index{predefined functions!stride (domain)@\chpl{stride} (domain)}
\begin{protohead}
proc $Domain$.stride: int(numBits(idxType)) where rank == 1
proc $Domain$.stride: rank*int(numBits(idxType))
\end{protohead}
\begin{protobody}
Returns the stride of the domain as the domain's stride type (for 1D
domains) or a tuple of the domain's stride type (for multidimensional
domains).
\end{protobody}

\index{domains!translate@\chpl{translate}}
\index{predefined functions!translate (domain)@\chpl{translate} (domain)}
\begin{protohead}
proc $Domain$.translate(off: integral): domain
proc $Domain$.translate(off: rank*integral): domain
\end{protohead}
\begin{protobody}
Returns a new domain that is the current domain translated
by \chpl{off} or \chpl{off(d)} for each dimension \chpl{d}.
\end{protobody}

%% \begin{protohead} **/
%% proc $Domain$.position(i: index($Domain$)): rank*idxType **/
%% \end{protohead} **/
%% \begin{protobody} **/
%% Returns a tuple holding the order of index i in each range defining **/
%% the domain. **/
%% \end{protobody} **/

\subsection{Methods on Irregular Domains}
\index{domains!methods!irregular}
\index{irregular domains!methods}

The following methods are available only on irregular domain types.

\index{domains!+@\chpl{+}}
\index{predefined functions!+ (domain)@\chpl{+} (domain)}
\begin{protohead}
proc +(d: domain, i: index(d))
proc +(i, d: domain) where i: index(d)
\end{protohead}
\begin{protobody}
Adds the given index to the given domain.  If the given index is already a
member of that domain, it is ignored.
\end{protobody}

\index{domains!+@\chpl{+}}
\index{predefined functions!+ (domain)@\chpl{+} (domain)}
\begin{protohead}
proc +(d1: domain, d2: domain)
\end{protohead}
\begin{protobody}
Merges the index sets of the two domain arguments.
\end{protobody}

\index{domains!-@\chpl{-}}
\index{predefined functions!- (domain)@\chpl{-} (domain)}
\begin{protohead}
proc -(d: domain, i: index(d))
\end{protohead}
\begin{protobody}
Removes the given index from the given domain.  It is an error if the domain
does not contain the given index.
\end{protobody}

\index{domains!-@\chpl{-}}
\index{predefined functions!- (domain)@\chpl{-} (domain)}
\begin{protohead}
proc -(d1: domain, d2: domain)
\end{protohead}
\begin{protobody}
Removes the indices in domain \chpl{d2} from those in \chpl{d1}.  It is an error
if \chpl{d2} contains indices which are not also in \chpl{d1}.
\end{protobody}

\index{domains!requestCapacity@\chpl{requestCapacity}}
\index{predefined functions!requestCapacity@\chpl{requestCapacity}}
\begin{protohead}
proc requestCapacity(s: int)
\end{protohead}
\begin{protobody}
Resizes the domain internal storage to hold at least \chpl{s} indices.
\end{protobody}

\cleardoublepage
\sekshun{Arrays}
\label{Arrays}
\index{arrays}

An \emph{array} is a map from a domain's indices to a collection of
variables of homogenous type.  Since Chapel domains support a rich
variety of index sets, Chapel arrays are also richer than the
traditional linear or rectilinear array types in conventional
languages.  Like domains, arrays may be distributed across multiple
locales without explicitly partitioning them using Chapel's Domain
Maps~(\rsec{Domain_Maps}).


\section{Array Types}
\label{Array_Types}
\index{arrays!types}

An array type is specified by the identity of the domain that it is
declared over and the element type of the array.  Array types are
given by the following syntax:

\begin{syntax}
array-type:
  [ domain-expression ] type-specifier
\end{syntax}
The \sntx{domain-expression} must specify a domain that the array can
be declared over.  If the \sntx{domain-expression} is a domain
literal, the curly braces around the literal may be omitted.

\begin{chapelexample}{decls.chpl}
In the code
\begin{chapel}
const D: domain(2) = {1..10, 1..10};
var A: [D] real;
\end{chapel}
\begin{chapelpost}
writeln(D);
writeln(A);
\end{chapelpost}
\begin{chapeloutput}
{1..10, 1..10}
0.0 0.0 0.0 0.0 0.0 0.0 0.0 0.0 0.0 0.0
0.0 0.0 0.0 0.0 0.0 0.0 0.0 0.0 0.0 0.0
0.0 0.0 0.0 0.0 0.0 0.0 0.0 0.0 0.0 0.0
0.0 0.0 0.0 0.0 0.0 0.0 0.0 0.0 0.0 0.0
0.0 0.0 0.0 0.0 0.0 0.0 0.0 0.0 0.0 0.0
0.0 0.0 0.0 0.0 0.0 0.0 0.0 0.0 0.0 0.0
0.0 0.0 0.0 0.0 0.0 0.0 0.0 0.0 0.0 0.0
0.0 0.0 0.0 0.0 0.0 0.0 0.0 0.0 0.0 0.0
0.0 0.0 0.0 0.0 0.0 0.0 0.0 0.0 0.0 0.0
0.0 0.0 0.0 0.0 0.0 0.0 0.0 0.0 0.0 0.0
\end{chapeloutput}
\chpl{A} is declared to be an arithmetic array over rectangular
domain \chpl{D} with elements of type \chpl{real}.  As a result, it
represents a 2-dimensional $10 \times 10$ real floating point
variables indexed using the indices $(1, 1), (1, 2), \ldots, (1, 10),
(2, 1), \ldots, (10, 10)$.
\end{chapelexample}

%
% should the following be moved elsewhere?  Should handle these
% param/type queries consistently between this chapter and domains
% (and ranges?)
%
\index{arrays!element type}
An array's element type can be referred to using the member symbol
\chpl{eltType}.

\begin{chapelexample}{eltType.chpl}
In the following example, \chpl{x} is declared to be of type
\chpl{real} since that is the element type of array \chpl{A}.
\begin{chapelpre}
const D: domain(2) = {1..10, 1..10};
\end{chapelpre}
\begin{chapel}
var A: [D] real;
var x: A.eltType;
\end{chapel}
\begin{chapelpost}
writeln(x.type:string);
writeln(A.eltType:string);
\end{chapelpost}
\begin{chapeloutput}
real(64)
real(64)
\end{chapeloutput}
\end{chapelexample}

\section{Array Values}
\label{Array_Values}
\index{arrays!values}
\index{arrays!initialization}
\index{initialization!arrays}

An array's value is the collection of its elements' values.
Assignments between array variables are performed by value as
described in~\rsec{Array_Assignment}.  Chapel semantics are defined so
that the compiler will never need to insert temporary arrays of the
same size as a user array variable.

\index{arrays!literals}

Array literal values can be either rectangular or associative, corresponding to
the underlying domain which defines its indices. 

\begin{syntax}
array-literal:
  rectangular-array-literal
  associative-array-literal
\end{syntax}

\subsection{Rectangular Array Literals}
\index{rectangular array literals}
\index{arrays!rectangular!literals}

Rectangular array literals are specified by enclosing a comma separated list of 
expressions representing values in square brackets. A 1-based domain will 
automatically be generated for the given array literal.  The type of the array's 
values will be the type of the first element listed.

\begin{syntax}
rectangular-array-literal:
  [ expression-list ]
\end{syntax}

\begin{chapelexample}{adecl-literal.chpl}
The following example declares a 5 element rectangular array literal 
containing strings, then subsequently prints each string element to the console.
\begin{chapel}
var A = ["1", "2", "3", "4", "5"];

for i in 1..5 do
  writeln(A[i]);
\end{chapel}
\begin{chapeloutput}
1
2
3
4
5
\end{chapeloutput}
\end{chapelexample}

\begin{future}
Provide syntax which allows users to specify the domain for a rectangular 
array literal.
\end{future}

\begin{future}
Determine the type of a rectangular array literal based on the most promoted 
type, rather than the first element's type.
\end{future}

\begin{chapelexample}{decl-with-anon-domain.chpl}
The following example declares a 2-element array \chpl{A} containing 3-element
arrays of real numbers.  \chpl{A} is initialized using array literals.
\begin{chapel}
var A: [1..2] [1..3] real = [[1.1, 1.2, 1.3], [2.1, 2.2, 2.3]];
\end{chapel}
\begin{chapelpost}
writeln(A.domain);
\end{chapelpost}
\begin{chapeloutput}
{1..2}
\end{chapeloutput}
\end{chapelexample}

\begin{openissue}
We would like to differentiate syntactically between array literals for an array
of arrays and a multi-dimensional array. 
\end{openissue}

\index{arrays!rectangular!default values}
An rectangular array's default value is for each array element to be initialized to
the default value of the element type.

\subsection{Associative Array Literals}
\index{associative array literals}
\index{arrays!associative!literals}

Associative array values are specified by enclosing a comma separated list of
index-to-value bindings within square brackets. It is expected that the indices 
in the listing match in type and, likewise, the types of values in the listing 
also match. 

\begin{syntax}
associative-array-literal:
  [ associative-expr-list ]

associative-expr-list:
  index-expr => value-expr
  index-expr => value-expr, associative-expr-list

index-expr:
  expression

value-expr:
  expression
\end{syntax}

\begin{openissue}
Currently it is not possible to use other associative domains as values within
an associative array literal.
\end{openissue}

\begin{chapelexample}{adecl-assocLiteral.chpl}
The following example declares a 5 element associative array literal which maps
integers to their corresponding string representation. The indices and their
corresponding values are then printed. 
\begin{chapel}
var A = [1 => "one", 10 => "ten", 3 => "three", 16 => "sixteen"];

for da in zip (A.domain, A) do
  writeln(da);
\end{chapel}
\begin{chapeloutput}
(1, one)
(16, sixteen)
(10, ten)
(3, three)
\end{chapeloutput}
\end{chapelexample}

\subsection{Runtime Representation of Array Values}
\label{Array_Runtime_Representation}
\index{arrays!runtime representation}
\index{arrays!domain maps}

The runtime representation of an array in memory is controlled by its
domain's domain map.  Through this mechanism, users can reason about
and control the runtime representation of an array's elements.  See
~\rsec{Domain_Maps} for more details.


\section{Array Indexing}
\label{Array_Indexing}
\index{arrays!indexing}
\index{indexing!arrays}

Arrays can be indexed using index values from the domain over which
they are declared.  Array indexing is expressed using either
parenthesis or square brackets.  This results in a reference to the
element that corresponds to the index value.

% NEED SYNTAX DIAGRAM HERE?

\begin{chapelexample}{array-indexing.chpl}
Given:
\begin{chapel}
var A: [1..10] real;
\end{chapel}
the first two elements of A can be assigned the value 1.2 and 3.4
respectively using the assignment:
\begin{chapel}
A(1) = 1.2;
A[2] = 3.4;
\end{chapel}
\begin{chapelpost}
writeln(A.domain);
writeln(A);
\end{chapelpost}
\begin{chapeloutput}
{1..10}
1.2 3.4 0.0 0.0 0.0 0.0 0.0 0.0 0.0 0.0
\end{chapeloutput}
\end{chapelexample}

Except for associative arrays, if an array is indexed using an index that
is not part of its domain's index set, the reference is considered
out-of-bounds and a runtime error will occur, halting the program.

\subsection{Rectangular Array Indexing}
\label{Rectangular_Array_Indexing}
\index{indexing!rectangular arrays}
\index{rectangular arrays!indexing}

Since the indices for multidimensional rectangular domains are tuples,
for convenience, rectangular arrays can be indexed using the list of
integer values that make up the tuple index.  This is semantically
equivalent to creating a tuple value out of the integer values and
using that tuple value to index the array.  For symmetry,
1-dimensional rectangular arrays can be accessed using 1-tuple indices
even though their index type is an integral value.  This is
semantically equivalent to de-tupling the integral value from the
1-tuple and using it to index the array.

\begin{chapelexample}{array-indexing-2.chpl}
Given:
\begin{chapel}
var A: [1..5, 1..5] real;
var ij: 2*int = (1, 1);
\end{chapel}
the elements of array A can be indexed using any of the following
idioms:
\begin{chapel}
A(ij) = 1.1;
A((1, 2)) = 1.2;
A(1, 3) = 1.3;
A[ij] = -1.1;
A[(1, 4)] = 1.4;
A[1, 5] = 1.5;
\end{chapel}
\begin{chapelpost}
writeln(ij);
writeln(A);
\end{chapelpost}
\begin{chapeloutput}
(1, 1)
-1.1 1.2 1.3 1.4 1.5
0.0 0.0 0.0 0.0 0.0
0.0 0.0 0.0 0.0 0.0
0.0 0.0 0.0 0.0 0.0
0.0 0.0 0.0 0.0 0.0
\end{chapeloutput}
\end{chapelexample}

\begin{chapelexample}{index-using-var-arg-tuple.chpl}
The code
\begin{chapel}
proc f(A: [], is...)
  return A(is);
\end{chapel}
\begin{chapelpost}
var B: [1..5] int;
[i in 1..5] B(i) = i;
var C: [1..5,1..5] int;
[(i,j) in {1..5,1..5}] C(i,j) = i+i*j;
writeln(f(B, 3));
writeln(f(C, 3, 3));
\end{chapelpost}
\begin{chapeloutput}
3
12
\end{chapeloutput}
defines a function that takes an array as the first argument and a
variable-length argument list.  It then indexes into the array using
the tuple that captures the actual arguments.  This function works
even for one-dimensional arrays because one-dimensional arrays can be
indexed into by 1-tuples.
\end{chapelexample}

\subsection{Associative Array Indexing}
\label{Associative_Array_Indexing}
\index{indexing!associative arrays}
\index{associative arrays!indexing}

Indices can be added to associative arrays in two different ways.

The first way is through the array's domain.
\begin{chapelexample}{assoc-add-index.chpl}
Given:
\begin{chapel}
var D : domain(string);
var A : [D] int;
\end{chapel}

the array A initially contains no elements. We can change that by adding
indices to the domain D:
\begin{chapel}
D.add("a");
D.add("b");
\end{chapel}

The array A can now be indexed with indices "a" and "b":

\begin{chapel}
A["a"] = 1;
A["b"] = 2;
var x = A["a"];
\end{chapel}
\end{chapelexample}

The second way is more concise, and has the same effect as the first method:
\begin{chapelexample}{assoc-add-index-2.chpl}
\begin{chapel}
var D : domain(string);
var A : [D] int;
\end{chapel}
For other array types, assigning to an index not in the array's domain
would incur an out-of-bounds error. For associative arrays such assignment will
add the index to the array's domain, and the array can be indexed with the
newly added indices:
\begin{chapel}
A["a"] = 1;
A["b"] = 2;
var x = A["a"];
\end{chapel}
Here, the indices "a" and "b" are implicitly added the domain D. Reading from
an index not in the array is still an out-of-bounds error.
\begin{chapel}
// writeln(A["c"]); // halts if "c" is not in A's domain
\end{chapel}
An important restriction for this method is that A may not share its domain
with another array. This restriction exists because it may be surprising to
seemingly modify one array, and to then see a change in another array. This
restriction is checked at runtime.
\end{chapelexample}


\section{Iteration over Arrays}
\label{Iteration_over_Arrays}
\index{arrays!iteration}
\index{iteration!array}

% FYI: Similar to text regarding tuple iteration.  Slightly less
% similar for domain iteration.
All arrays support iteration via standard \chpl{for}, \chpl{forall}
and \chpl{coforall} loops.  These loops iterate over all of the array
elements as described by its domain.  A loop of the form:

% This is difficult to capture in a test program
\begin{chapel}
[for|forall|coforall] a in A do
  ...a...
\end{chapel}

is semantically equivalent to:

% This is difficult to capture in a test program
\begin{chapel}
[for|forall|coforall] i in A.domain do
  ...A[i]...
\end{chapel}

The iterator variable for an array iteration is a reference to the
array element type.


\section{Array Assignment}
\label{Array_Assignment}
\index{arrays!assignment}
\index{assignment!array}

Array assignment is by value.  Arrays can be assigned arrays, ranges,
domains, iterators, or tuples.

\begin{chapelexample}{assign.chpl}
If \chpl{A} is an lvalue of array type and \chpl{B} is an expression
of either array, range, or domain type, or an iterator, then the
assignment
\begin{chapelpre}
var A: [1..3] int;
var B: [1..3] int;
A = -1;
B = 1;
\end{chapelpre}
\begin{chapelnoprint}
writeln(A);
writeln(B);
\end{chapelnoprint}
\begin{chapel}
A = B;
\end{chapel}
\begin{chapelnoprint}
writeln(A);
writeln(B);
A = -2;
B = 2;
writeln(A);
writeln(B);
\end{chapelnoprint}
is equivalent to
\begin{chapel}
forall (a,b) in zip(A,B) do
  a = b;
\end{chapel}
\begin{chapelpost}
writeln(A);
writeln(B);
\end{chapelpost}
\begin{chapeloutput}
-1 -1 -1
1 1 1
1 1 1
1 1 1
-2 -2 -2
2 2 2
2 2 2
2 2 2
\end{chapeloutput}
If the zipper iteration is illegal, then the assignment is illegal.
Notice that the assignment is implemented with the semantics of
a \chpl{forall} loop.
\end{chapelexample}

Arrays can be assigned tuples of values of their element type if the
tuple contains the same number of elements as the array.  For
multidimensional arrays, the tuple must be a nested tuple such that
the nesting depth is equal to the rank of the array and the shape of
this nested tuple must match the shape of the array.  The values are
assigned element-wise.

% Is the above true for unordered array types?  Should it be?

Arrays can also be assigned single values of their element type.  In
this case, each element in the array is assigned this value.
\begin{chapelexample}{assign-2.chpl}
If \chpl{e} is an expression of the element type of the array or a
type that can be implicitly converted to the element type of the
array, then the assignment
\begin{chapelpre}
var A: [1..4] uint;
writeln(A);
var e: uint = 77;
\end{chapelpre}
\begin{chapel}
A = e;
\end{chapel}
\begin{chapelnoprint}
writeln(A);
e = 33;
\end{chapelnoprint}
is equivalent to
\begin{chapel}
forall a in A do
  a = e;
\end{chapel}
\begin{chapelpost}
writeln(A);
\end{chapelpost}
\begin{chapeloutput}
0 0 0 0
77 77 77 77
33 33 33 33
\end{chapeloutput}
\end{chapelexample}

\section{Array Slicing}
\label{Array_Slicing}
\index{arrays!slicing}
\index{slicing!array}

An array can be sliced using a domain that has the same type as the
domain over which it was declared.  The result of an array slice is an
alias to the subset of the array elements from the original array
corresponding to the slicing domain's index set.
 
\begin{chapelexample}{slicing.chpl}
Given the definitions
\begin{chapelpre}
config const n = 2;
\end{chapelpre}
\begin{chapel}
var OuterD: domain(2) = {0..n+1, 0..n+1};
var InnerD: domain(2) = {1..n, 1..n};
var A, B: [OuterD] real;
\end{chapel}
\begin{chapelnoprint}
writeln(OuterD);
writeln(InnerD);
B = 1;
\end{chapelnoprint}
the assignment given by
\begin{chapel}
A[InnerD] = B[InnerD];
\end{chapel}
\begin{chapelpost}
writeln(A);
writeln(B);
\end{chapelpost}
\begin{chapeloutput}
{0..3, 0..3}
{1..2, 1..2}
0.0 0.0 0.0 0.0
0.0 1.0 1.0 0.0
0.0 1.0 1.0 0.0
0.0 0.0 0.0 0.0
1.0 1.0 1.0 1.0
1.0 1.0 1.0 1.0
1.0 1.0 1.0 1.0
1.0 1.0 1.0 1.0
\end{chapeloutput}
assigns the elements in the interior of \chpl{B} to the elements in
the interior of \chpl{A}.
\end{chapelexample}

\subsection{Rectangular Array Slicing}
\label{Rectangular_Array_Slicing}
\index{arrays!slicing!rectangular}
\index{slicing!arrays!rectangular}

A rectangular array can be sliced by any rectangular domain that is a
subdomain of the array's defining domain.  If the subdomain
relationship is not met, an out-of-bounds error will occur.  The
result is a subarray whose indices are those of the slicing domain and
whose elements are an alias of the original array's.

Rectangular arrays also support slicing by ranges directly.  If each
dimension is indexed by a range, this is equivalent to slicing the
array by the rectangular domain defined by those ranges.  These
range-based slices may also be expressed using partially unbounded or
completely unbounded ranges.  This is equivalent to slicing the
array's defining domain by the specified ranges to create a subdomain
as described in~\rsec{Array_Slicing} and then using that subdomain to slice
the array.

\subsection{Rectangular Array Slicing with a Rank Change}
\label{Rectangular_Array_Slicing_With_Rank_Change}
\index{arrays!slicing!rectangular!rank change}

For multidimensional rectangular arrays, slicing with a rank change is
supported by substituting integral values within a dimension's range
for an actual range.  The resulting array will have a rank less than
the rectangular array's rank and equal to the number of ranges that are
passed in to take the slice.

\begin{chapelexample}{array-decl.chpl}
Given an array
\begin{chapelpre}
config const n = 4;
\end{chapelpre}
\begin{chapel}
var A: [1..n, 1..n] int;
\end{chapel}
\begin{chapelpost}
writeln(A);
\end{chapelpost}
\begin{chapeloutput}
0 0 0 0
0 0 0 0
0 0 0 0
0 0 0 0
\end{chapeloutput}
the slice \chpl{A[1..n, 1]} is a one-dimensional array whose elements
are the first column of \chpl{A}.
\end{chapelexample}


\section{Count Operator}
\label{Count_Operator_Arrays}
\index{arrays!count operator}
\index{operators!# (on arrays)@\chpl{#} (on arrays)}
The \chpl{#} operator can be applied to dense rectangular arrays with
a tuple argument whose size matches the rank of the array (or
optionally an integer in the case of a 1D array).  The operator is
equivalent to applying the \chpl{#} operator to the array's domain and
using the result to slice the array as described in
Section~\ref{Rectangular_Array_Slicing}.


\section{Array Arguments to Functions}
\label{Array_Arguments_To_Functions}
\index{arrays!actual arguments}
\index{arguments!array}

Arrays are passed to functions by reference.  Formal arguments that
receive arrays are aliases of the actual arguments.

% Do we really need to say this?  Should it be said here -- seems like
% there is no normal rule and that the cases should be described in
% the function intents section.
%
%  The ordinary rule
%that disallows assignment to formal arguments of default intent does not
%apply to arrays.

When a formal argument has array type, the element type of the array
can be omitted and/or the domain of the array can be queried or
omitted.  In such cases, the argument is generic and is discussed
in~\rsec{Formal_Arguments_of_Generic_Array_Types}.

If a formal array argument specifies a domain as part of its type
signature, the domain of the actual argument must represent the same
index set.  If the formal array's domain was declared using an
explicit domain map, the actual array's domain must use an equivalent
domain map.


\subsection{Array Promotion of Scalar Functions}
\label{Array_Promotion_of_Scalar_Functions}
\index{arrays!promotion}
\index{promotion!arrays}

Array promotion of a scalar function is defined over the array type
and the element type of the array.  The domain of the returned array,
if an array is captured by the promotion, is the domain of the array
that promoted the function.  In the event of zipper promotion over
multiple arrays, the promoted function returns an array with a domain
that is equal to the domain of the first argument to the function that
enables promotion.  If the first argument is an iterator or a range,
the result is a one-based one-dimensional array.

\begin{chapelexample}{whole-array-ops.chpl}
Whole array operations is a special case of array promotion of scalar
functions.  In the code
\begin{chapelpre}
var A, B, C: [1..3] real;
A = -1;
B = 2;
C = 3;
\end{chapelpre}
\begin{chapel}
A = B + C;
\end{chapel}
\begin{chapelpost}
writeln(A);
\end{chapelpost}
\begin{chapeloutput}
5.0 5.0 5.0
\end{chapeloutput}
if \chpl{A}, \chpl{B}, and \chpl{C} are arrays, this code assigns each
element in \chpl{A} the element-wise sum of the elements in \chpl{B}
and \chpl{C}.
\end{chapelexample}

%
% TODO: should have an example of promoting an actual function here
%


\section{Array Aliases}
\label{Array_Aliases}
\index{arrays!aliases}
\index{=> (array)@\chpl{=>} (array)}
\index{operators!=> (array)@\chpl{=>} (array)}

Array slices alias the data in arrays rather than copying it.  Such
array aliases can be captured and optionally reindexed with the array
alias operator \chpl{=>}.  The syntax for capturing an alias to an
array requires a new variable declaration:
\begin{syntax}
array-alias-declaration:
  identifier reindexing-expression[OPT] => array-expression ;

reindexing-expression:
  : [ domain-expression ]

array-expression:
  expression
\end{syntax}
The identifier is an alias to the array specified in
the \sntx{array-expression}.

The optional \sntx{reindexing-expression} allows the domain of the
array alias to be reindexed.  The shape of the domain in
the \sntx{reindexing-expression} must match the shape of the domain of
the \sntx{array-expression}.  Indexing via the alias is governed by
the new indices.

\begin{chapelexample}{reindexing.chpl}
In the code
\begin{chapel}
var A: [1..5, 1..5] int;
var AA: [0..2, 0..2] => A[2..4, 2..4];
\end{chapel}
\begin{chapelpost}
A = -11;
writeln(A);
AA = -66;
writeln(AA);
writeln(A);
\end{chapelpost}
\begin{chapeloutput}
-11 -11 -11 -11 -11
-11 -11 -11 -11 -11
-11 -11 -11 -11 -11
-11 -11 -11 -11 -11
-11 -11 -11 -11 -11
-66 -66 -66
-66 -66 -66
-66 -66 -66
-11 -11 -11 -11 -11
-11 -66 -66 -66 -11
-11 -66 -66 -66 -11
-11 -66 -66 -66 -11
-11 -11 -11 -11 -11
\end{chapeloutput}
an array alias \chpl{AA} is created to alias the interior of
array \chpl{A} given by the slice \chpl{A[2..4, 2..4]}.  The
reindexing expression changes the indices defined by the domain of the
alias to be zero-based in both dimensions.  Thus \chpl{AA(1,1)} is
equivalent to \chpl{A(3,3)}.
\end{chapelexample}

%
% TODO: need to insert something about using alias operator in
% constructors as well.  Ran out of time for 1.1 release
%


\section{Sparse Arrays}
\label{Sparse_Arrays}
\index{arrays!sparse}

Sparse arrays in Chapel are those whose domain is a sparse array.  A
sparse array differs from other array types in that it stores a single
value corresponding to multiple indices.  This value is commonly
referred to as the \emph{zero value}, but we refer to it as the
\emph{implicitly replicated value} or \emph{IRV} since it can take
on any value of the array's element type in practice including
non-zero numeric values, a class reference, a record or tuple value,
etc.

An array declared over a sparse domain can be indexed using any of the
indices in the sparse domain's parent domain.  If it is read using an
index that is not part of the sparse domain's index set, the IRV value
is returned.  Otherwise, the array element corresponding to the index
is returned.

Sparse arrays can only be written at locations corresponding to
indices in their domain's index set.  In general, writing to other
locations corresponding to the IRV value will result in a runtime
error.

By default a sparse array's IRV is defined as the default value for
the array's element type.  The IRV can be set to any value of the
array's element type by assigning to a pseudo-field named \chpl{IRV}
in the array.

\begin{chapelexample}{sparse-error.chpl}
The following code example declares a sparse array, \chpl{SpsA} using
the sparse domain \chpl{SpsD} (For this example, assume that
\chpl{n}$>$1).  Line~2 assigns two indices to \chpl{SpsD}'s index set
and then lines 3--4 store the values 1.1 and 9.9 to the corresponding
values of \chpl{SpsA}.  The IRV of \chpl{SpsA} will initially be 0.0
since its element type is \chpl{real}.  However, the fifth line sets
the IRV to be the value 5.5, causing \chpl{SpsA} to represent the
value 1.1 in its low corner, 9.9 in its high corner, and 5.5
everywhere else.  The final statement is an error since it attempts to
assign to \chpl{SpsA} at an index not described by its domain,
\chpl{SpsD}.

\begin{chapelpre}
config const n = 5;
const D = {1..n, 1..n};
\end{chapelpre}
\begin{chapel}
var SpsD: sparse subdomain(D);
var SpsA: [SpsD] real;
SpsD = ((1,1), (n,n));
SpsA(1,1) = 1.1;
SpsA(n,n) = 9.9;
SpsA.IRV = 5.5;
SpsA(1,n) = 0.0;  // ERROR!
\end{chapel}
\begin{chapeloutput}
sparse-error.chpl:9: error: halt reached - attempting to assign a 'zero' value in a sparse array: (1, 5)
\end{chapeloutput}
\end{chapelexample}



\section{Association of Arrays to Domains}
\label{Association_of_Arrays_to_Domains}
\index{domains!association with arrays}
\index{arrays!association with domains}

%
% Be sure to talk about resetting array values & assigning IRVs
%

When an array is declared, it is linked during execution to the domain
identity over which it was declared.  This linkage is invariant for
the array's lifetime and cannot be changed.

When indices are added or removed from a domain, the change impacts
the arrays declared over this particular domain.  In the case of
adding an index, an element is added to the array and initialized to
the IRV for sparse arrays, and to the default value for the element
type for dense arrays.  In the case of removing an index, the element
in the array is removed.

When a domain is reassigned a new value, its arrays are also impacted.
Values that correspond to indices in the intersection of the old and
new domain are preserved in the arrays.  Values that could only be
indexed by the old domain are lost.  Values that can only be indexed
by the new domain have elements added to the new array, initialized to
the IRV for sparse arrays, and to the element type's default value for
other array types.

For performance reasons, there is an expectation that a method will be
added to domains to allow non-preserving assignment, \emph{i.e.}, all
values in the arrays associated with the assigned domain will be lost.
Today this can be achieved by assigning the array's domain an empty
index set (causing all array elements to be deallocated) and then
re-assigning the new index set to the domain.

An array's domain can only be modified directly, via the domain's name
or an alias created by passing it to a function via default intent.  In
particular, the domain may not be modified via the array's
\chpl{.domain} method, nor by using the domain query syntax on a
function's formal array
argument~(\rsec{Formal_Arguments_of_Generic_Array_Types}).

\begin{rationale}
When multiple arrays are declared using a single domain, modifying the
domain affects all of the arrays.  Allowing an array's domain to be
queried and then modified suggests that the change should only affect
that array.  By requiring the domain to be modified directly, the user
is encouraged to think in terms of the domain distinctly from a
particular array.

In addition, this choice has the beneficial effect that arrays
declared via an anonymous domain have a constant domain.  Constant
domains are considered a common case and have potential compilation
benefits such as eliminating bounds checks.  Therefore making this
convenient syntax support a common, optimizable case seems prudent.
\end{rationale}


\section{Predefined Functions and Methods on Arrays}
\label{Predefined_Functions_and_Methods_on_Arrays}
\index{arrays!predefined functions}
\index{predefined functions!arrays}
\index{functions!arrays!predefined}

There is an expectation that this list of predefined methods will grow.

\index{arrays!eltType@\chpl{eltType}}
\index{predefined functions!eltType (array)@\chpl{eltType} (array)}
\begin{protohead}
proc $Array$.eltType type
\end{protohead}
\begin{protobody}
Returns the element type of the array.
\end{protobody}

\index{arrays!rank@\chpl{rank}}
\index{predefined functions!rank (array)@\chpl{rank} (array)}
\begin{protohead}
proc $Array$.rank param
\end{protohead}
\begin{protobody}
Returns the rank of the array.
\end{protobody}

\index{arrays!domain@\chpl{domain}}
\index{predefined functions!domain (array)@\chpl{domain} (array)}
\begin{protohead}
proc $Array$.domain: this.domain
\end{protohead}
\begin{protobody}
Returns the domain of the given array.  This domain is constant,
implying that the domain cannot be resized by assigning to its domain
field, only by modifying the domain directly.
\end{protobody}

\index{arrays!numElements@\chpl{numElements}}
\index{predefined functions!numElements (array)@\chpl{numElements} (array)}
\begin{protohead}
proc $Array$.numElements: this.domain.dim_type
\end{protohead}
\begin{protobody}
Returns the number of elements in the array.
\end{protobody}

\index{arrays!reshape@\chpl{reshape}}
\index{predefined functions!reshape (array)@\chpl{reshape} (array)}
\begin{protohead}
proc reshape(A: $Array$, D: $Domain$): $Array$
\end{protohead}
\begin{protobody}
Returns a copy of the array containing the same values but in the
shape of the new domain.  The number of indices in the domain must
equal the number of elements in the array.  The elements of the array
are copied into the new array using the default iteration orders over
both arrays.
\end{protobody}

\index{arrays!size@\chpl{size}}
\index{predefined functions!size (array)@\chpl{size} (array)}
\begin{protohead}
proc $Array$.size: this.domain.dim_type
\end{protohead}
\begin{protobody}
Same as $Array$.numElements.
\end{protobody}

\cleardoublepage
\sekshun{Iterators}
\label{Iterators}
\index{functions!iterators}
\index{iterators}

An \emph{iterator} is a function that can generate, or \emph{yield}, multiple values (consecutively or in parallel) via its yield statements.

\begin{openissue}
The parallel iterator story is under development.  It is expected that
the specification will be expanded regarding parallel iterators soon.
\end{openissue}

\section{Iterator Definitions}
\label{Iterator_Function_Definitions}
\index{iterators!definition}

The syntax to declare an iterator is given
by:
\begin{syntax}
iterator-declaration-statement:
  privacy-specifier[OPT] `iter' iterator-name argument-list[OPT] return-intent[OPT] return-type[OPT] where-clause[OPT]
  iterator-body

iterator-name:
  identifier

iterator-body:
  block-statement
  yield-statement
\end{syntax}

The syntax of an iterator declaration is similar to a procedure declaration, with
some key differences:
\begin{itemize}
\item The keyword \chpl{iter} is used instead of the keyword \chpl{proc}.
\item The name of the iterator cannot overload any operator.
\item \chpl{yield} statements may appear in the body of an iterator, but not in
a procedure.
\item A \chpl{return} statement in the body of an iterator is not allowed to have an expression.
\end{itemize}

\section{The Yield Statement}
\label{The_Yield_Statement}
\index{yield@\chpl{yield}}
\index{iterators!yield@\chpl{yield}}

The yield statement can only appear in iterators.  The syntax of the
yield statement is given by
\begin{syntax}
yield-statement:
  `yield' expression ;
\end{syntax}

When an iterator is executed and a \chpl{yield} is encountered, the value of the yield
expression is returned.  However, the state of execution of the iterator is
saved.  On its next invocation, execution resumes from the point immediately
following that \chpl{yield} statement and with the saved state of execution.
A yield statement in a variable iterator must contain an lvalue expression.

When a \chpl{return} is encountered, the iterator finishes without yielding another
index value.  The \chpl{return} statements appearing in an iterator are not
permitted to have a return expression.
An iterator also completes after the last
statement in the iterator is executed.
An iterator need not contain any yield statements.
% TODO: the current implementation requires at least one yield

% TODO specify how the return type is established/checked for an iterator,
% analogously to \label{Return_Types}.

\section{Iterator Calls}
\label{Iterator_Calls}
\index{iterators!calls}

Iterators are invoked using regular call expressions:
\begin{syntax}
iteratable-call-expression:
  call-expression
\end{syntax}

All details of iterator calls, including argument passing, function
resolution, the use of parentheses versus brackets to delimit the parameter
list, and so on,
are identical to procedure calls as described in Chapter~\ref{Functions}.

However, the result of an iterator call depends upon its context, as described below.

\subsection{Iterators in For and Forall Loops}
\label{Iterators_in_For_and_Forall_Loops}
\index{iterators!in for loops}
\index{iterators!in forall loops}

When an iterator is accessed via a for or forall loop, the iterator is
evaluated alongside the loop body in an interleaved manner.  For each
iteration, the iterator yields a value and the body is executed.

\subsection{Iterators as Arrays}
\label{Iterators_as_Arrays}
\index{iterators!and arrays}

If an iterator function is captured into a new variable declaration or
assigned to an array, the iterator is iterated over in total and the
expression evaluates to a one-dimensional arithmetic array that
contains the values returned by the iterator on each iteration.
\begin{chapelexample}{as-arrays.chpl}
Given this iterator
\begin{chapel}
iter squares(n: int): int {
  for i in 1..n do
    yield i * i;
}
\end{chapel}
\begin{chapelpost}
writeln(squares(5));
\end{chapelpost}
the expression \chpl{squares(5)} evaluates to
\begin{chapelprintoutput}{}
1 4 9 16 25
\end{chapelprintoutput}
\end{chapelexample}

\subsection{Iterators and Generics}
\label{Iterators_and_Generics}
\index{iterators!and generics}

An iterator call expression can be passed to a generic function argument that
has neither a declared type nor default value
(\rsec{Formal_Arguments_without_Types}).
In this case the iterator is passed without being evaluated.
Within the generic function the corresponding formal argument
can be used as an iterator, e.g. in for loops.
The arguments to the iterator call expression, if any, are evaluated
at the call site, i.e. prior to passing the iterator to the generic function.

\subsection{Recursive Iterators}
\label{Recursive_Iterators}
\index{iterators!recursive}

Recursive iterators are allowed. A recursive iterator invocation is
typically made by iterating over it in a loop.


\begin{chapelexample}{recursive.chpl}
A post-order traversal of a tree data structure could be written like this:
\begin{chapelnoprint}
class Tree {
  var data: string;
  var left, right: Tree;
  proc deinit() {
    if left then delete left;
    if right then delete right;
  }
}

var tree = new Tree("a", new Tree("b"), new Tree("c", new Tree("d"), new Tree("e")));
\end{chapelnoprint}
\begin{chapel}
iter postorder(tree: Tree): string {
  if tree != nil {
    for child in postorder(tree.left) do
      yield child;
    for child in postorder(tree.right) do
      yield child;
    yield tree.data;
  }
}
\end{chapel}
\begin{chapelnoprint}
proc Tree.writeThis(x)
{
  var first = true;
  for node in postorder(tree) {
    if first then first = false;
      else x.write(" ");
    write(node);
  }
}
writeln("Tree Data");
writeln(tree);
delete tree;
\end{chapelnoprint}
By contrast, using calls \chpl{postorder(tree.left)}
and \chpl{postorder(tree.right)} as stand-alone statements would
result in generating temporary arrays containing the outcomes of these
recursive calls, which would then be discarded.
\begin{chapeloutput}
Tree Data
b d e c a
\end{chapeloutput}
\end{chapelexample}

\subsection{Iterator Promotion of Scalar Functions}
\label{Iterator_Promotion_of_Scalar_Functions}
\index{iterators!promotion}
\index{promotion!iterator}

Iterator calls may be passed to a scalar function argument whose type
matches the iterator's yielded type.  This results in a promotion of the
scalar function as described in~\rsec{Promotion}.

\begin{chapelexample}{iteratorPromotion.chpl}
Given a function \chpl{addOne(x:int)} that accepts \chpl{int} values
and an iterator \chpl{firstN()} that yields \chpl{int} values,
\chpl{addOne()} can be called with \chpl{firstN()} as its actual argument.
This pattern creates a new iterator that yields the result of applying
\chpl{addOne()} to each value yielded by \chpl{firstN()}.

\begin{chapel}
proc addOne(x:int) {
  return x + 1;
}
iter firstN(n:int) {
  for i in 1..n {
    yield i;
  }
}
for number in addOne(firstN(10)) {
  writeln(number);
}
\end{chapel}
\begin{chapeloutput}
2
3
4
5
6
7
8
9
10
11
\end{chapeloutput}

\end{chapelexample}

\section{Parallel Iterators}
\label{Parallel_Iterators}
\index{parallel iterators}
\index{iterators!parallel}

Iterators used in explicit forall-statements or -expressions must be
parallel iterators.  Reductions, scans and promotion over serial
iterators will be serialized.

Parallel iterators are defined for standard constructs in Chapel such
as ranges, domains, and arrays, thereby allowing these constructs to
be used with forall-statements and -expressions.

The left-most iteratable expression in a forall-statement or
-expression determines the number of tasks, the iterations each task
executes, and the locales on which these tasks execute.  For ranges,
default domains, and default arrays, these values can be controlled
via configuration constants~(\rsec{data_parallel_knobs}).

Domains and arrays defined using distributed domain maps will
typically implement forall loops with multiple tasks on multiple
locales.  For ranges, default domains, and default arrays, all tasks
are executed on the current locale.

A more detailed definition of parallel iterators is forthcoming.

\cleardoublepage
\sekshun{Generics}
\label{Generics}
\index{generics}

Chapel supports generic functions and types that are parameterizable
over both types and parameters.  The generic functions and types look
similar to non-generic functions and types already discussed.

\section{Generic Functions}
\label{Generic_Functions}
\index{functions!generic}
\index{generics!functions}

A function is generic if any of the following conditions hold:
\begin{itemize}
\item
Some formal argument is specified with an intent of \chpl{type} or
\chpl{param}.
\item
Some formal argument has no specified type and no default value.
\item
Some formal argument is specified with a queried type.
\item
The type of some formal argument is a generic type, e.g., \chpl{List}.
Queries may be inlined in generic types, e.g., \chpl{List(?eltType)}.
\item
The type of some formal argument is an array type where either the
element type is queried or omitted or the domain is queried or
omitted.
\end{itemize}
These conditions are discussed in the next sections.

\subsection{Formal Type Arguments}
\label{Formal_Type_Arguments}
\index{intents!type@\chpl{type}}

If a formal argument is specified with intent \chpl{type}, then a type
must be passed to the function at the call site.  A copy of the
function is instantiated for each unique type that is passed to this
function at a call site.  The formal argument has the semantics of a
type alias.
\begin{chapelexample}{build2tuple.chpl}
The following code defines a function that takes two types at the call
site and returns a 2-tuple where the types of the components of the
tuple are defined by the two type arguments and the values are
specified by the types default values.
\begin{chapel}
proc build2Tuple(type t, type tt) {
  var x1: t;
  var x2: tt;
  return (x1, x2);
}
\end{chapel}
This function is instantiated with ``normal'' function call syntax
where the arguments are types:
\begin{chapel}
var t2 = build2Tuple(int, string);
t2 = (1, "hello");
\end{chapel}
\begin{chapelpost}
writeln(t2);
\end{chapelpost}
\begin{chapeloutput}
(1, hello)
\end{chapeloutput}
\end{chapelexample}

\subsection{Formal Parameter Arguments}
\label{Formal_Parameter_Arguments}
\index{intents!param@\chpl{param}}

If a formal argument is specified with intent \chpl{param}, then a
parameter must be passed to the function at the call site.  A copy of
the function is instantiated for each unique parameter that is passed
to this function at a call site.  The formal argument is a parameter.
\begin{chapelexample}{fillTuple.chpl}
The following code defines a function that takes an integer parameter
\chpl{p} at the call site as well as a regular actual argument of
integer type \chpl{x}.  The function returns a homogeneous tuple of
size \chpl{p} where each component in the tuple has the value of
\chpl{x}.
\begin{chapel}
proc fillTuple(param p: int, x: int) {
  var result: p*int;
  for param i in 1..p do
    result(i) = x;
  return result;
}
\end{chapel}
\begin{chapelpost}
writeln(fillTuple(3,3));
\end{chapelpost}
\begin{chapeloutput}
(3, 3, 3)
\end{chapeloutput}
The function call \chpl{fillTuple(3, 3)} returns a 3-tuple where each
component contains the value \chpl{3}.
\end{chapelexample}

\subsection{Formal Arguments without Types}
\label{Formal_Arguments_without_Types}
\index{formal arguments!without types}

If the type of a formal argument is omitted, the type of the formal
argument is taken to be the type of the actual argument passed to the
function at the call site.  A copy of the function is instantiated for
each unique actual type.
\begin{chapelexample}{fillTuple2.chpl}
The example from the previous section can be extended to be generic on
a parameter as well as the actual argument that is passed to it by
omitting the type of the formal argument \chpl{x}.  The following code
defines a function that returns a homogeneous tuple of size \chpl{p}
where each component in the tuple is initialized to \chpl{x}:
\begin{chapel}
proc fillTuple(param p: int, x) {
  var result: p*x.type;
  for param i in 1..p do
    result(i) = x;
  return result;
}
\end{chapel}
\begin{chapelpost}
var x = fillTuple(3, 3.14);
writeln(x);
writeln(x.type:string);
\end{chapelpost}
\begin{chapeloutput}
(3.14, 3.14, 3.14)
3*real(64)
\end{chapeloutput}
In this function, the type of the tuple is taken to be the type of the
actual argument.  The call \chpl{fillTuple(3, 3.14)} returns a 3-tuple
of real values \chpl{(3.14, 3.14, 3.14)}.  The return type is
\chpl{(real, real, real)}.
\end{chapelexample}

\subsection{Formal Arguments with Queried Types}
\label{Formal_Arguments_with_Queried_Types}
\index{formal arguments!with queried types}

If the type of a formal argument is specified as a queried type, the
type of the formal argument is taken to be the type of the actual
argument passed to the function at the call site.  A copy of the
function is instantiated for each unique actual type.  The queried
type has the semantics of a type alias.
\begin{chapelexample}{fillTuple3.chpl}
The example from the previous section can be rewritten to use a
queried type for clarity:
\begin{chapel}
proc fillTuple(param p: int, x: ?t) {
  var result: p*t;
  for param i in 1..p do
    result(i) = x;
  return result;
}
\end{chapel}
\begin{chapelpost}
var x = fillTuple(3, 3.14);
writeln(x);
writeln(x.type:string);
\end{chapelpost}
\begin{chapeloutput}
(3.14, 3.14, 3.14)
3*real(64)
\end{chapeloutput}
\end{chapelexample}

\subsection{Formal Arguments of Generic Type}
\label{Formal_Arguments_of_Generic_Type}
\index{formal arguments!generic}

If the type of a formal argument is a generic type, the type of the
formal argument is taken to be the type of the actual argument passed
to the function at the call site with the constraint that the type of
the actual argument is an instantiation of the generic type.  A copy
of the function is instantiated for each unique actual type.
\begin{example}
The following code defines a function \chpl{writeTop} that takes an
actual argument that is a generic stack
(see~\rsec{Example_Generic_Stack}) and outputs the top element of the
stack.  The function is generic on the type of its argument.
\begin{chapel}
proc writeTop(s: Stack) {
  write(s.top.item);
}
\end{chapel}
\end{example}

Types and parameters may be queried from the top-level types of formal
arguments as well.  In the example above, the formal argument's type
could also be specified as \chpl{Stack(?type)} in which case the
symbol \chpl{type} is equivalent to \chpl{s.itemType}.

Note that generic types which have default values for all of their
generic fields, \emph{e.g. range}, are not generic when simply
specified and require a query to mark the argument as generic.  For
simplicity, the identifier may be omitted.
\begin{example}
The following code defines a class with a type field that has a
default value.  Function \chpl{f} is defined to take an argument of
this class type where the type field is instantiated to the default.
Function \chpl{g}, on the other hand, is generic on its argument
because of the use of the question mark.
\begin{chapel}
class C {
  type t = int;
}
proc f(c: C) {
  // c.type is always int
}
proc g(c: C(?)) {
  // c.type may not be int
}
\end{chapel}
\end{example}

\index{where@\chpl{where}!implicit}
The generic type may be specified with some queries and some exact
values.  These exact values result in \emph{implicit where clauses}
for the purpose of function resolution.
\begin{example}
Given the class definition
\begin{chapel}
class C {
  type t;
  type tt;
}
\end{chapel}
then the function definition
\begin{chapel}
proc f(c: C(?t,real)) {
  // body
}
\end{chapel}
is equivalent to
\begin{chapel}
proc f(c: C(?t,?tt)) where tt == real {
  // body
}
\end{chapel}
\end{example}
For tuples with query arguments, an implicit where clause is always
created to ensure that the size of the actual tuple matches the
implicitly specified size of the formal tuple.
\begin{example}
The function definition
\begin{chapel}
proc f(tuple: (?t,real)) {
  // body
}
\end{chapel}
is equivalent to
\begin{chapel}
proc f(tuple: (?t,?tt)) where tuple.size == 2 && tt == real {
  // body
}
\end{chapel}
\end{example}

\begin{chapelexample}{query.chpl}
Type queries can also be used to constrain the types of other function arguments
and/or the return type.  In this example, the type query on the first argument
establishes type constraints on the other arguments and also determines the
return type.

The code
\begin{chapel}
writeln(sumOfThree(1,2,3));
writeln(sumOfThree(4.0,5.0,3.0));

proc sumOfThree(x: ?t, y:t, z:t):t {
   var sum: t;
   
   sum = x + y + z;
   return sum;
}
\end{chapel}
produces the output
\begin{chapelprintoutput}{}
6
12.0
\end{chapelprintoutput}
\end{chapelexample}

\index{integral (generic type)@\chpl{integral} (generic type)}
\index{numeric (generic type)@\chpl{numeric} (generic type)}
\index{enumerated (generic type)@\chpl{enumerated} (generic type)}
The generic types \chpl{integral}, \chpl{numeric} and \chpl{enumerated}
are generic types that can only be instantiated with, respectively, the
signed and unsigned integral types, all of the numeric types, and
enumerated types.

\subsection{Formal Arguments of Generic Array Types}
\label{Formal_Arguments_of_Generic_Array_Types}
\index{formal arguments!array}

If the type of a formal argument is an array where either the domain
or the element type is queried or omitted, the type of the formal
argument is taken to be the type of the actual argument passed to the
function at the call site.  If the domain is omitted, the domain of
the formal argument is taken to be the domain of the actual argument.

A queried domain may not be modified via the name to which it is bound
(see~\rsec{Association_of_Arrays_to_Domains} for rationale).

\section{Function Visibility in Generic Functions}
\label{Function_Visibility_in_Generic_Functions}
\index{generics!function visibility}

Function visibility in generic functions is altered depending on the
instantiation.  When resolving function calls made within generic
functions, the visible functions are taken from any call site at which
the generic function is instantiated for each particular
instantiation.  The specific call site chosen is arbitrary and it is
referred to as the \emph{point of instantiation}.

For function calls that specify the module
explicitly~(\rsec{Explicit_Naming}), an implicit use of the specified
module exists at the call site.

\begin{chapelexample}{point-of-instantiation.chpl}
Consider the following code which defines a generic
function \chpl{bar}:
\begin{chapel}
module M1 {
  record R {
    var x: int;
    proc foo() { }
  }
}

module M2 {
  proc bar(x) {
    x.foo();
  }
}

module M3 {
  use M1, M2;
  proc main() {
    var r: R;
    bar(r);
  }
}
\end{chapel}
\begin{chapeloutput}
\end{chapeloutput}
In the function \chpl{main}, the variable \chpl{r} is declared to be
of type \chpl{R} defined in module \chpl{M1} and a call is made to the
generic function \chpl{bar} which is defined in module \chpl{M2}.
This is the only place where \chpl{bar} is called in this program and
so it becomes the point of instantiation for \chpl{bar} when the
argument \chpl{x} is of type \chpl{R}.  Therefore, the call to
the \chpl{foo} method in \chpl{bar} is resolved by looking for visible
functions from within \chpl{main} and going through the use of
module \chpl{M1}.
\end{chapelexample}

If the generic function is only called indirectly through dynamic
dispatch, the point of instantiation is defined as the point at which
the derived type (the type of the implicit \chpl{this} argument) is
defined or instantiated (if the derived type is generic).

\begin{rationale}
Visible function lookup in Chapel's generic functions is handled
differently than in C++'s template functions in that there is no split
between dependent and independent types.

Also, dynamic dispatch and instantiation is handled differently.
Chapel supports dynamic dispatch over methods that are generic in some
of its formal arguments.

Note that the Chapel lookup mechanism is still under development and
discussion.  Comments or questions are appreciated.
\end{rationale}

\section{Generic Types}
\label{Generic_Types}
\index{generics!types}
\index{types!generic}
\index{generics!classes}
\index{classes!generic}
\index{generics!records}
\index{records!generic}

Generic types are generic classes and generic records.
A class or record is generic if it contains one or more
\index{generics!fields}
\index{fields!generic}
generic fields. A generic field is one of:
\begin{itemize}
\item a specified or unspecified type alias,
\item a parameter field, or
\item a \chpl{var} or \chpl{const} field that has no type and no initialization
expression.
\end{itemize}

For each generic field, the class or record is parameterized over:
\begin{itemize}
\item the type bound to the type alias,
\item the value of the parameter field, or
\item the type of the \chpl{var} or \chpl{const} field, respectively.
\end{itemize}
Correspondingly, the class or record is instantiated with a set
of types and parameter values, one type or value per generic field.

% Here are the aspects to be defined for each kind of generic field:
% - what it makes the class/record generic over
% - the type constructor arg that gets created
% - the default constructor arg that gets created
% - the requirements on the corresponding user-defined constructor arg
% - for each of the above args:
%    - what kind of actual it accepts (type, param, value)
%    - what is the semantics;
%      i.e. how it corresponds to the class/record's genericity
%    - what is the arg's default, if any
% 
% In the presentation below, some of these aspects are discussed
% in the field-kind-specific subsections, some in the constructor-specific
% subsections, some in both.  I.e. there is an overlap between
% field-kind and constructor subsections; that should be OK but feel free
% to clean up.
% 
% It would be cool to summarize that in a table
% (one dimension: field kinds; the other dimension: aspects).

\subsection{Type Aliases in Generic Types}
\label{Type_Aliases_in_Generic_Types}
\index{type aliases!in classes or records}
\index{fields!type alias}

If a class or record defines a type alias, the class or record
is generic over the type that is bound to that alias.
% Type aliases defined in a class or a record can be unspecified type
% aliases; type aliases that are not bound to a type.  If a class or
% record contains an unspecified type alias, the aliased type must be
% specified whenever the type is used.
Such a type alias is accessed as if it were a field;
similar to a parameter field, it cannot be assigned
except in its declaration.

The type alias becomes an argument with intent \chpl{type} to
the compiler-generated constructor (\rsec{Generic_Compiler_Generated_Constructors})
for its class or record. This makes the compiler-generated constructor generic.
The type alias also becomes an argument with intent \chpl{type} to
the type constructor (\rsec{Type_Constructors}).
If the type alias declaration binds it to a type, that type
becomes the default for these arguments, otherwise they have no defaults.

The class or record is instantiated by binding the type alias
to the actual type passed to the corresponding argument of
a user-defined (\rsec{Generic_User_Constructors})
or compiler-generated constructor or type constructor.
If that argument has a default, the actual type can be omitted, in
which case the default will be used instead.

\begin{chapelexample}{NodeClass.chpl}
The following code defines a class called \chpl{Node} that implements
a linked list data structure.  It is generic over the type of the
element contained in the linked list.
\begin{chapel}
class Node {
  type eltType;
  var data: eltType;
  var next: Node(eltType);
}
\end{chapel}
\begin{chapelpost}
var n: Node(real) = new Node(real, 3.14);
writeln(n.data);
writeln(n.next);
writeln(n.next.type:string);
delete n;
\end{chapelpost}
\begin{chapeloutput}
3.14
nil
Node(real(64))
\end{chapeloutput}
The call \chpl{new Node(real, 3.14)} creates a node in the linked list
that contains the value \chpl{3.14}.  The \chpl{next} field is set to
nil.  The type specifier \chpl{Node} is a generic type and cannot be
used to define a variable.  The type specifier \chpl{Node(real)}
denotes the type of the \chpl{Node} class instantiated over
\chpl{real}.  Note that the type of the \chpl{next} field is specified
as \chpl{Node(eltType)}; the type of \chpl{next} is the same type as
the type of the object that it is a field of.
\end{chapelexample}

\subsection{Parameters in Generic Types}
\label{Parameters_in_Generic_Types}
\index{parameters!in classes or records}
\index{fields!parameter}

If a class or record defines a parameter field, the class or record
is generic over the value that is bound to that field.
% A parameter defined in a class or record is accessed as if it were a
% field.  This access returns a parameter.  
The parameter becomes an argument with intent \chpl{param} to the
compiler-generated constructor (\rsec{Generic_Compiler_Generated_Constructors})
for that class or record.  This makes the compiler-generated
constructor generic.  The parameter also becomes an argument
with intent \chpl{param} to the type  constructor (\rsec{Type_Constructors}).
If the parameter declaration has an initialization expression, that expression
becomes the default for these arguments, otherwise they have no defaults.

The class or record is instantiated by binding the parameter
to the actual value passed to the corresponding argument of
a user-defined (\rsec{Generic_User_Constructors})
or compiler-generated constructor or type constructor.
If that argument has a default, the actual value can be omitted, in
which case the default will be used instead.

\begin{chapelexample}{IntegerTuple.chpl}
The following code defines a class called \chpl{IntegerTuple} that is
generic over an integer parameter which defines the number of
components in the class.
\begin{chapel}
class IntegerTuple {
  param size: int;
  var data: size*int;
}
\end{chapel}
\begin{chapelpost}
var x = new IntegerTuple(3);
writeln(x.data);
delete x;
\end{chapelpost}
\begin{chapeloutput}
(0, 0, 0)
\end{chapeloutput}
The call \chpl{new IntegerTuple(3)} creates an instance of the
\chpl{IntegerTuple} class that is instantiated over parameter
\chpl{3}.  The field \chpl{data} becomes a 3-tuple of integers.  The
type of this class instance is \chpl{IntegerTuple(3)}.  The type
specified by \chpl{IntegerTuple} is a generic type.
\end{chapelexample}

\subsection{Fields without Types}
\label{Fields_without_Types}
\index{fields!variable and constant, without types}
\index{variables!in classes or records}
\index{constants!in classes or records}

If a \chpl{var} or \chpl{const} field in a class or record has no specified type or
initialization expression, the class or record is generic over the
type of that field.  The field becomes an argument with default intent to
the compiler-generated constructor (\rsec{Generic_Compiler_Generated_Constructors}).
That argument has no specified type and no default
value. This makes the compiler-generated constructor generic.
The field also becomes an argument with \chpl{type} intent and no default
to the type constructor (\rsec{Type_Constructors}).
Correspondingly, an actual value must always be passed to the default
constructor argument and an actual type to the type constructor argument.

The class or record is instantiated by binding the type of the field
to the type of the value passed to the corresponding argument
of a user-defined (\rsec{Generic_User_Constructors}) or compiler-generated constructor (\rsec{Generic_Compiler_Generated_Constructors}).
When the type constructor is invoked, the class or record is instantiated
by binding the type of the field to the actual type passed to
the corresponding argument.

\begin{chapelexample}{fieldWithoutType.chpl}
The following code defines another class called \chpl{Node} that
implements a linked list data structure.  It is generic over the type
of the element contained in the linked list.  This code does not
specify the element type directly in the class as a type alias but
rather omits the type from the \chpl{data} field.
\begin{chapel}
class Node {
  var data;
  var next: Node(data.type) = nil;
}
\end{chapel}
A node with integer element type can be defined in the call to the
constructor.  The call \chpl{new Node(1)} defines a node with the
value \chpl{1}.  The code
\begin{chapel}
var list = new Node(1);
list.next = new Node(2);
\end{chapel}
\begin{chapelpost}
writeln(list.data);
writeln(list.next.data);
delete list.next;
delete list;
\end{chapelpost}
\begin{chapeloutput}
1
2
\end{chapeloutput}
defines a two-element list with nodes containing the values \chpl{1}
and \chpl{2}.  The type of each object could be specified
as \chpl{Node(int)}.
\end{chapelexample}

\subsection{The Type Constructor}
\label{Type_Constructors}
\index{generics!type constructor}
\index{constructors!type constructors}

A type constructor is automatically created for each class or record.
A type constructor is a type function (\rsec{Type_Return_Intent}) that has
the same name as the class or record.  It takes one argument per the
class's or record's generic field, including fields inherited from the
superclasses, if any.
The formal argument has intent \chpl{type} for a type alias field and is a
parameter for a parameter field. It accepts the type to be bound
to the type alias and the value to be bound to the parameter, respectively.
For a generic \chpl{var} or \chpl{const} field, the corresponding
formal argument also has intent \chpl{type}. It accepts the type
of the field, as opposed to a value as is the case for a parameter field.
The formal arguments occur in the same order as the fields are
declared and have the same names as the corresponding fields.
Unlike the compiler-generated constructor, the type constructor has only
those arguments that correspond to generic fields.

A call to a type constructor accepts actual types and parameter values
and returns the type of the class or record that is instantiated
appropriately for each field
(\rsec{Type_Aliases_in_Generic_Types}, \rsec{Parameters_in_Generic_Types},
\rsec{Fields_without_Types}).
\index{generics!instantiated type}
Such an instantiated type must be used as the type of a variable,
array element, non-generic formal argument, and in other cases
where uninstantiated generic class or record types are not allowed.

When a generic field declaration has an initialization expression
or a type alias is specified, that initializer becomes the default value
for the corresponding type constructor argument.  Uninitialized
fields, including all generic \chpl{var} and \chpl{const} fields,
and unspecified type aliases result in arguments with no defaults;
actual types or values for these arguments must always be provided
when invoking the type constructor.

\subsection{Generic Methods}
\label{Generic_Methods}
\index{generics!methods}
\index{methods!generic}

All methods bound to generic classes or records, including
constructors, are generic over the implicit \chpl{this} argument.
This is in addition to being generic over any other argument that is generic.

\subsection{The Compiler-Generated Constructor}
\label{Generic_Compiler_Generated_Constructors}
\index{generics!constructors!compiler-generated}
\index{constructors!compiler-generated!for generic classes or records}

If no user-defined constructors are supplied for a given generic class, the
compiler generates one following in a manner similar to that for concrete
classes (\rsec{The_Compiler_Generated_Constructor}).
However, the compiler-generated constructor for a generic class or record
(\rsec{The_Compiler_Generated_Constructor}) is generic over each argument that
corresponds to a generic field, as specified above.
The argument has intent \chpl{type} for a type alias field and is a
parameter for a parameter field. It accepts the type to be bound
to the type alias and the value to be bound to the parameter, respectively.
This is the same as for the type constructor.
For a generic \chpl{var} or \chpl{const} field, the corresponding
formal argument has the default intent and accepts the value
for the field to be initialized with. The type of the field
is inferred automatically to be the type of the initialization value.

The default values for the generic arguments of the compiler-generated constructor
are the same as for the type constructor (\rsec{Type_Constructors}).
For example, the arguments corresponding to the generic \chpl{var}
and \chpl{const} fields, if any, never have defaults, so the corresponding
actual values must always be provided.

\subsection{User-Defined Constructors}
\label{Generic_User_Constructors}
\index{generics!constructors!user-defined}
\index{constructors!user-defined!for generic classes or records}

If a generic field of a class does not have an initialization expression
or a type alias is unspecified, each user-defined constructor for that
class must provide a formal argument whose name
matches the name of the field.

If the name of a formal argument in a user-defined constructor matches the name
of a generic field that does not have an initialization
expression, is a type alias, or is a parameter field, the field is
automatically initialized at the beginning of the constructor invocation
to the actual value of that argument.
This is done by passing that formal argument to the implicit invocation
of the compiler-generated constructor during default-initialization (\rsec{Default_Initialization}).

%%  The following story is nicer but it's not how it is implemented:
%If the name of a formal argument in a class constructor
%matches the name of a generic field, the field is automatically initialized
%to the actual value for that argument upon the constructor invocation.
%If the generic field does not have an initialization expression,
%such a matching formal argument must be provided in each constructor
%for that class.

\begin{chapelexample}{constructorsForGenericFields.chpl}
In the following code:
\begin{chapel}
class MyGenericClass {
  type t1;
  param p1;
  const c1;
  var v1;
  var x1: t1; // this field is not generic

  type t5 = real;
  param p5 = "a string";
  const c5 = 5.5;
  var v5 = 555;
  var x5: t5; // this field is not generic

  proc MyGenericClass(c1, v1, type t1, param p1) { }
  proc MyGenericClass(type t5, param p5, c5, v5, x5,
                     type t1, param p1, c1, v1, x1) { }
}  // class MyGenericClass

var g1 = new MyGenericClass(11, 111, int, 1);
var g2 = new MyGenericClass(int, "this is g2", 3.3, 333, 3333,
                            real, 2, 222, 222.2, 22);
\end{chapel}
\begin{chapelpost}
writeln(g1.p1);
writeln(g1.p5);
writeln(g1);
writeln(g2.p1);
writeln(g2.p5);
writeln(g2);
delete g1;
delete g2;
\end{chapelpost}
\begin{chapeloutput}
1
a string
{c1 = 11, v1 = 111, x1 = 0, c5 = 5.5, v5 = 555, x5 = 0.0}
2
this is g2
{c1 = 222, v1 = 222.2, x1 = 0.0, c5 = 5.5, v5 = 555, x5 = 0}
\end{chapeloutput}
The arguments \chpl{t1}, \chpl{p1}, \chpl{c1}, and \chpl{v1} are
required in all constructors for \chpl{MyGenericClass}. They can appear
in any order. Both \chpl{MyGenericClass} constructors initialize the
corresponding fields implicitly because these fields do not have initialization
expressions. The second constructor also initializes implicitly
the fields \chpl{t5} and \chpl{p5} because they are a type field
and a parameter field. It does not initialize the fields \chpl{c5}
and \chpl{v5} because they have initialization expressions, or
the fields \chpl{x1} and \chpl{x5} because they are not generic fields
(even though they are of generic types).
\end{chapelexample}

\begin{openissue}
The design of constructors, especially for generic classes, is
under development, so the above specification may change.
\end{openissue}

\pagebreak
\section{User-Defined Compiler Diagnostics}
\label{User_Defined_Compiler_Errors}
\index{user-defined compiler diagnostics}
\index{user-defined compiler errors}
\index{user-defined compiler warnings}
\index{compilerError}
\index{compilerWarning}

The special compiler diagnostic function calls \chpl{compilerError}
and \chpl{compilerWarning} generate compiler diagnostic of the
indicated severity if the function containing these calls may be
called when the program is executed and the function call is not
eliminated by parameter folding.

The compiler diagnostic is defined by the actual arguments which must
be string parameters.  The diagnostic points to the spot in the Chapel
program from which the function containing the call is called.
Compilation halts if a \chpl{compilerError} is encountered whereas it
will continue after encountering a \chpl{compilerWarning}.

\begin{chapelexample}{compilerDiagnostics.chpl}
The following code shows an example of using user-defined compiler
diagnostics to generate warnings and errors:
\begin{chapel}
proc foo(x, y) {
  if (x.type != y.type) then
    compilerError("foo() called with non-matching types: ", 
                  x.type:string, " != ", y.type:string);
  writeln("In 2-argument foo...");
}

proc foo(x) {
  compilerWarning("1-argument version of foo called");
  writeln("In generic foo!");
}
\end{chapel}
\begin{chapelpost}
foo(3.4);
foo("hi");
foo(1, 2);
foo(1.2, 3.4);
foo("hi", "bye");
\end{chapelpost}
\begin{chapeloutput}
compilerDiagnostics.chpl:12: warning: 1-argument version of foo called
compilerDiagnostics.chpl:13: warning: 1-argument version of foo called
In generic foo!
In generic foo!
In 2-argument foo...
In 2-argument foo...
In 2-argument foo...
\end{chapeloutput}

The first routine generates a compiler error whenever the compiler
encounters a call to it where the two arguments have different types.
It prints out an error message indicating the types of the arguments.
The second routine generates a compiler warning whenever the compiler
encounters a call to it.

Thus, if the program foo.chpl contained the following calls:

\begin{numberedchapel}
foo(3.4);
foo("hi");
foo(1, 2);
foo(1.2, 3.4);
foo("hi", "bye");
foo(1, 2.3);
foo("hi", 2.3);
\end{numberedchapel}

\noindent compiling the program would generate output like:

\begin{commandline}
foo.chpl:1: warning: 1-argument version of foo called with type: real
foo.chpl:2: warning: 1-argument version of foo called with type: string
foo.chpl:6: error: foo() called with non-matching types: int != real
\end{commandline}

\end{chapelexample}

\section{Creating General and Specialized Versions of a Function}
\label{Creating_General_and_Specialized_Versions_of_a_Function}
\index{specific instantiations}
\index{generic specialization}
\index{generic functions and special versions}

The Chapel language facility supports three mechanisms for using generic
functions along with concrete functions. These mechanisms allow users to
create a general generic implementation and also a special implementation
for specific concrete types.

The first mechanism applies to functions.
According to the function resolution rules described in
\rsec{Function_Resolution}, functions accepting concrete arguments are
selected in preference to those with a totally generic argument. So,
creating a second version of a generic function that declares a concrete
type will cause the concrete function to be used where possible:

\begin{chapelexample}{specializeGenericFunction.chpl}
\begin{chapel}
proc foo(x) {
  writeln("in generic foo(x)");
}
proc foo(x:int) {
  writeln("in specific foo(x:int)");
}

var myReal:real;
foo(myReal); // outputs "in generic foo(x)"
var myInt:int;
foo(myInt); // outputs "in specific foo(x:int)"
\end{chapel}
\begin{chapeloutput}
in generic foo(x)
in specific foo(x:int)
\end{chapeloutput}
\end{chapelexample}

This program will run the generic foo function if the argument is a real,
but it runs the specific version for int if the argument is an int.

The second mechanism applies when working with methods on generic types.
When declaring a secondary method, the receiver type can be a
parenthesized expression. In that case, the compiler will evaluate the
parenthesized expression at compile time in order to find the concrete
receiver type. Then, the resolution rules described above will cause the
concrete method to be selected when applicable. For example:

\begin{chapelexample}{specializeGenericMethod.chpl}
\begin{chapel}

record MyNode {
  var field;  // since no type is specified here, MyNode is a generic type
}

proc MyNode.foo() {
  writeln("in generic MyNode.foo()");
}
proc (MyNode(int)).foo() {
  writeln("in specific MyNode(int).foo()");
}

var myRealNode = new MyNode(1.0);
myRealNode.foo(); // outputs "in generic MyNode.foo()"
var myIntNode = new MyNode(1);
myIntNode.foo(); // outputs "in specific MyNode(int).foo()"
\end{chapel}
\begin{chapeloutput}
in generic MyNode.foo()
in specific MyNode(int).foo()
\end{chapeloutput}
\end{chapelexample}

The third mechanism is to use a where clause. Where clauses limit a
generic method to particular cases. Unlike the previous two cases, a
where clause can be used to declare special implementation of a function
that works with some set of types - in other words, the special
implementation can still be a generic function.  See also
\rsec{Where_Expressions}.

\section{Example: A Generic Stack}
\label{Example_Generic_Stack}
\index{generics!examples!stack}
\begin{chapelexample}{genericStack.chpl}
\begin{chapel}
class MyNode {
  type itemType;              // type of item
  var item: itemType;         // item in node
  var next: MyNode(itemType); // reference to next node (same type)
}

record Stack {
  type itemType;             // type of items
  var top: MyNode(itemType); // top node on stack linked list

  proc push(item: itemType) {
    top = new MyNode(itemType, item, top);
  }

  proc pop() {
    if isEmpty then
      halt("attempt to pop an item off an empty stack");
    var oldTop = top;
    var oldItem = top.item;
    top = top.next;
    delete oldTop;
    return oldItem;
  }

  proc isEmpty return top == nil;
}
\end{chapel}
\begin{chapelpost}
var s: Stack(int);
s.push(1);
s.push(2);
s.push(3);
while !s.isEmpty do
  writeln(s.pop());
\end{chapelpost}
\begin{chapeloutput}
3
2
1
\end{chapeloutput}
\end{chapelexample}

\cleardoublepage
\sekshun{Input and Output}
\label{Input_and_Output}
% got one right below: \index{input and output}
\index{input/output}

\section{See Library Documentation}

Chapel includes an extensive library for input and output that is
documented in the standard library documentation. See
\url{http://chapel.cray.com/docs/latest/modules/standard/IO.html}
and
\url{http://chapel.cray.com/docs/latest/modules/internal/ChapelIO.html}.


\cleardoublepage
\sekshun{Task Parallelism and Synchronization}
\label{Task_Parallelism_and_Synchronization}
\index{synchronization}

Chapel supports both task parallelism and data parallelism.  This
chapter details task parallelism as follows:
\begin{itemize}
\item \rsec{Task_parallelism} introduces tasks and task parallelism.
\item \rsec{Begin} describes the begin statement, an unstructured way
to introduce concurrency into a program.
\item \rsec{Synchronization_Variables} describes synchronization
variables, an unstructured mechanism for synchronizing tasks.
\item \rsec{Atomic_Variables} describes atomic variables, a mechanism
for supporting atomic operations.
\item \rsec{Cobegin} describes the cobegin statement, a structured way to
introduce concurrency into a program.
\item \rsec{Coforall} describes the coforall loop, another structured way to
introduce concurrency into a program.
\item \rsec{Task_Intents} specifies how variables from outer scopes
are handled within \chpl{begin}, \chpl{cobegin} and \chpl{coforall}
statements.
\item \rsec{Sync_Statement} describes the sync statement, a structured
way to control parallelism.
\item \rsec{Serial} describes the serial statement, a structured way to suppress
parallelism.
\item \rsec{Atomic_Statement} describes the atomic statement, a construct to
support atomic transactions.
\end{itemize}

\section{Tasks and Task Parallelism}
\label{Task_parallelism}
\index{task parallelism}
\index{parallelism!task}

A Chapel \emph{task} is a distinct context of execution that may be
running concurrently with other tasks.  Chapel provides a simple
construct, the \chpl{begin} statement, to create tasks, introducing
concurrency into a program in an unstructured way.  In addition,
Chapel introduces two type qualifiers, \chpl{sync} and \chpl{single},
for synchronization between tasks.

Chapel provides two constructs, the \chpl{cobegin} and \chpl{coforall} statements,
to introduce concurrency in a more structured way.  These constructs
create multiple tasks but do not continue until these tasks have
completed.  In addition, Chapel provides two constructs, the \chpl{sync} and
\chpl{serial} statements, to insert synchronization and suppress parallelism.
All four of these constructs can be implemented through judicious uses
of the unstructured task-parallel constructs described in the previous
paragraph.

\index{task parallelism!task creation}
\index{task creation}
Tasks are considered to be created when execution reaches the start
of a \chpl{begin}, \chpl{cobegin}, or \chpl{coforall} statement.
When the tasks are actually executed depends on the Chapel
implementation and run-time execution state.

\index{task function}
\index{task parallelism!task function}
A task is represented as a call to a \emph{task function}, whose body
contains the Chapel code for the task. Variables defined in outer
scopes are considered to be passed into a task function by default intent,
unless a different \emph{task intent} is specified explicitly
by a \sntx{task-intent-clause}.

% Should this be placed more prominently right before this section?
Accesses to the same variable from different tasks are subject
to the Memory Consistency Model (\rsec{Memory_Consistency_Model}).
Such accesses can result from aliasing due to \chpl{ref} argument intents
or task intents, among others.

\section{The Begin Statement}
\label{Begin}
\index{begin@\chpl{begin}}
\index{statements!begin@\chpl{begin}}

The begin statement creates a task to execute a statement.  The syntax
for the begin statement is given by
\begin{syntax}
begin-statement:
  `begin' task-intent-clause[OPT] statement
\end{syntax}
Control continues concurrently with the statement following the
begin statement.

\begin{chapelexample}{beginUnordered.chpl}
The code
\begin{chapel}
begin writeln("output from spawned task");
writeln("output from main task");
\end{chapel}
\begin{chapelprediff}
\#!/usr/bin/env sh
testname=$1
outfile=$2
sort $outfile > $outfile.2
mv $outfile.2 $outfile
\end{chapelprediff}
\begin{chapeloutput}
output from main task
output from spawned task
\end{chapeloutput}
executes two \chpl{writeln} statements that output the strings to the
terminal, but the ordering is purposely unspecified.  There is no
guarantee as to which statement will execute first.  When the
begin statement is executed, a new task is created that will execute
the \chpl{writeln} statement within it.  However, execution will
continue immediately after task creation with the next statement.
\end{chapelexample}

A begin statement creates a single task function,
whose body is the body of the begin statement.
The handling of the outer variables within the task function and
the role of \sntx{task-intent-clause} are defined in \rsec{Task_Intents}.

Yield and return statements are not allowed in begin blocks.  Break
and continue statements may not be used to exit a begin block.

%
% TODO: Future environment about task teams.
%

\section{Synchronization Variables}
\label{Synchronization_Variables}
\index{synchronization variables!sync@\chpl{sync}}
\index{synchronization variables!sync@\chpl{single}}
\index{sync@\chpl{sync}}
\index{single@\chpl{single}}

Synchronization variables have a logical state associated with the
value.  The state of the variable is either {\em full} or {\em empty}.
Normal reads of a synchronization variable cannot proceed until the
variable's state is full.  Normal writes of a synchronization variable
cannot proceed until the variable's state is empty.

Chapel supports two types of synchronization variables: sync and
single.  Both types behave similarly, except that a single variable
may only be written once.  Consequently, when a sync variable is read,
its state transitions to empty, whereas when a single variable is
read, its state does not change.  When either type of synchronization
variable is written, its state transitions to full.

\chpl{sync} and \chpl{single} are type qualifiers and precede
the type of the variable's value in the declaration.  Sync and single
are supported for all Chapel primitive types (~\rsec{Primitive_Types})
except complex.  They are also supported for enumerated types
(~\rsec{Enumerated_Types}) and variables of class type
(~\rsec{Class_Types}).  For sync variables of class type, the
full/empty state applies to the reference to the class object, not to
its member fields.

\begin{rationale}
It is only well-formed to apply full-empty semantics to types that
have no more than a single logical value.  Booleans, integers, real
and imaginary numbers, enums, and class references all meet this
criteria.  Since it is possible to read/write the individual elements
of a complex value, it's not obvious how the full-empty semantics
would interact with such operations.  While one could argue that
record types with a single field could also be included, the user can
more directly express such cases by declaring the field itself to be
of sync type.
\end{rationale}

If a task attempts to read or write a synchronization variable that is
not in the correct state, the task is suspended.  When the variable
transitions to the correct state, the task is resumed.  If there are
multiple tasks blocked waiting for the state transition, one is
non-deterministically selected to proceed and the others continue to
wait if it is a sync variable; all tasks are selected to proceed
if it is a single variable.

%
% TODO: The following should really be a 'type-expression' right?
% i.e., if there was a user type alias 'eltType', one should be
% able to do 'sync eltType'...
%

A synchronization variable is specified with a sync or single type
given by the following syntax:
\begin{syntax}
sync-type:
  `sync' type-expression

single-type:
  `single' type-expression
\end{syntax}

If a synchronization variable declaration has an initialization
expression, then the variable is initially full, otherwise it is
initially empty.

\begin{chapelexample}{beginWithSyncVar.chpl}
The code
\begin{chapel}
class Tree {
  var isLeaf: bool;
  var left, right: unmanaged Tree;
  var value: int;

  proc sum():int {
    if (isLeaf) then
       return value;

    var x(*\texttt{\$}*): sync int;
    begin x(*\texttt{\$}*) = left.sum();
    var y = right.sum();
    return x(*\texttt{\$}*) + y;
  }
}
\end{chapel}
\begin{chapelpost}
var tree: unmanaged Tree = new unmanaged Tree(false, new unmanaged Tree(false, new unmanaged Tree(true, nil, nil, 1),
                                                 new unmanaged Tree(true, nil, nil, 1), 1),
                                 new unmanaged Tree(false, new unmanaged Tree(true, nil, nil, 1),
                                                 new unmanaged Tree(true, nil, nil, 1), 1), 1);
writeln(tree.sum());
proc Tree.deinit() {
  if isLeaf then return;
  delete left;
  delete right;
}
delete tree;
\end{chapelpost}
\begin{chapeloutput}
4
\end{chapeloutput}
the sync variable \chpl{x$\mbox{\texttt{\$}}$} is assigned by an
asynchronous task created with the begin statement.  The task
returning the sum waits on the reading of \chpl{x$\mbox{\texttt{\$}}$}
until it has been assigned.  By convention, synchronization variables
end in \texttt{\$} to provide a visual cue to the programmer
indicating that the task may block.
\end{chapelexample}

\begin{chapelexample}{syncCounter.chpl}
Sync variables are useful for tallying data from multiple tasks as
well.  If all updates to an initialized sync variable are via compound
assignment operators (or equivalently, traditional assignments that
read and write the variable once), the full/empty state of the sync
variable guarantees that the reads and writes will be interleaved
in a manner that makes the updates atomic.  For example, the code:
\begin{chapel}
var count(*\texttt{\$}*): sync int = 0;
cobegin {
  count(*\texttt{\$}*) += 1;
  count(*\texttt{\$}*) += 1;
  count(*\texttt{\$}*) += 1;
}
\end{chapel}
\begin{chapelpost}
writeln("count is: ", count(*\texttt{\$}*).readFF());
\end{chapelpost}
\begin{chapeloutput}
count is: 3
\end{chapeloutput}
creates three tasks that increment \chpl{count$\mbox{\texttt{\$}}$}.
If \chpl{count$\mbox{\texttt{\$}}$} were not a sync variable, this code
would be unsafe because two tasks could then read the same value
before either had written its updated value, causing one of the
increments to be lost.
\end{chapelexample}

\pagebreak
\begin{chapelexample}{singleVar.chpl}
The following code implements a simple split-phase barrier using a
single variable.
\begin{chapelpre}
config const n = 44;
proc work(i) {
  // do nothing
}
\end{chapelpre}
\begin{chapel}
var count(*\texttt{\$}*): sync int = n;  // counter which also serves as a lock
var release(*\texttt{\$}*): single bool; // barrier release

forall t in 1..n do begin {
  work(t);
  var myc = count(*\texttt{\$}*);  // read the count, set state to empty
  if myc!=1 {
    write(".");
    count(*\texttt{\$}*) = myc-1;  // update the count, set state to full
    // we could also do some work here before blocking
    release(*\texttt{\$}*);
  } else {
    release(*\texttt{\$}*) = true;  // last one here, release everyone
    writeln("done");
  }
}
\end{chapel}
\begin{chapeloutput}
...........................................done
\end{chapeloutput}
In each iteration of the forall loop after the work is completed, the
task reads the \chpl{count$\mbox{\texttt{\$}}$} variable, which is
used to tally the number of tasks that have arrived.  All tasks except
the last task to arrive will block while trying to read the
variable \chpl{release$\mbox{\texttt{\$}}$}.  The last task to arrive
will write to \chpl{release$\mbox{\texttt{\$}}$}, setting its state to
full at which time all the other tasks can be unblocked and run.
\end{chapelexample}

\index{synchronization types!formal arguments}
If a formal argument with a default intent either has a synchronization
type or the formal is generic (\rsec{Formal_Arguments_of_Generic_Type})
and the actual has a synchronization type, the actual must be an
lvalue and is passed by reference. In these cases the formal itself
is an lvalue, too. The actual argument is not read or written during
argument passing; its state is not changed or waited on. The qualifier
\chpl{sync} or \chpl{single} without the value type can be used to
specify a generic formal argument that requires a \chpl{sync}
or \chpl{single} actual.

\index{synchronization types!actual arguments}
When the actual argument is a \chpl{sync} or \chpl{single} and the
corresponding formal has the actual's base type or is implicitly
converted from that type, a normal read of the actual is performed
when the call is made, and the read value is passed to the formal.


\subsection{Predefined Single and Sync Methods}
\label{Functions_on_Synchronization_Variables}
\index{synchronization variables!predefined methods on}

The following methods are defined for variables of sync and single
type.

\index{readFE (sync var)@\chpl{readFE} (sync var)}
\index{predefined functions!readFE (sync var)@\chpl{readFE} (sync var)}
\begin{protohead}
proc (sync t).readFE(): t
\end{protohead}
\begin{protobody}
Returns the value of the sync variable.  This method blocks until the
sync variable is full.  The state of the sync variable is set to empty
when this method completes.
This method implements the normal read of a \chpl{sync} variable.
\end{protobody}

\index{readFF (sync var)@\chpl{readFF} (sync var)}
\index{predefined functions!readFF (sync var)@\chpl{readFF} (sync var)}
\begin{protohead}
proc (sync t).readFF(): t
proc (single t).readFF(): t
\end{protohead}
\begin{protobody}
Returns the value of the sync or single variable.  This method blocks
until the sync or single variable is full.  The state of the sync or
single variable remains full when this method completes.
This method implements the normal read of a \chpl{single} variable.
\end{protobody}

\pagebreak
\index{readXX (sync var)@\chpl{readXX} (sync var)}
\index{predefined functions!readXX (sync var)@\chpl{readXX} (sync var)}
\begin{protohead}
proc (sync t).readXX(): t
proc (single t).readXX(): t
\end{protohead}
\begin{protobody}
Returns the value of the sync or single variable.  This method is non-blocking
and the state of the sync or single variable is unchanged when this method
completes.
\end{protobody}

\index{writeEF (sync var)@\chpl{writeEF} (sync var)}
\index{predefined functions!writeEF (sync var)@\chpl{writeEF} (sync var)}
\begin{protohead}
proc (sync t).writeEF(v: t)
proc (single t).writeEF(v: t)
\end{protohead}
\begin{protobody}
Assigns \chpl{v} to the value of the sync or single variable.  This
method blocks until the sync or single variable is empty.  The state
of the sync or single variable is set to full when this method
completes.
This method implements the normal write of a \chpl{sync} or \chpl{single}
variable.
\end{protobody}

\index{writeFF (sync var)@\chpl{writeFF} (sync var)}
\index{predefined functions!writeFF (sync var)@\chpl{writeFF} (sync var)}
\begin{protohead}
proc (sync t).writeFF(v: t)
\end{protohead}
\begin{protobody}
Assigns \chpl{v} to the value of the sync variable.  This method
blocks until the sync variable is full.  The state of the sync
variable remains full when this method completes.
\end{protobody}

\index{writeXF (sync var)@\chpl{writeXF} (sync var)}
\index{predefined functions!writeXF (sync var)@\chpl{writeXF} (sync var)}
\begin{protohead}
proc (sync t).writeXF(v: t)
\end{protohead}
\begin{protobody}
Assigns \chpl{v} to the value of the sync variable.  This method is
non-blocking and the state of the sync variable is set to full when
this method completes.
\end{protobody}

\index{reset (sync var)@\chpl{reset} (sync var)}
\index{predefined functions!reset (sync var)@\chpl{reset} (sync var)}
\begin{protohead}
proc (sync t).reset()
\end{protohead}
\begin{protobody}
Assigns the default value of type \chpl{t} to the value of the sync
variable.  This method is non-blocking and the state of the sync
variable is set to empty when this method completes.
\end{protobody}

\index{isFull (sync var)@\chpl{isFull} (sync var)}
\index{predefined functions!isFull (sync var)@\chpl{isFull} (sync var)}
\begin{protohead}
proc (sync t).isFull: bool
proc (single t).isFull: bool
\end{protohead}
\begin{protobody}
Returns \chpl{true} if the sync or single variable is full and \chpl{false}
otherwise.  This method is non-blocking and the state of the sync or single
variable is unchanged when this method completes.
\end{protobody}

Note that \chpl{writeEF} and \chpl{readFE}/\chpl{readFF} methods
(for \chpl{sync} and \chpl{single} variables, respectively) are
implicitly invoked for normal writes and reads of synchronization variables.


\begin{chapelexample}{syncMethods.chpl}
Given the following declarations
\begin{chapelpre}
{ // }
\end{chapelpre}
\begin{chapel}
var x(*\texttt{\$}*): sync int;
var y(*\texttt{\$}*): single int;
var z: int;
\end{chapel}
the code
\begin{chapel}
x(*\texttt{\$}*) = 5;
y(*\texttt{\$}*) = 6;
z = x(*\texttt{\$}*) + y(*\texttt{\$}*);
\end{chapel}
\begin{chapelnoprint}
writeln((x(*\texttt{\$}*).readXX(), y(*\texttt{\$}*).readFF(), z));
// {
}
{ // }
var x(*\texttt{\$}*): sync int;
var y(*\texttt{\$}*): single int;
var z: int;
\end{chapelnoprint}
is equivalent to
\begin{chapel}
x(*\texttt{\$}*).writeEF(5);
y(*\texttt{\$}*).writeEF(6);
z = x(*\texttt{\$}*).readFE() + y(*\texttt{\$}*).readFF();
\end{chapel}
\begin{chapelpost}
writeln((x(*\texttt{\$}*).readXX(), y(*\texttt{\$}*).readFF(), z));
// {
}
\end{chapelpost}
\begin{chapeloutput}
(5, 6, 11)
(5, 6, 11)
\end{chapeloutput}
\end{chapelexample}



\section{Atomic Variables}
\label{Atomic_Variables}
\index{atomic variables!atomic@\chpl{atomic}}
\index{atomic@\chpl{atomic}}

Atomic variables are variables that support atomic operations. Chapel
currently supports atomic operations for bools, all supported sizes of
signed and unsigned integers, as well as all supported sizes of reals.

\begin{rationale}
The choice of supported atomic variable types as well as the atomic
operations was strongly influenced by the C11 standard.
\end{rationale}

Atomic is a type qualifier that precedes the variable's type in
the declaration. Atomic operations are supported for bools, and all
sizes of ints, uints, and reals.

An atomic variable is specified with an atomic type given by the
following syntax:

\begin{syntax}
atomic-type:
  `atomic' type-expression
\end{syntax}

\subsection{Predefined Atomic Methods}
\label{Functions_on_Atomic_Variables}
\index{atomic variables!predefined methods on}

The following methods are defined for variables of atomic type. Note
that not all operations are supported for all atomic types. The
supported types are listed for each method.

\index{atomic types!memory order}
Most of the predefined atomic methods accept an optional argument
named \chpl{order} of type memory\_order. The \chpl{order} argument is
used to specify the ordering constraints of atomic operations. The
supported memory\_order values are:
\begin{itemize}
\item{memory\_order\_relaxed}
\item{memory\_order\_acquire}
\item{memory\_order\_release}
\item{memory\_order\_acq\_rel}
\item{memory\_order\_seq\_cst}
\end{itemize}

\begin{openissue}
  The memory\_order values were taken directly from the C11
  specification.  We expect to review and better define the supported
  values with work on Chapel's memory consistency model
  (see~\ref{Memory_Consistency_Model}).
\end{openissue}

Unless specified, the default for the memory\_order parameter is
memory\_order\_seq\_cst.

\begin{note}
Not all architectures or implementations may support all memory\_order
values.  In these cases, the implementation should default to a more
conservative ordering than specified.
\end{note}

\index{read (atomic var)@\chpl{read} (atomic var)}
\index{predefined functions!read (atomic var)@\chpl{read} (atomic var)}
\begin{protohead}
proc (atomic t).read(memory_order order): t
\end{protohead}
\begin{protobody}
Reads and returns the stored value. Defined for all atomic types.
\end{protobody}

\pagebreak
\index{peek (atomic var)@\chpl{peek} (atomic var)}
\index{predefined functions!peek (atomic var)@\chpl{peek} (atomic var)}
\begin{protohead}
proc (atomic t).peek(): t
\end{protohead}
\begin{protobody}
Reads and returns the stored value using memory\_order\_relaxed.
Defined for all atomic types.
\end{protobody}

\index{write (atomic var)@\chpl{write} (atomic var)}
\index{predefined functions!write (atomic var)@\chpl{write} (atomic var)}
\begin{protohead}
proc (atomic t).write(v: t, memory_order order)
\end{protohead}
\begin{protobody}
Stores \chpl{v} as the new value. Defined for all atomic types.
\end{protobody}

\index{poke (atomic var)@\chpl{poke} (atomic var)}
\index{predefined functions!poke (atomic var)@\chpl{poke} (atomic var)}
\begin{protohead}
proc (atomic t).poke(v: t)
\end{protohead}
\begin{protobody}
Stores \chpl{v} as the new value using memory\_order\_relaxed.
Defined for all atomic types.
\end{protobody}

\index{exchange (atomic var)@\chpl{exchange} (atomic var)}
\index{predefined functions!exchange (atomic var)@\chpl{exchange} (atomic var)}
\begin{protohead}
proc (atomic t).exchange(v: t, memory_order order): t
\end{protohead}
\begin{protobody}
Stores \chpl{v} as the new value and returns the original
value. Defined for all atomic types.
\end{protobody}

\index{compareExchangWeak (atomic var)@\chpl{compareExchangeWeak} (atomic var)}
\index{predefined functions!compareExchangeWeak (atomic var)@\chpl{compareExchangeWeak} (atomic var)}
\index{compareExchangStrong (atomic var)@\chpl{compareExchangeStrong} (atomic var)}
\index{predefined functions!compareExchangeStrong (atomic var)@\chpl{compareExchangeStrong} (atomic var)}
\index{compareExchange (atomic var)@\chpl{compareExchange} (atomic var)}
\index{predefined functions!compareExchange (atomic var)@\chpl{compareExchange} (atomic var)}
\begin{protohead}
proc (atomic t).compareExchangeWeak(e: t, v: t, memory_order order): bool
proc (atomic t).compareExchangeStrong(e: t, v: t, memory_order order): bool
proc (atomic t).compareExchange(e: t, v: t, memory_order order): bool
\end{protohead}
\begin{protobody}
Stores \chpl{v} as the new value, if and only if the original value is
equal to \chpl{e}. Returns \chpl{true} if \chpl{v} was
stored, \chpl{false} otherwise. The 'weak' variation may
return \chpl{false} even if the original value was equal to \chpl{e},
if, for example, the value could not be updated
atomically. \chpl{compareExchange} is equivalent to
\chpl{compareExchangeStrong}.  Defined for all atomic types.
\end{protobody}

\index{add (atomic var)@\chpl{add} (atomic var)}
\index{predefined functions!add (atomic var)@\chpl{add} (atomic var)}
\index{sub (atomic var)@\chpl{sub} (atomic var)}
\index{predefined functions!sub (atomic var)@\chpl{sub} (atomic var)}
\index{or (atomic var)@\chpl{or} (atomic var)}
\index{predefined functions!or (atomic var)@\chpl{or} (atomic var)}
\index{and (atomic var)@\chpl{and} (atomic var)}
\index{predefined functions!and (atomic var)@\chpl{and} (atomic var)}
\index{xor (atomic var)@\chpl{xor} (atomic var)}
\index{predefined functions!xor (atomic var)@\chpl{xor} (atomic var)}
\begin{protohead}
proc (atomic t).add(v: t, memory_order order)
proc (atomic t).sub(v: t, memory_order order)
proc (atomic t).or(v: t, memory_order order)
proc (atomic t).and(v: t, memory_order order)
proc (atomic t).xor(v: t, memory_order order)
\end{protohead}
\begin{protobody}
Applies the appropriate operator (\verb@+@, \verb@-@, \verb@|@,
\verb@&@, \verb@^@) to the original value and \chpl{v} and stores the result.
All of the methods are defined for integral atomic types. Only add and
sub (\verb@+@, \verb@-@) are defined for \chpl{real} atomic types.
None of the methods are defined for the \chpl{bool} atomic type.
\end{protobody}

\begin{future}
In the future we may overload certain operations such as \verb@+=@ to
call the above methods automatically for atomic variables.
\end{future}

\index{fetchAdd (atomic var)@\chpl{fetchAdd} (atomic var)}
\index{predefined functions!fetchAdd (atomic var)@\chpl{fetchAdd} (atomic var)}
\index{fetchSub (atomic var)@\chpl{fetchSub} (atomic var)}
\index{predefined functions!fetchSub (atomic var)@\chpl{fetchSub} (atomic var)}
\index{fetchOr (atomic var)@\chpl{fetchOr} (atomic var)}
\index{predefined functions!fetchOr (atomic var)@\chpl{fetchOr} (atomic var)}
\index{fetchAnd (atomic var)@\chpl{fetchAnd} (atomic var)}
\index{predefined functions!fetchAnd (atomic var)@\chpl{fetchAnd} (atomic var)}
\index{fetchXor (atomic var)@\chpl{fetchXor} (atomic var)}
\index{predefined functions!fetchXor (atomic var)@\chpl{fetchXor} (atomic var)}
\begin{protohead}
proc (atomic t).fetchAdd(v: t, memory_order order): t
proc (atomic t).fetchSub(v: t, memory_order order): t
proc (atomic t).fetchOr(v: t, memory_order order): t
proc (atomic t).fetchAnd(v: t, memory_order order): t
proc (atomic t).fetchXor(v: t, memory_order order): t
\end{protohead}
\begin{protobody}
Applies the appropriate operator (\verb@+@, \verb@-@, \verb@|@,
\verb@&@, \verb@^@) to the original value and \chpl{v}, stores the result, and
returns the original value. All of the methods are defined for
integral atomic types. Only add and sub (\verb@+@, \verb@-@) are
defined for \chpl{real} atomic types.  None of the methods are defined
for the \chpl{bool} atomic type.
\end{protobody}


\index{testAndSet (atomic bool)@\chpl{testAndSet} (atomic bool)}
\index{predefined functions!testAndSet (atomic bool)@\chpl{testAndSet}
(atomic bool)}
\begin{protohead}
proc (atomic bool).testAndSet(memory_order order): bool
\end{protohead}
\begin{protobody}
Stores \chpl{true} as the new value and returns the old
value. Equivalent to \chpl{exchange(true)}. Only defined for
the \chpl{bool} atomic type.
\end{protobody}

\index{clear (atomic bool)@\chpl{clear} (atomic bool)}
\index{predefined functions!clear (atomic bool)@\chpl{clear} (atomic bool)}
\begin{protohead}
proc (atomic bool).clear(memory_order order)
\end{protohead}
\begin{protobody}
Stores \chpl{false} as the new value. Equivalent
to \chpl{write(false)}. Only defined for the \chpl{bool} atomic type.
\end{protobody}

\index{waitFor (atomic var)@\chpl{waitFor} (atomic var)}
\index{predefined functions!waitFor (atomic var)@\chpl{waitFor} (atomic var)}
\begin{protohead}
proc (atomic t).waitFor(v: t)
\end{protohead}
\begin{protobody}
Waits until the stored value is equal to \chpl{v}. The implementation
may yield the running task while waiting.  Defined for all atomic types.
\end{protobody}



\section{The Cobegin Statement}
\label{Cobegin}
\index{cobegin@\chpl{cobegin}}
\index{statements!cobegin@\chpl{cobegin}}

The cobegin statement is used to introduce concurrency within a
block.  The \chpl{cobegin} statement syntax is
\begin{syntax}
cobegin-statement:
  `cobegin' task-intent-clause[OPT] block-statement
\end{syntax}

A new task and a corresponding task function are created for each statement
in the \sntx{block-statement}.  Control
continues when all of the tasks have finished.
The handling of the outer variables within each task function and
the role of \sntx{task-intent-clause} are defined in \rsec{Task_Intents}.

Return statements are not allowed in cobegin blocks.  Yield statement
may only be lexically enclosed in cobegin blocks in parallel
iterators~(\rsec{Parallel_Iterators}).  Break and continue statements
may not be used to exit a cobegin block.


\begin{chapelexample}{cobeginAndEquivalent.chpl}
The cobegin statement
\begin{chapelpre}
var s1, s2: sync int;
proc stmt1() { s1; }
proc stmt2() { s2; s1 = 1; }
proc stmt3() { s2 = 1; }
\end{chapelpre}
\begin{chapel}
cobegin {
  stmt1();
  stmt2();
  stmt3();
}
\end{chapel}
is equivalent to the following code that uses only begin statements
and single variables to introduce concurrency and synchronize:
\begin{chapel}
var s1(*\texttt{\$}*), s2(*\texttt{\$}*), s3(*\texttt{\$}*): single bool;
begin { stmt1(); s1(*\texttt{\$}*) = true; }
begin { stmt2(); s2(*\texttt{\$}*) = true; }
begin { stmt3(); s3(*\texttt{\$}*) = true; }
s1(*\texttt{\$}*); s2(*\texttt{\$}*); s3(*\texttt{\$}*);
\end{chapel}
\begin{chapeloutput}
\end{chapeloutput}
Each begin statement is executed concurrently but control does not
continue past the final line above until each of the single variables
is written, thereby ensuring that each of the functions has finished.
\end{chapelexample}

\section{The Coforall Loop}
\label{Coforall}
\index{coforall@\chpl{coforall}}
\index{statements!coforall@\chpl{coforall}}

The coforall loop is a variant of the cobegin statement in loop form.
The syntax for the coforall loop is given by
\begin{syntax}
coforall-statement:
  `coforall' index-var-declaration `in' iteratable-expression task-intent-clause[OPT] `do' statement
  `coforall' index-var-declaration `in' iteratable-expression task-intent-clause[OPT] block-statement
  `coforall' iteratable-expression task-intent-clause[OPT] `do' statement
  `coforall' iteratable-expression task-intent-clause[OPT] block-statement
\end{syntax}

The \chpl{coforall} loop creates a separate task for each iteration of
the loop.  Control continues with the statement following
the \chpl{coforall} loop after all tasks corresponding to the
iterations of the loop have completed.

The single task function created for a \chpl{coforall} and invoked by
each task contains the loop body.
The handling of the outer variables within the task function and
the role of \sntx{task-intent-clause} are defined in \rsec{Task_Intents}.

Return statements are not allowed in coforall blocks.  Yield statement
may only be lexically enclosed in coforall blocks in parallel
iterators~(\rsec{Parallel_Iterators}).  Break and continue statements
may not be used to exit a coforall block.

\begin{chapelexample}{coforallAndEquivalent.chpl}
The coforall statement
\begin{chapelpre}
iter iterator() { for i in 1..3 do yield i; }
proc body() { }
\end{chapelpre}
\begin{chapel}
coforall i in iterator() {
  body();
}
\end{chapel}
is equivalent to the following code that uses only begin statements
and sync and single variables to introduce concurrency and
synchronize:
\begin{chapel}
var runningCount(*\texttt{\$}*): sync int = 1;
var finished(*\texttt{\$}*): single bool;
for i in iterator() {
  runningCount(*\texttt{\$}*) += 1;
  begin {
    body();
    var tmp = runningCount(*\texttt{\$}*);
    runningCount(*\texttt{\$}*) = tmp-1;
    if tmp == 1 then finished(*\texttt{\$}*) = true;
  }
}
var tmp = runningCount(*\texttt{\$}*);
runningCount(*\texttt{\$}*) = tmp-1;
if tmp == 1 then finished(*\texttt{\$}*) = true;
finished(*\texttt{\$}*);
\end{chapel}
\begin{chapeloutput}
\end{chapeloutput}
Each call to \chpl{body()} executes concurrently because it is in a
begin statement.  The sync
variable \chpl{runningCount$\mbox{\texttt{\$}}$} is used to keep track
of the number of executing tasks plus one for the main task.  When
this variable reaches zero, the single
variable \chpl{finished$\mbox{\texttt{\$}}$} is used to signal that
all of the tasks have completed.  Thus control does not continue past
the last line until all of the tasks have completed.
\end{chapelexample}


\section{Task Intents}
\label{Task_Intents}
\index{task intents}
\index{task parallelism!task functions}
\index{task parallelism!task intents}

% Would be nice to give this arrangement a name. Could say
% "the task intent rule", although that sounds a bit like
% a colloquialism.
If a variable is referenced within the lexical scope of a
\chpl{begin}, \chpl{cobegin}, or \chpl{coforall} statement
and is declared outside that statement, it is considered
to be passed as an actual argument to the corresponding task function
at task creation time. All references to the variable
within the task function implicitly refer to the task function's
corresponding formal argument.

Each formal argument of a task function has the default intent by default.
For variables of primitive and class types, this has the effect
of capturing the value of the variable at task creation time
and referencing that value instead of the original variable
within the lexical scope of the task construct.

A formal can be given another intent explicitly by listing it
with that intent in the optional \sntx{task-intent-clause}.
For example, for variables of most types, the \chpl{ref} intent allows
the task construct to modify the corresponding original variable
or to read its updated value after concurrent modifications.

The syntax of the task intent clause is:

\begin{syntax}
task-intent-clause:
  `with' ( task-intent-list )

task-intent-list:
  task-intent-item
  task-intent-item, task-intent-list

task-intent-item:
  formal-intent identifier
  task-private-var-decl
\end{syntax}

where the following intents can be used as a \sntx{formal-intent}:
\chpl{ref}, \chpl{in}, \chpl{const}, \chpl{const in}, \chpl{const ref}.
\sntx{task-private-var-decl} is defined in \rsec{Task_Private_Variables}.
In addition, \sntx{task-intent-item} may define a \chpl{reduce} intent.
Reduce intents are described in the \emph{Reduce Intents} technical note
in the online documentation:
\\ %formatting
\mbox{$$ $$ $$} %indent
\url{https://chapel-lang.org/docs/technotes/reduceIntents.html}

% TODO for task intents:
% * Introduce a 'task-formal-intent' syntactic rule
%   that expands to the legal intents - far preferable than
%   qualifying which intents are legal in the text of this paragraph.
% * That's assuming we do not support 'out' and 'inout',
%   which is still up for debate; otherwise leave as-is.
% * Do we want to allow default intents? Current implementation allows them.

The implicit treatment of outer scope variables as the task function's
formal arguments applies to both module level and local variables.
It applies to variable references within the lexical scope
of a task construct, but does not extend to its dynamic scope, i.e.,
to the functions called from the task(s) but declared outside of
the lexical scope.
The loop index variables of a \chpl{coforall} statement are not
subject to such treatment within that statement; however, they are
subject to such treatment within nested task constructs, if any.


\begin{rationale}
The primary motivation for task intents is to avoid some races on
scalar/record variables, which are possible when one task modifies a
variable and another task reads it. Without task intents,
for example, it would be easy to introduce and overlook a bug
illustrated by this simplified example:

  \begin{chapel}
  {
    var i = 0;
    while i < 10 {
      begin {
        f(i);
      }
      i += 1;
    }
  }
  \end{chapel}

If all the tasks created by the \chpl{begin} statement start executing
only after the \chpl{while} loop completes, and \chpl{i} within the
\chpl{begin} is treated as a reference to the original \chpl{i},
there will be ten tasks executing \chpl{f(10)}. However, the user most
likely intended to generate ten tasks executing
\chpl{f(0)}, \chpl{f(1)}, ..., \chpl{f(9)}.
Task intents ensure that, regardless of the timing of task execution.

Another motivation for task intents is that referring to a captured
copy in a task is often more efficient than referring to the original
variable. That's because the copy is a local constant, e.g. it could
be placed in a register when it fits.  Without task intents,
references to the original variable would need to be implemented using
a pointer dereference. This is less efficient and can hinder optimizations
in the surrounding code, for example loop-invariant code motion.

Furthermore, in the above example the scope where \chpl{i} is declared
may exit before all the ten tasks complete.  Without task intents,
the user would have to protect \chpl{i} to make sure its lexical scope
doesn't exit before the tasks referencing it complete.

We decided to treat \chpl{cobegin} and \chpl{coforall} statements the
same way as \chpl{begin}. This is for consistency and to make the
race-avoidance benefit available to more code.

We decided to apply task intents to module level variables, in addition
to local variables. Again, this is for consistency. One could view module
level variables differently than local variables (e.g. a module level
variable is ``always available''), but we favored consistency over such
an approach.

We decided not to apply task intents to ``closure'' variables, i.e.,
the variables in the dynamic scope of a task construct. This is to
keep this feature manageable, so that all variables subject to task
intents can be obtained by examining just the lexical scope of the
task construct. In general, the set of closure variables can be hard
to determine, unwieldy to implement and reason about, it is unclear
what to do with extern functions, etc.

We do not provide \chpl{inout} or \chpl{out} as task intents because they
will necessarily create a data race in a \chpl{cobegin} or \chpl{coforall}.
% that does not necessarily apply to a 'begin'
\chpl{type} and \chpl{param} intents are not available either
as they do not seem useful as task intents.
\end{rationale}

\begin{future}
For a given intent, we would also like to provide a blanket clause,
which would apply the intent to all variables.
An example of syntax for a blanket \chpl{ref} intent would be \chpl{ref *}.
\end{future}


\section{The Sync Statement}
\label{Sync_Statement}
\index{sync@\chpl{sync}}
\index{statements!sync@\chpl{sync}}

The sync statement acts as a join of all dynamically encountered
begins from within a statement.  The syntax for the sync statement is
given by
\begin{syntax}
sync-statement:
  `sync' statement
  `sync' block-statement
\end{syntax}

Return statements are not allowed in sync statement blocks.  Yield
statement may only be lexically enclosed in sync statement blocks in
parallel iterators~(\rsec{Parallel_Iterators}).  Break and continue
statements may not be used to exit a sync statement block.

\begin{chapelexample}{syncStmt1.chpl}
The sync statement can be used to wait for many dynamically created
tasks.
\begin{chapelpre}
config const n = 9;
proc work() {
  write(".");
}
\end{chapelpre}
\begin{chapel}
sync for i in 1..n do begin work();
\end{chapel}
\begin{chapelpost}
writeln("done");
\end{chapelpost}
\begin{chapeloutput}
.........done
\end{chapeloutput}
The for loop is within a sync statement and thus the tasks created
in each iteration of the loop must complete before the continuing past
the sync statement.
\end{chapelexample}

\begin{chapelexample}{syncStmt2.chpl}
The sync statement
\begin{chapelpre}
proc stmt1() { }
proc stmt2() { }
\end{chapelpre}
\begin{chapel}
sync {
  begin stmt1();
  begin stmt2();
}
\end{chapel}
is similar to the following cobegin statement
\begin{chapel}
cobegin {
  stmt1();
  stmt2();
}
\end{chapel}
\begin{chapeloutput}
\end{chapeloutput}
except that if begin statements are dynamically encountered
when \chpl{stmt1()} or \chpl{stmt2()} are executed, then the former
code will wait for these begin statements to complete whereas the
latter code will not.
\end{chapelexample}

\section{The Serial Statement}
\label{Serial}
\index{serial@\chpl{serial}}
\index{statements!serial@\chpl{serial}}

The \chpl{serial} statement can be used to dynamically disable
parallelism.  The syntax is:
\begin{syntax}
serial-statement:
  `serial' expression[OPT] `do' statement
  `serial' expression[OPT] block-statement
\end{syntax}
where the optional \sntx{expression} evaluates to a boolean value.  If
the expression is omitted, it is as though 'true' were specified.
Whatever the expression's value, the statement following it is
evaluated. If the expression is true, any dynamically encountered code
that would normally create new tasks within the statement is instead
executed by the original task without creating any new ones.  In
effect, execution is serialized.  If the expression is false, code
within the statement will generates task according to normal Chapel
rules.

\begin{chapelexample}{serialStmt1.chpl}
In the code
\begin{chapelpre}
config const lo = 9;
config const hi = 23;
proc work(i) {
  if \_\_primitive("task\_get\_serial") then
    writeln("serial ", i);
}
\end{chapelpre}
\begin{chapel}
proc f(i) {
  serial i<13 {
    cobegin {
      work(i);
      work(i);
    }
  }
}

for i in lo..hi {
  f(i);
}
\end{chapel}
\begin{chapelpost}
\end{chapelpost}
\begin{chapeloutput}
serial 9
serial 9
serial 10
serial 10
serial 11
serial 11
serial 12
serial 12
\end{chapeloutput}
the serial statement in procedure f() inhibits concurrent execution of
work() if the variable i is less than 13.
\end{chapelexample}

\begin{chapelexample}{serialStmt2.chpl}
The code
\begin{chapelpre}
proc stmt1() { write(1); }
proc stmt2() { write(2); }
proc stmt3() { write(3); }
proc stmt4() { write(4); }
var n = 3;
\end{chapelpre}
\begin{chapel}
serial {
  begin stmt1();
  cobegin {
    stmt2();
    stmt3();
  }
  coforall i in 1..n do stmt4();
}
\end{chapel}
is equivalent to
\begin{chapel}
stmt1();
{
  stmt2();
  stmt3();
}
for i in 1..n do stmt4();
\end{chapel}
\begin{chapelpost}
writeln();
\end{chapelpost}
\begin{chapeloutput}
123444123444
\end{chapeloutput}
because the expression evaluated to determine whether to serialize
always evaluates to true.
\end{chapelexample}

\section{Atomic Statements}
\label{Atomic_Statement}
\index{atomic transactions}
\index{atomic statement}
\index{atomic@\chpl{atomic}}
\index{statements!atomic@\chpl{atomic}}

\begin{openissue}
  This section describes a feature that is a work-in-progress.  We seek feedback
  and collaboration in this area from the broader community.
\end{openissue}

The \emph{atomic statement} is used to specify that a statement should appear
to execute atomically from other tasks' point of view.
In particular, no task will see memory in a state that would reflect that
the atomic statement had begun executing but had not yet completed.

\begin{openissue}
  This definition of the atomic statement provides a notion of {\em
    strong atomicity} since the action will appear atomic to any task
  at any point in its execution.  For performance reasons, it could be
  more practical to support {\em weak atomicity} in which the
  statement's atomicity is only guaranteed with respect to other
  atomic statements.  We may also pursue using atomic type qualifiers
  as a means of marking data that should be accessed atomically inside
  or outside an atomic section.
\end{openissue}

The syntax for the atomic statement is given by:
\begin{syntax}
atomic-statement:
  `atomic' statement
\end{syntax}

%\begin{chapelexample}{atomicStmt}
\begin{example}
The following code illustrates the use of an atomic statement
to perform an insertion into a doubly-linked list:

\begin{chapelpre}
class Node {
  var data: int;
  var next: Node;
  var prev: Node;
}
var head = new Node(1);
head.insertAfter(new Node(4));
head.insertAfter(new Node(2));

var obj = new Node(3);
head.next.insertAfter(obj);
\end{chapelpre}
\begin{chapel}
proc Node.insertAfter(newNode: Node) {
  atomic {
    newNode.prev = this;
    newNode.next = this.next;
    if this.next then this.next.prev = newNode;
    this.next = newNode;
  }
}
\end{chapel}
\begin{chapelpost}
writeln(head.data, head.next.data, head.next.next.data, head.next.next.next.data);
proc Node.remove() {
  if this.prev then this.prev = this.next;
  if this.next then this.next = this.prev;
  return this;
}
while (head) {
  next = head.next;
  delete head;
  head = next;
}
\end{chapelpost}
\begin{chapeloutput}
atomic.chpl:13: warning: atomic keyword is ignored (not implemented)
1234
\end{chapeloutput}
The use of the atomic statement in this routine prevents other tasks
from viewing the list in a partially-updated state in which the
pointers might not be self-consistent.
\end{example}
%\end{chapelexample}

\cleardoublepage
\sekshun{Data Parallelism}
\label{Data_Parallelism}
\index{data parallelism}
\index{parallelism!data}

Chapel provides two explicit data-parallel constructs (the
forall-statement and the forall-expression) and several idioms that
support data parallelism implicitly (whole-array assignment, function
and operator promotion, reductions, and scans).

This chapter details data parallelism as follows:
\begin{itemize}
\item \rsec{Forall} describes the forall statement.
\item \rsec{Forall_Expressions} describes forall expressions
\item \rsec{Forall_Intents} specifies how variables from outer scopes
are handled within forall statements and expressions.
\item \rsec{Promotion} describes promotion.
\item \rsec{Reductions_and_Scans} describes reductions and scans.
\item \rsec{data_parallel_knobs} describes the configuration constants for
controlling default data parallelism.
\end{itemize}       

\section{The Forall Statement}
\label{Forall}
\index{forall@\chpl{forall} (see also statements, forall)}
\index{loops!forall (see also statements, forall)}
\index{data parallelism!forall}
\index{statements!forall}

The forall statement is a concurrent variant of the for statement
described in~\rsec{The_For_Loop}.

\subsection{Syntax}
\label{forall_syntax}
\index{statements!forall!syntax}

The syntax of the forall statement is given by
\begin{syntax}
forall-statement:
  `forall' index-var-declaration `in' iteratable-expression task-intent-clause[OPT] `do' statement
  `forall' index-var-declaration `in' iteratable-expression task-intent-clause[OPT] block-statement
  `forall' iteratable-expression task-intent-clause[OPT] `do' statement
  `forall' iteratable-expression task-intent-clause[OPT] block-statement
  [ index-var-declaration `in' iteratable-expression task-intent-clause[OPT] ] statement
  [ iteratable-expression task-intent-clause[OPT] ] statement
\end{syntax}
As with the for statement, the indices may be omitted if they are
unnecessary and the \chpl{do} keyword may be omitted before a block
statement.  The square bracketed form is a syntactic convenience.

The handling of the outer variables within the forall statement and
the role of \sntx{task-intent-clause} are defined in \rsec{Forall_Intents}.

\subsection{Execution and Serializability}
\label{forall_semantics}
\index{statements!forall!semantics}

The forall statement evaluates the loop body once for each element
yielded by the \sntx{iteratable-expression}.  Each instance of the
forall loop's body may be executed concurrently with the others, but
this is not guaranteed.  In particular, the loop must be serializable.
Details regarding concurrency and iterator implementation are
described in~\ref{Parallel_Iterators}.

This differs from the semantics of the \chpl{coforall} loop, discussed
in~\rsec{Coforall}, where each iteration is guaranteed to run using a
distinct task.  The \chpl{coforall} loop thus has potentially higher
overhead than a forall loop with the same number of iterations, but in
cases where concurrency is required for correctness, it is essential.

\index{leading the execution of a loop}
\index{data parallelism!leader iterator}
In practice, the number of tasks that will be used to evaluate
a \chpl{forall} loop is determined by the object or iterator that
is \emph{leading} the execution of the loop, as is the mapping of
iterations to tasks.

This concept will be formalized in future drafts of the Chapel
specification; for now, please refer
to \chpl{CHPL_HOME/examples/primers/leaderfollower.chpl} for a brief
introduction or to \emph{User-Defined Parallel Zippered Iterators in
Chapel}, published in the PGAS 2011 workshop.

Control continues with the statement following the forall loop only
after every iteration has been completely evaluated.  At this point,
all data accesses within the body of the forall loop will be
guaranteed to be completed.

The following statements may not be lexically enclosed in forall
statements: break statements, and return statements.  Yield statement
may only be lexically enclosed in forall statements in parallel
iterators~(\rsec{Parallel_Iterators}).

\begin{chapelexample}{forallStmt.chpl}
In the code
\begin{chapelpre}
config const N = 5;
var a: [1..N] int;
var b = [i in 1..N] i;
\end{chapelpre}
\begin{chapel}
forall i in 1..N do
  a(i) = b(i);
\end{chapel}
the user has stated that the element-wise assignments can execute
concurrently.  This loop may be executed serially with a single task,
or by using a distinct task for every iteration, or by using a number
of tasks where each task executes a number of iterations.  This loop
can also be written as
\begin{chapel}
[i in 1..N] a(i) = b(i);
\end{chapel}
\begin{chapelpost}
writeln(a);
\end{chapelpost}
\begin{chapeloutput}
1 2 3 4 5
\end{chapeloutput}
\end{chapelexample}

\subsection{Zipper Iteration}
\label{forall_zipper}
\index{statements!forall!zipper iteration}

Zipper iteration has the same semantics as described
in~\rsec{Zipper_Iteration} and~\rsec{Parallel_Iterators} for parallel
iteration.


\section{The Forall Expression}
\label{Forall_Expressions}
\index{data parallelism!forall expressions}
\index{forall expressions (see also expressions, forall)}
\index{expressions!forall}

The forall expression is a concurrent variant of the for expression
described in~\rsec{For_Expressions}.

\subsection{Syntax}
\label{forall_expr_syntax}
\index{expressions!forall!syntax}

The syntax of a forall expression is given by
\begin{syntax}
forall-expression:
  `forall' index-var-declaration `in' iteratable-expression task-intent-clause[OPT] `do' expression
  `forall' iteratable-expression task-intent-clause[OPT] `do' expression
  [ index-var-declaration `in' iteratable-expression task-intent-clause[OPT] ] expression
  [ iteratable-expression task-intent-clause[OPT] ] expression
\end{syntax}
As with the for expression, the indices may be omitted if they are
unnecessary.  The \chpl{do} keyword is always required in the
keyword-based notation.  The bracketed form is a syntactic
convenience.

The handling of the outer variables within the forall expression and
the role of \sntx{task-intent-clause} are defined in \rsec{Forall_Intents}.

\subsection{Execution and Serializability}
\label{Forall_Expression_Execution_and_Serializability}
\index{expressions!forall!semantics}

The forall expression executes a forall loop (\rsec{Forall}),
evaluates the body expression on each iteration of the loop, and
returns the resulting values as a collection.  The size and shape of
that collection are determined by the iteratable-expression.

\begin{chapelexample}{forallExpr.chpl}
The code
\begin{chapel}
writeln(+ reduce [i in 1..10] i**2);
\end{chapel}
\begin{chapeloutput}
385
\end{chapeloutput}
applies a reduction to a forall-expression that evaluates the square
of the indices in the range \chpl{1..10}.
\end{chapelexample}

The forall expression follows the semantics of the forall statement as
described in~\ref{forall_semantics}.

\subsection{Zipper Iteration}
\index{expressions!forall!zipper iteration}
Forall expression also support zippered iteration semantics as
described in~\rsec{Zipper_Iteration} and~\rsec{Parallel_Iterators} for
parallel iteration.

\subsection{Filtering Predicates in Forall Expressions}
\label{Filtering_Predicates_Forall}
\index{expressions!forall!and conditional expressions}
\index{expressions!forall!filtering}

A filtering predicate is an if expression that is immediately enclosed
by a forall expression and does not have an
else clause.  Such an if expression filters the iterations of the
forall expression.  The iterations for which the condition does not
hold are not reflected in the result of the forall expression.

\begin{chapelexample}{forallFilter.chpl}
The following expression returns every other element starting with the
first:
\begin{chapelpre}
var s: [1..10] int = [i in 1..10] i;
var result =
\end{chapelpre}
\begin{chapel}
[i in 1..s.numElements] if i % 2 == 1 then s(i)
\end{chapel}
\begin{chapelpost}
;
writeln(result);
\end{chapelpost}
\begin{chapeloutput}
1 3 5 7 9
\end{chapeloutput}
\end{chapelexample}


\section{Forall Intents}
\label{Forall_Intents}
\index{forall intents}
\index{data parallelism!forall intents}

If a variable is referenced within the lexical scope of a
forall statement or expression and is declared outside
that statement or expression, it is subject to \emph{forall intents},
analogously to task intents (\rsec{Task_Intents})
for task-parallel constructs. That is, the variable is considered
to be passed as an actual argument to
each task function created by the object or iterator leading
the execution of the loop. If no tasks are created,
it is considered to be an actual argument to the leader
iterator itself. All references to the variable
within the forall statement or expression implicitly refer
to the corresponding formal argument of the task function
or the leader iterator.

Each formal argument of a task function or iterator has the default
intent by default.  For variables of primitive, enum, class, record
and union types, this has the effect of capturing the value of the
variable at task creation time.  Within the lexical scope of the
forall statement or expression, the variable name references the
captured value instead of the original value.

A formal can be given another intent explicitly by listing it
with that intent in the optional \sntx{task-intent-clause}.
For example, for variables of most types, the \chpl{ref} intent allows
the body of the forall loop to modify the corresponding original
variable or to read its updated value after concurrent modifications.
The \chpl{in} intent is a way to obtain task-private variables
in a forall loop.

\begin{rationale}
A forall statement or expression may create tasks in its implementation.
Forall intents affect those tasks in the same way that task intents
affect the behavior of a task construct such as a \chpl{coforall} loop.
\end{rationale}

\begin{craychapel}
An initial implementation of "reduce" intents is also available,
which permits users to reduce values across iterations of a forall loop.
They are described in the \emph{Reduce Intents} page
under \emph{Technical Notes}
in Cray Chapel online documentation here:
\\ %formatting
\mbox{$$ $$ $$} %indent
\url{http://chapel.cray.com/docs/latest/}
\end{craychapel}


\section{Promotion}
\label{Promotion}
\index{promotion}

A function that expects one or more scalar arguments but is called
with one or more arrays, domains, ranges, or iterators is promoted if
the element types of the arrays, the index types of the domains and/or
ranges, or the yielded types of the iterators can be resolved to the
type of the argument.  The rules of when an overloaded function can be
promoted are discussed in~\rsec{Function_Resolution}.

In addition to scalar functions, operators and casts are also
promoted.

\begin{chapelexample}{promotion.chpl}
Given the array
\begin{chapel}
var A: [1..5] int = [i in 1..5] i;
\end{chapel}
and the function
\begin{chapel}
proc square(x: int) return x**2;
\end{chapel}
then the call \chpl{square(A)} results in the promotion of
the \chpl{square} function over the values in the array \chpl{A}.  The
result is an iterator that returns the
values \chpl{1}, \chpl{4}, \chpl{9}, \chpl{16}, and \chpl{25}.
\begin{chapelnoprint}
for s in square(A) do writeln(s);
\end{chapelnoprint}
\begin{chapeloutput}
1
4
9
16
25
\end{chapeloutput}
\end{chapelexample}

%%
%% sungeun: 10/2011
%% I axed the following paragraph because comments from the peanut
%% gallery suggested it was confusing, too much implementation
%% details, etc.  One idea was to add something about defining
%% iterators to support promotion.  Not sure where that would
%% eventually go.
%%
%% If a promoted function returns a value, the promoted function becomes
%% an iterator that is controlled by a loop over the iterator (or array,
%% domain, or range) that it is promoted by.  If the function does not
%% return a value, the function is controlled by a loop over the iterator
%% that it is promoted by, but the promotion does not become an iterator.

Whole array operations are a form of promotion as applied to operators
rather than functions.


\subsection{Zipper Promotion}
\label{Zipper_Promotion}
\index{promotion!zipper iteration}

Promotion also supports zippered iteration semantics as described
in~\rsec{Zipper_Iteration} and~\rsec{Parallel_Iterators} for parallel
iteration.

Consider a function \chpl{f} with formal
arguments \chpl{s1}, \chpl{s2},~... that are promoted and formal
arguments \chpl{a1}, \chpl{a2},~... that are not promoted.  The call
\begin{chapel}
f(s1, s2, ..., a1, a2, ...)
\end{chapel}
is equivalent to
\begin{chapel}
[(e1, e2, ...) in zip(s1, s2, ...)] f(e1, e2, ..., a1, a2, ...)
\end{chapel}
The usual constraints of zipper iteration apply to zipper promotion so
the promoted actuals must have the same shape.

\begin{chapelexample}{zipper-promotion.chpl}
Given a function defined as
\begin{chapel}
proc foo(i: int, j: int) {
  return (i,j);
}
\end{chapel}
and a call to this function written
\begin{chapel}
writeln(foo(1..3, 4..6));
\end{chapel}
then the output is
\begin{chapelprintoutput}{}
(1, 4) (2, 5) (3, 6)
\end{chapelprintoutput}
\end{chapelexample}

\subsection{Whole Array Assignment}
\label{Whole_Array_Assignment}
\index{whole array assignment}
\index{arrays!assignment}
\index{assignment!whole array}

Whole array assignment is a considered a degenerate case of promotion
and is implicitly parallel.  The assignment statement
\begin{chapel}
LHS = RHS;
\end{chapel}
is equivalent to
\begin{chapel}
forall (e1,e2) in zip(LHS,RHS) do
  e1 = e2;
\end{chapel}

\subsection{Evaluation Order}
\label{Evaluation_Order}
\index{data parallelism!evaluation order}
The semantics of whole array assignment and promotion are different
from most array programming languages.  Specifically, the compiler
does not insert array temporaries for such operations if any of the
right-hand side array expressions alias the left-hand side expression.

%
% sungeun 4/8/2011
% Did not convert this one due to non-deterministic output
%
\begin{example}
If \chpl{A} is an array declared over the indices \chpl{1..5}, then
the following codes are not equivalent:
\begin{chapel}
A[2..4] = A[1..3] + A[3..5];
\end{chapel}
and
\begin{chapel}
var T = A[1..3] + A[3..5];
A[2..4] = T;
\end{chapel}
This follows because, in the former code, some of the new values that
are assigned to \chpl{A} may be read to compute the sum depending on
the number of tasks used to implement the data parallel statement.
\end{example}



\section{Reductions and Scans}
\label{Reductions_and_Scans}
\index{reductions}
\index{scans}
\index{data parallelism!reductions}
\index{data parallelism!scans}

Chapel provides reduction and scan expressions that apply operators to
aggregate expressions in stylized ways.  Reduction expressions
collapse the aggregate's values down to a summary value.  Scan
expressions compute an aggregate of results where each result value
stores the result of a reduction applied to all of the elements in the
aggregate up to that expression.  Chapel provides a number of predefined
reduction and scan operators, and also supports a mechanism for the
user to define additional reductions and
scans (Chapter~\ref{User_Defined_Reductions_and_Scans}).

\subsection{Reduction Expressions}
\label{reduce}
\index{reduction expressions}
\index{expressions!reduction}

A reduction expression applies a reduction operator to an aggregate
expression, collapsing the aggregate's dimensions down into a result
value (typically a scalar or summary expression that is independent of
the input aggregate's size).  For example, a sum reduction computes
the sum of all the elements in the input aggregate expression.

The syntax for a reduction expression is given by:
\begin{syntax}
reduce-expression:
  reduce-scan-operator `reduce' iteratable-expression
  class-type `reduce' iteratable-expression

reduce-scan-operator: one of
  + $ $ $ $ * $ $ $ $ && $ $ $ $ || $ $ $ $ & $ $ $ $ | $ $ $ $ ^ $ $ $ $ `min' $ $ $ $ `max' $ $ $ $ `minloc' $ $ $ $ `maxloc'
\end{syntax}

Chapel's predefined reduction operators are defined
by \sntx{reduce-scan-operator} above.  In order, they are: sum,
product, logical-and, logical-or, bitwise-and, bitwise-or,
bitwise-exclusive-or, minimum, maximum, minimum-with-location, and
maximum-with-location.  The minimum reduction returns the minimum
value as defined by the \verb@<@ operator.  The maximum reduction
returns the maximum value as defined by the \verb@>@ operator.  The
minimum-with-location reduction returns the lowest index position with
the minimum value (as defined by the \verb@<@ operator).  The
maximum-with-location reduction returns the lowest index position with
the maximum value (as defined by the \verb@>@ operator).

The expression on the right-hand side of the \chpl{reduce} keyword
can be of any type that can be iterated over, provided
the reduction operator can be applied to the values yielded
by the iteration. For example, the bitwise-and
operator can be applied to arrays of boolean or integral types to
compute the bitwise-and of all the values in the array.

For the minimum-with-location and maximum-with-location reductions,
the argument on the right-hand side of the \chpl{reduce} keyword
must be a 2-tuple. Its first component is the collection
of values for which the minimum/maximum value is to be computed.  The
second argument component is a collection of indices with the same size and
shape that provides names for the locations of the values in the first
component.  The reduction returns a tuple containing the
minimum/maximum value in the first argument component and the value
at the corresponding location in the second argument component.

\begin{chapelexample}{reduce-loc.chpl}
The first line below computes the smallest element in an array
\chpl{A} as well as its index, storing the results in \chpl{minA} and
\chpl{minALoc}, respectively.  It then computes the largest element in
a forall expression making calls to a function \chpl{foo()}, storing
the value and its number in \chpl{maxVal} and \chpl{maxValNum}.
\begin{chapelnoprint}
config const n = 10;
const D = {1..n};
var A: [D] int = [i in D] i % 7;
proc foo(x) return x % 7;
\end{chapelnoprint}
\begin{chapel}
var (minA, minALoc) = minloc reduce zip(A, A.domain); 
var (maxVal, maxValNum) = maxloc reduce zip([i in 1..n] foo(i), 1..n);
\end{chapel}
\begin{chapelnoprint}
writeln((minA, minALoc));
writeln((maxVal, maxValNum));
\end{chapelnoprint}
\begin{chapeloutput}
(0, 7)
(6, 6)
\end{chapeloutput}
\end{chapelexample}

User-defined reductions are specified by preceding the
keyword \chpl{reduce} by the class type that implements the reduction
interface as described in~\rsec{User_Defined_Reductions_and_Scans}.

\subsection{Scan Expressions}
\label{scan}
\index{scan expressions}
\index{expressions!scan}

A scan expression applies a scan operator to an aggregate expression,
resulting in an aggregate expression of the same size and shape.  The
output values represent the result of the operator applied to all
elements up to and including the corresponding element in the input.

The syntax for a scan expression is given by:
\begin{syntax}
scan-expression:
  reduce-scan-operator `scan' iteratable-expression
  class-type `scan' iteratable-expression
\end{syntax}

The predefined scans are defined by \sntx{reduce-scan-operator}.  These
are identical to the predefined reductions and are described
in~\rsec{reduce}.

The expression on the right-hand side of the scan can be of any type
that can be iterated over and to which the operator can be applied.

%
% sungeun: 4/8/2011
% Did not convert this one yet due to warning about serializing scans
%
\begin{example}
Given an array
\begin{chapel}
var A: [1..3] int = 1;
\end{chapel}
that is initialized such that each element contains one, then the code
\begin{chapel}
writeln(+ scan A);
\end{chapel}
outputs the results of scanning the array with the sum operator.  The
output is
\begin{chapelprintoutput}{}
1 2 3
\end{chapelprintoutput}
\end{example}

User-defined scans are specified by preceding the keyword \chpl{scan}
by the class type that implements the scan interface as described
in Chapter~\ref{User_Defined_Reductions_and_Scans}.

\section{Configuration Constants for Default Data Parallelism}
\label{data_parallel_knobs}
\index{data parallelism!knobs for default data parallelism}
\index{data parallelism!configuration constants}
\index{dataParTasksPerLocale@\chpl{dataParTasksPerLocale}}
\index{dataParIgnoreRunningTasks@\chpl{dataParIgnoreRunningTasks}}
\index{dataParMinGranularity@\chpl{dataParMinGranularity}}

The following configuration constants are provided to control the
degree of data parallelism over ranges, default domains, and default
arrays:

\begin{center}
\begin{tabular}{|l|l|l|}
\hline
{\bf Config Const} & {\bf Type} & {\bf Default} \\
\hline
\chpl{dataParTasksPerLocale} & \chpl{int} &
top level \chpl{.maxTaskPar}~(see~\rsec{Locale_Methods}) \\
\chpl{dataParIgnoreRunningTasks} & \chpl{bool} & \chpl{true} \\
\chpl{dataParMinGranularity} & \chpl{int} & \chpl{1} \\
\hline
\end{tabular}
\end{center}

The configuration constant \chpl{dataParTasksPerLocale} specifies the
number of tasks to use when executing a forall loop over a range,
default domain, or default array.  The actual number of tasks may be
fewer depending on the other two configuration constants.  A value of
zero results in using the default value.

The configuration constant \chpl{dataParIgnoreRunningTasks}, when
true, has no effect on the number of tasks to use to execute the
forall loop.  When false, the number of tasks per locale is decreased
by the number of tasks that are already running on the locale, with a
minimum value of one.

The configuration constant \chpl{dataParMinGranularity} specifies the
minimum number of iterations per task created.  The number of tasks is
decreased so that the number of iterations per task is never less than
the specified value.

For distributed domains and arrays that have these same configuration
constants (\eg, Block and Cyclic distributions), these same
module level configuration constants are used to specify their
default behavior within each locale.

\cleardoublepage
\sekshun{Locales}
\label{Locales_Chapter}

Chapel provides high-level abstractions that allow programmers to
exploit locality by controlling the affinity of both data and tasks to
abstract units of processing and storage capabilities
called \emph{locales}.  The \emph{on-statement} allows for the
migration of tasks to \emph{remote} locales.

\index{local}
\index{remote}
\index{locales!local}
\index{locales!remote}
Throughout this section, the term \emph{local} will be used to
describe the locale on which a task is running, the data located on
this locale, and any tasks running on this locale.  The
term \emph{remote} will be used to describe another locale, the data
on another locale, and the tasks running on another locale.

\section{Locales}
\label{Locales}
\index{locales}

A \emph{locale} is a portion of the target parallel architecture that
has processing and storage capabilities.  Chapel implementations
should typically define locales for a target architecture such that
tasks running within a locale have roughly uniform access to values
stored in the locale's local memory and longer latencies for accessing
the memories of other locales.  As an example, a cluster of multicore
nodes or SMPs would typically define each node to be a locale.  In
contrast a pure shared memory machine would be defined as a single
locale.

\subsection{Locale Types}
\label{The_Locale_Type}
\index{locale@\chpl{locale}}
\index{types!locale@\chpl{locale}}

The identifier \chpl{locale} is a class type that abstracts a
locale as described above.  Both data and tasks can be associated with
a value of locale type.  A Chapel implementation may define subclass(es)
of \chpl{locale} for a richer description of the target architecture.

\subsection{Locale Methods}
\label{Locale_Methods}
\index{locales!methods}
\index{predefined functions!locale}

The locale type supports the following methods:

%% \begin{protohead}
%% proc locale.blockedTasks() : int ;
%% \end{protohead}
%% \begin{protobody}
%% Returns the number of tasks on this locale which are blocked at the time of the
%% call.
%% \end{protobody}

\index{predefined functions!callStackSize@\chpl{callStackSize}}
\index{locales!callStackSize@\chpl{callStackSize}}
\begin{protohead}
proc locale.callStackSize: uint(64);
\end{protohead}
\begin{protobody}
Returns the per-task call stack size used when creating tasks on the
locale in question.  A value of 0 indicates that the call stack size
is determined by the system.
\end{protobody}

\index{predefined functions!id@\chpl{id}}
\index{locales!id@\chpl{id}}
\begin{protohead}
proc locale.id: int;
\end{protohead}
\begin{protobody}
Returns a unique integer for each locale, from 0 to the number of
locales less one.
\end{protobody}

%% \begin{protohead}
%% proc locale.idleThreads() : uint ;
%% \end{protohead}
%% \begin{protobody}
%% Returns the number of threads which are currently idle on this locale.
%% \end{protobody}

\index{predefined functions!maxTaskPar@\chpl{maxTaskPar}}
\index{locales!maxTaskPar@\chpl{maxTaskPar}}
\begin{protohead}
proc locale.maxTaskPar: int(32);
\end{protohead}
\begin{protobody}
Returns an estimate of the maximum parallelism available for tasks
on a given locale.
\end{protobody}

\index{predefined functions!name@\chpl{name}}
\index{locales!name@\chpl{name}}
\begin{protohead}
proc locale.name: string;
\end{protohead}
\begin{protobody}
Returns the name of the locale.
\end{protobody}

\index{predefined functions!numCores@\chpl{numCores}}
\index{locales!numCores@\chpl{numCores}}
\begin{protohead}
proc locale.numCores: int;
\end{protohead}
\begin{protobody}
Returns the number of logical CPUs available on a given locale.
\end{protobody}

\index{predefined functions!physicalMemory@\chpl{physicalMemory}}
\index{locales!physicalMemory@\chpl{physicalMemory}}
\begin{protohead}
use Memory;
proc locale.physicalMemory(unit: MemUnits=MemUnits.Bytes, type retType=int(64)): retType;
\end{protohead}
\begin{protobody}
Returns the amount of physical memory available on a given locale in
terms of the specified memory units (Bytes, KB, MB, or GB) using a
value of the specified return type.
\end{protobody}

%% \begin{protohead}
%% proc locale.queuedTasks() : uint ;
%% \end{protohead}
%% \begin{protobody}
%% Returns the number of tasks on this locale which are currently on the task queue.
%% \end{protobody}

%% \begin{protohead}
%% proc locale.runningTasks() : uint ;
%% \end{protohead}
%% \begin{protobody}
%% Returns the number of tasks on this locale that are currently running.
%% \end{protobody}

%% \begin{protohead}
%% proc locale.totalThreads() : uint ;
%% \end{protohead}
%% \begin{protobody}
%% Returns the total number of threads (active + idle) that currently exist on this
%% locale.
%% \end{protobody}

\subsection{The Predefined Locales Array}
\label{Predefined_Locales_Array}
\index{Locales@\chpl{Locales}}
\index{numLocales@\chpl{numLocales}}
\index{execution environment}

Chapel provides a predefined environment that stores information about
the locales used during program execution.  This {\em execution
environment} contains definitions for the array of locales on which
the program is executing (\chpl{Locales}), a domain for that array
(\chpl{LocaleSpace}), and the number of locales (\chpl{numLocales}).
\begin{chapel}
config const numLocales: int;
const LocaleSpace: domain(1) = [0..numLocales-1];
const Locales: [LocaleSpace] locale;
\end{chapel}
When a Chapel program starts, a single task executes \chpl{main}
on \chpl{Locales(0)}.

Note that the Locales array is typically defined such that distinct
elements refer to distinct resources on the target parallel
architecture.  In particular, the Locales array itself should not be
used in an oversubscribed manner in which a single processor resource
is represented by multiple locale values (except during development).
Oversubscription should instead be handled by creating an aggregate of
locale values and referring to it in place of the Locales array.

\begin{rationale}
This design choice encourages clarity in the program's source text and
enables more opportunities for optimization.

For development purposes, oversubscription is still very useful and
this should be supported by Chapel implementations to allow
development on smaller machines.
\end{rationale}

\begin{example}
The code
\begin{chapel}
const MyLocales: [0..numLocales*4] locale 
               = [loc in 0..numLocales*4] Locales(loc%numLocales);
on MyLocales[i] ...
\end{chapel}
defines a new array \chpl{MyLocales} that is four times the size of
the \chpl{Locales} array.  Each locale is added to
the \chpl{MyLocales} array four times in a round-robin fashion.
\end{example}

\subsection{The {\em here} Locale}
\label{here}
\index{here@\chpl{here}}
\index{locales!here@\chpl{here}}

A predefined constant locale \chpl{here} can be used anywhere in a
Chapel program.  It refers to the locale that the current task is
running on.

\begin{example}
The code
\begin{chapel}
on Locales(1) {
  writeln(here.id);
}
\end{chapel}
results in the output \chpl{1} because the \chpl{writeln} statement is
executed on locale 1.
\end{example}

The identifier \chpl{here} is not a keyword and can be overridden.

\subsection{Querying the Locale of an Expression}
\label{Querying_the_Locale_of_a_Variable}
\index{locale@\chpl{locale}}

The locale associated with an expression (where the expression is
stored) is queried using the following syntax:
\begin{syntax}
locale-access-expression:
  expression . `locale'
\end{syntax}
When the expression is a class, the access returns the locale on which
the class object exists rather than the reference to the class.  If
the expression is a value, it is considered local.  The implementation
may warn about this behavior.  If the expression is a locale, it is
returned directly.

\begin{example}
Given a class C and a record R, the code
\begin{chapel}
on Locales(1) {
  var x: int;
  var c: C;
  var r: R;
  on Locales(2) {
    on Locales(3) {
      c = new C();
      r = new R();
    }
    writeln(x.locale.id);
    writeln(c.locale.id);
    writeln(r.locale.id);
  }
}
\end{chapel}
results in the output
\begin{chapelprintoutput}{}
1
3
1
\end{chapelprintoutput}
The variable \chpl{x} is declared and exists on \chpl{Locales(1)}.
The variable \chpl{c} is a class reference.  The reference exists
on \chpl{Locales(1)} but the object itself exists
on \chpl{Locales(3)}.  The locale access returns the locale where the
object exists.  Lastly, the variable \chpl{r} is a record and has
value semantics.  It exists on \chpl{Locales(1)} even though it is
assigned a value on a remote locale.
\end{example}

Global (non-distributed) constants are replicated across all locales.
\begin{example}
For example, the following code:
%
% We can't yet specify multiple .good files or .numlocales files, so
% add this test later when we can.
%
\begin{chapel}
const c = 10;
for loc in Locales do on loc do
    writeln(c.locale.id);
\end{chapel}
outputs
\begin{chapelprintoutput}{}
0
1
2
3
4
\end{chapelprintoutput}
when running on 5 locales.
\end{example}


\section{The On Statement}
\label{On}
\index{on@\chpl{on}}
\index{statements!on@\chpl{on}}

The on statement controls on which locale a block of code should be
executed or data should be placed.  The syntax of the on statement is
given by
\begin{syntax}
on-statement:
  `on' expression `do' statement
  `on' expression block-statement
\end{syntax}
The locale of the expression is automatically queried as described
in~\rsec{Querying_the_Locale_of_a_Variable}.  Execution of the
statement occurs on this specified locale and then continues after
the \chpl{on-statement}.

Return statements may not be lexically enclosed in on statements.
Yield statements may only be lexically enclosed in on statements in
parallel iterators~\rsec{Parallel_Iterators}.

\subsection{Remote Variable Declarations}
\label{remote_variable_declarations}
\index{variable declarations!remote}

By default, when new variables and data objects are created, they are
created in the locale where the task is running.  Variables can be
defined within an \sntx{on-statement} to define them on a particular
locale such that the scope of the variables is outside
the \sntx{on-statement}.  This is accomplished using a similar syntax
but omitting the \chpl{do} keyword and braces.  The syntax is given
by:
\begin{syntax}
remote-variable-declaration-statement:
  `on' expression variable-declaration-statement
\end{syntax}

\cleardoublepage
\sekshun{Domain Maps}
\label{Domain_Maps}
\index{domain maps}

\index{mapped!domain maps}
A domain map specifies the implementation of the domains and arrays
that are \emph{mapped} using it. That is, it defines how domain indices
and array elements are mapped to locales, how they are stored in
memory, and how operations such as accesses, iteration, and slicing
are performed.  Each domain and array is mapped using some domain map.

\index{layouts (see also domain maps, layouts)}
\index{distributions (see also domain maps, distributions)}
\index{domain maps!layout}
\index{domain maps!distribution}
A domain map is either a \emph{layout} or a \emph{distribution}.
A layout describes domains and arrays that exist on a single locale,
whereas a distribution describes domains and arrays that are
partitioned across multiple locales.

A domain map is represented in the program with an instance of
a \emph{domain map class}.
Chapel provides a set of standard domain map classes.
Users can create domain map classes as well.

Domain maps are presented as follows:
\begin{itemize}

\item domain maps for domain types \rsec{Domain_Maps_For_Types},
      domain values \rsec{Domain_Maps_For_Values}, and
      arrays \rsec{Domain_Maps_For_Arrays}

\item domain maps are not retained upon domain assignment
      \rsec{Domain_Maps_Not_Assigned}

\item standard layouts and distributions, such as Block and Cyclic,
are documented under \emph{Standard Library}
in Cray Chapel online documentation here:
\\ %formatting
\mbox{$$ $$ $$} %indent
% Should this be "craychapel"?
\url{http://chapel.cray.com/docs/latest/}

\item specification of user-defined domain maps is forthcoming;
please refer to the \emph{Domain Map Standard Interface} page
under \emph{Technical Notes}
in Cray Chapel online documentation here:
\\ %formatting
\mbox{$$ $$ $$} %indent
% Should this be "craychapel"?
\url{http://chapel.cray.com/docs/latest/}

\end{itemize}


\section{Domain Maps for Domain Types}
\label{Domain_Maps_For_Types}
\index{domain maps for domain types}
\index{types!domains!domain maps for}

Each domain type has a domain map associated with it.
This domain map is used to map all domain values of this type
(\rsec{Domain_Maps_For_Values}).

If a domain type does not have a domain map specified for it
explicitly as described below,
a default domain map is provided by the Chapel implementation.
Such a domain map will typically be a layout that maps the entire domain
to the locale on which the domain value is created or
the domain or array variable is declared.
% or: "the locale on which the current task is running, i.e., 'here'"

\begin{craychapel}
The default domain map provided by the Cray Chapel compiler
is such a layout. The storage for the representation of a domain's
index set is placed on the locale where the domain variable is declared.
The storage for the elements of arrays declared over domains with
the default map is placed on the locale where the array variable
is declared.
Arrays declared over rectangular domains with this default map
are laid out in memory in row-major order.
\end{craychapel}

\index{dmap value}
\index{dmapped clause}
\index{domain maps!dmap value}
\index{domain maps!dmapped clause}
A domain map can be specified explicitly by
providing a \emph{dmap value} in a \chpl{dmapped} clause:

\begin{syntax}
mapped-domain-type:
  domain-type `dmapped' dmap-value

dmap-value:
  expression
\end{syntax}

A dmap value consists of an instance of a domain map class
wrapped in an instance of the predefined record \chpl{dmap}.
The domain map class is chosen and instantiated by the user.
% The above sentence strive to emphasize that here the users need to make a
% choice according to their needs, vs. 'dmap' which is prescribed by the lang.
\chpl{dmap} behaves like a generic record with a single generic field,
which holds the domain map instance.

\begin{example}
The code
\begin{chapel}
use BlockDist;
var MyBlockDist: dmap(Block(rank=2));
\end{chapel}
declares a variable capable of storing dmap values
for a two-dimensional Block distribution.
The Block distribution is described in more detail here:
\\ %formatting
\mbox{$$ $$ $$} %indent
% Should this be "craychapel"?
\url{http://chapel.cray.com/docs/latest/}
\end{example}

\begin{example}
The code
\begin{chapel}
use BlockDist;
var MyBlockDist: dmap(Block(rank=2)) = new dmap(new Block({1..5,1..6}));
\end{chapel}
creates a dmap value wrapping a two-dimensional Block distribution with a
bounding box of \chpl{\{1..5, 1..6\}} over all of the locales.
\end{example}

\begin{example}
The code
\begin{chapel}
use BlockDist;
var MyBlockDist = new dmap(new Block({1..5,1..6}));
type MyBlockedDom = domain(2) dmapped MyBlockDist;
\end{chapel}
defines a two-dimensional rectangular domain type
that is mapped using a Block distribution.
\end{example}

The following syntactic sugar is provided within the \chpl{dmapped} clause.
If a \chpl{dmapped} clause starts with the name of a domain map class,
it is considered to be a constructor expression as if preceded by
\chpl{new}. The resulting domain map instance is wrapped in a newly-created
instance of \chpl{dmap} implicitly.

\begin{example}
The code
\begin{chapel}
use BlockDist;
type BlockDom = domain(2) dmapped Block({1..5,1..6});
\end{chapel}
is equivalent to
\begin{chapel}
use BlockDist;
type BlockDom = domain(2) dmapped new dmap(new Block({1..5,1..6}));
\end{chapel}
\end{example}


\section{Domain Maps for Domain Values}
\label{Domain_Maps_For_Values}
\index{domain maps!for domain values}
\index{values!domains!domain maps for}

A domain value is always mapped using the domain map of that value's type.
The type inferred for a domain literal (\rsec{Rectangular_Domain_Values})
has a default domain map.

\begin{example}
In the following code
\begin{chapel}
use BlockDist;
var MyDomLiteral = {1..2,1..3};
var MyBlockedDom: domain(2) dmapped Block({1..5,1..6}) = MyDomLiteral;
\end{chapel}
\chpl{MyDomLiteral} is given the inferred type of the domain literal
and so will be mapped using a default map.
MyBlockedDom is given a type explicitly, in accordance to which
it will be mapped using a Block distribution.
\end{example}

A domain value's map can be changed explicitly with a \chpl{dmapped} clause,
in the same way as a domain type's map.

\begin{syntax}
mapped-domain-expression:
  domain-expression `dmapped' dmap-value
\end{syntax}

\begin{example}
In the following code
\begin{chapel}
use BlockDist;
var MyBlockedDomLiteral1 = {1..2,1..3} dmapped new dmap(new Block({1..5,1..6}));
var MyBlockedDomLiteral2 = {1..2,1..3} dmapped Block({1..5,1..6});
\end{chapel}
both \chpl{MyBlockedDomLiteral1} and \chpl{MyBlockedDomLiteral2}
will be mapped using a Block distribution.
\end{example}


\section{Domain Maps for Arrays}
\label{Domain_Maps_For_Arrays}
\index{domain maps!for arrays}
\index{arrays!domain maps}

Each array is mapped using the domain map of the domain
over which the array was declared.

\begin{example}
In the code
\begin{chapel}
use BlockDist;
var Dom: domain(2) dmapped Block({1..5,1..6}) = {1..5,1..6};
var MyArray: [Dom] real;
\end{chapel}
the domain map used for \chpl{MyArray} is the Block
distribution from the type of \chpl{Dom}.
\end{example}

\section{Domain Maps Are Not Retained upon Domain Assignment}
\label{Domain_Maps_Not_Assigned}
\index{domain maps!domain assignment}
\index{domains!assignment}
\index{assignment!domain}

Domain assignment (\rsec{Domain_Assignment}) transfers only the index
set of the right-hand side expression. The implementation of the
left-hand side domain expression, including its domain map, is
determined by its type and so does not change upon a domain assignment.

\begin{example}
In the code
\begin{chapel}
use BlockDist;
var Dom1: domain(2) dmapped Block({1..5,1..6}) = {1..5,1..6};
var Dom2: domain(2) = Dom1;
\end{chapel}
\chpl{Dom2} is mapped using the default distribution, despite
\chpl{Dom1} having a Block distribution.
\end{example}

\begin{example}
In the code
\begin{chapel}
use BlockDist;
var Dom1: domain(2) dmapped Block({1..5,1..6}) = {1..5,1..6};
var Dom2 = Dom1;
\end{chapel}
\chpl{Dom2} is mapped using the same distribution as \chpl{Dom1}.
This is because the declaration of \chpl{Dom2} lacks an explicit
type specifier and so its type is defined to be the type of its
initialization expression, \chpl{Dom1}. So in this situation
the effect is that the domain map does transfer upon
an initializing assignment.
\end{example}

\cleardoublepage
\input{User_Defined_Reductions_and_Scans}
\cleardoublepage
\sekshun{Memory Consistency Model}
\label{Memory_Consistency_Model}
\index{memory consistency model}

\begin{openissue}
  This chapter is a work-in-progress and represents an area where we
  are particularly interested in feedback from, and collaboration
  with, the broader community.
\end{openissue}

Chapel's memory consistency model is well-defined for programs that
are {\em data-race-free}.  Such programs are sequentially consistent.
For other programs, no specific guarantees can be made about the
program's execution.

\begin{rationale}
  Chapel presents a memory consistency model that is less strict than
  Java's.  It does so because it does not strive to provide the same
  dynamic security requirements as Java does.
\end{rationale}

Accessing a synchronization variable (\chpl{sync} or \chpl{single}, \rsec{Synchronization_Variables}) is
the only way to correctly synchronize between two Chapel tasks.
Such reads and writes serve as memory fences, preventing
reordering of reads and writes to traditional variables across the
synchronization variable's access.  When the same memory location
is written by one task and read by another, the ordering of the
read relative to the write is undefined unless there is an intervening
access to a synchronization variable.

%\begin{chapelexample}{syncFenceFlag}
\begin{example}
  In this example, a synchronization variable is used to (a) ensure that
  all writes to an array of unsynchronized variables are complete, (b)
  signal that fact to a second task, and (c) pass along the number of
  values that are valid for reading.

  The program
\begin{chapel}
var A: [1..100] real;
var done(*\texttt{\$}*): sync int;           // initially empty
cobegin {
  { // Reader task
    const numToRead = done(*\texttt{\$}*);   // block until writes are complete
    for i in 1..numToRead do
      writeln("A[", i, "] = ", A[i]);
  }
  {  // Writer task
    const numToWrite = 23;     // an arbitrary number
    for i in 1..numToWrite do
      A[i] = i/10.0;
    done(*\texttt{\$}*) = numToWrite;        // fence writes to A and signal done
  }
}
\end{chapel}
  produces the output
\begin{chapelprintoutput}{}
A[1] = 0.1
A[2] = 0.2
A[3] = 0.3
A[4] = 0.4
A[5] = 0.5
A[6] = 0.6
A[7] = 0.7
A[8] = 0.8
A[9] = 0.9
A[10] = 1.0
A[11] = 1.1
A[12] = 1.2
A[13] = 1.3
A[14] = 1.4
A[15] = 1.5
A[16] = 1.6
A[17] = 1.7
A[18] = 1.8
A[19] = 1.9
A[20] = 2.0
A[21] = 2.1
A[22] = 2.2
A[23] = 2.3
\end{chapelprintoutput}
%\end{chapelexample}
\end{example}


\begin{chapelexample}{syncSpinWait.chpl}
One consequence of Chapel's memory consistency model is that a task cannot spin-wait on a
variable waiting for another task to write to that variable.  The behavior of
the following code is undefined:

\begin{chapelpre}
if false { // }
\end{chapelpre}
\begin{chapel}
var x: int;
cobegin with (ref x) {
  while x != 1 do ;  // spin wait
  x = 1;
}
\end{chapel}
\begin{chapelnoprint}
// {
}
\end{chapelnoprint}
In contrast, spinning on a synchronization variable has well-defined
behavior:
\begin{chapel}
var x(*\texttt{\$}*): sync int;
cobegin {
  while x(*\texttt{\$}*).readXX() != 1 do ;  // spin wait
  x(*\texttt{\$}*).writeXF(1);
}
\end{chapel}
\begin{chapeloutput}
\end{chapeloutput}

In this code, the first statement in the cobegin statement executes a
loop until the variable is set to one.  The second statement in the
cobegin statement sets the variable to one.  Neither of these
statements block.
\end{chapelexample}


\begin{future}
Upon completion, Chapel's atomic statement~(\rsec{Atomic_Statement}) will serve as
an additional means of correctly synchronizing between tasks.
\end{future}


\cleardoublepage
\sekshun{Interoperability}
\label{Interoperability}
\index{interoperability}

Chapel's interoperability features support cooperation between Chapel
and other languages.  They provide the ability to create software
systems that incorporate both Chapel and non-Chapel components.
Thus, they support the reuse of existing software components while
leveraging the unique features of the Chapel language.

Interoperability can be broken down in terms of the exchange of types, variables
and procedures, and whether these are imported or exported.  An overview of
procedure importing and exporting is provided in~\rsec{Interop_Overview}.
Details on sharing types, variables and procedures are supplied
in \rsec{Shared_Language_Elements}.
%  The creation and use of Chapel libraries is
%treated in~\rsec{Interop_Libraries}.  

\begin{future}

At present, the backend language for Chapel is C, which makes it relatively
easy to call C libraries from Chapel and vice versa.  To support a variety of
platforms without requiring recompilation, it may be desirable to move
to an intermediate-language model.

In that case, each supported platform must minimally support that virtual
machine.  However, in addition to increased portability, a virtual machine
model may expose elements of the underlying machine's programming model
(hardware task queues, automated garbage collection, etc.) that are not easily
rendered in C.  In addition, the virtual machine model can support run-time task
migration.

\end{future}

The remainder of this chapter documents Chapel support of interoperability through
the existing C-language backend.

\section{Interoperability Overview}
\label{Interop_Overview}
\index{interoperability!overview}

The following two subsections provide an overview of calling externally-defined
(C) routines in Chapel, and setting up Chapel routines so they can be called
from external (C) code.

\subsection{Calling External Functions}
\label{Calling_External_Functions}
\index{interoperability!external functions!calling}

To use an external function in a Chapel program, it is necessary to inform the
Chapel compiler of that routine's signature through an external function
declaration.  This permits Chapel to bind calls to that function signature
during function resolution.  The user must also supply a definition for the
referenced function by naming a C source file, an object file or an object
library on the \chpl{chpl} command line. 

An external procedure declaration has the following syntax:
\begin{syntax}
external-procedure-declaration-statement:
  `extern' external-name[OPT] `proc' function-name argument-list return-intent[OPT] return-type[OPT]
\end{syntax}

Chapel will call the external function using the parameter types supplied in
the \chpl{extern} declaration.  Therefore, in general, the type of each argument
in the supplied \sntx{argument-list} must be the Chapel equivalent of the
corresponding external type.  

The return value of the function can be used by Chapel only if its type is
declared using the optional \sntx{return-type} specifier.  If it is omitted,
Chapel assumes that no value is returned, or equivalently that the function
returns \chpl{void}.

At present, external iterators are not supported.  

\begin{future}
The overloading of function names is
also not supported directly in the compiler.  However, one can use
the \sntx{external-name} syntax to supply a name to be used by the linker.  In
this way, function overloading can be implemented ``by hand''.  This syntax also
supports calling external C++ routines: The \sntx{external-name} to use is the
mangled function name generated by the external compilation
environment\footnote{In UNIX-like programming environments, \chpl{nm} and \chpl{grep}
can be used to find the mangled name of a given function within an object file
or object library.}.
\end{future}

\begin{future}
Dynamic dispatch (polymorphism) is also unsupported in this version.  But this
is not ruled out in future versions.  Since Chapel already supports type-based
procedure declaration and resolution, it is a small step to translate a
type-relative extern method declaration into a virtual method table entry.  The
mangled name of the correct external function must be supplied for each
polymorphic type available.  However, most likely the generation of \chpl{.chpl}
header files from C and C++ libraries can be fully automated.
\end{future}

There are three ways to supply to the Chapel compiler the definition of an
external function: as a C source file (\chpl{.c} or \chpl{.h}), as an object
file and as an object library.  It is platform-dependent whether static
libraries (archives), dynamic libraries or both are supported.  See
the \chpl{chpl} man page for more information on how these file types are handled.

\subsection{Calling Chapel Functions}
\label{Calling_Chapel_Functions}
\index{interoperability!Chapel functions!calling}

To call a Chapel procedure from external code, it is necessary to expose the
corresponding function symbol to the linker.  This is done by adding
the \chpl{export} linkage specifier to the function definition.
The \chpl{export} specifier ensures that the corresponding procedure will be
resolved, even if it is not called within the Chapel program or library being
compiled.

An exported procedure declaration has the following syntax:
\begin{syntax}
exported-procedure-declaration-statement:
  `export' external-name[OPT] `proc' function-name argument-list return-intent[OPT] return-type[OPT]
    function-body

external-name:
  identifier
  string-literal
\end{syntax}

If the optional \sntx{external-name} is
supplied, then it is used verbatim as the exported function symbol.  Otherwise,
the Chapel name of the procedure is exported.  The rest of the procedure
declaration is the same as for a non-exported function.  An exported procedure can be
called from within Chapel as well.  Currently, iterators cannot be exported.

\begin{future}
Currently, exported functions cannot have generic, \chpl{param} or type arguments.
This is because such functions actually represent a family of functions,
specific versions of which are instantiated as need during function resolution.

Instantiating all possible versions of a template function is not
practical in general.  However, if explicit instantiation were supported in
Chapel, an explicit instantiation with the export linkage specifier would
clearly indicate that the matching template function was to be instantiated with
the given \chpl{param} values and argument types.
\end{future}

\section{Shared Language Elements}
\label{Shared_Language_Elements}
\index{interoperability!sharing}

This section provides details on how to share Chapel types, variables and
procedures with external code.  It is written assuming that the intermediate
language is C.

\subsection{Shared Types}

This subsection discusses how specific types are shared between Chapel and
external code.  

\subsubsection{Referring to Standard C Types}
\label{Referring_to_Standard_C_Types}
\index{interoperability!standard C types}
\index{interoperability!C types!standard}

In Chapel code, all standard C types must be expressed in terms of their Chapel
equivalents.  This is true, whether the entity is exported, imported or private.
Standard C types and their corresponding Chapel types are shown in the following
table.

\begin{tabular}{rlrlrl}
C Type & Chapel Type & C Type & Chapel Type & C Type & Chapel Type \\
\hline
\tt int8\_t  & \tt int(8)  & \tt uint8\_t  & \tt uint(8)  & \tt \_real32 & \tt real(32) \\
\tt int16\_t & \tt int(16) & \tt uint16\_t & \tt uint(16) & \tt \_real64 & \tt real(64) \\
\tt int32\_t & \tt int(32) & \tt uint32\_t & \tt uint(32) & \tt \_imag32 & \tt imag(32) \\
\tt int64\_t & \tt int(64) & \tt uint64\_t & \tt uint(64) & \tt \_imag64 & \tt imag(64) \\
\tt chpl\_bool & \tt bool & \tt  const char* & \tt c\_string \\
\tt \_complex64 & \tt complex(64) & \tt \_complex128 & \tt complex(128) \\
\end{tabular}

Standard C types are built-in.  Their Chapel equivalents do not have to be
declared using the \chpl{extern} keyword.

In C, the ``colloquial'' integer type names \chpl{char}, \chpl{signed
char}, \chpl{unsigned char}, (\chpl{signed}) \chpl{short}
(\chpl{int}), \chpl{unsigned short} (\chpl{int}),
(\chpl{signed}) \chpl{int}, \chpl{unsigned int},
(\chpl{signed}) \chpl{long} (\chpl{int}), \chpl{unsigned long} (\chpl{int}), (\chpl{signed}) \chpl{long
long} (\chpl{int}) and \chpl{unsigned long long} (\chpl{int}) may have an
implementation-defined width.\footnote{However, most implementations have settled
on using 8, 16, 32, and 64 bits (respectively) to
represent \chpl{char}, \chpl{short}, \chpl{int} and \chpl{long}, and \chpl{long
long} types}.  When referring to C types in a Chapel program, the burden of
making sure the type sizes agree is on the user.  A Chapel implementation must
ensure that all of the C equivalents in the above table are defined and have the
correct representation with respect to the corresponding Chapel type.

\subsubsection{Referring to External C Types}
\label{Referring_to_External_C_Types}
\index{interoperability!external C types}
\index{interoperability!C types!external}

An externally-defined type can be referenced using a external type declaration
with the following syntax.
\begin{syntax}
external-type-alias-declaration-statement:
  `extern' `type' type-alias-declaration-list ;
\end{syntax}

In each \sntx{type-alias-declaration}, if the \sntx{type-specifier} part is
supplied, then Chapel uses the supplied type specifier internally.  Otherwise,
it treats the named type as an opaque type.  The definition for an external type
must be supplied by a C header file named on the \chpl{chpl} command line.

Fixed-size C array types can be described within Chapel using its
homogenous tuple type.  For example, the C typedef
\begin{chapel}
typedef double vec[3];
\end{chapel}
can be described in Chapel using
\begin{chapel}
extern type vec = 3*real(64);
\end{chapel}


\subsubsection{Referring to External C Structs}
\label{Referring_to_External_C_Structs}
\index{interoperability!C structs!external}

External C struct types can be referred to within Chapel by prefixing
a Chapel \chpl{record} definition with the \chpl{extern} keyword.
\begin{syntax}
external-record-declaration-statement:
  `extern' external-name[OPT] simple-record-declaration-statement
\end{syntax}

For example, consider an external C structure defined in \chpl{foo.h} called \chpl{fltdbl}.
\begin{chapel}
    typedef struct _fltdbl {
      float x;
      double y;
    } fltdbl;
\end{chapel}
This type could be referred to within a Chapel program using
\begin{chapel}
   extern record fltdbl {
     var x: real(32);
     var y: real(64);
   }
\end{chapel}
\noindent
and defined by supplying \chpl{foo.h} on the \chpl{chpl} command line.

Within the Chapel declaration, some or all of the fields from the C
structure may be omitted.  The order of these fields need not match
the order they were specified within the C code.  Any fields that are
not specified (or that cannot be specified because there is no
equivalent Chapel type) cannot be referenced within the Chapel code.  Some
effort is made to preserve the values of the undefined fields when copying
these structs but Chapel cannot guarantee the contents or memory story of
fields of which it has no knowledge.

If the optional \sntx{external-name} is supplied, then it is used verbatim as
the exported struct symbol.

A C header file containing the struct's definition in C must be specified on the
chpl compiler command line.  Note that only typdef'd C structures are supported
by default.  That is, in the C header file, the \chpl{struct} must be supplied
with a type name through a \chpl{typedef} declaration. If this is not true, you
can use the \sntx{external-name} part to apply the \chpl{struct} specifier.
As an example of this, given a C declaration of:

% Not a spec test, we would need to be able to provide .c and .h files
\begin{chapel}
  struct Vec3 {
    double x, y, z;
  };
\end{chapel}

in Chapel you would refer to this \chpl{struct} via

\begin{chapel}
  extern "struct Vec3" record Vec3 {
    var x, y, z: real(64);
  }
\end{chapel}


\subsubsection{Referring to External Structs Through Pointers}
\label{Referring_to_External_Structs_Through_Pointers}
\index{interoperability!C Structs!external!pointers to}

An external type which is a pointer to a \chpl{struct} can be referred to from
Chapel using an external \chpl{class} declaration.  External class declarations
have the following syntax.
\begin{syntax}
external-class-declaration-statement:
  `extern' external-name[OPT] simple-class-declaration-statement
\end{syntax}
External class declarations are similar to external record declarations as
discussed above, but place additional requirements on the C code.

For example, given the declaration
\begin{chapel}
  extern class D {
    var x: real;
  }
\end{chapel}
\noindent
the requirements on the corresponding C code are:
\begin{enumerate}
\item There must be a struct type that is typedef'd to have the name \chpl{\_D}.
\item A pointer-to-\chpl{\_D} type must be typedef'd to have the name \chpl{D}.
\item The \chpl{\_D} struct type must contain a field named \chpl{x} of type \chpl{double}.
\end{enumerate}
\noindent
Like external records/structs, it may also contain other fields
that will simply be ignored by the Chapel compiler.

The following C typedef would fulfill the external Chapel class
declaration shown above:
\begin{chapel}
   typedef struct __D {
     double x;
     int y;
   } _D, *D;
\end{chapel}
where the Chapel compiler would not know about the 'y' field and
therefore could not refer to it or manipulate it.

If the optional \sntx{external-name} is supplied, then it is used verbatim as
the exported class symbol.


\subsubsection{Opaque Types}
\label{Opaque_Types}
\index{interoperability!opaque types}

It is possible refer to external pointer-based C types that cannot be
described in Chapel by using the "opaque" keyword.  As the name implies,
these types are opaque as far as Chapel is concerned and cannot be
used for operations other than argument passing and assignment.

For example, Chapel could be used to call an external C function that
returns a pointer to a structure (that can't or won't be described as
an external class) as follows:
\begin{chapel}
    extern proc returnStructPtr(): opaque;

    var structPtr: opaque = returnStructPtr();
\end{chapel}

However, because the type of \chpl{structPtr} is opaque, it can be used only in
assignments and the arguments of functions expecting the same underlying type.
\begin{chapel}
    var copyOfStructPtr = structPtr;

    extern proc operateOnStructPtr(ptr: opaque);
    operateOnStructPtr(structPtr);
\end{chapel}
\noindent
Like a \chpl{void*} in C, Chapel's \chpl{opaque} carries no information
regarding the underlying type.  It therefore subverts type safety, and should be
used with caution.

\subsection{Shared Data}
\label{Shared_Data}
\index{interoperability!shared data}

This subsection discusses how to access external variables and constants.

A C variable or constant can be referred to within Chapel by prefixing
its declaration with the extern keyword.  For example:
\begin{chapel}
    extern var bar: foo;
\end{chapel}
\noindent
would tell the Chapel compiler about an external C variable named
\chpl{bar} of type \chpl{foo}.  Similarly, 
\begin{chapel}
   extern const baz: int(32);
\end{chapel}
would refer to an external 32-bit integer constant named \chpl{baz} in the
C code.  In practice, external consts can be used to provide Chapel
definitions for \#defines and enum symbols in addition to traditional C
constants.

\begin{craychapel}
Note that since params must be known to Chapel at compile-time (and
because the Chapel compiler doesn't have the ability to parse C code),
external params are not supported.
\end{craychapel}

\subsection{Shared Procedures}
\label{Shared_Procedures}
\index{interoperability!shared procedures}

This subsection provides additional detail and examples for calling external
procedures from Chapel and for exporting Chapel functions for external use.

\subsubsection{Calling External C Functions}
\label{Calling_External_C_Functions}
\index{interoperability!external functions!calling}

To call an external C function, a prototype of
the routine must appear in the Chapel code.  This is accomplished by providing
the Chapel signature of the function preceded by the \chpl{extern} keyword.  For
example, for a C function foo() that takes no arguments and returns
nothing, the prototype would be:
\begin{chapel}
       extern proc foo();
\end{chapel}

To refer to the return value of a C function, its type must be supplied through
a \sntx{return-type} clause in the prototype.\footnote{The return type cannot be
inferred, since an \chpl{extern} procedure declaration has no body.}

If the above function returns a C \chpl{double}, it would be declared as:
\begin{chapel}
       extern proc foo(): real;
\end{chapel}
Similarly, for external functions that expect arguments, the types of those
arguments types may be declared in Chapel using explicit argument type specifiers.

The types of function arguments may be omitted from the external procedure
declaration, in which case they are inferred based on the Chapel callsite.
For example, the Chapel code
\begin{chapel}
       extern proc foo(x: int, y): real;
       var a, b: int;
       foo(a, b);
\end{chapel}
\noindent
would imply that the external function foo takes two 64-bit integer values
and returns a 64-bit real.  External function declarations with omitted type
arguments can also be used call external C macros.

External function arguments can be declared using the \sntx{default-expression}
syntax.  In this case, the default argument will be supplied by the Chapel
compiler if the corresponding actual argument is omitted at the callsite.  For example:
\begin{chapel}
       extern proc foo(x: int, y = 1.2): real;
       foo(0);
\end{chapel}
Would cause external function foo() to be invoked with the arguments 0
and 1.2.

C varargs functions can be declared using
Chapel's \sntx{variable-argument-expression} syntax (\chpl{...}).  For example,
the C \chpl{printf} function can be declared in Chapel as
\begin{chapel}
       extern proc printf(fmt: c_string, vals...?numvals): int;
\end{chapel}

External C functions or macros that accept type arguments can also be
prototyped in Chapel by declaring the argument as a type.  For
example:
\begin{chapel}
       extern foo(type t);
\end{chapel}
Calling such a routine with a Chapel type will cause the type
identifier (e.g., 'int') to be passed to the routine.\footnote{In practice,
this will typically only be useful if the external function is a macro
or built-in that can handle type identifiers.}

\subsection{Calling Chapel Procedures Externally}
\label{Calling_Chapel_Procedures_Externally}
\index{interoperability!Chapel procedures!calling}

To call a Chapel procedure from external code, the procedure name must be
exported using the \chpl{export} keyword.  An exported procedure taking no
arguments and returning void can be declared as:
\begin{chapel}
export proc foo();
\end{chapel}
If the procedure body is omitted, the procedure declaration is a prototype; the
body of the procedure must be supplied elsewhere.  In a prototype, the return
type must be declared; otherwise, it is assumed to be \chpl{void}.  If the body
is supplied, the return type of the exported procedure is inferred from the
type of its return expression(s).  

If the optional \sntx{external-name} is supplied, that is the name used in
linking with external code.  For example, if we declare
\begin{chapel}
export "myModule_foo" proc foo();
\end{chapel}
\noindent
then the name \chpl{foo} is used to refer to the procedure within chapel code,
whereas a call to the same function from C code would appear
as \chpl{myModule_foo();}.  If the external name is omitted, then its internal
name is also used externally.

When a procedure is exported, all of the types and functions on which it depends
are also exported.  Iterators cannot be explicitly exported.  However, they are
inlined in Chapel code which uses them, so they are exported in effect.

\subsection{Argument Passing}
\label{Interop_Argument_Passing}
\index{interoperability!argument passing}

The manner in which arguments are passed to an external function can be
controlled using argument intents.  The following table shows the correspondence
between Chapel intents and C argument type declarations.  These correspondences
pertain to both imported and exported function signatures.

\begin{tabular}{rl}
Chapel & C \\
\hline
\tt T & \tt const T \\
\tt in T & \tt T \\
\tt inout T & \tt T* \\
\tt out T & \tt T* \\
\tt ref T & \tt T* \\
\tt param & \tt \\
\tt type & \tt char*\\
\end{tabular}

Currently, \chpl{param} arguments are not allowed in an extern function
declaration, and \chpl{type} args are passed as a string containing the name of
the actual type being passed.  Note that the level of indirection is changed
when passing arguments to a C function using \chpl{inout}, \chpl{out},
or \chpl{ref} intent.  The C code implementing that function must dereference
the argument to extract its value.

\cleardoublepage
\appendix
\input{Syntax}
\cleardoublepage
\markboth{Chapel Language Specification}{Index}
\documentclass[10pt,twoside,titlepage]{../../spec}
\usepackage{amsmath}
\usepackage{amssymb}
\usepackage{color}
\usepackage{times}
%\usepackage{fullpage}
\usepackage{graphicx}
\usepackage{listings}
\usepackage{longtable}
\usepackage[nottoc]{tocbibind}
\usepackage{multirow}
%
% These are special environments for adding extra information about
% code snippets which can be later extracted and used to generate test
% codes for automated testing.
%
% During LaTeX compilation, the environments defined in this file throw
% away all text within the scope of the environment, with the
% exception of 'chapelprintoutput' which prints the output (and is
% also extracted for testing purposes).
%
% Usage:
%
% - chapelexample (REQUIRED) {f.chpl}
%   This marks the start of a test.  This environment requires a
%   single argument that is the name of the Chapel test program.  This
%   filename will appear in the spec.
%
% - chapelpre
%   Any Chapel code in this scope is put *before* the code in the
%   chapel|chapelcode scope.
%
% - chapelcode|chapel
%   This is the part of the code that is in the spec.
%
% - chapelnoprint
%   This is the part of the code that goes in the test with chapelcode
%   and chapel, but does not appear in the spec.
%   
% - chapelpost
%   Any Chapel code in this scope is put *after* the code in the
%   chapel|chapelcode scope.
%
% - chapelfuture
% - chapelcompopts
% - chapelexecopts
%   The lines in these scopes are put directly into the appropriate file.
%
% - chapeloutput|chapelprintoutput (REQUIRED)
%   These environment provide the test output (.good files).  There can be
%   multiple such environments, and the filename is specified by a LaTeX
%   style comment preceding the contents of the output.  The
%   'chapelprintoutput' scope is also outputted in the spec itself and
%   thus may contain LaTeX formatting (see GENERAL CAVEATS below)
%   
% - chapelwideoutput
%   Provides the test output for no-local tests, if that differs from the
%   normal test output.  The content of this environment is dumped into a
%   <test>.no-local.good file, along with a copy of the content of 
%   chplprintoutput.
%

%
% GENERAL CAVEATS:
%
% - Because the chapelprintoutput environment must used LaTeX
%   formatting, the script that extracts the tests must removed any
%   LaTeX specific formatting.
%
% - Using a backslash or other special LaTeX characters may also be
%   needed (e.g., \_ or \#) in the other environments for LaTex
%   parsing purposes.  Such characters are considered fragile and may
%   lead to unexpected results.
%

%
% Gobble up the text in this new box.  The text in each environment is
% dropped on the floor during LaTeX compilation.
%
\newsavebox{\teststuff}

%
% Any additional lines needed for the code snippet to run/compile
% (before and after the chapel code segment)
%
\newenvironment{chapelpre} {\begin{lrbox}{\teststuff}
\begin{minipage}{6in}}
{\end{minipage}\end{lrbox}}

\newenvironment{chapelnoprint} {\begin{lrbox}{\teststuff}
\begin{minipage}{6in}}
{\end{minipage}\end{lrbox}}

\newenvironment{chapelpost} {\begin{lrbox}{\teststuff}
\begin{minipage}{6in}}
{\end{minipage}\end{lrbox}}


%
% .future file
%
\newenvironment{chapelfuture} {\begin{lrbox}{\teststuff}
\begin{minipage}{6in}}
{\end{minipage}\end{lrbox}}

%
% .compopts file
%
\newenvironment{chapelcompopts} {\begin{lrbox}{\teststuff}
\begin{minipage}{6in}}
{\end{minipage}\end{lrbox}}

%
% .execopts file
%
\newenvironment{chapelexecopts} {\begin{lrbox}{\teststuff}
\begin{minipage}{6in}}
{\end{minipage}\end{lrbox}}


%
% .good file
% To get more than one file, use a LaTeX style comment to name the
% .good file
%
\newenvironment{chapeloutput} {\begin{lrbox}{\teststuff}
\begin{minipage}{6in}}
{\end{minipage}\end{lrbox}}

%
% .no-local.good file
% (The naming feature mentioned above does not yet work, so this
% environment is a Q&D way to get a .no-local.good file.)
%
\newenvironment{chapelwideoutput} {\begin{lrbox}{\teststuff}
\begin{minipage}{6in}}
{\end{minipage}\end{lrbox}}

%
% .prediff file
%
\newenvironment{chapelprediff} {\begin{lrbox}{\teststuff}
\begin{minipage}{6in}}
{\end{minipage}\end{lrbox}}

%
% .good file that is printed in the text of the Spec
% To get more than one file, use a LaTeX style comment to name the
% .good file
%
%\lstnewenvironment{chapelprintoutput} 
% (See chapel_listing.tex for the implementation.)

\lstdefinelanguage{chapel}
  {
    morekeywords={
      align, as, atomic,
      begin, bool, break, by,
      class, cobegin, coforall, complex, config, const, continue,
      delete, dmapped, do, domain,
      else, enum, except, export, extern,
      false, for, forall,
      if, imag, in, index, inline, inout, int, iter,
      label, lambda, let, local, locale,
      module,
      new, nil, noinit,
      on, only, opaque, otherwise, out,
      param, private, proc, public,
      range, real, record, reduce, ref, require, return,
      scan, select, serial, single, sparse, string, subdomain, sync,
      then, true, type,
      uint, union, use,
      var,
      when, where, while, with,
      yield,
      zip
    },
    sensitive=false,
    mathescape=true,
    morecomment=[l]{//},
    morecomment=[s]{/*}{*/},
    morestring=[b]",
}

\lstset{
    basicstyle=\footnotesize\ttfamily,
    keywordstyle=\bfseries,
    commentstyle=\em,
    showstringspaces=false,
    flexiblecolumns=false,
    numbers=left,
    numbersep=5pt,
    numberstyle=\tiny,
    numberblanklines=false,
    stepnumber=0,
    escapeinside={(*}{*)},
    language=chapel,
  }

%\newcommand{\chpl}[1]{\lstinline[language=chapel,basicstyle=\ttfamily,keywordstyle=\bfseries]!#1!}
\newcommand{\chpl}[1]{\lstinline[language=chapel,basicstyle=\small\ttfamily,keywordstyle=]!#1!}
\newcommand{\varname}[1]{\emph{#1}}
\newcommand{\typename}[1]{\emph{#1}}
\newcommand{\fnname}[1]{\chpl{#1}}

\lstnewenvironment{chapel}{\lstset{language=chapel,xleftmargin=2pc,stepnumber=0}}{}
\lstnewenvironment{invisible}{\lstset{language=chapel,xleftmargin=2pc,stepnumber=0,keywordstyle=\bfseries\color{white},basicstyle=\small\ttfamily\color{white}}}{}
\lstnewenvironment{chapel0}{\lstset{language=chapel,stepnumber=0}}{}

\lstnewenvironment{numberedchapel}{\lstset{language=chapel,xleftmargin=15pt,stepnumber=1}}{}

\lstnewenvironment{chapelcode}{\lstset{language=chapel,stepnumber=1}}{}

% Uses the same listing style as the {chapel} environment, but keyword
% formatting is turned off.  The argument is ignored in LaTeX
% but used to name the .good file during test extraction.
% The argument must be supplied but may be empty.
% If empty it defaults to null, which signals the test extractor to 
% autogenerate the .good file name as ``<test_name>.good''.
\lstnewenvironment{chapelprintoutput}[1]
  {\lstset{language=chapel,xleftmargin=2pc,stepnumber=0,keywordstyle=}}{}

\lstnewenvironment{commandline}{\lstset{keywordstyle=,xleftmargin=2pc}}{}

\lstnewenvironment{protohead}{\lstset{language=chapel,xleftmargin=0pc,belowskip=-10pt,stepnumber=0}}{}

\newenvironment{protobody}{\begin{description}\item[\quad\quad] }{\end{description}}

\input{../../syntax_listing}

%% High section numbers require different number widths
\usepackage[titles]{tocloft}
\setlength{\cftchapnumwidth}{1.3em}
\setlength{\cftsecnumwidth}{2.6em}
\setlength{\cftsubsecnumwidth}{3.9em}
\setlength{\cftsubsubsecnumwidth}{5.2em}
\setlength{\cftsubsecindent}{1.7em}
\setlength{\cftsubsubsecindent}{4.3em}

\usepackage{ifpdf}
\ifpdf
\usepackage[pdftex,
            bookmarks,
            plainpages=false,
            breaklinks,
            pdftitle={Chapel Language Specification},
            pdfauthor={Cray Inc, 901 Fifth Avenue Suite 1000, Seattle, WA 98164},
            pdfsubject={Chapel High Productivity Language}
           ]{hyperref}
\else
\usepackage[ps2pdf]{hyperref}
\fi

\newcommand{\ie}{\emph{i.e.}}
\newcommand{\eg}{\emph{e.g.}}

\newenvironment{TODO} {
\begin{quote}
{\it TODO:}
}{
\end{quote}
}

\newenvironment{example}{
\begin{quote}
{\it Example}.
}{
\end{quote}
}

\newenvironment{chapelexample}[1]{
\begin{quote}
{\it Example (#1)}.
}{
\end{quote}
}

\newenvironment{note}{
\begin{quote}
{\it Implementors' note}.
}{
\end{quote}
}

\newenvironment{rationale}{
\begin{quote}
{\it Rationale}.
}{
\end{quote}
}

\newenvironment{openissue}{
\begin{quote}
{\it Open issue}.
}{
\end{quote}
}

\newenvironment{future}{
\begin{quote}
{\it Future}.
}{
\end{quote}
}

\newenvironment{craychapel}{
\begin{quote}
{\it Cray's Chapel Implementation}.
}{
\end{quote}
}

\newenvironment{suggestionbox}{
\begin{quote}
{\it Suggestions?}
}{
\end{quote}
}

\newcommand{\rsec}[1]
           {\S\ref{#1}}

% courtesy: http://www.iam.ubc.ca/~newbury/tex/page-set-up.html
\newcommand{\sekshun}[1]
           {
             \chapter{#1}
             \markboth{Chapel Language Specification}{#1}
           }

\oddsidemargin 0.5in
\evensidemargin 0.0in
\textwidth 6in
\headheight 0.2in
\topmargin 0in
\headsep 0.3in
\textheight 8.5in

\title{Chapel Language Specification\\Version 0.91}

\author{Cray Inc\\
901 Fifth Avenue, Suite 1000\\
Seattle, WA 98164}

\date{May 21, 2012}

\setcounter{secnumdepth}{3}
\setcounter{tocdepth}{3}

\begin{document}

\pagestyle{myheadings}
\markboth{Chapel Language Specification}{Chapel Language Specification}
%\pagenumbering{roman}

\setlength{\parindent}{0in}
\setlength{\parskip}{4mm plus2mm minus1mm}

\sekshun{Special Functions}

Special functions are functions for which the compiler provides default definitions.  
Together, these special functions support
value-initialization, default-construction,
copy-construction, destruction, assignment, fieldwise-construction and equality comparisons.
The first four of these are required by the Chapel
compiler.  The remainder are provided as a convenience
to the user.  The compiler-generated version of a special function may always be
overridden by an explicity user-defined version.

Of the special functions required by the compiler, all but assignment are associated with
a state transition in the object lifecycle.  These associations are detailed in the
following subsection.  The remaining sections describe each of the special functions in
turn.  Each section details the interface, semantics and visibility rules for each
function.  For the required special functions, language elements that cause
the compiler to generate a call to that special function are described.


\subsection{The Object Lifecycle}

In general, the states an object can be in are:
\begin{enumerate}
\item Undefined
\item Uninitialized
\item Value-Initialized
\item Field-Initialized
\item Fully-Initialized (a.k.a. Constructed)
\item Destroyed
\item Reclaimed
\end{enumerate}

Before an object has been allocated, it is in the ``undefined'' state.  This is
equivalent to representing its storage with a null pointer; the object {\it per
se} does not exist.

Once memory sufficient to represent the object has been allocated, the object
moves to the ``uninitialized'' state.  The name of the object (or the class
variable) refers to actual memory, but that memory is in an unknown state.

Value-initialization moves the object to the ``value-initialized'' state.  Each
field within an aggregate object is in an initial state consistent with its type,
but neither the fields nor the object as a whole have been ``constructed''.
Value-initialization may be applied knowing only the type of the object --- or
just the types of the fundamental leaves of an aggregate type.  In particular,
neither the constructors nor the field initializers are consulted in creating a
value-initialized object.\footnote{This can be stated more simply using just the
  restriction against consulting constructors, since field-initialization
  depends upon construction.}

Field-initialization consists of bringing each field in an aggregate type into a
state consistent with the field declarations.  If the field is declared without
a specified initializer, then the default constructor (for the type of that
field) is called; otherwise, a constructor for the type of that field is called,
passing the initialization expression for that field as its operand.

Following field-initialization, the actions specified in the body of the
constructor are called.  In many cases, the body of the constructor is empty,
because the actions of value- and field-initialization bring the object into a
state consistent with the desired class invariants.  In any case, after the body
of one of its constructors has run to completeion, the object is
fully-initialized (i.e. constructed).

When an object is destroyed it transitions to the destroyed state.  This state
is equivalent to ``uninitialized'', meaning that its contents cannot be relied
upon.  An implementation may reuse a destroyed object without first returning it
to the heap through deallocation.\footnote{After its destruction, an object may
  still be accessible through the name or reference used to call the
  destructor.  But since its contents cannot be relied upon, it represents a
  programming error to attempt to read or write an object in this state.}

Once an object has been destroyed, an implementation may reuse or reclaim it.
This may happen immediately or at some unspecified future time.  After it has
been reclaimed, the object returns to the undefined state.



\section{Value Initialization}

Value-initialization is provided by the \chpl{_defaultOf()} function.
This function is called wherever the compiler needs to provide value-initialization
for an instance of the corresponding type.  

Value-initialization brings the object into a known initial state before
construction is applied.  For fundamental types (including numerics, strings,
enums and ranges) as well as sync/single variables and atomics, this is the
value specified in section 8.1.1 [Default Initialization].  The remaining type
classes (classes, records, arrays and tuples) will be handled specifically below.
% TODO: Default initialziation as discussed in that section should actually be
% interpreted as default construction.  Default construction is
% value-initialization followed by the actions specified in the constructor
% body.  But since the body of the constructor for every fundamental type is
% empty, the distinction between value-initialization and default-initialization
% is moot.
% The TODO involves separating the concepts of value-initialization and
% default-construction.  Default-initialization is then naturally a combination
% of the two.
For records, arrays and tuples, value-initialization consists of
value-initialization applied element-wise (recursively).  For class variables,
value-initialization sets the variable equal to \chpl{nil}.

Value-initialization is most important in connection with externally defined
types.  It provides the means to bring an object associated with an external
type to a value-initialized state.  Other languages sometimes specify this value
for some or all types; \chpl{_defaultOf} supplies a hook, so the authors of
libraries supporting interoperability can satisfy the external language's
value-initialization expectations.

Although the language currently allows it, overriding the definition
of \chpl{_defaultOf} for fundamental types is strongly discouraged.  For
example, redefining the value returned by
\begin{chapel}
proc _defaultOf(type t) : t where t==bool(?w) ;
\end{chapel}
from \chpl{false} to \chpl{true} would change the initial value of a
Boolean program-wide, which would then break any code that uses the value of any
Boolean variable that is not explicitly initialized.

In terms of the object lifecycle, value-initialization performs the second step.  

\subsection{Interface}

The \chpl{_defaultOf} function has
the following signature.
\begin{chapel}
proc _defaultOf(type t) : t ;
\end{chapel}

\begin{future}

Currently, the interface to \chpl{_defaultOf()} takes a single type argument and
returns an instance that is properly value-initialized.  This means that
creation of an initialized object in general involves either assignment or
copy-construction.  Otherwise, the construction of an object cannot be described
in terms of the language itself.

In the future, it is desired to restate the interface to \chpl{_defaultOf()} as
a member function.  This will permit objects to be initialized in-place, which
will simplify the generated code and improve run-time efficiency.

Rendering the \chpl{_defaultOf()} function in method form will also support
calling \chpl{_defaultOf()} before entering the constructor, rather than calling
it as part of the constructor.  This delineation is important, because the
recursive definitions of \chpl{_defaultOf()} and the compiler-supplied
constructors can then be kept distinct (as opposed to being mutually dependent).

\end{future}

\subsection{Semantics}

Value initializtion is more fundamental than construction.  It is equivalent to
the language concept of default initialization (also known as guaranteed
initialization).  

Value-initialization and construction of an object take place in sequence.  When
an object is created, it starts out as uninitialized storage.  Value
initialization places the object (or each field of an aggregate object,
recursively) into a known initial state.  Implicitly, it is possible to
determine the initial value for each field in object based solely on its type, whereas
construction may establish relationships between fields that require treating
the object as a whole (and may involve external knowledge as well).

\begin{note}
At present, the order in which value-initialization is carried out among the
fields is unspecified.  This is intended to give an implementation considerable
flexibility in implementing value-initialization.

Unless overridden, value-initialization can most-often be found (at compile
time) to be equivalent to zero-initialization.  If this is the case, the
provision for value-initialization can be satisfied by emitting the equivalent
of a \chpl{memset(var, 0, sizeof(var));} or performing the object allocation
using \chpl{calloc()}.
\end{note}

\subsection{Visibility}

The \chpl{_defaultOf()} function must be invoked with an operand that evaluates
to a concrete type $T$.  Generic instantiation rules are applied to all
user-defined definitions of \chpl{_defaultOf()} with compatible
signatures\footnote{That is, functions named \chpl{_defaultOf} that can accept a type
  argument as their first argument and whose other arguments all have default
  values associated with them.} to generate a candidate set.  If the candidate
set is non-empty, normal resolution rules are applied to select the best
candidate.

\subsection{Usage}

To enforce value-initialization within a constructor, a call to \chpl{_defaultOf()} is
inserted by the compiler at the start of every (user-supplied as well as
compiler-generated) constructor.  The \chpl{_defaultOf()} function should not be
invoked directly from user code.

\begin{rationale}

The \chpl{_defaultOf()} function may assume that its operand is not yet
value-initialized.  This condition can only occur within a constructor.  But since
\chpl{_defaultOf()} is automatically invoked at (or preferably, before) the
beginning of every constructor, it would be pointless to invoke it explicitly in
that context. Fields in an aggregate type being constructed have also already been
value-initialized through recursive calls to \chpl{_defaultOf()}.

\end{rationale}



\section{Default Construction}

Default construction for a given type is provided by any constructor compatible
with that type that can be called with an empty argument list (i.e. ``no
arguments'' or ``zero arguments'').  In particular, a generic constructor may be
compatible with many concrete types, and a constructor with one or more formals
may be called with no arguments so long as each argument has an associated
default value.

\subsection{Interface}

An explicit default constructor for a type $T$ has the following signature.
\begin{chapel}
proc T.T();
\end{chapel}
When the compiler supplies a default constructor, it will have exactly this
signature.  However, as explained above, any user-defined signature that can be
called with an empty argument list will serve as a default constructor.

\subsection{Semantics}

The intended semantics of the default constructor are to take an object from the
value-initialized state to a fully-constructed state consistent with the
corresponding type.  For a user-defined type, this typically means adjusting the
initialization of the object to satisfy the class (or record) invariants.

Construction proceeds in two phases: field-initialization and
instance-initialization.  Field-initialization moves the object from the
value-initialized state to the field-initialized state.  This is done by
recursively applying either the default constructor or the copy constructor to
each contained field, depending on whether the field is declared without an
initializer or with an initializer, respectively.  Instance-initialization
creates further relationships between the fields in an object, to establish the
invariants that are common to all fully-initialized objects in that class.
Details on the syntax and semantics of user-defined constructors are provided in
Section 15.3 [Class Constructors].

To remain consistent with current behavior, field-initialization is performed
sequentially, in the order that the variables are declared in the class or
record definition.  This naturally satisfies the author's expectations, by
analogy with the order of initialization of successive declarations 
in a procedure definition.

When present, the compiler-generated constructor for a class or record type
consists of field-initialization followed by an empty constructor body.  That
is, instance-initialization is trivial.
\footnote{At present, there is no way to specify that the compiler should not generate a
default constructor if no user-callable version is provided.  Thus, in order to
prevent the creation of a default-valued object, the user must currently define
a default constructor that generates a compile-time error.}
This choice is reasonable for the
compiler's default behavior, since the compiler cannot infer any relationships
among the fields based solely on their types.  If such relationships
(invariants) are to be established, they must be provided explicitly by the
class author.

\begin{note}

Following field-initialization, each field in an aggregate object is considered
to be initialized.  Therefore, updates to these fields in the body of a
constructor are performed using assignment.

\end{note}

\subsection{Visibility}

If no user-defined constructors are supplied for a given type, then the compiler
provides an all-fields constructor as detailed below.  Because the all-fields
constructor may be invoked with an empty argument list, it implements the
default constructor in this case.

If the user-defined type defines at least one constructor, then the compiler
will provide a default constructor (as defined above).  During resolution of a
constructor called with no arguments, the user-defined constructors for that
concrete type will be considered first (including ones contributed from
compatible generic types).  If no matching constructor is found in that set,
then the compiler-generated default constructor is considered.

\subsection{Usage}

The compiler implicitly invokes the default constructor when a variable is
declared without an explicit initializer.  Also, it is used in the recursive
definition of the default constructor for an aggregate type.



\section{Copy Construction}

Copy construction for a given type is provided by any constructor compatible
with that type that can be called with an argument list consisting of a single
object of that type.  The selected definition may involve generic types for both
the object and the operand.  The definition may also involve other arguments, so
long as all but one of the arguments has a default value associated with it, and
the type of the remaining formal argument can bind to an actual argumet of the
object type.

\subsection{Interface}

The copy constructor for a type $T$ has the following signature
\begin{chapel}
proc T.T(const ref T);
\end{chapel}
When the compiler supplies a copy constructor, it will have exactly this
signature.  However, as explained above, any user-defined signature that can be
called with an argument list containing just one argument coercible to the
object type will serve.

\subsection{Semantics}

The intended semantics of the copy constructor are to create a fully-constructed
object that is a verbatim copy of the operand.  The compiler uses copy
construction to implement pass-by-value and return-by-value semantics.  This
corresponds to formal arguments whose concrete intent resolves to \chpl{in}, and
to records and other value types when they are returned by value.\footnote{The
return intent \chpl{ref} causes the return value to be returned by reference.}

As with default construction, copy construction proceeds in two phases:
field-initialization followed by instance-initialization.  Field initialization
is exactly the same as for default construction described above.  Instance
initialization then provides the copy semantics and provides the invariants
required by the class design.  As with the body of the default constructor
updates to fields within the object are treated as assignments (initialization
already having been performed during field-initialization).

When present, the compiler-generated copy constructor for a class or record type
performs a field-by-field copy of the fields in the operand into the
corresponding fields in the object being initialized.  

\begin{note}
It may be observed that the compiler-generated copy constructor overwrites every
field, even though they have already been initialized during
field-initialization.  So long as the semantics remain unchanged, the compiler
may skip field-initialization and use copy-initialization instead to establish
the value of each field.

This optimization cannot be applied in the general case, where the behavior of
default-initialization, copy-initialization or assignment has been overridden by
the user for the type of a given field.  But it may safely be applied where
assignment and the default constructor (for default-initialization) or the copy
constructor (for copy-initialization) have been supplied by the compiler for the
corresponding field type.
\end{note}

\subsection{Visibility}

The compiler provides copy constructor for every type, defined as specified
above.  During resolution of a constructor call, the compiler-supplied copy
constructor is considered only if there are no candidates among the user-defined
constructors applicable to that object type.

\subsection{Usage}

The compiler implicitly invokes the copy constructor to initialize a formal
argument with in intent from the corresponding actual argument.  It is also used
to initialize the returned value temporary variable (in the calling context)
from the return value expression in a called routine.



\section{All-Fields Construction}

The all-fields constructor is provided by the compiler as a convenience.  It
supports rapid prototyping by allowing client code to specify the initial states
of fields within an object being constructed, without requiring the class author
to supply this capability explicitly.

The all-fields constructor is provided only if no user-specified constructors
have been defined for that type.  

\begin{rationale}

It is a useful prototyping feature to be able to structure data (as in a C
\chpl{struct}) without having to write a constructor and accessor functions to
support its initialization and use.

On the other hand, allowing client code direct access to the fields in a class
or record defeats encapsulation.  The compromise is to provide the all-fields
constructor when no user-defined constructors are present, but to hide this
after any user-defined constructor is provided.  The operative assumption is
that a user who is adding constructors to a class will be aware of the
all-fields constructor and the conditions under which it is provided.

\end{rationale}

\subsection{Interface}

For a class or record type, the signature of the all-fields constructor contains
one formal argument for each field in the aggregate, arranged in declaration
order.  The name of each formal is exactly the same as the name of the
corresponding field.  Each formal is provided with a default value in the same
manner as for field-initialization.  That is, default construction provides the
initial value for formals whose fields are declared without an initializer, and
copy construction provides the initial value for formals whose fields are
declared with an initializer. 

\subsection{Semantics}

All-fields construction proceeds in two phases: field-initialization followed by
instance-initialization.  Field-initialization is provided as described above.
Instance initialization consists of overwriting each field with the value of the
corresponding formal.

\begin{note}

As with the copy constructor, it can be observed that field-initialization is
essentially moot, because the semantics of the all-fields constructor cause each
field to be overwritten in the body of the constructor.

Similarly, if assignment and copy construction or default construction
(corresponding to whether the associated field has an initializer or not) are
not overridden by the user, the difference between performing field
initialization or not for that field will not be observable by the user.  Under
those conditions, the initialization of that field may be elided.

\end{note}

\subsection{Visibility}

The all-fields constructor is visible only if no user-defined constructors have
been provided for the corresponding class or record type.

\subsection{Usage}

The compiler does not generate any calls to the all-fields constructor.



\section{Assignment}

Assignment is provided by the procedure named ``='' whose left and right
operands are compatible with the object type.  Assignment is used to replace the
value of a field or variable that has already been initialized.  It is used by
the compiler to update the value of a field in the body of a copy constructor or
the all-fields constructor.  It is also used to update the value of the return
value temporary prior to its return to the caller.

\begin{future}

In addition to the two-argument form, it is desirable to support assignment as a
method.  Given that the canonical form for assignment has been updated to pass
the left operand by reference, the two forms are equivalent in behavior and
efficiency.

The return value variable is useful in determining the type of the return value
of a routine where that is unspecified.  However, by using the semantics that a
return statement directly calls a copy-constructor to transfer the result to the
calling context would eliminate an unnecessary assignment from the semantics of
that statement.

\end{future}

\subsection{Interface}

The canonical two-argument form for assignment is:
\begin{chapel}
proc =(ref lhs, rhs) : void ;
\end{chapel}

\subsection{Semantics}

Excluding class variables, the semantics of assignment are to replace the
contents of the lhs operand with the contents of the rhs operand.  For class
variables, the class instance reference held by the lhs operand is replaced by a
copy of the reference held by the rhs operand.  The object referenced by the rhs
thus becomes shared through the lhs, and the object formerly accessible through
the lhs operand is no longer accessible through that name (though it may be
accessible through other paths).

For fundamental types (excluding strings),
assignment can be accomplished by a bitwise copy.  
Depending on how strings are implemented, a bitwise copy may or may not suffice.

When provided, the compiler-supplied version of record assignment consists of a
(recursive) field-wise assignment from the fields in the right operand to the
fields in the left operand.

\begin{note}

As with the copy and all-fields constructors, it can be observed that a bitwise
copy will suffice as long as none of the fields (recursively) have types that
have user-defined assignment operators associated with them.  The definition of
the compiler-generated record assignment operator can be seen to be equivalent
ot a bitwise copy.

\end{note}

\subsection{Visibility}

A compiler-generated assignment operator is provided for every type except class
types.  However, a compatible user-defined assignment operator will always
override the compiler-generated one.

\subsection{Usage}

The compiler currently generates assignments to implement the bodies of the copy
constructor and all-fields constructor.  Also, as mentioned above, it is used
to establish the value of the return value temporary, prior to its return to the
caller.



\section{Destruction}

Destruction is provided by the destructor.  Its purpose is to move a
fully-constructed object into the destroyed state.  Before transitioning to this
state, an object should release all resources dynamically attached to it during
the course of execution.

Once an object is destroyed, it is available for reclamation, so its contents
should not be relied upon.  It is a programming error to read or write an object
that has already been destroyed.

\subsection{Interface}

The destructor for a type \chpl{T} is a method that takes no arguments and has
the name \chpl{\~T}.

\begin{chapel}
proc T.~T();
\end{chapel}

\begin{openissue}
Polymorphic behavior for destructors is not yet specified, but is essential for
supporting inhomogeneous containers.  It may be sufficient to specify that
destructors are always dynamically dispatched, though this carries some
unnecessary overhead when the destructor can be statically bound.
\end{openissue}

\subsection{Semantics}

The semantics of the destructor are to release any resources owned by the object
before the object itself is release.  Once released, the object may (though need
not necessarily) be reclaimed.  Reclamation involves placing the memory formerly
allocated to the object back on the heap (for dynamically allocated objects).
Reclamation and reuse of the memory happens automatically when the object is
allocated on the stack.

Since the compiler cannot know about additional resources allocated to the
object during execution, the compiler-generated destructor is trivial.

\subsection{Visibility}

The compiler-generated destructor is supplied for every type.  However during
resolution, if a compatible user-defined destructor is available, it is always
selected in preference to the compiler-generated one.

\subsection{Usage}

The compiler calls the destructor for non-class objects when execution exits the
scope in which the name bound to each such object is defined.  Variables defined
at global (module) scope are destroyed before the program terminates.  For class
objects, the destructor is invoked explicitly using the \chpl{delete} operator.



% TODO: This section is at least partially redundant with the section 15.4.
% Probably, the scope of this appendix should be reduced to UMM-related functions and any
% new material here merged into section 15.4..
\section{Field Accessors}

For each field in an aggregate type, the compiler provides a field accessor
function.  For read-only fields, the accessor returns the value of that field.

For read/write fields, the accessor also possesses a hidden param bool argument
named \chpl{setter}.  When the corresponding hidden actual is \chpl{true}, the function
returns a reference to the field that can be updated.  When the corresponding hidden
actual is \chpl{false}, the function returns the value of that field.

Read-only fields are fields that are declared using the \chpl{param} or \chpl{type}
keyword.  The implicit base class member \chpl{super} in a derived class or record type is
also a read-only field.

\subsection{Interface}

All field accessor functions are parentheses-less functions, meaning that they must be
called without an argument list (not even an empty one).  The signature of a field
accessor for a read-only field is:
\begin{chapel}
proc ro_field : t ;
\end{chapel}
\noindent
The signature of a field accessor function for a read/write field is equivalent to:
\begin{chapel}
pragma "no parens"
proc rw_field(param setter : bool) : ref t ;
\end{chapel}
\noindent
In both signatures, the return type \chpl{t} is the same as the type of the named field.

When calling the accessor for a read-only field, or the accessor for a read-write field in
an rvalue context, it is called using the normal parentheses-less form.  In that context,
the reference result is implicitly dereferenced to yield a value result.  For example, the
code
\begin{chapel}
record R { var a = 3 ; }
writeln(r.a);
\end{chapel}
\noindent
will print out a \chpl{3}.

When calling the accessor for a read-write field in an lvalue context, the compiler
supplies an implicit \chpl{setter} argument with the value \chpl{true}.  For example,
\begin{chapel}
r.a = 7;
\end{chapel}
\noindent
is implicitly converted to
\begin{chapel}
r.a(setter=true) = 7;
\end{chapel}

\subsection{Semantics}

A getter returns the field of the same name.  If the field is a param, it is
returned as a param.  If the field is a type field, it is returned as a type.
If the field is the superclass object, it is returned by value (i.e. the object
itself, rather than a reference to that object).  Otherwise, the field is
returned by reference.  In this last case, if the reference is used in an rvalue context,
it is implicitly dereferenced to yield a value (i.e. an rvalue).

\subsection{Visibility}

For each field in an aggregate type (class, record or union), the compiler supplies a
default version of the field accessor function.  The semantics of this function are to
always return the value of the named field (for read-only fields) or a reference to the
named field (for read-write fields).  In the default version for read-write fields,
the \chpl{setter} param argument is ignored.

The compiler-supplied version can be overridden by a user-defined version.  In the case of
a read-write field, the \chpl{setter} argument can be used to perform different actions
depending on whether the field is to be written or read.

\subsection{Usage}

The compiler implicitly calls the getter when a field in an aggregate is read,
and implicitly calls the setter when a field is written.  That is, when used in
an rvalue context, the expression \chpl{c.a} is translated internally into
a call to
\begin{chapel}
proc c.a : ref a.type where setter==false;
\end{chapel}
and when used in an lvalue context is translated internally to a call to
\begin{chapel}
proc c.a : ref a.type where setter==true;
\end{chapel}



\section{Casts}

Casts alter the type of an object, and where necessary change its representation as well.
They are called implicitly by the compiler to implement coercions.

\subsection{Interface}

A cast is a procedure that takes a type argument and a value argument.  The return value
is the input value cast to the specified type.
\begin{chapel}
proc _cast(type t, x) : t ;
\end{chapel}

\subsection{Semantics}

The \chpl{_cast} function performs whatever transformations are necessary to convert the
input value \chpl{x} to the specified target type.  The return value is of the target
type.  Casts may involve a loss of information (down-cast) or the insertion of default
information (up-cast).  In either case, since the result needs to be a valid object of
\chpl{type t}, the actions performed by an up-cast are closely associated with the target
\chpl{type t}.  If a down-cast is the reverse of an up-cast, it may also be most
appropriately associated with the definition of \chpl{type t}.

\subsection{Visibility}

The compiler provides default implementations of the \chpl{_cast} function for:
\begin{itemize}
\item Converting integral types to enumerated types.
\item Converting a string to an enumerated type.
\item Converting a enumerator into a string.
\item Converting a record to one of its subtypes.
\end{itemize}

The functions for casting from an integer or string into an enumerated type may be
overridden by a user-defined version.  At present, the record cast function cannot be
overridden, nor can the function for converting an enumerator to a string.

\subsection{Usage}

The compiler calls the integral type to enumerated cast when an integer is bound to a
formal argument with an enumeration type (including the RHS argument of an assignment).
The compiler call the string to enumerated cast when converting a string (e.g. from input)
to an enumerated type.  
The compiler calls the enumerator to string cast when converting
an enumerator to a string (either internally or in formatted I/O).  
The compiler calls the record cast function when a record is coerced to one of its
subtypes.



\section{Hash Functions}

Record hash functions provide a default hashing function for records.  It is used in the
implementation of the DefaultAssociative domain to create a hash address from a key that
is a record.

\subsection{Interface}

The default hash function has the signature:
\begin{chapel}
proc chpl__defaultHash(x) : int(64) ;
\end{chapel}

\subsection{Semantics}

The default hash function produces a 64-bit hash value based on the contents of the given
record.  The hash value is not guaranteed to be unique.  That is, once a hash bucket is
located, elements in that bucket must still be compared for an exact match with the key.

\subsection{Visibility}

The compiler-generated default hash function cannot currently be overridden.

\subsection{Usage}

This function is provided as a convenience to the implementation of the DefaultAssociative
domain type.  It is used for generating hash addresses from keys to support hashed
insertion and lookup.
% Note: Since a field iterator is now available, the compiler-generated version of this
% could probably be replaced by a generic version in module code.


\section{Record Equality}

The compiler provides record equality \chpl{==} and inequalty \chpl{!=} functions as a
convenience.

\subsection{Interface}

The record equality and inequality functions have the signatures:
\begin{chapel}
proc ==(a : R, b : R) : bool ;
proc !=(a : R, b : R) : bool ;
\end{chapel}

\subsection{Semantics}

The record equality function performs a recursive field-wise comparison between the
operands and return \chpl{true} if they all compare equal and \chpl{false} otherwise.  The
output of the record inequality function is symmetrical, returning \chpl{false} if all of
the fields compare equal and \chpl{true} otherwise.

\subsection{Visibility}

Both record equality and record inequality functions may be overridden.

\subsection{Usage}

The compiler and module code may call record equality and inequality functions from
several locations.




\end{document}

\end{document}
